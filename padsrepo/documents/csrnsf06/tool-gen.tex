
\subsubsection{Automatic Tool Generation}

The current \pads{}\ system can automatically generate tools for
creating statistical summaries of data sources, querying
data sources and outputting data in standard formats such as
\xml.  Unfortunately, it is not easy to extend the system with new generic tools,
nor does the current tool generation process allow all tools to be generated for all formats.
In this section, we propose several innovative ways of improving our
automatic tool generation infrastructure.

\paragraph*{Generic Interface Generation}
Currently, in order to generate any tool, the \pads{} compiler will 
produce a new version of a parser (or printer) that reads (or writes)
data in the format in question.  As control traverses the ad hoc data, within the parser (printer) code,
parts of the data are passed to explicit processing functions that 
convert to XML, gather statistical data, generate histogram buckets, etc.
As it stands, there is different traversal code for each different tool.
In order to add a new tool to the PADS tool suite, it is necessary to modify the compiler
to generate a new sort of traversal.  Hence, the barrier to adding new tools
is very high.  Unfortunately, this prevents independent researchers from adding more and more
useful tools to the suite of tools that apply to any PADS description.

We plan to study software architectures that will allow the compiler to generate
flexible and highly efficient, but tool-independent traversals for a particular data format.  
These tool-independent traversals will match a generic tool interface.  If designed correctly, 
tool implementers will be able to write separate tool-specific code and then link that code
against a given tool-independent traversal.  Our revised design will take advantage of
recent research in the functional programming languages community on type-directed 
programming~\cite{crary+:intensional-types,LPJ03,syb2,SYB3}
as we have observed that each of our tools is actually an example 
of a type-directed program -- in other words, the structure of the PADS type directs run-time 
execution of the tool.

\paragraph*{Attribute-based Tool Generation}
One important reason that automatic tool generation is brittle
is that many of the tools must assume that the data passed to them fits a generic
shape: it must be an optional {\em header} followed by
a {\em body} that consists of a sequence of records 
terminated by the end of source.  
When the data does not fit the assumed shape, automatic
tool generation cannot be used.  Moreover, in order to specify 
the header and body sections, a programmer must write a complicated series of
C macros that specialize the tool.  This process is obviously highly 
error-prone and extremely inflexible.  We need a more flexible 
tool generation paradigm that allows data analysts to manipulate
any data format they might come across, and also to
analyze any subparts of a data set they deem important.

We propose to develop a more principled and substantially more robust
approach to automatic tool generation by extending \pads{}\ with a
high-level attribute system.  {\em Attributes} are tokens that may be
attached to \pads{}\ specifications and that communicate semantic
information to tools such as the statistical analyzer and the XML generator.  
For example, if a data analyst wanted to use the statistical
summary tool, he or she might attach the {\cd{summary}} attribute to
the specific parts of a PADS description that describe data that must
be summarized.  Likewise, to generate \xml{}\ from a portion of the
data, the analyst might attach the {\cd{xml}} attribute to the
appropriate component of the \pads{} description.  In addition to
enabling more robust tool generation, attributes will make our
generated tools much more flexible.  Analysts will be able to use
attributes to select portions of the data that they wish to analyze or
manipulate.

In order to make this new attribute-based tool generation system work,
we must do considerable research in language design and we must
re-architect some of the current tool base.  The language design is
nontrivial because attribute specifications must be separated from the
basic data format description.  The attributes need to the separated
because they change frequently and depend upon the current needs of
the data analyst, whereas the basic data format description is
independent of any given tool or any given analyst, and ideally
persists indefinitely.


% \paragraph*{Exploiting Semantic Constraints}
% In addition to being brittle, the current \pads{}\ tools suffer from
% performance penalties because there is no automatic way to communicate
% semantic information from the data source to the tool.  The query
% engine tool, in particular, could benefit tremendously from semantic
% information such as the property that a given field acts as a {\em
% key} for a certain record (\ie\  each record of this type in the data
% source has a unique key field) or the property that an array of
% records is {\em sorted}.

% We plan to extend the basic attribute system described above with
% user-defined attributes that are associated with semantic constraints
% such as the \cd{key} attribute or the \cd{sorted} attribute.  When we
% generate tools from a \pads{} description and associated attributes,
% the generated parser will check the semantic constraints as it reads
% new data.  On the other hand, a downstream tool such as the query
% engine will be able to assume that the semantic constraints hold
% (provided the parse descriptor indicates the data is error-free) and
% it will be able to exploit this knowledge to optimize its query plan.
% Overall, we believe attributes with semantic constraints will provide
% an efficient and robust means of communicating information between
% tools.  However, once again, we must do much more research to develop
% the right language design and effective tool interfaces.


%\label{subsec:general}

% We start by showing in \figref{figure:dibbler-filter} a simple use of 
% the core library to clean and normalize \dibbler{} data. After initializing
% the \pads{} library handle and opening the data source, the code sets
% the mask to check all conditions in the \dibbler{} description except the
% sorting of the timestamps.  We have omitted from the figure the code to read and write the header. 
% The code then echoes error records to one file and cleaned ones to another.
% The raw data has two different representations of unavailable phone numbers:
% simply omitting the number altogether, which corresponds to the \cd{NONE}
% branch of the \kw{Popt}, or having the value \cd{0} in the data.  
% The function \cd{cnvPhoneNumbers} unifies these two representations 
% by converting the zeroes into \cd{NONE}s.  The function \cd{entry_t_verify}
% ensures that our computation hasn't broken any of the semantic properties
% of the in-memory representation of the data.
% \begin{figure}[t]
% \begin{small}
% \begin{center}
% \begin{centercode}
P\_t                  *p;
entry\_t              entry;
entry\_t\_pd           pd;
entry\_t\_m            mask;
    ...
P\_open(&p, 0, 0);
P\_io\_fopen(p, "sirius/data/2004.11.11");
    ...
entry\_t\_m\_init(p, &mask, P\_CheckAndSet);
mask.events.compoundLevel = P\_Set;
    ...
\kw{while} (!P\_io\_at\_eof(p)) \{
  entry\_t\_read(p, &mask, &pd, &entry);
  \kw{if} (pd.nerr > 0) \{
    entry\_t\_write2io(p, ERR\_FILE, &pd, &entry);
  \} \kw{else} \{
    cnvPhoneNumbers(&entry);
    \kw{if} (entry\_t\_verify(&entry)) \{
      entry\_t\_write2io(p, CLEAN\_FILE, &pd, &entry);
    \} \kw{else} \{
      error(2, "Data transform failed.");
    \}
  \}
\}
\end{centercode}

% \caption{Code fragment to filter and normalize \dibbler{} data.}
% \label{figure:dibbler-filter}
% \end{center}
% \end{small}
% \end{figure}

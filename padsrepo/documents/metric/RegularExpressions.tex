\section{Complexity Metric}

The PADS structure has two levels, namely \textit{Base types} and
\textit{structured types}. The base types are already somewhat complex.
Many of the base types are defined using \textit{regular expressions}.
Structured types provide further structure on top of the base types.
Structured types include \textbf{Pstruct}, \textbf{Punion},
\textbf{Parray}, and \textbf{Penum}.

The complexity metric must take into account both the structure
described by the PADS structured types, and the structure inherent in
the base types and their underlying regular expressions. Regular
expressions can be arbitrarily complex, however PADS base types do use
only relatively simple regular expressions. Some of the PADS base
types use a rich variety of these simple regular expressions. For
example, the time base type is based on the C-library \textit{tm}
function, which supports at least 40 kinds of time and data
specification.

To get started, we will make some simplifying assumptions about the
PADS structures, and the underlying regular expressions. We need to
verify that these assumptions match the initial test cases for PADS.

\subsection{Requirements}

\begin{description}

\item [Compositionality]
The metric must be compositional in terms of the PADS structure.
For example, based on the complexity assigned to each of PADS
types $P_1$, $P_2$, and $P_3$, it should be possible to compute
the complexity for the PADS structure having $P_1$, $P_2$, and $P_3$
as fields.

\item [MDL Principle]
The metric must permit trade-offs between the complexity of the
description as a PADS type, and the complexity of the data under
the description. In practice, this means that the complexity metric
should have terms corresponding to different PADS constructs such
as \textbf{Pstruct}, \textbf{Punion}.

\end{description}

\subsection{Simplifying assumptions}

\begin{description}

\item [Pre]

\end{description}

\subsection{Calculating the metric}

\begin{description}

\item [Fixed Sequence]

\item [Unbounded Iteration]

\item [Selection]

\end{description}

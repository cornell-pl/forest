\subsection{Repository Specifications}
\label{ssec:repo-spec}

Our prototype \pads{} language can describe the syntax and semantic properties
of individual files.  However, diagnosing problems or simply monitoring
the health of applications in networked systems
will usually involve navigating and analyzing {\em sets of files},
not just individual logs.  After all, even individual
applications can generate complex sets of log files.  

As an example, consider the Coral content
distribution system~\cite{coral}, a typical distributed application.
Coral is currently running on PlanetLab system~\cite{planetlab}, a
testbed with 800+ machines distributed world-wide with presence in the
US, Canada, Europe and Asia.  The Coral system periodically collects
four different log files from all machines in its global network.
Coral implementers use this information to understand and debug
it's performance and identify security threats.  They regularly probe the
logs to answer specific questions such as:
\begin{itemize}
\item ``What are 10 the most popular domains and how many requests per 
  domain?'' or
\item ``How many requests were sent by the ten most busy client IPs? (An important signal of misuse)'' or
\item ``How does load look between different nodes?''
\end{itemize}

In order to answer these questions, they have written custom tools
to walk over their repository of log files.  This is a time-consuming and 
error-prone process and their software tools cannot be reused for new 
applications. Moreover, whenever they have new questions to ask, they may
need to write new tools from scratch.

In order to save time, effort, and hence money, we propose to
extend the \pads{} language so that it can describe entire
repositories of application data in a compact, modular fashion.  We
also propose to extend the \pads{} compiler so that it can generate
reuseable interfaces for parsing, traversal, error detection,
transformation and querying of this application data.
To understand how we might extend \pads{} to enable analysis of
entire repositories, let us return to the example of the Coral application, 
mentioned above.  Coral stores it's information in a collection of
log files organized as follows.

\begin{itemize}
\item {\bf Machine Names:}  At the top-level, there is a single directory for
each machine on which a Coral end-point is running.  Hence there are 800+ 
top-level directories.  Each directory
is named using the name of the relevant machine.  
\item {\bf Date-Time:}  Underneith each machine-level directory there is
a set of subdirectories -- one subdirectory for each date (and time) that
monitoring data was acquired.  
\item {\bf Application data:}  Underneith each date-time directory there
are four different kinds of log files:
\begin{itemize}
\item {\bf\tt coralwebsrv.log:}  A log file similar in structure to the log
file fragment presented in Figure~\ref{figure:clf-records} and described in 
Figure~\ref{figure:clf}.
\item {\bf\tt corald.log:}  Logs that describe to which "cluster" of the 
three-level coral hierarchy a node currently belongs, as well as some 
statistics about that nodes' view of the cluster.
\item {\bf\tt coraldnssrv.log:}  Logs of information related to the type of 
DNS query (A, NS, AAAA), the domain being queried, as well as record of 
which A and NS records were returned to that query.
\item {\bf\tt probed.log:}  Logs of traceroutes to clients.
\end{itemize}
\end{itemize}

Figure~\ref{figure:coral-repo} illustrates how we might specify the structure
of the entire repository in an extension of the \pads{} language.  First,
we include a series of single-document descriptions ({\tt corald.pads},
{\tt coraldns.pads}, {\tt coralweb.pads} and {\tt probe.pads}).  These 
included descriptions will be similar to the description shown in
Figure~\ref{figure:clf}.  Next, we begin description of the repository
structure from the bottom up.  The directory specification
called {\tt  host\_info} takes two parameters, {\tt h} and {\tt t}
where {\tt h} is intended to be the associated host and {\tt t} the
associated time the data was collected.  The first two lines of 
{\tt  host\_info} declare {\em logical fields} named {\tt host} and
{\tt time}.  These logical fields are useful for associating meta-data with
specific log files.  The generated interface will provide support for querying
against such generated meta-data.  The following four lines all have the structure:

\begin{code}
<lname> is <filename> as <description> <| <constraints> |>;
\end{code}

The {\tt <lname>} is the logical name of the field.  This logical name
will show up in the generated programmatic interface and may also
serve as a name against which queries may be written.  The 
 {\tt <filename>} component describes the name of the file in the file
system.  Inside {\tt  host\_info} declaration the {\tt <filename>} components
are simple constant strings.  However, they may also be expressed as regular expressions,
or more generally using \pads{} types.  The {\tt <description>} component
provides the name of another description that specifies the structure of the file(s) in question.
Finally, {\tt <constraints>} are used specify any additional information about the
file(s) -- ownership, permissions, creation time, {\em etc.}

The declarations of the {\tt times} directory and the {\tt repository} directory
have a similar structure to the {\tt host\_info} directory.  One difference is that
the {\tt <filename>} component is more sophisticated.  For example, the fragment
{\tt (t:Ptime)} from the {\tt times} directory specification indicates that the
set of file names that match this part of the specification must have a structure
specified by \pads{} type {\tt Ptime}.  Moreover, if a filename does have that structure,
the name will be bound to the variable {\tt t}, which can then be passed as a parameter
to other parts of the description or used in constraints (not shown).

\begin{figure}
\begin{code}
\#include corald.pads     /- imported pads description 
\#include coraldns.pads   /- imported pads description 
\#include coralweb.pads   /- imported pads description 
\#include probe.pads      /- imported pads description 
{\ }
Pdirectory host\_info (h,t) \{
  host = h;
  time = t;
  corald   is  "corald.log"       as corald\_source   <| (perm = "-rwx------") |>;
  coraldns is  "coraldnssrv.log"  as coraldns\_source <| (perm = "-rwx------") |>;
  coralweb is  "coralwebsrv.log"  as coralweb\_source <| (perm = "-rwx------") |>; 
  probe    is  "probed.log"       as probe\_source    <| (perm = "-rwx------") |>;
 \}
{\ }
Pdirectory times (h) \{
  time     is (t:Ptime)           as host_info(h,t)
\}
{\ }
Pdirectory repository \{
   hosts   is (h:Phostname)       as times(h);
\}
\end{code}
\caption{Coral Repository Description}
\label{figure:coral-repo}
\end{figure}
\documentclass{sigplanconf}

\usepackage{xspace,amsmath,
            times,
            alltt,multicol,url}

\newcommand{\cut}[1]{}
\newcommand{\reminder}[1]{{\it #1 }}
\newcommand{\poplversion}[1]{#1}
\newcommand{\trversion}[1]{}

\newcommand{\appref}[1]{Appendix~\ref{#1}}
\newcommand{\secref}[1]{Section~\ref{#1}}
\newcommand{\tblref}[1]{Table~\ref{#1}}
\newcommand{\figref}[1]{Figure~\ref{#1}}
\newcommand{\listingref}[1]{Listing~\ref{#1}}
%\newcommand{\pref}[1]{{page~\pageref{#1}}}

\newcommand{\eg}{{\em e.g.}}
\newcommand{\cf}{{\em cf.}}
\newcommand{\ie}{{\em i.e.}}
\newcommand{\etc}{{\em etc.\/}}
\newcommand{\naive}{na\"{\i}ve}
\newcommand{\role}{r\^{o}le}
\newcommand{\forte}{{fort\'{e}\/}}
\newcommand{\appr}{\~{}}

%\newcommand{\bftt}[1]{{\ttfamily\bfseries{}#1}}
\newcommand{\kw}[1]{\bftt{#1}}
\newcommand{\pads}{\textsc{pads}}
\newcommand{\padsc}{\textsc{pads/c}}
\newcommand{\ipads}{\textsc{ipads}}
\newcommand{\padsl}{\textsc{padsl}}
\newcommand{\blt}{\textsc{blt}}
\newcommand{\ddc}{\textsc{ddc}$^{\alpha}$}
\newcommand{\ddcold}{\textsc{ddc}}
\newcommand{\padsml}{\textsc{pads/ml}}
\newcommand{\padsmlbig}{\textsc{PADS/ML}}
\newcommand{\ddl}{\textsc{ddl}}
\newcommand{\C}{\textsc{c}}
\newcommand{\perl}{\textsc{perl}}
\newcommand{\ml}{\textsc{ml}}
\newcommand{\smlnj}{\textsc{sml/nj}}
\newcommand{\ocaml}{\textsc{o'caml}}
\newcommand{\java}{\textsc{java}}
\newcommand{\xml}{\textsc{xml}}
\newcommand{\xquery}{\textsc{xquery}}
\newcommand{\datascript}{\textsc{datascript}}
\newcommand{\packettypes}{\textsc{packettypes}}
\newcommand{\erlang}{\textsc{Erlang}}

\newcommand{\dibbler}{Sirius}
\newcommand{\ningaui}{Altair}
\newcommand{\darkstar}{Regulus}

%% \newcommand{\IParray}[4]{{\tt Parray} \; #1 \; \[#2, #3, #4\]}

\newcommand{\figHeight}[4]{\begin{figure}[tb]
	\centerline{
	            \epsfig{file=#1,height=#4}}
	\caption{#2}
	\label{#3}
	\end{figure}}


\begin{document}
\authorpermission
\conferenceinfo{TLDI'07} {January 16, 2007, Nice, France.}
\copyrightyear{2007}
\copyrightdata{1-59593-393-X/07/0001}

\title{Typing Ad Hoc Data}
\authorinfo{Kathleen Fisher\\
            AT\&T Labs Research\\ 
            \mono{kfisher@research.att.com}}



\maketitle{}
\begin{abstract}
Traditionally, types describe the internal data manipulated by
programs.  To accommodate the variety of desired data structures,
language designers and type theorists have developed a wide variety of
types and type constructors.  But not all useful data is in programs;
in fact, enormous amounts of it sit on disks or stream by on wires in
a dizzying array of encodings and formats.  It turns out that many of
the types developed for internal data can be used to describe external
data: tuples, records, unions, options, and lists come to mind as
obvious examples.  Perhaps more surprisingly, recursive types,
singletons, functions, parametric polymorphism, and dependent types
are relevant as well.  Using types to describe external data leads
naturally to the insight that we can reuse the same type to define an
internal data structure and to generate parsing and printing functions
to map between the two representations.  The PADS
project~\cite{padsproject} has exploited this idea, building data
description languages based on the type structure of C
(PADS/C)~\cite{fisher+:pads} and on ML
(PADS/ML)~\cite{mandelbaum+:pads-ml} and exploring the theoretical
basis for such languages with the Data Description Calculus
(DDC)~\cite{fisher+:next700ddl}.
Other groups have also leveraged this
insight, most closely the work on DataScript~\cite{gpce02} and
PacketTypes~\cite{sigcomm00}. 
Continuing the analogy, it turns out that other concepts from the
types world are also relevant to ad hoc data processing, including
generic programming, type inference, type isomorphisms, and subtyping.

In this talk, I will describe the domain of ad hoc data processing and
explain how types enable precise descriptions of such data.  I will
then explore the question of type inference, describing quantitative
techniques we are currently developing to construct a description of
ad hoc data given example instances.


\end{abstract}

\category{D.3.3}{Language Constructs and Features}{Data types and structures}

\terms
Languages

\keywords
Data description languages, domain-specific languages, ad hoc data,
dependent types


\section*{Acknowledgments}
Many people have worked on the PADS project, including 
Mark Daly, Mary Fern\'andez,
Robert Gruber, Yitzhak Mandelbaum, and David Walker. 

%\bibliographystyle{abbrv}
%\bibliography{../popl07/pads}
\begin{thebibliography}{1}

\bibitem{padsproject}
{PADS} project.
\newblock \url{http://www.padsproj.org}.

\bibitem{gpce02}
G.~Back.
\newblock {D}ata{S}cript - {A} specification and scripting language for binary
  data.
\newblock In {\em Generative Programming and Component Engineering}, volume
  2487, pages 66--77. Lecture Notes in Computer Science, 2002.

\bibitem{fisher+:pads}
K.~Fisher and R.~Gruber.
\newblock {PADS}: A domain specific language for processing ad hoc data.
\newblock In {\em {ACM} Conference on Programming Language Design and
  Implementation}, pages 295--304. {ACM} Press, June 2005.

\bibitem{fisher+:next700ddl}
K.~Fisher, Y.~Mandelbaum, and D.~Walker.
\newblock The next 700 data description languages.
\newblock In {\em {ACM} Symposium on Principles of Programming Languages},
  pages 2 -- 15, Jan. 2006.

\bibitem{mandelbaum+:pads-ml}
Y.~Mandelbaum, K.~Fisher, D.~Walker, M.~Fernandez, and A.~Gleyzer.
\newblock {PADS/ML}: {A} functional data description language.
\newblock In {\em {ACM} Symposium on Principles of Programming Languages}.
  {ACM} Press, Jan. 2007.

\bibitem{sigcomm00}
P.~Mc{C}ann and S.~Chandra.
\newblock Packet{T}ypes: {A}bstract specificationa of network protocol
  messages.
\newblock In {\em {ACM} Conference of Special Interest Group on Data
  Communications}, pages 321--333. {ACM} Press, August 2000.

\end{thebibliography}

%\appendix
\section{Language Syntax}
{\allowdisplaybreaks
\noindent
{\bf Syntax of data descriptions and other types}
\label{app:syntax-dd}
\begin{bnf}
\name{Constants} \meta{k} \::= \mcd{true} \| \mcd{false} \| \mcd{()} \| ...
\\
\name{Type Variables} \meta{\alpha}
\\
\name{Type Names} \meta{t}
\\
\name{\Core{} Types} \meta{T} \::= 
  \alpha 
\| {Pbase} 
\| M 
\nlalt \ppair x {T_1} {T_2} 
\| \precord {\nont{ffts}} 
\| \nont{tas}\;t(M) 
\nlalt \pset x T M 
\nlalt \parray T {M_{sep}} {M_{term}} 
\\
\name{\Core{} Datatypes} \meta{D} \::= 
  \mcd{datatype}\; \nont{tps}\; t(x{:}F) = \nont{b} \nlalt
  \mcd{type}\; \nont{tps}\; t(x{:}F) = T
\\
\name{Type Parameters} \meta{tps} \::= \cdot \| \alpha \| (\nont{tvs})
\\
\name{} \meta{tvs} \::= \alpha \| \alpha,\, \nont{tvs}
\\
\name{Type Arguments} \meta{tas} \::= \cdot \| T \| (\nont{ts})
\\
\name{} \meta{ts} \::= T \| T,\, \nont{tss}
\\
\name{} \meta{b} \::= \nont{cs} \| \mcd{case}\; M\; \mcd{of}\; \nont{ccs}
\\
\name{} \meta{cs} \::= c\;\mcd{of}\;T \| c\;\mcd{of}\;T \cvb \nont{cs}
\\
\name{} \meta{ccs} \::= 
  \nont{pat} \Rightarrow c\;\mcd{of}\;T \nlalt
  \nont{pat} \Rightarrow c\;\mcd{of}\;T \cvb \nont{ccs}
\\
\name{\Core{} Field Types} \meta{ffts} \::= \nont{fft} \| \nont{fft};\;\nont{ffts}
\\
\name{\Core{} Field Type} \meta{fft} \::= T \| x = T
\end{bnf}
%\begin{bnf}
\name{Constants} \meta{k} \::= \mcd{true} \| \mcd{false} \| \mcd{()} \| ...
\\
\name{Type Variables} \meta{\alpha}
\\
\name{Type Names} \meta{t}
\\
\name{\Core{} Types} \meta{T} \::= 
  \alpha 
\| {Pbase} 
\| M 
\nlalt \ppair x {T_1} {T_2} 
\| \precord {\nont{ffts}} 
\| \nont{tas}\;t(M) 
\nlalt \pset x T M 
\nlalt \parray T {M_{sep}} {M_{term}} 
\\
\name{\Core{} Datatypes} \meta{D} \::= 
  \mcd{datatype}\; \nont{tps}\; t(x{:}F) = \nont{b} \nlalt
  \mcd{type}\; \nont{tps}\; t(x{:}F) = T
\\
\name{Type Parameters} \meta{tps} \::= \cdot \| \alpha \| (\nont{tvs})
\\
\name{} \meta{tvs} \::= \alpha \| \alpha,\, \nont{tvs}
\\
\name{Type Arguments} \meta{tas} \::= \cdot \| T \| (\nont{ts})
\\
\name{} \meta{ts} \::= T \| T,\, \nont{tss}
\\
\name{} \meta{b} \::= \nont{cs} \| \mcd{case}\; M\; \mcd{of}\; \nont{ccs}
\\
\name{} \meta{cs} \::= c\;\mcd{of}\;T \| c\;\mcd{of}\;T \cvb \nont{cs}
\\
\name{} \meta{ccs} \::= 
  \nont{pat} \Rightarrow c\;\mcd{of}\;T \nlalt
  \nont{pat} \Rightarrow c\;\mcd{of}\;T \cvb \nont{ccs}
\\
\name{\Core{} Field Types} \meta{ffts} \::= \nont{fft} \| \nont{fft};\;\nont{ffts}
\\
\name{\Core{} Field Type} \meta{fft} \::= T \| x = T
\end{bnf}
%%% Local Variables: 
%%% mode: latex
%%% TeX-master: "paper"
%%% End: 


\noindent
{\bf Syntax of terms}
\label{app:syntax-terms}
\begin{bnf}
\name{Types} \meta{F} \::= 
  T           \descr{type of \pvalue{}} 
\nlalt \nont{base} \descr{values of ordinary base types} 
\nlalt \mcd{PD}    \descr{PD type}
\nlalt F * F       \descr{ordinary pairs} 
\nlalt \{\nont{fts}\}     \descr{ordinary records} 
\nlalt F \-> F     \descr{functions}
\nlalt \pstream F  \descr{streams}
\\
\name{Field Types} \meta{fts} \::= x = F \| x=F,\;\nont{fts}
\\
%\end{bnf}
%\begin{bnf}
\name{Parse Descriptors} \meta{pd} \::=   
  G \| B \| N \| S \| U
\\
\name{\Core{} Terms} \meta{N} \::=  
       Pbase[M_1](M_2)                \descr{base type constructor}
\nlalt \langle M \rangle              \descr{unit value (with singleton type M)}
\nlalt (x{=}{M_1} \mathrel{**} {M_2}) \descr{pair}
\nlalt \lcr \nont{fs} \rcr            \descr{record}
\nlalt c[M_1](M_2)                    \descr{data type constructor}
\nlalt \{x = {M_1} \cvb {M_2}\}       \descr{constrained type, with
  $M_2$ the constraint}
\nlalt \mcd{Parray}(M, M_{sep}, M_{term})   \descr{array; first element is stream}
\end{bnf}

\newpage

\begin{bnf}
\name{Terms} \meta{M} \::= 
       x                        \descr{variable}
\nlalt N                        \descr{\core{} terms}
\nlalt k                        \descr{constants}
\nlalt \nont{pd}                      \descr{pd value}
\nlalt (M_1 * M_2)              \descr{ordinary pair}
\nlalt \{\nont{fs}\}        \descr{ordinary record}
\nlalt \tfun {x_1}{x_2}{F_1}{F_2}{M}     \descr{recursive function x1 with arg x2}
\nlalt \mcd{nil}                \descr{empty stream}
\nlalt M_1 \mathrel{::} M_2     \descr{cons}
\nlalt \mcd{case}\;M\;\mcd{of}\;\nont{ms} \descr{deconstructors}
\nlalt M_1\;(M_2)               \descr{function application}
\nlalt \mcd{op}\;M                     \descr{additional uninteresting operations}
\nlalt \mcd{let}\;x = M_1\;\mcd{in}\;M_2           \descr{computation in host language}
\nlalt \mcd{cast}\;(M : T)             \descr{type annot/dependent cast?}
\\
\name{Fields} \meta{fs} \::= x = M \| x = M; \nont{fs}
\\
\name{Matches}\meta{ms} \::= 
  \nont{pat} \Rightarrow M \| \nont{pat} \Rightarrow M \cvb \nont{ms}
\end{bnf}
%{\small
\begin{verbatim}
M ::=  x                        // variable
     | Pbase[M1](M2)            // base type constructor; M1 is
                                   an argument to the type; M2 computes the rep)
     | c(M)                     // data type constructor with parameter M
     | (x:M1 ** M2)             // pair
     | {fields}                 // record
     | {x = M1 | M2}            // set-type; M2 is the predicate
     | Parray(M, Msep, Mterm)   // array; first element is stream
     | k                        // constants
     | let x = M in M           // computation in host language
     | <M>                      // unit value given singleton type M
     | (M1 * M2)                // ordinary pair
     | nil                      // empty list
     | M1 :: M2                 // cons
     | case M of MS             // deconstructors
     | fun x1(x2:F1):F2 = M     // recursive function x1 with arg x2
     | M1 (M2)                  // function application
     | cast (M : T)             // type annot/dependent cast?
     | op M                     // additional uninteresting operations
     
fields ::= x = M | x = M; fields

pd ::=   G    // good
     |   B    // bad
     |   N    // nested error
     |   S    // semantic error
     |   U    // unknown
\end{verbatim}
}

  
\newpage

\noindent
{\bf Syntax of patterns}
\label{app:syntax-pat}
\begin{bnf}
% \name{Parse Descriptors} \meta{pd} \::=   
%          G    \descr{good}
% \nlalt   B    \descr{bad}
% \nlalt   N    \descr{nested error}
% \nlalt   S    \descr{semantic error}
% \nlalt   U    \descr{unknown}
\name{\Core{} Patterns} \meta{fpat} \::=
x \| \nont{Pbase}(\nont{pat})
\nlalt \langle \nont{pat} \rangle             \descr{singleton}
\nlalt (\nont{fpat} \mathrel{**} \nont{fpat})   \descr{\core{} pair}
\nlalt \lcr \nont{ffps} \rcr                  \descr{\core{} record}
\nlalt c(\nont{fpat})                          \descr{constructor}
\nlalt \{\nont{fpat} \cvb \nont{cpat}\}        \descr{type constaint}
\nlalt \mcd{Parray}(\nont{pat}, x_{sep}, x_{term}) \descr{array with stream, sep and term.}
\nlalt \nont{fpat}\langle\langle\nont{pdpat}\rangle\rangle
\\
\name{Patterns}\meta{pat} \::= 
       \nont{fpat} \descr{\core{} pattern}
\nlalt k \| \nont{pdpat}                      \descr{constants and parse descriptors}
\nlalt (\nont{pat} * \nont{pat})              \descr{normal pair}
\nlalt \{\nont{fps}\}                        \descr{record}
\nlalt \mcd{nil} \| \nont{pat}_1 \mathrel{::} \nont{pat}_2 \descr{stream}
\\
\name{\Core{} Field Pattern} \meta{ffps} \::= x = \nont{fpat} \| x = \nont{fpat};\;\nont{ffps}
\\
\name{Constraint Pattern} \meta{cpat} \::= x \| \mcd{true} \| \mcd{false}
\\
\name{PD Pattern} \meta{pdpat}\::= x \| \nont{pd}
\\
\name{Field Pattern} \meta{fps} \::= x = \nont{pat} \| x = \nont{pat};\;\nont{fps}
\end{bnf}
%{\small
\begin{bnf}
% \name{Parse Descriptors} \meta{pd} \::=   
%          G    \descr{good}
% \nlalt   B    \descr{bad}
% \nlalt   N    \descr{nested error}
% \nlalt   S    \descr{semantic error}
% \nlalt   U    \descr{unknown}
\name{\Core{} Patterns} \meta{fpat} \::=
x \| \nont{Pbase}(\nont{pat})
\nlalt \langle \nont{pat} \rangle             \descr{singleton}
\nlalt (\nont{fpat} \mathrel{**} \nont{fpat})   \descr{\core{} pair}
\nlalt \lcr \nont{ffps} \rcr                  \descr{\core{} record}
\nlalt c(\nont{fpat})                          \descr{constructor}
\nlalt \{\nont{fpat} \cvb \nont{cpat}\}        \descr{type constaint}
\nlalt \mcd{Parray}(\nont{pat}, x_{sep}, x_{term}) \descr{array with stream, sep and term.}
\nlalt \nont{fpat}\langle\langle\nont{pdpat}\rangle\rangle
\\
\name{Patterns}\meta{pat} \::= 
       \nont{fpat} \descr{\core{} pattern}
\nlalt k \| \nont{pdpat}                      \descr{constants and parse descriptors}
\nlalt (\nont{pat} * \nont{pat})              \descr{normal pair}
\nlalt \{\nont{fps}\}                        \descr{record}
\nlalt \mcd{nil} \| \nont{pat}_1 \mathrel{::} \nont{pat}_2 \descr{stream}
\\
\name{\Core{} Field Pattern} \meta{ffps} \::= x = \nont{fpat} \| x = \nont{fpat};\;\nont{ffps}
\\
\name{Constraint Pattern} \meta{cpat} \::= x \| \mcd{true} \| \mcd{false}
\\
\name{PD Pattern} \meta{pdpat}\::= x \| \nont{pd}
\\
\name{Field Pattern} \meta{fps} \::= x = \nont{pat} \| x = \nont{pat};\;\nont{fps}
\end{bnf}
}

%%% Local Variables: 
%%% mode: latex
%%% TeX-master: "paper"
%%% End: 


\noindent
{\bf Syntax of programs}
\label{app:syntax-prog}
\begin{bnf}
\name{Program} \meta{prog} \::= 
  M              
 \nlalt D\; \nont{prog}          \descr{type declaration}
 \nlalt \mcd{val}\;x = M\;\mcd{prog} \descr{value declaration}
\end{bnf}
%{\small
\begin{verbatim}
prog ::= M              
       | D prog          // type declaration
       | val x = M prog  // value declaration
\end{verbatim}
}

}
%%% Local Variables: 
%%% mode: latex
%%% TeX-master: "paper"
%%% End: 


\end{document}

%%% Local Variables:
%%% mode: outline-minor
%%% End:


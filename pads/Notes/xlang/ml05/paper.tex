\documentclass{entcs} 
\usepackage{entcsmacro}
\usepackage{graphicx}
\usepackage{code,amsmath,amssymb}
\usepackage{math-cmds}

\newcommand{\cut}[1]{}

\newcommand{\appref}[1]{Appendix~\ref{#1}}
\newcommand{\secref}[1]{Section~\ref{#1}}
\newcommand{\tblref}[1]{Table~\ref{#1}}
\newcommand{\figref}[1]{Figure~\ref{#1}}
\newcommand{\listingref}[1]{Listing~\ref{#1}}
%\newcommand{\pref}[1]{{page~\pageref{#1}}}

\newcommand{\eg}{{\em e.g.}}
\newcommand{\cf}{{\em cf.}}
\newcommand{\ie}{{\em i.e.}}
\newcommand{\etc}{{\em etc.\/}}
\newcommand{\naive}{na\"{\i}ve}
\newcommand{\role}{r\^{o}le}
\newcommand{\forte}{{fort\'{e}\/}}

\newcommand{\bftt}[1]{{\ttfamily\bfseries{}#1}}
\newcommand{\kw}[1]{{\tt #1 }}
\newcommand{\pads}{\textsc{pads}}
\newcommand{\padsc}{\textsc{pads/c}}
\newcommand{\padsl}{\textsc{padsl}}
\newcommand{\C}{\textsc{C}}
\newcommand{\ocaml}{\textsc{O'Caml}}
\newcommand{\java}{\textsc{Java}}


% A couple of exemplary definitions:
\def\lastname{Fernandez, Fisher, Mandelbaum and Walker}
\begin{document}
\begin{frontmatter}
  \title{\datatype{}: A Language for Describing and Transforming Ad Hoc Data} 
  \author{Mary Fernandez,
    Kathleen Fisher\thanksref{attemail}}
  \address{AT\&T\\ 
    Florham Park,NJ USA} 
  \author{Yitzhak Mandelbaum,
    David Walker\thanksref{premail}}
  \address{Department of Computer Science\\ 
    Princeton University\\
    Princeton,NJ USA} 
  \thanks[attemail]{Email:\texttt{\normalshape
        mff,kfisher@research.att.com}}
  \thanks[premail]{Email:\texttt{\normalshape
        yitzhakm,dpw {\it at} cs {\it dot} princeton {\it dot} edu}}
\begin{abstract}
\datatype{} is a declarative data description language paired with an 
error-aware transformation language.  This pairing provides rich support for
programming with ad hoc data sources.  Such data is common in a wide range of domains including networking, financial analysis, biology, and physics.
\datatype{}'s data description language is based on polymorphic, recursive, and dependent data types.  From such descriptions, the compiler generates robust parsing and pretty printing code. The transformation language supports \textit{error-aware computation} by automatically maintaining an association between data and a description of the errors in that data.  
\end{abstract}
\begin{keyword}
Ad hoc data, semi-structured data, data description, data transformation, pattern matching, dependent types.
\end{keyword}
\end{frontmatter}

\section{Introduction}
\label{sec:intro}

\datascript{}~\cite{gpce02}. \packettypes{}~\cite{sigcomm00}. \padsc{}~\cite{fisher+:pads}
and \padsml{}~\cite{mandelbaum+:padsml}. Bro\cite{paxson:bro}. These
are but a few of the many languages designed for describing data
formats. In his classic paper {\em The Next 700 Programming
  Languages}, 1966~\cite{landin:700}, Landin asserts that principled
programming language design involves thinking in terms of ``families
of languages'' and choosing from a ``well-mapped space.''  However,
when it comes to the domain of processing ad hoc data, there is no
well-mapped space and no systematic understanding of the family of
languages one might be dealing with.

In our previous work, we developed the data description calculus
\ddcold{} to capture the core features of many existing data
description languages~\cite{fisher+:next700ddl}, like \padsc{},
\packettypes{} and \datascript{}. Given the broad applicability of
\ddcold{}, we wanted to use it to define the semantics of
\padsml{}. However, the polymorphic types that we wished to include in
\padsml{} can not be formalized with \ddcold{}.  In addition, both
\padsc{} and \padsml{} generate tools from data format descriptions to
{\em print} data in the specified format. For
many applications, printing data correctly can be as important as
parsing it correctly. Yet, our previous work
specified only the type and parsing semantics of \ddcold{}. 

In this work, we address both of these limitations of
\ddcold{}. First, we extend \ddcold{} with abstractions over types to
create \ddc. In the process, we also improve the \ddc\ theory, as
noted in \secref{sec:ddc-sem}. The new \ddc provides basis for
specifying the semantics of \padsml{}. Second, we specify the a
printing semantics for the new \ddc{}.  We used this new
semantics to guide the \padsml{} implementation of printing.
\secref{sec:ddc} presents the extended \ddc{} calculus, focusing on
the semantics of polymorphic types for parsing and the key elements of
the printing semantics.  We show that both parsers and printers in the
\ddc{} are type correct and furthermore that parsers produce pairs of
parsed data and parse descriptors in {\em canonical form}, and that
printers, given data in canonical form, print successfully.

In summary, this work makes the following key contributions:
\begin{itemize}
\item We have defined the formal semantics of both \padsml{} parsers 
and printers. 
\item We have proven our generated code is type safe and
well-behaved as defined by a canonical forms theorem.
\end{itemize}

%%% Local Variables: 
%%% mode: latex
%%% TeX-master: "paper"
%%% End: 


\section{Describing Data in \datatypebig{} }
\label{sec:data-description}

A \datatype{} description specifies the physical layout and semantic
properties of an ad hoc data source.  
These descriptions are composed of types: 
base types describe atomic data, while structured
types describe compound data built from simpler pieces.  Examples of
base types include 8-bit unsigned integers (\cd{Puint8}), 32-bit
integers (\cd{Pint32}), binary 32-bit integers (\cd{Pbint32}), dates
(\cd{Pdate}), strings (\cd{Pstring}), and IP addresses (\cd{Pip}).
Semantic conditions for such base types include checking that the
resulting number fits in the indicated space, \ie, 16-bits for
\cd{Pint16}.

Base types (and other types) may be parameterized by values.  This
mechanism serves both to reduce the number of base types and to permit
the format and properties of later portions of the data to depend upon
earlier portions.  For example, the base type \cd{Puint16_FW(3)}
specifies an unsigned two byte integer physically represented by
exactly three characters. The base type \cd{Pstring} takes 
a string indicating the set of acceptable terminator characters. Hence, 
\cd{Pstring(";|")} can be terminated by either a semicolon or a
vertical bar.


To describe more complex data, \datatype{} provides a collection of
type constructors derived from the type structure of functional
programming languages such as Haskell and ML.  The following
subsections will explain these structured types through a series 
of real-world examples.  Readers eager to see the complete syntax
of types should flip forward to Appendix~\ref{app:syntax-dd}.

\subsection{Simple Structured Descriptions}

The bread and butter of any \datatype{} description are the
simple structured types: tuple, record and array types for expressing
sequences of data; sum types, realized here as datatypes, for
expressing alternatives; and singleton types for expressing
the placement of particular characters in the data.  In this section, we
explain each of these items by building a description 
of the \dibbler{} data presented in \figref{figure:dibblerml}.

The \dibbler{} data source is used to record summaries of
phone orders produced at AT\&T.  Each summary
involves a date and one record per order.
Each order record contains a header followed by a sequence of events.
The header has 13 pipe separated fields: the order number, AT\&T's
internal order number, the order version, four different telephone
numbers associated with the order, the zip code of the order, a
billing identifier, the order type, a measure of the complexity of the
order, an unused field, and the source of the order data.  Many of
these fields are optional, in which case nothing appears between the
pipe characters.  The billing identifier may not be available at the
time of processing, in which case the system generates a unique
identifier, and prefixes this value with the string ``no\_ii'' to
indicate the number was generated. The event sequence represents the
various states a service order goes through; it is represented as a
new-line terminated, pipe separated list of state, timestamp pairs.
There are over 400 distinct states that an order may go through during
provisioning.  The sequence is sorted in order of increasing timestamps. 
%\figref{figure:dibbler-records} shows a small example of
%this format.
%156 different states for one order
%-rw-r--r--    1 angusm   dibbler   2187472314 Jun  9  2003 /fs/dibblerd/tlf/data/out_sum.stream
%2171.364u 31.379s 40:41.54 90.2% 0+0k 2+0io 2pf+0w
%53 had trailing t or } after zip code
%It may be apparent from this paragraph that English is a poor
%language for describing data formats!


\begin{figure*}
\begin{small}
%\begin{center}
\begin{verbatim}
0|1005022800
9152|9152|1|9735551212|0||9085551212|07988|no_ii152272|EDTF_6|0|APRL1|DUO|
9153|9153|1|0|0|0|0||152268|LOC_6|0|FRDW1|DUO|
\end{verbatim}
\caption{Miniscule example of \dibbler{} data.}
\label{figure:dibbler-records}
%\end{center}
\end{small}
\end{figure*}

\suppressfloats

\begin{figure}
\begin{code}
\kw{type} summary\_header = <"0|"> * Puint32 * <NL>
\mbox{}
\kw{datatype} dib\_ramp = 
  Ramp of Puint64 
| GenRamp of <"no\_ii"> * Puint64
\mbox{}
\kw{type} order\_header = \{
       order\_num      : Puint32;
 '|';  att\_order\_num  : Puint32;             
 '|';  ord\_version    : Puint32;         
 '|';  service\_tn     : pn\_t \kw{Popt};
 '|';  billing\_tn     : pn\_t \kw{Popt};          
 '|';  nlp\_service\_tn : pn\_t \kw{Popt};
 '|';  nlp\_billing\_tn : pn\_t \kw{Popt};
 '|';  zip\_code       : Pzip \kw{Popt};
 '|';  ramp           : dib\_ramp; 
 '|';  order\_sort     : Pstring('|');
 '|';  order\_details  : Puint32;             
 '|';  unused         : Pstring('|');
 '|';  Pstring('|')   : stream;
 '|';
\}
\mbox{}
\kw{type} event  = Pstring('|') *  <'|'> * Puint32
\mbox{}
\kw{type} events = event Parray('|',NL)
\mbox{}
\kw{type} source = summary\_header * (order\_header * events) Parray(NL,EOF)
\end{code}
\caption{\datatype{} description for \dibbler{} provisioning data.}
\label{figure:dibblerml}
\end{figure}


\figref{figure:dibblerml} gives a \datatype{} description for the
\dibbler{} phone order summaries in our syntax.  Overall, the description 
is a sequence of type definitions.  It is probably
easiest to understand the data source by reading these definitions
bottom up.

The last type definition \cd{source} is intended to be a definition of
an entire \dibbler{} data source.  It states that a
\cd{source} is a \cd{summary\_header} followed by a sequence of
objects made up of an \cd{order\_header} followed by \cd{events}.  The
tuple type constructor (\cd{T1 ** T2}) and the array type constructor
(\cd{T Parray(sep,term)}) both specify sequences of objects in a data
source.  The \cd{Parray} type depends upon two value parameters,
(\cd{sep} and \cd{term}).  The first parameter describes the syntactic
separators that may be found between elements of the array.  In this
case \cd{Peor}, the {\em end-of-record} character, may be found 
between each element
of the array.  In this case, the end-of-record character 
should be set to a newline character as there is one record per
line in the file.   
The second parameter is the
terminator for the array.  In this case, the terminator is \cd{Peof}, the
end-of-file marker.  It is often necessary to have additional
termination conditions for arrays, such as termination dependent on
the number of elements read so far, but we omitted this additional option 
for expository purposes.

The definition of \cd{events} indicates that this part of the
\dibbler{} data will contain a sequence of \cd{event}s separated by
vertical bars and terminated by an end-of-record character.  
Each \cd{event} is a
string terminated by a vertical bar, followed by a vertical bar and
ending with an unsigned 32-bit integer.  The interesting part of this
sequence is the presence of the type \cd{'|'}.  In type-theoretic
terms, this is a {\em singleton type}.  It states that one should
expect exactly the character \cd{'|'} in the input stream at this
point.  Other singletons appear in the summary header type as
\cd{"0|"} and \cd{Peor}.  Any expression that appears embedded in a type
as a singleton must be effect-free.

The type \cd{order\_header} is a record type that indicates
the data format involves the sequence of items described by
the fields of the record.  Notice that there are two different
sorts of fields: anonymous fields (a second form of singleton type) 
containing directives to parse a particular character (\cd{'|'}) or
string (\cd{"0|"})
and fields with names.  
% The second named field,
% \cd{att\_order\_num}, reveals two other proposed features of 
% \datatype: dependency and constraints.  Here,
% \cd{att\_order\_num} is constrained to be less than
% \cd{order\_num}, the value parsed in an earlier field.
% This is a relatively simple constraint on the correctness of the
% ad hoc data format.  In practice, constraints can become very rich
% involving properties such as sortedness of records in an array,
% definitions of expected characters,
% restrictions on date and time ranges, constraints on IP address
% domains, restrictions on phone number area codes and virtually 
% infinite variety of other possibilities. 

The last interesting feature in the \dibbler{} example is the
datatype definition of \cd{dib\_ramp}.  It describes
two alternatives for a portion of data, either an integer alone
or the fixed string \cd{"no\_ii"} followed by an integer.
To parse data in this format, the parser will
attempt to parse the first branch and only if it
fails will it attempt to parse the second branch.
Notice how this semantics differs from similar constructs in
regular expressions and context-free grammars, which
non-deterministically choose between alternatives.
By making alternatives deterministic, we avoid
the need to implement general backtracking search.  
Fortunately, we have yet to come across an ad hoc data source
where we wish we had nondeterminisc choice.\footnote{\pads{}
can recognize string data based on regular expressions.
Non-determinism here has been useful, but as it
has been confined to parsing elements of the \cd{Pstring} 
base type, it has had no impact on the overall parsing 
algorithm.}

\begin{figure}
\begin{code}
val COMMA  = ','
val COLON  = ':'
val SEMI   = ';'
val LPAREN = '('
val RPAREN = ')'
\mbox{}
type entry = Pstring(COLON) ** COLON ** Pfloat32
\mbox{}
datatype tree =
    Tree of LPAREN ** tree Parray(COMMA,RPAREN) ** "):" ** Puint32
  | Tip of entry
\mbox{}
type source = (tree ** SEMI) Parray(NL,EOF)
\mbox{}
{\rm Tiny fragment of Newick data:} 
\mbox{}
(((erHomoC:0.28006,erCaelC:0.22089):0.40998,(erHomoA:0.32304,
(erpCaelC:0.58815,((erHomoB:0.5807,erCaelB:0.23569):0.03586,
erCaelA:0.38272):0.06516):0.03492):0.14265):0.63594,(TRXHomo:0.65866,
TRXSacch:0.38791):0.32147,TRXEcoli:0.57336);
\end{code}
\caption{Simplified tree-shaped Newick data}
\label{fig:newick}
\end{figure}

\subsection{Recursive Descriptions}

The original \pads{} design contained analogues of all the 
simple type constructors described in the previous section.
Hence, although the syntax of types we just saw
is more compact and more elegant than the C-style syntax 
of \pads, we have yet to add any expressive power. 
The first real step forward for \datatype{} is the introduction
of recursion through the data type mechanism.  While
recursion never appeared in the telecommunications and 
networking data sources \pads{} was initially designed for, it
appears essential for describing a number of biological
data sources we have encountered.
  
A representative example of such data comes courtesy of Steven
Kleinstein, program coordinator of Princeton's Picasso project for
interdisciplinary research in computational sciences.  Kleinstein is
in the process of building a simulator to study the
proliferation of B lymphocytes during an immune response.  Data
needed for his simulations is represented in a variant of the
Newick format, which is a flat
representation of trees used by 
many biologists~\cite{newick}.  In Newick, 
leaves of the tree are string labels followed by a colon and a number.
A parent node in the tree introduces a collection of children by
placing a sequence of trees within parens.  Following the parens is a
colon and a number, as is the case for the leaf node.
The numbers represent the ``distance'' 
that separates the child from the parent.  In
Kleinstein's case, the distance is the number of mutations that occur
in the antibody receptor genes of B lymphocytes.   Each line
of a file may contain a different tree, terminated by a semi-colon.

\figref{fig:newick} gives a description of Newick and a short fragment 
of example data.  Despite the relative complexity of the structure of the data,
the description is remarkably concise and elegant.  Notice that the data type
definition of \cd{tree} is recursive --- we have not been able to find any
effective description of this data source without it.

\begin{figure}
  \centering
  \small
\begin{verbatim}
 2:3004092508||5001|dns1=abc.com;dns2=xyz.com|c=slow link;w=lost packets
 |INTERNATIONAL
 3:|3004097201|5074|dns1=bob.com;dns2=alice.com|src_addr=192.168.0.10;
 dst_addr=192.168.23.10;start_time=1234567890;end_time=1234568000;
 cycle_time=17412|SPECIAL
\end{verbatim}  
  \caption{Simplified network monitoring data.  This 
data containts two alarm records.  Extra newlines 
were inserted mid-record so the data would fit on a page.}
  \label{fig:darkstar-records}
\end{figure}

\subsection{Parameterized Descriptions}

The last key feature of \datatype{} is the ability to define
parameterized descriptions.  In \pads, description parameters
were limited to values.  In \datatype, descriptions
maybe parameterized by both values and types.  Type parameterization
allows descriptions to be
reused at a variety of different types.  This feature 
helps make data descriptions
more concise and allows programmers to define convenient libraries
of reusable description components.

Our final example, shown in
Figure~\ref{fig:darkstar-records}, will illustrate 
the usefulness of descriptions parameterized by
both types and values and introduce a couple of other
useful features as well.   This example displays
alarm data recorded by a network link monitor
used by the \darkstar{} project at AT\&T.  Each alarm signals some
problem with a network link.  

Now, at this point, we could write out an English description
of the \darkstar{} data, and indeed, in an earlier version of this paper
we did so.  However, English is a terrible specification language
for ad hoc data, and every description we have attempted to give has not 
only been imprecise, it has also made for some very 
boring reading.  Hence, we dispense with the informal
English and head straight for for the \datatype{} description,
which is shown in \figref{fig:darkstar-ml}.

% In our example,
% there are two classes of alarms, indicated with the integers 2 and 3.
% The alarm class is specified first, followed by a colon. When a class
% 2 alarm occurs, the ``start'' timestamp is recorded, while when a
% class 3 alarm occurs, the ``clear'' timestamp is recorded.  Next, we
% have an integer indicating the exact type of the alarm.  Following
% this, we expect two pairs of strings, with the strings of each pair
% separated by '='. The first string is a name and the second a string
% value associated with that name. The pairs are separated by a
% semicolon.  The first pair has name ``dns1'' and the second ``dns2'',
% indicating the DNS names of the parties on either end of the network
% link. Next, the record contains information about the alarm. The
% format of this information differs depending on the type of alarm. For
% alarm 5074, we include a specific set of details about the failure.
% For other alarms, there appears a sequence of name-string pairs of
% unpspecfied length.  Finally, the alarm indicates the service class of
% the network. This value is one of three strings: ``DOMESTIC'',
% ``INTERNATIONAL,'' and ``SPECIAL.'' Each field of the alarm is
% separated by a vertical bar, except where otherwise specified.

% The details included for alarm 5074 are as follows: a source and
% detination IP address, both encoded as name-IP pairs, with names
% ``src\_addr'' and ``dest\_addr.'' These are followed by two
% name-timestamp pairs, with names ``start\_time'' and ``end\_time,''
% indicating the start and end times of the problem. The timestamps are
% 10-digits indicating the number of seconds since the epoch, in
% GMT. The last field is a name-integer pair. Note that each field 
% is separated by a semicolon.

% At this point, the reader might consider for themselves whether or not
% English is an effective specification language for ad hoc data.

\begin{figure}
  \centering
  \small
  \begin{code}
type timestamp = Ptimestamp_explicit_FW(10, "%S", GMT)
\mbox{}
type 'a pnvp(f:Pstring -> bool) =
      \{ name : \{name : Pstring("=") | f name\}; 
        '='; 
        val : 'a \}
type 'a nvp(name:string) = 'a pnvp(fn s = (s = name))
type nvp_a = Pstring(";|") pnvp(fn s = true)
\mbox{}
type details = \{
      source      : Pip nvp("src_addr");
';';  dest        : Pip nvp("dest_addr");
';';  start_time  : timestamp nvp("start_time");
';';  end_time    : timestamp nvp("end_time");
';';  cycle_time  : Puint32 nvp("cycle_time")
\}
\mbox{}
datatype info(alarm_code : Puint32) =
  case alarm_code of 
    5074 => Details of details
  | _    => Generic of nvp_a Parray(';', '|')
\mbox{}
datatype service = DOMESTIC | INTERNATIONAL | SPECIAL
\mbox{}
type raw_alarm = \{
       alarm    : \{ alarm : Puint32 | alarm == 2 orelse alarm == 3\};
 ':';  start    : timestamp Popt;
 '|';  clear    : timestamp Popt;
 '|';  code     : Puint32;
 '|';  src_dns  : Pstring(";|") nvp("dns1");
 ';';  dest_dns : Pstring(";|") nvp("dns2");
 '|';  info     : info(code);
 '|';  service  : service;
\}
\mbox{}
fun checkCorr(ra:raw_alarm):bool =
  case ra of 
    \{alarm=2; start=SOME(_); clear=NONE;...\} => true
  | \{alarm=3; start=NONE;    clear=SOME(_);...\} => true
  |  _ => false
\mbox{}
type alarm = \{x:raw_alarm | checkCorr x\}
\mbox{}
type source = alarm Parray(Peor, Peof)
\end{code}

  \caption{Description of \darkstar{} data.}
  \label{fig:darkstar-ml}
\end{figure}

One of the interesting facets of this data source is the fact that
it contains multiple different types of name-value pairs.  In the
second definition from the top, \cd{pnvp}, we take
advantage of both the type and value parameterization of types to
encode all the different 
variations.\footnote{This fragment of code contains both
types \cd{Pstring} and \cd{string}.  These types are subtly different
as the first is the type of an \pvalue{} containing both a string 
representation and a parse descriptor whereas the second is an 
ordinary string.  The reader can safely ignore the distinction for now.}    
The type parameters
are specified to the left of type name, as is customary in ML.
To the right of the type name in parentheses, we give the
value parameter and its type. 
In the case of \cd{pnvp}, there is a type parameter name \cd{'a}
and a value parameter named \cd{f}.  Informally,
\cd{'a pnvp(f)} is a name-value pair where the value has type \cd{'a}
and the name must satisfy the predicate \cd{f}.  More precisely,
\cd{pnvp} is defined to be
a record with three fields.  The first field of the record
has been given a constrained type.  In general, constrained types have the
form \cd{\{x:T | M\}} where \cd{M} is an arbitrary pure boolean 
expression.  Data \cd{d} satisfies this description if it satisfies
\cd{T} and boolean \cd{M} evaluates to true when the \pvalue{} \cd{d}
is substituted for \cd{x}.  If the boolean evaluates to false, the
data contains a semantic error.  

The \cd{pnvp} type is used in both of the following type definitions.
In the case of the \cd{nvp} definition, the predicate is instantiated
with a test for a specific string but the type parameter remains 
abstract.  In the \cd{nvp\_a} 
definition, the name can be arbitrary, but the value must have 
type string. Later in the description, \cd{nvp}'s
are used with ip addresses, timestamps and integers. 

% The source type is an array of \cd{alarm}s, where each alarm is a
% \cd{raw\_alarm}, constrained to ensure that the alarm number is
% properly correlated with the timestamps.  We check this correlation
% with the function \cd{checkCorr}.  The type \cd{raw\_alarm} closely
% follows the description above. We highlight a few important features.
% First, we note that the type of the field \cd{info} depends on the
% alarm code, reflecting the text above. More interestingly, the type
% \cd{info} is implemented with a switched datatype, deciding how to
% parse based on the parameter \cd{alarm\_code}.  Next, we note that the
% description includes five different types of name-value pairs. We take
% advantage of both the type and value parameterization of types to
% encode all of these pair types based on one common description,
% \cd{pnvp}. This type is polymorphic in the type of the value and takes
% an arbitrary constraint \cd{c} as an argument. The type \cd{nvp} is
% polymorphic in the type of the value, but takes the expected name of
% the string as an argument. 

The \darkstar{} description also introduces two new forms of
datatypes.  The first new form appears in the \cd{info} type.
This datatype is parameterized by an \cd{alarm\_code}.
Rather than 
determine the branch of the datatype to choose
based on the data about to be parsed, the decision is made based on
the \cd{alarm\_code}, data which has likely been extracted from some previous
point in the source.  More specifically, if the alarm code is
\cd{5074}, the format specification given by the 
\cd{Details} constructor will be used to parse
the current data.  Otherwise, the format given by the \cd{Generic} constructor
will be used to parse the current data.  

The second new form of datatype appears in the  \cd{service} type.
Here, the constructors specify no types for their arguments.  
Our convention in this case
is to look for string data that matches the name of the
constructor. For example, the constructor \cd{DOMESTIC} 
is associated with the string
``DOMESTIC.'' This type could have been specified (more verbosely)
using the ordinary form for datatypes together with singletons:
\begin{code}
datatype service =
    DOMESTIC      of "DOMESTIC"
  | INTERNATIONAL of "INTERNATIONAL"
  | SPECIAL       of "SPECIAL"
\end{code}


\subsection{Complete Syntax}

Appendix~\ref{app:syntax-dd} presents the complete syntax of
\datatype{} data descriptions.

\cut{
%\subsection{The \pads{} Weltanshauung}
\subsection{Error Representation}

The data description primitives of the \pads{} language are types.
However, more than describing data alone, these types describe a
transformation from data in an external format to data in an internal
format.  Yet, in \pads{}, data does not live alone. A critical benefit
of the parsing is the meta-data gained during the transformation
process.  Therefore, the result of a parse is a pair consisting of a
canonical in-memory representation of the data and meta-data for the
parse.  We refer to the meta-data as the {\em parse descriptor} (PD).
The parse descriptor may hold a variety of bits of information
including the position of the data in the file and a characterization
of the possible errors in the data.  There are several different sorts
of errors that may arise.  Syntactic errors occur when a parser cannot
read a valid item of the right type from the file (eg: of parser
attempts to read an integer to find the character '?'  instead).
Semantic errors occur when a parser can read an item but it does not
satisfy the appropriate semantic condition (eg: the parse finds an
integer in the file, but the integer is $-1$ when it should be greater
than zero).  Furthermore, the \pads{} system does not view errors as
fatal. Through the parse descriptor, \pads{} is able to record the
presence of errors and continue (performing recovery as necessary),
instead of throwing an exception or halting the parse as may be done
in more conventional, ``data-only'' systems. Here, the parse
descriptors play a critical role in the functioning of the system,
beyond their otherwise ``passive,'' descriptive role.
}
%%% Local Variables: 
%%% mode: latex
%%% TeX-master: "paper"
%%% End: 


\section{Transforming Data in \datatype{}}
\label{sec:data-transformation}

\datatype{} is more than a data description language. It is also a
functional programming language with extensive support for data
transformation. Most importantly, \datatype{} is not merely a
combination of these two language elements - data description and
transformation - but a synthesis.  The meta-data learned during the
process of parsing the data based on its description is not lost once
the parsing is finished, but instead plays a prominent role in the
programmatic elements of the language. As we are particularly
concerned with error-related meta-data, we term this synthesis
``error-aware computing.''

\subsection{Language Design for Error-aware Computing}

A typical \datatype{} program begins with parsing a data source. The
result of the parse is a pair of a representation of the parsed data
and a parse descriptor. Many transformations and analyses of data are
concerned with both the data and the meta-data. Therefore, the
language directly supports manipulating them together. To begin,
\datatype{} integrates the two together into one element, which we
term a \pvalue{}. This integrated value pairs a parse descriptor with
a data element at every level of a data structure. That is, the
subcomponents of a data structure (if any) are themselves \pvalue{}.
Next, associated with the set of \pvalue{} are data constructors and
patterns (destructors) designed to enable the programmer to program
with \pvalue{} in a natural way.  Finally, \datatype{} carefully
controls the use of \pvalue{} to maintain the invariant that a
\pvalue{}'s parse descriptor accurately describes the representation.
Critical to this last feature is \datatype{}'s type system, which is
designed with this invariant in mind. We will further discuss the type
system towards the end of the section, after describing some features
of the language itself.

\begin{figure}
  \centering
  {\small
\begin{verbatim}
M ::=  x                        // variable
     | Pbase[M1](M2)            // base type constructor; M1 is
                                   an argument to the type; M2 computes the rep)
     | c(M)                     // data type constructor with parameter M
     | (x:M1 ** M2)             // pair
     | {fields}                 // record
     | {x = M1 | M2}            // set-type; M2 is the predicate
     | Parray(M, Msep, Mterm)   // array; first element is stream
     | k                        // constants
     | let x = M in M           // computation in host language
     | <M>                      // unit value given singleton type M
     | (M1 * M2)                // ordinary pair
     | nil                      // empty list
     | M1 :: M2                 // cons
     | case M of MS             // deconstructors
     | fun x1(x2:F1):F2 = M     // recursive function x1 with arg x2
     | M1 (M2)                  // function application
     | cast (M : T)             // type annot/dependent cast?
     | op M                     // additional uninteresting operations
     
fields ::= x = M | x = M; fields

pd ::=   G    // good
     |   B    // bad
     |   N    // nested error
     |   S    // semantic error
     |   U    // unknown
\end{verbatim}
}

  \caption{Syntax of Constructors}
  \label{fig:syntax-terms}
\end{figure}

\begin{figure}
  \centering
  {\small
\begin{bnf}
% \name{Parse Descriptors} \meta{pd} \::=   
%          G    \descr{good}
% \nlalt   B    \descr{bad}
% \nlalt   N    \descr{nested error}
% \nlalt   S    \descr{semantic error}
% \nlalt   U    \descr{unknown}
\name{\Core{} Patterns} \meta{fpat} \::=
x \| \nont{Pbase}(\nont{pat})
\nlalt \langle \nont{pat} \rangle             \descr{singleton}
\nlalt (\nont{fpat} \mathrel{**} \nont{fpat})   \descr{\core{} pair}
\nlalt \lcr \nont{ffps} \rcr                  \descr{\core{} record}
\nlalt c(\nont{fpat})                          \descr{constructor}
\nlalt \{\nont{fpat} \cvb \nont{cpat}\}        \descr{type constaint}
\nlalt \mcd{Parray}(\nont{pat}, x_{sep}, x_{term}) \descr{array with stream, sep and term.}
\nlalt \nont{fpat}\langle\langle\nont{pdpat}\rangle\rangle
\\
\name{Patterns}\meta{pat} \::= 
       \nont{fpat} \descr{\core{} pattern}
\nlalt k \| \nont{pdpat}                      \descr{constants and parse descriptors}
\nlalt (\nont{pat} * \nont{pat})              \descr{normal pair}
\nlalt \{\nont{fps}\}                        \descr{record}
\nlalt \mcd{nil} \| \nont{pat}_1 \mathrel{::} \nont{pat}_2 \descr{stream}
\\
\name{\Core{} Field Pattern} \meta{ffps} \::= x = \nont{fpat} \| x = \nont{fpat};\;\nont{ffps}
\\
\name{Constraint Pattern} \meta{cpat} \::= x \| \mcd{true} \| \mcd{false}
\\
\name{PD Pattern} \meta{pdpat}\::= x \| \nont{pd}
\\
\name{Field Pattern} \meta{fps} \::= x = \nont{pat} \| x = \nont{pat};\;\nont{fps}
\end{bnf}
}

%%% Local Variables: 
%%% mode: latex
%%% TeX-master: "paper"
%%% End: 

  \caption{Syntax of Patterns}
  \label{fig:syntax-pat}
\end{figure}

\begin{figure}
  \centering
  {\small
\begin{verbatim}
prog ::= M              
       | D prog          // type declaration
       | val x = M prog  // value declaration
\end{verbatim}
}

  \caption{Syntax of Programs}
  \label{fig:syntax-prog}
\end{figure}

In essence, data transformation in \datatype{} is performed by
deconstructing values using patterns and reconstructing them after
applying a transformation to their contents. For each type capable of
describing data, their is a corresponding pattern and constructor for
manipulating a \pvalue{} that corresponds to that type. We will see
and explain these patterns and constructors in the upcoming examples.
However, we do encourage interested readers to reference
\figref{fig:syntax-dd}, \figref{fig:syntax-terms},
\figref{fig:syntax-pat} and \figref{fig:syntax-prog} for a formal
presentation of the syntax. Data constructors are included under terms
\cd{M}, and patterns under \cd{pat}. We also note that while
\datatype{} focuses on supporting \pvalue{}s, the language also
includes standard constructs such as functions, pairs, and normal base
values, for example, \cd{int}.

In addition, although programmers cannot construct a parse descriptor
and pair it with a representation to create a \pvalue{}, there is a
generic pattern for all \pvalue{}s, \patreadpd {pat} {pdpat}, to
extract and read a value's parse descriptor (note that while the right
subpattern matches only the parse descriptor, the pattern on the left
matches the entire pair). There are currently four possible
descriptions in a parse descriptor: \pdgood{}, \pdbad{},\pdnest{} and
\pdsem{}. The description \pdgood{} describes data that is error free;
\pdbad{} describes data that is entirely invalid; \pdnest{} describes
data with an error in one or more subcomponents; \pdsem{} describes
data with a semantic error.  Note that in the actual language, parse
descriptors will carry more information including the location in the
source file (or a default location if generated by any user), but for
the purposes of this paper we have simplified the representation to
the one described.

{\em is this par. important?}
Notice that when the programmer
determines that a value's parse descriptor is \cd{G}ood, this implies
the entire substructure is error free and need not be checked further
for errors.  This greatly facilitates programming with data that is
expected to be error-free in the common case --- a programmer can
simply check the top-level PD and if it is indeed \cd{G}ood, she need
not clutter the rest of her code with error-checking preteens.  In
addition, when the top-level PD is \cd{G}ood, the value representation
might perhaps be optimized as the PDs for the substructures are
unnecessary.
  
The direct access to parse descriptors enables two powerful features:
error querying and error repair. For error querying, the programmer
can write a program to walk over the data source and collect
information about the errors in the data such as how many errors
occured, where they occured, etc. Given the expressiveness of the
language, the queryist is limited only by his imagination. It would be
an interesting future project to implement a generic query engine for
a well known query language such as XQuery. Lest this seem like an
unreasonable goal, a popular implementation of XQuery - Galax - is
already written in O'Caml.

Alternatively, the programmer might be interested in cleaning the data
in preparation for use in another application or even just further
transformation. Again, based on the parse descriptor, the programmer
can know the location and nature of errors in the data. This
information, combined with any relevant domain-specific knowledge
about the data can allow the programmer to effectively repair broken
records or remove them entirely. The fine-grained nature of the
meta-data can be important in minimizing how much work must be done in
such a cleaning process.

\begin{figure}
\begin{code}
type streams = entry stream * entry stream
\mbox{}
fun splitEntry (e:entry * streams) : streams =
  case e of
    (<<entry, G>> * (good * bad)) => (entry::good * bad)
  | (<<entry, _>> * (good * bad)) => (good * entry::bad)
\mbox{}    
fun splitSource (s:source) : streams =
    let (hdr * Parray(entries,_,_)) = s in 
    stream\_fold splitEntry (nil * nil) entries
\mbox{}
fun app (fds : fd * fd * fd) : unit =  
  let (fdin * fdgood * fdbad) = fds            in
  let input               = Pread::source fdin in
  let (good*bad)          = splitSource input  in
  Pwrite::(entry Parray(NL,EOF)) (Parray(good,NL,EOF)* fdgood);
  Pwrite::(entry Parray(NL,EOF)) (Parray(bad,NL,EOF)* fdbad)
\end{code}
\caption{Error filter for \dibbler{} data}
\label{fig:ex-data-clean}
\end{figure}

As an example of data cleaning, we provide the program in
\figref{fig:ex-data-clean}. It takes the \dibbler{} data source and
walks through all of the entries in the source checking for entries
with errors.  The good entries are placed in one output stream and the
bad entries are placed in another.  The idea is that the good entries
may then be further processed or directly loaded into a database
without corrupting the valuable data therein.  A human might examine
the bad entries off-line to determine the cause of errors and to
figure out how to fix the corrupted entries.

The \cd{splitEntry} function describes how to check an entry to
determine whether it is \cd{G}ood (syntactically and semantically
valid) or not.  Here, \cd{<<pat,pdpat>>} is a pattern that the
programmer uses to extract information about the parse descriptor from
a value.  \cd{pdpat} is the pattern for PDs and \cd{pat} is a pattern
for the value's substructures. Also, \cd{(good * bad)} deconstructs a pair,
while \cd{(good * entry::bad)} constructs one.

The \cd{splitSource} function pattern matches against the input
source, extracting the stream of entries, and iteratively applying the
\cd{splitEntry} function.  In this example, \cd{Parray(entry,\_,\_)}
is a pattern for array values.  An array value is a stream (in this
case \cd{entry}) coupled with a separator and a terminator.  Also in
this example, we assume \cd{fold\_stream} iteratively applies a
function to a stream.  

The function \cd{app} is responsible for reading in data from the file
descriptors it is given, splitting the data into good and bad and
writing the data back out to files.  The process of reading and
writing is dependent on the type descriptions of the ad hoc data.
Here, we write \cd{read::T} to read a data source formatted according
to \cd{T} and \cd{write::T} to write a format according to \cd{T}.
The read function will parse a file and generates an ad hoc data value
with type \cd{T}.  The write function will write an ad hoc data value
with type \cd{T} to a file.

\begin{figure}
  \centering
\begin{code}
fun checkAlarm (a : alarm): alarm opt =
    case a of 
	\{\{alarm=<<_,S>>;...\} |_\} => SOME(a)
      | _ => NONE
\mbox{}
fun collectAlarms(as : alarm stream): alarm stream =
    Stream.compact (Stream.map checkAlarm as)  
\end{code}
  \caption{Querying for Erroneous Alarm Values}
  \label{fig:ex-error-query}
\end{figure}

We present a furthre example of error querying in
\figref{fig:ex-error-query}, based on the \darkstar{} data
description. The function \cd{checkAlarm} takes an alarm set-type,
extracts the underlying raw alarm and checks its \cd{alarm} field for
a semantic error.  If the \cd{alarm} field contained a semantic error,
we return the entire alarm (wrapped in \cd{SOME}). Otherwise, we
return \cd{NONE}. The pattern \cd{\{pat | setpat\}} for set-types
matches \cd{pat} against the underlying value, and \cd{setpat} against
a boolean indicating whether the constraint was satisfied. As we don't
care about the constraint on the alarm itself, we specify a wild-card
pattern.  Also, the pattern for records is quite similar to that found
in ML. We note, however, that literal fields in a record type do not
appear as part of the pattern. 

The function \cd{collectAlarms}, then,
collects all alarm records in the alarm stream \cd{as}, whose
\cd{alarm} field has a semantic error.

\begin{figure}
  \centering
\begin{code}
fun fixAlarmVal (al : \{x:Puint32 | x==2 orelse x==3\}): 
  \{x:Puint32 | x==2 orelse x==3\} =
    case al of
      \{20 | _\} => \{x = Puint32(2) | x == 2 orelse x == 3\}
    | \{30 | _\} => \{x = Puint32(3) | x == 2 orelse x == 3\}
    | _ => al
\mbox{}
fun fixAlarm (a : alarm): alarm opt =
    case a of 
	\{\{alarm=<<al,S>>; start=s; clear=c; 
          code=cd; src_dns=sd; dest_dns=dd; 
	  info=i; service=s\} |_\} 
          => \{x=\{alarm=fixAlarmVal al; 
                 start=s; clear=c; 
                 code=cd; src_dns=sd; dest_dns=dd; 
                 info=i; service=s\}
              | checkCorr x\} 
      | _ => a
\mbox{}
fun fixAlarms(as : alarm stream): alarm stream =
    Stream.map checkAlarm as
\end{code}
  \caption{Repairing Erroneous Alarms}
  \label{fig:ex-error-repair}
\end{figure}

For an example of error repair in which the erroneous records are
fixed, please see \figref{fig:ex-error-repair}. Here, the data analyst
has discovered that some process producing alarm records was
erroneously using the values $20$ and $30$ for the alarm classes,
instead of the required $2$ and $3$. The program finds all alarm
records with semantics errors in the alarm field and corrects those
with value $20$ or $30$, leaving other untouched. We highlight three
new features. In function \cd{fixAlarmVal}, the bodies of the case
branches use the constructor for set types (\cd{\{x=M | M'\}}) and the
constructor for base types (\cd{Pbase(M)}.  In the pattern
\cd{<<_,S>>} the variable \cd{al} is bound to the entire \pvalue{} of
the \cd{alarm} \pvalue{}, rather than just the data. {\em should we
  explain this decision here?  its simpler for the language and for
  the programmer - don't have to worry about prevalues.}

\begin{figure}
  \centering
  \begin{code}
type d_alarm = \{
       alarm    :  \{ alarm : Puint32 | alarm == 2 
                                        orelse alarm == 3\};
 ':';  start :  date_time Popt;
 '|';  clear :  date_time Popt;
 '|';  code: Puint32;
 '|';  src_dns  :  nsp("dns1");
 ';';  dest_dns :  nsp("dns2");
 '|';  details  : details;
 '|';  service  :  service;
\}
\mbox{}
type g_alarm = \{
       alarm    :  \{ alarm : Puint32 | alarm == 2 
                                        orelse alarm == 3\};
 ':';  start :  date_time Popt;
 '|';  clear :  date_time Popt;
 '|';  code: Puint32;
 '|';  src_dns  :  nsp("dns1");
 ';';  dest_dns :  nsp("dns2");
 '|';  generic  : nsp_a Parray(';', '|');
 '|';  service  :  service;
\}
\mbox{}
fun splitAlarm (ra : raw_alarm): 
    d_alarm opt * g_alarm list opt
  = case ra of
        \{alarm=a; start=s; clear=c; code=cd; 
         src_dns=sd; dest_dns=dd; 
         info=Details(d); service=s\} 
        => 
        (SOME \{alarm=a; start=s; clear=c; 
               code=cd; src_dns=sd; dest_dns=dd; 
               details=d; service=s\}, 
         NONE)
      | \{alarm=a; start=s; clear=c; code=cd; 
         src_dns=sd; dest_dns=dd; 
         info=Generic(g); service=s\} 
        => 
        (NONE,
         SOME \{alarm=a; start=s; clear=c; 
               code=cd; src_dns=sd; dest_dns=dd; 
               generic=g; service=s\})    
  \end{code}
  \label{fig:ex-no-err-check}
  \caption{Shredding \darkstar{} Data Based on the {\tt info} Field}
\end{figure}

\begin{figure}
  \centering
  \begin{code}
type time = 
  \{time: Ptimestamp_explicit_FW(8, "%H:%M:%S", GMT);
   ':'; timezone: Pstring_FW(3)\}
\mbox{}
fun normalizeTimeToGMT(t : time): Ptimestamp_FW(8) =
    case t of
      \{time=t;timezone="GMT"\} => t
    | \{time=t;timezone="EST"\} => t + (5 * 60 * 60)
    | \{time=t;timezone="PST"\} => t + (8 * 60 * 60)
    | ...    
  \end{code}
  \caption{Normalizing Timestamps}
  \label{fig:ex-normalize}
\end{figure}

In addition to their primary focus, the examples above illustrated
another crucial aspect of our design of \datatype{} - error handling
is ``pay as you go''.  Despite the essential pairing of data and
meta-data, programmers can choose exactly when and where in the
transformation to attend to errors, or even to safely ignore them. The
programmer can deconstruct data without examining the parse
descriptor. That is, use of the pattern \patreadpd{pat}{pdpat} is not
required. This feature allows structural manipulation of the data (see
\darkstar{} example) without regard to errors. For example, the
program in \figref{fig:ex-no-err-check} shreds the \darkstar{} data
source into two, based on the \cd{info} field, without any reference
to parse descriptors at all.

However, such structural manipulation is not enough to fully shield
the programmer from dealing with errors. When manipulating base
values, there might be no structure to manipulate. If the parse failed
for that value, the representation will be \btm{}, and no meaningful
value can be extracted. Therefore, we have also designed our operators
to be able to safely operate on \btm{}, by propogating it from inputs
to output. For example, addition of one or more \btm{} values results
in \btm{}. We use this property in the example in
\figref{fig:ex-normalize}, where we normalize timestamp-timezone pairs
into simple timestamps in GMT time.

The many advantages of accurate parse descriptors highlight the
importance of maintaining accurate meta-data. Therefore, we need
language support for pairing newly created data with an accurate parse
descriptor, based on the type of the pair. Furthermore, we would like
to be able to check whether an existing value matches a particular
data description (i.e. type), for example, in a cast operation.
However, data descriptions can contain rich constraints that can't
necessarily be checked at compile time.

To this end, we include a mechanism for translating the constraints
found in the data descriptions into dynamic checks in the style of
contracts~\cite{contracts}. For example, in
\figref{fig:ex-error-repair}, the compiler will insert a dynamic check
at the end of function \cd{fixAlarmVal} to ensure that the constraint
on the set-type is satisfied. However, unlike previous work on
contracts, our system does not raise an exception when a dynamic check
fails, but, instead, records the failure in the parse descriptor of
the datum being checked.  The exact nature and location of these
checks will be detailed in the semantics of the language.

\subsection{Semantics Preview}

As \datatype{} is a work-in-progress, we have not fully worked out the
details of the semantics.  Therefore, we present here an overview of
the main concepts involved, together with some of the judgment forms.
...

% assigns \pvalue{}s to their own class of types.
% For this purpose, programmers can never construct PDs themselves; they
% are always constructed automatically by the compiler system.  It will
% be impossible for a programmer to corrupt the relationship between
% representation and PD accidentally. Additionally, there is no
% mechanism for accessing a representation directly.  Instead, it can
% only be accessed by deconstructing it with one of the patterns. While
% this is not critical for the safety of the language, we feel it is an
% important element of the language design that parsed data always be
% paired with its meta-data.
% Furthermore, note that while the right subpattern
% matches only the parse descriptor, the pattern on the left matches the
% entire pair, preventing direct access to the data representation, as
% discussed above.


%\cut{
\begin{itemize}
\item Compiler
  \begin{itemize}
  \item Camlp4, tcc, and fmlc (generate typedefs and prefeed defs) 
  \end{itemize}

\item Runtime system
  \begin{itemize}
  \item data structure of feed items: iData, meta data, etc
  \item implementation of feed/stream: lazy list
  \item fetching mechanism: eager fetching vs. lazy consumption, 
    http\_client library, batch fetching
  \item parse using padsML easy lib
  \item concurrency
  \item error handling
  \item discussion of selected combinators: local pairing, 
    dependent pairing (separate thread/queue)
  \end{itemize}

\item Tools library
  \begin{itemize}
  \item use of generic tool framework and feeds runtime lib
  \item use of several external ocaml libs: rrdtools, xml\_light
  \end{itemize}

\item Future work (shall we include???)
  \begin{itemize}
  \item expose meta data to the surface language
  \item a second (simplified) prefeed def with type defs only
  \end{itemize}

\item Experiments
  \begin{itemize}
  \item performance metrics: throughput, network/system latency
  \item setup (mac powerbook g4, 100Mb ethernet connection, 
    comon spec, comon nodes, random selection of nodes)
  \item two tables and graphs: throughput peaks at 
    200 nodes (chunk size), sys latency almost constant,
    system is scalable to comon (842 nodes)
  \end{itemize}
\end{itemize}
}

The \padsd{} implementation has three parts: the compiler, the runtime
system, and the built-in tools library. We describe these
parts in turn and then evaluate the overall system performance and design.

%%In this section, we describe
%%these parts and evaluate the performance of the system. We conclude
%%with a discussion of our choice to design a language
%%extension to \ocaml{} as opposed to providing a library.
%We will show that the system can easily scale
%to support PlanetLab-sized applications with 
%hundreds of nodes. 

\paragraph*{The Compiler.}
The \padsd{} compiler consists of
\cd{tcc}, the tool configuration compiler for .tc files, 
and 
\cd{fmlc}, the compiler for feed declarations (.fml files). 
Both compilers convert their sources into \ocaml{} code, which is then
compiled
and linked to the runtime libraries.  We implemented both tools with
\camlp{}, the \ocaml{} preprocessor. 

% \begin{figure}[t]
% \centering
% \begin{codebox}
% let simple_comon =
% \{\kw{frep} = fun ff ->
%  ff.\kw{all}
%  \{Combinators.\kw{format} = Comon_format.Source.parse;
%   \kw{print} = Comon_format.Source.print;
%   \kw{format_rep} = Comon_format.Source.tyrep; 
%   \kw{incremental} = false;
%   \kw{header_format} = None; 
%   \kw{locations} = sites;
%   \kw{schedule} =
%     Schedule.{\kw every} (Time.now (), 10., 
%                     Schedule.default_duration, 60.);
%   \kw{has_records} = Comon_format.__PML__has_records; 
%   \kw{pp} = None\}\}
% \end{codebox}
% \caption{Code fragment of compiled simple\_comon feed}\label{fig:compiledcomon}
% \end{figure}

%A source program is parsed into an abstract
%syntax tree defined by the Camlp4 extended syntax, and the
%code generation is done through the quotation system. 
The \cd{fmlc} compiler performs code generation in two steps.
First, the code generator emits the
type declarations for each feed.  
%% FIXME: is what is meant by representations clear here? (ksf)
Second, it generates representations for each feed description.  
The compiler constructs these representations
by extracting elements from the concurrently
generated \padsml{} libraries
and using polymorphic combinators to build structured 
descriptions.  
%Figure \ref{fig:compiledcomon} shows a fragment of
%the compiled code for
%the simple CoMon feed in Figure \ref{fig:simplecomon}.
%%While a programmer could use our combinator library directly, the
%%surface syntax provides a simple veneer that reduces the barrier to
%%entry substantially.
% in a
%lazy fashion (that is, only generate the declaration if the
%feed is used in the rest of the description).
%In the second step, 
%,
%also known as the ``prefeed''. Schedules which are
%definite times such as ``2008/09/30:12:00:00'' or ``5 mins'' are
%converted to floating point number of seconds at compile time
%to make the code more efficient.
%All O'Caml expressions embedded in the fml file are
%included without change in the compiled code. 

\paragraph*{The Runtime System.}
We implement each \padsd{} feed as a lazy list of feed items. 
Following the semantics in \secref{sec:semantics}, 
a feed item is a (meta-data, payload) pair, 
although the implementation has a more refined notion of meta-data
that includes more detailed error information.
% for HTTP errors, late item arrivals, parse errors, \etc{}

% Late arrival, (3)
% %{{\small{ 
% \begin{verbatim}
% 1: Misc HTTP error
% 2: Late arrival
% 3: Ssh host required
% 4: Remote command required for ssh
% 5: Bad message
% \end{verbatim}
% %}}} \normalsize

% % This means the feed 
% % is actually a function {\tt next}. It is evaluated 
% % only when the user program attempts to take an
% % item from the feed. The function {\tt next} returns an
% % item plus a new {\tt next} function which represents
% % the tail of the feed.
% %a data item of polymorphic type 'a and 
% %a meta data structure that corresponds 'a. 
% %The type of a feed is also the type of its data item. 
% %The type of a base feed has an option type.
% The meta data is a tree structure in which
% each node is tagged with a meta header. 
% The tree structure (also known as the meta body) resembles 
% the structure of the data item, i.e.  if the data item is a pair, 
% then the meta body is also a pair of meta data. Each leaf
% of the meta body corresponds to a base feed. Meta information
% such as the scheduled time, arrival time, location and 
% errors is stored in the leaves. The header records the summary 
% of meta information within the subtree. 

The \padsd{} runtime system is a multi-threaded concurrent
system that follows the  master-worker implementation strategy. 
%Each base feed is created and maintained by a separate 
%worker thread, and a master thread drives the combination of 
%base feeds into compound feeds.  
Each worker thread either fetches data from a specified
location and parses the data into an internal representation (the {\em rep}),
%(known as the {\em rep})
or synthesizes its data by calling a
generator function.  Using error conditions, location, scheduled time
and arrival time, the worker generates the appropriate meta-data,
pairs it with the rep and pushes the feed item onto a queue. 
%And then it generates the idata from the rep and the meta data,
%and pushes the idata into the concurrent queue.
%The workers communicated with
%the master through a concurrent queue.
The master thread pops the feed item from the queue on demand, \ie{},
when the user program requests the data. 
The worker thread is {\em eager}, which guarantees that all 
data will be fetched and archived, but the master thread is 
{\em lazy}, which allows application programs to process only relevant
data. 

%The concurrency control is implemented by the O'Caml threads
%library using standard mutex and condition variables.

We used the \ocamlnet{} library~\cite{ocamlnet2} to implement
the fetching engine. It batches concurrent fetch requests into groups
of 200, a size which balances maximizing throughput with avoiding
overwhelming the operating system with too many open sockets.


%Given a list of locations to fetch from at any one time, 
%the system converts the list into batches and fetches up to 200 locations
%per batch. The choice of 200 is a trade-off between maximizing
%the throughput and avoiding overwhelming the operating system
%with too many open sockets. 
%If the system fails to fetch an
%item due to network or system error, a feed item is still created
%with the appropriate error code written in the meta-data. 

% The \padsd{} system does not 
% automatically filter out erroneous items because errors often 
% provides very important information to users who monitors 
% distributed systems. Users could optional choose to filter
% out bad feed items using the Feed library functions.

% Compound feeds are created within the generic tool framework by
% combining base feeds using various combinator functions.
% These functions takes idatas from two feeds and creates a new idata often
% by comparing the timestamps in the two idatas. While this is done
% lazily in most of the combinators, it is not the case for
% dependent pairs. This is because the dependent feed needs to be
% created {\em eagerly} when each item from the {\em depending}
% feed arrives. Therefore we add another layer of ``pseudo-fetching"
% for the dependent feed. Here a separate thread is created for each
% dependent feed, which actively takes data from the depending
% feed, creates dependent items, and push to another concurrent
% queue of its own.  
 
\paragraph*{Tools Library.}
As explained in Section~\ref{sec:programming}, we implemented the
\padsd{} off-the-shelf tool suite using our generic tool framework. 
%Many of the tools rely upon format-specific libraries generated 
%from format specifications.
Some tools depend upon auxiliary tools.  
For instance, the feed selector calls a data selector built using
the \padsml{} generic tool framework \cite{padsml-padl} for base feeds.
%For instance, the
%feed selector relies upon a data selector generated by type-directed
%compilation of the \padsml{} description.
Other tools depend upon external libraries. For instance, the
\cd{feed2rrd} tool requires the RRD round-robin database~\cite{rrdtool} and
the \cd{feed2rss} tool uses the XML-Light package~\cite{xmllight} for
parsing and printing XML.
%The system also provides an Feed interface to a number of
%useful runtime functions such as map and fold for advanced
%users to program against the feeds.

\cut{
\begin{table*}[th]
\begin{center}
\begin{tabular}{|l|r|r|r|r|r|r|r|r|r|r|r|r|}\hline
Num of nodes&	50&	100&	150&	200&	250&	300&	350&	400&	450&	500&	550&	600 \\ \hline\hline
Net latency per node (secs)&	9&	4&	4&	4&	8.6&	5.3&	19.1&	19.5&	14.4&	7.8&	12&	13.3 \\ \hline
Sys latency per node (secs)&	0&	0&	0&	0.3&	0.2&	0.4&	0.3&	0.1&	0.3&	0.4&	0.2&	0.7 \\ \hline
%Total Latency (secs)&	9&	4.04&	4&	4.3&	8.8&	5.8&	19.4&	19.6&	14.7&	8.2&	12.3&	14 \\ \hline
Total fetch time (secs)&	9&	5&	4&	5&	23&	9&	22&	23&	26&	14&	27&	28 \\ \hline	
Throughput (items/sec)&	5.6&	20&	37.5&	40&	10.9&	33.3&	15.9&	17.4&	17.3&	35.7&	20.4&	21.4 \\ \hline
\end{tabular}
\end{center}
\caption{Performance of CoMon without archiving}
\label{tab:comon-noarch}
\end{table*}


\begin{table*}
\begin{center}
\begin{tabular}{|l|r|r|r|r|r|r|r|r|r|r|r|r|}\hline
Num of nodes&	50&	100&	150&	200&	250&	300&	350&	400&	450&	500&	550&	600 \\ \hline\hline
Net latency per node (secs)&	16&	4&	4&	4&	18.9&	6&	20.6&	22&	8.4&	13&	21.8&	21.3 \\ \hline
Sys latency per node (secs)&	0.8&	1.28&	1.4&	1.8&	1.9&	1.5&	1.6&	1.3&	1.9&	1.7&	1.7&	2.2 \\ \hline
%Total Latency (secs)&	16.8&	5.28&	5.4&	5.8&	20.8&	7.5&	22.2&	23.3&	10.3&	14.7&	23.56&	23.5 \\ \hline
Total fetch time (secs)&	17&	6&	7&	7&	27&	12&	27&	30&	19&	33&	43&	43 \\ \hline
Throughput (items/sec)&	2.9&	16.7&	21.4&	28.6&	9.3&	25&	13&	13.3&	23.7&	15.2&	12.8&	14 \\ \hline
\end{tabular}
\end{center}
\caption{Performance of CoMon with archiving}
\label{tab:comon-arch}
\end{table*}
}

\paragraph*{Experiments.} \label{sec:experiments}

\begin{figure}[t]
\begin{center}
\epsfig{file=throughput.eps, width=\columnwidth}
\epsfig{file=latency.eps, width=\columnwidth}
\caption{Average throughput and latencies per node}
\label{fig:throughput}
\shrink
\end{center}
\end{figure}

To assess performance, we measure 
the average time to fetch a data item (termed {\em network latency}), 
the average time to prepare the data item for consumption
after fetching it (termed {\em system latency}),
and the {\em throughput} of the system for the CoMon feed
description in \figref{fig:feedcomon}. 
The throughput measures the average
number of items fetched and processed per second. 
%We picked
%the comon example because it is a real life
%application that involves fetching from large number of 
%nodes (800+) at the same time, which can be viewed as a stress test. 

All the experiments were conducted on a Mac Powerbook G4 computer
with a 1.67GHz CPU and 2GB memory running Mac OS X 10.4.
%connected to the Internet via 11Mb/s wireless ethernet.
In each experiment, we randomly selected 16 subsets of PlanetLab
nodes, with increasing size from 50 to 800 in increments of 50.
%%In each experiment, we randomly selected 50 nodes,
%%100 nodes, up to 800 PlanetLab nodes
%\footnote{List taken from http://www.cs.princeton.edu/~vivek/node\_list\_all. 
%Nodes could come and go sporadically.} 
%%to form 16 node sets. 
For each set, we applied the profiler tool for the CoMon feed
twice, once without archiving and once with it, to measure the
throughput and latencies as the system fetched from these node lists. 
We repeated the experiment ten times and calculated the average values.

\figref{fig:throughput} shows the average throughput
and the average network and system latencies.
The throughput is maximized when fetching from 200 nodes because
the system supports up to 200 concurrent fetches.
%Generally, 
%the throughput goes up with the number of nodes
%until 200 and then starts declining and saturating
%as it approaches 800. Locally, 
%the throughput hits peaks at multiples of 200 since the system supports 
%up to 200 concurrent fetches.
%is the size of the
%fetching batch. At multiples of 200, the concurrent
%fetching is maximally utilized. 
%An anomaly occurred at 400 nodes, as a number of nodes were unreachable
%because of DNS failures.
% at the time of
%the experiment. 
Archiving adds to the overhead of the system and hence
reduces the throughput and increases network and system latencies. 
Note that while network latency increases with the number of nodes,  
system latency remains almost constant and relatively
low, showing that the \padsd{} runtime system adds
little overhead to the inevitable network fetching
cost. Despite the random network delays in these experiments,
the network latency is generally linear in the number of nodes. 
The system, which we have not tried to optimize, was able to fetch
data from 
800~nodes and archive the results in under 70~seconds, well under the
5 minute turnaround time currently supported by CoMon. 
Taken together, these results suggest that \padsd{} is
capable of supporting PlanetLab-scale monitoring. 

%Tables \ref{tab:comon-noarch} and \ref{tab:comon-arch}
%shows the results from two different scenarios:
%one in which the user program simply creates the comon feed without applying
%any tools on it, and one in which the comon feed is created
%and archived. In both scenarios, the system was
%able to fetch from up to 600 nodes within one minute, which is
%significantly shorter than the 5-minute turnaround time in the real
%CoMon system.
%\begin{figure}[th]
%\begin{center}
%\epsfig{file=latency.eps, width=\columnwidth}
%\caption{Average latencies per node}
%\label{fig:latency}
%\end{center}
%\end{figure}

\cut{
We plot the average throughput in Figure \ref{fig:throughput}
and the system and network latencies in Figure \ref{fig:latency}. 
As expected, both the throughput and latency suffer a little with 
archiving taking place. The throughput peaks when fetching from
200 nodes because 200 is the size of the fetching batch and
at 200 nodes, the concurrent fetching engine is maximally utilized.
The experiment for 450 nodes exhibits an anomaly as the archived
experiment takes less time than the un-archived experiment.
This is probably due to sudden delays in some of the nodes when the
no-archiving experiment is run. One obvious take-away from
Figure \ref{fig:latency} is that while the network latency may
vary greatly depending on the network condition, the system latency
stays almost constant at relatively low levels. This shows that
the \padsd{} system runtime adds little overhead to the 
inevitable network cost, which means the system could scale to
large applications. 
}

\paragraph*{Language or Library.}
A natural question that frequently arises for domain-specific languages
is whether the system is better
implemented as a library or as a language extension.  The strongest reason
for us to implement our system as a language extension is that
O'Caml  (and C, and SML, and, in fact, most functional and imperative languages) 
have poor support for generic, type-directed programming.  Unfortunately, 
many of
our key tools, including our parsers, printers, database loaders, selectors,
etc, are generic programs defined over the types of the feeds that our 
specifications
generate.  By defining a language extension, we are free to invoke a
compiler to assemble the code fragments comprising the needed applications
in a type-correct way.  

Now, in theory, the compiler is not 100\% essential to the generation of
our generic programs, but in practice, it is an enormous advantage to the 
average programmer.  
After spending months studying this issue, the best alternative we have devised
that does not use compiler support is to require that
programmers write their specification code inside of functors parameterized
in the appropriate way.  These functors can then be passed off to other 
functors implementing appropriate tool interfaces.  However this 
functor programming style is extremely hard to learn, to use and to 
explain.
%\footnote{Some of the smarter members of the group have attempted
%to explain it multiple times to one of the dumber members (David Walker)
%and he still has trouble understanding how it works.}  
Avoiding these
complications by creating a language-level interface seems to be a good,
practical solution to the problem.  For more insight into the precise
issues at hand, we recommend reading related work on the construction of
the \padsml{} infrastructure~\cite{padsml-padl} 
as well as
Hinze's work~\cite{hinz:icfp04} on 
generic programming.

Two secondary issues influencing our choice of language over library
are that (1) we could choose a pleasing and concise syntax for both
our feed and tool specifications and (2) this approach allows
smooth integration with \padsml{}, which itself is a successful 
language extension.  On the latter point, developing a system in which
data locality, temporal availability, format and properties are
all specified in one place and in one seemlessly integrated syntax was
an important goal.  We believe it improves the user programming
experience significantly.

\cut{%%%%%%%%%%%%%%%%%%

\subsection{Language or Library}
% reason (4) managing functors & modules?
Our feed language is a veneer on \ocaml{} built with the \camlp{}
preprocessor.   
A natural question is whether the system would be better
implemented as a library rather than a language extension.  For the reasons
described in the following paragraphs, we chose to present our work as a
language. 

\textbf{\textit{Automatic elimination of boilerplate code.}}
The compiler eliminates boilerplate code by
(a) generating both type declarations and values from 
descriptions (particularly record types and datatypes), something
that cannot be done in a library, (b) packaging definitions
in modules for name-space management and functor usage, 
(c) automatically filling in defaults for values omitted from
configuration files, and (d) generating complete, stand-alone executables from
declarative descriptions and configurations. 
%Much of the boilerplate 
%elimination is achieved by parsing fml and config files, filling in 
%defaults and automatically generating driver programs for the naive 
%user that string together generic tools.  Other boilerplate is avoided
%by generating both types and values automatically from descriptions
%(particularly datatype descriptions) and packaging them inside modules.

\textbf{\textit{Syntax and simplicity of coding style.}}  The underlying
interfaces are {\em very} higher-order, which, without surface sugar,
would force a complex coding style on the off-the-shelf user.
For instance, almost every line of a description would be translated
to an increasingly nested combinator application, and every variable binding
would induce a use of higher-order abstract syntax. 

\textbf{\textit{Generic programming.}} \ocaml{} (and most other 
potential host languages)
has no direct support for
the generic programming needed to implement the tool suite.  After considerable
study, the most effective way we have found to provide the required 
generic programming interface involves judicious use of unsafe casts
under the covers.  By generating type representations using the compiler,
we guarantee these casts cannot go wrong.

% \textbf{\textit{Compile-time schedule analysis.}} The synchronous
% pair and list feed comprehension combinators are intended for subfeeds
% that share the same schedule.  Using them otherwise is likely an error.
% We anticipate future work on schedule analysis to detect such errors 
% and to help us optimize our implementation.  
% %Such analysis is easier in language framework.  


\textbf{\textit{Integration with \pads{}.}}  Core
\pads{}~\cite{fisher+:pads,fisher+:popl06,fisher+:dirttoshovels,mandelbaum+:pads-ml} has had success as a language extension on top of C as well as
\ocaml{}.  Its purpose is to describe and document properties of ad hoc
data sources as well as to facilitate generation of local, single-source 
tools.  Extending such descriptions to include source location, 
availability and 
access mode helps complete the documentation in a single centralized 
specification and through a uniform notation.  It gives off-the-shelf
users everything they need in a single language.  Forcing a division 
of the specification into part library/part language would ruin its
cohesiveness, particularly in the context of
dependent feeds where there is tight interplay between access mode, location,
schedule and format.


%Having made our case for a language extension, the bulk of our implementation
%{\em is} a collection of libraries.  Hence, the intrepid hacker may eschew our
%surface syntax and program directly against the interfaces underneath.
%Whatever the programmer chooses, the central contribution of this work
%are the abstractions we provide.  Moreover
Though we believe our current design is well motivated,
we also believe the ideas presented here can transcend
their current implementation.  By defining a compact feed calculus 
with a precise semantics, we allow the possibility for
others to embed our abstractions directly in a language
such as Haskell that provides superior support for generic programming. 
%%and the ability to avoid boilerplate by automatically deriving 
%%implementations from type classes.

}%%%%%%%%%%%%%%%%%%%%


\section{Related Work}
\label{sec:related-work}

Given the importance of ad hoc data, it is perhaps surprising that
more tools do not support it.  Many, many tools simply assume that the
data they manipulate is in the right format from the beginning.  If it
is not, it is up to the user to get the data in the correct format
themselves --- he or she receives little or no help with the problem.
For instance, \xml{} and relational databases expect their inputs are
already in \xml{} or standard formats such as CSV (Comma-Separated
Values).  The development of \datatype{} is completely complementary
to research on relational and semi-structured databases as \datatype
will be explicitly designed to help with the problem (among others) of
loading one of these databases with data that is not currently in the
expected format.

As \datatype{} supports both data description and transformation, we
will divide related work between these two functions. It is
interesting to note that, to the best of our knowledge, there are no
other languages that synthesize these two functions as \datatype{}
does.

\paragraph{Data Description}

One might wonder why we do not choose to base our descriptions on
regular expressions or context-free grammars. First, regular
expressions and context-free grammars, while excellent formalisms for
describing programming language syntax, are not ideal for describing
the sort of ad hoc data we have discussed in this paper.  The main
reason for this is that regular expressions and context free grammars
do not support dependency and do not support deep semantic constraints
that are important for ensuring data integrity.

ASN.1~\cite{asn} and related systems~\cite{asdl} allow the user to
specify the {\em logical} in-memory representation and generate a {\em
  physical} on-disk format, but this doesn't help when given a
particular, fixed physical on-disk format.  \datatype{} helps solve
the latter problem.

More closely related work includes \erlang{}'s bit
syntax~\cite{erlang} and the \packettypes{}~\cite{sigcomm00} and
\datascript{} languages~\cite{gpce02}, all of which allow declarative
descriptions of physical data.  These projects were motivated by
parsing protocols, \textsc{TCP/IP} packets, and \java{} jar-files,
respectively.  Like \datatype{}, these languages have a type-directed
approach to describing ad hoc data and permit the user to define
semantic constraints.  In contrast to our work, these systems handle
only binary data and assume the data is error-free or halt parsing if
an error is detected.  Parsing non-binary data poses additional
challenges because of the need to handle delimiter values and to
express richer termination conditions on sequences of
data. Furthermore, these languages provide bindings for imperative
languages only, while we have focused here on data descriptions in a
functional setting.


\paragraph{Data Transformation}

The closest work to our own is the XDuce
language~\cite{hosoya+:xduce-journal} for processing {XML}. XDuce is a
statically typed, functional and domain-specific. It includes {XML}
documents as basic data values, and data constructors and
deconstructors (pattern matching) targeted specifically at {XML}
data. However, {XML} and \datatype{} types are syntactically quite
different ... and {XML} parsing must be perfect in order to succeed ...
 
CDuce~\cite{benzaken+:cduce} and Xtatic~\cite{gapeyev+:XtaticRuntime}
both aim to integrate the advantages of the XDuce language into a
general purpose progamming language. CDuce is a functional language,
while Xstatic is an imperative, OO language based on {C\#}.

Harmony and lenses~\cite{foster+:lenses} is about transformations, but
focuses on the issue of transforming abstract views and then
resynching them with the whole data structure. hence the need for
bi-directional.

% ... yes there are similarities and we want to take advantage of them but, I
% would argue we **do** have a new approach -- one founded on {\em error-aware
% computing}.  All these XML languages assume that their input is error-free
% and simply throw up their hands if not.  This is one feature that makes ad
% hoc data completely different from XML.  Our language focuses on handling
% this novel form of complexity.  This is a major new area of research...

% And of course our language of types and patterns is different in the details
% as well.  

% The tone of the writing should emphasize that the similarities, where they
% exist, are positive (no need to be defensive at all).  We will study these
% other languages and extract every bit of information we can from them.  We
% do not have to worry about being defensive about overlap because there are
% clearly so many novel elements of the design.  We just must make the novelty
% clear.

%%% Local Variables: 
%%% mode: latex
%%% TeX-master: "paper"
%%% End: 


\section{Conclusions}
\label{sec:conclusion}

\reviewer{ What about concurrent updates by other Forest programs or
  by users hand-editing the filestore?  Can "invariants" become
  violated as a result of such concurrency?}

In this paper, we proposed the idea of extending a modern, high-level
programming language with tightly integrated features for
processing coherent file system fragments, which we call
\filestores{}.  To demonstrate the
potential of this idea, we designed \forest{}, a
domain-specific language 
embedded in \haskell{}
for describing and managing \filestores{}.  

The \forest{} design has been informed by both
theoretical analysis and practical experience.
On the theoretical side, 
we developed a formal
calculus that models the core \forest{} functionality and
we proved that our calculus obeys round-tripping laws
derived from previous work on bi-directional programming paradigms.
On the practical side, 
we illustrated the utility of our design
by describing several example \filestores{}, and showing how
to use these descriptions
to build simple, lightweight \haskell{} scripts that query,  
analyze, and transform the example data in useful ways.
We also provided evidence that
\forest{} has effective support for building
generic, description-directed tools by 
implementing a number of such 
tools ourselves, including a filesystem visualizer,
a generic query interface, an access control checker, and
(circularly) a simple description inference engine.
An ancillary benefit of this 
engineering work was that it
served as an extensive case study in domain-specific
language design, and, as such, inspired changes in the design of
\template{}.  

For further information about \forest{}, we direct readers to
the \forest{} web site~\cite{forest-web-site}, where they may find
our fully functional open source implementation and a number of
additional examples.




\bibliographystyle{abbrv}
\bibliography{pads}
\end{document}

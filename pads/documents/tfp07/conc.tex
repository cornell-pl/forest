\section{Conclusions}
\label{sec:future}

Our work on \ddc\ has suggested a number of possible directions for
future work, of which we will briefly describe two. First, we have
begun to consider which properties we might expect to hold of the
interaction between the parser and printer of any given description.

{\em Add more detail here.}

%\edcom{M: Scan is primitive. speculate on using fancier algorithm.}
Second, we would like to enhance our support for expressing error
recovery mechanisms in \ddc. The $\pscann$ type provides a very simple
error recovery mechanism that is similar to the {\em local} error
recovery mechanisms of many early versions of the
\yacc{} parser generator~\cite{appel:mci}.  These mechanisms operate,
in essence, by deleting input tokens until a particular {\em
  synchronizing} token is found. However, the choice of where and when
to attempt error recovery, and which synchronizing tokens to use, is
not made automatically, but must be specified within the grammar
itself with special error recovery rules.  Yet, more advanced error
recovery mechanisms exist that take a substantially different approach
to error recovery. For example, {\em global error repair} ``finds the
smallest set of insertions and deletions that would turn the source
string into a syntactically correct string, even if the insertions or
deletions are not at a point where an LL or LR parser would first
report an error''~\cite{appel:mci}. In addition, global error repair
does not depend on explicit error recovery rules, but instead uses a
single, uniform mechanism for the entire grammar.

%Burke-Fisher~\cite{burke-fisher} 

We would like to support such global error repair in the \ddc\
framework. However, adding a new set of type constructors to \ddc\
would be insufficient, as it would still require that error recovery
be specified as part of the description and would be limited to local
recovery due to the orthogonal nature of types.  Instead, support for
a global mechanism would likely require that we parameterize the
parsing semantics itself by an error recovery mechanism. Furthermore,
as the exact operation of the error repair, including the choice of
which tokens to insert or delete, depends on the particular
description, we hypothesize that the error recovery mechanism itself
should be specified as an interpretation of \ddc.


{\em add genuine conclusion.}

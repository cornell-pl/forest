\section{Introduction}
\label{sec:intro}

{\em Data description languages} are a class of domain specific
languages for specifying data formats, from billing records to TCP
packets.  Examples of such languages include \datascript{},
\packettypes{}, \padsc{}, \padsml{},
Bro~\cite{gpce02,sigcomm00,fisher+:pads,mandelbaum+:padsml,paxson:bro},
among others.  In his classic paper {\em The Next 700 Programming
  Languages}, 1966~\cite{landin:700}, Landin asserts that principled
programming language design involves thinking in terms of ``families
of languages'' and choosing from a ``well-mapped space.''  However,
when it comes to the domain of specifying (ad hoc) data formats, there
is no well-mapped space and no systematic understanding of the family
of languages one might be dealing with.

In previous work, we developed the data description calculus
\ddcold{}, a calculus of simple, orthogonal type constructors,
designed to capture the core features of many existing data
description languages~\cite{fisher+:next700ddl}. Recently, we designed
a new data description language,
\padsml{}~\cite{mandelbaum+:padsml}. Yet, \ddcold{} can not describe
an essential new feature of \padsml{}: {\em polymormorphic types},
which are abstractions of types over other types.  In addition, both
\padsc{} and \padsml{} generate tools from data format descriptions to
{\em print} data in the specified format. For many applications,
printing data correctly can be as important as parsing it
correctly. Yet, our previous work specified only the type and parsing
semantics of \ddcold{}.

In this work, we address both of these limitations of \ddcold{} with
the introduction of \ddc{}. First, we extend \ddcold{} with
abstractions over types, which provides a basis for specifying the
semantics of \padsml{}. In the process, we also improve upon the
\ddcold{} theory by making a couple of subtle changes. For example, we
are able to eliminate the complicated ``contractiveness'' constraint
from our earlier work. Second, we specify a printing semantics for the
\ddc{}.

The main practical benefit of the calculus has been as a guide for our
implementation. Before working through the formal semantics, we
struggled to disentangle the invariants related to polymorphism. After
we had defined the calculus, we were able to implement type
abstractions as \ocaml{} functors in approximately a week.  Our new
printing semantics was also very important for helping us define and
check the correctness of our printer implementation.  We hope the
calculus will serve as a guide for implementations of \pads{} in
other host languages.  In this paper, we give an overview of the
calculus.  A companion technical report contains a complete formal
specification~\cite{fisher+:popl-sub-long}.

In summary, this work makes the following key contributions:
\begin{itemize}
\item We have defined a new data description calculus with which to
  formally specify semantics of both \padsml{} parsers and printers.
\item We have proven that \ddc{} parsers and printers are type safe
  and well-behaved as defined by a canonical forms theorem.
\end{itemize}

Sections~\ref{sec:ddc-syntax} through \ref{sec:ddc-sem} present the
extended \ddc{} calculus, focusing on the semantics of polymorphic
types for parsing and the key elements of the printing semantics.
Then, in \secref{sec:meta-theory}, we show that both parsers and
printers in the \ddc{} are type correct and furthermore that parsers
produce pairs of parsed data and parse descriptors in {\em canonical
  form}, and that printers, given data in canonical form, print
successfully. We discuss related work in \secref{sec:related}, and
conclude in \secref{sec:conc}.

%%% Local Variables: 
%%% mode: latex
%%% TeX-master: "paper"
%%% End: 

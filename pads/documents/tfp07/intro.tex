\section{Introduction}
\label{sec:intro}

\datascript{}~\cite{gpce02}. \packettypes{}~\cite{sigcomm00}. \padsc{}~\cite{fisher+:pads}
and \padsml{}~\cite{mandelbaum+:padsml}. Bro\cite{paxson:bro}. These
are but a few of the many languages designed for describing data
formats. In his classic paper {\em The Next 700 Programming
  Languages}, 1966~\cite{landin:700}, Landin asserts that principled
programming language design involves thinking in terms of ``families
of languages'' and choosing from a ``well-mapped space.''  However,
when it comes to the domain of processing ad hoc data, there is no
well-mapped space and no systematic understanding of the family of
languages one might be dealing with.

In our previous work, we developed the data description calculus
\ddcold{} to capture the core features of many existing data
description languages~\cite{fisher+:next700ddl}, like \padsc{},
\packettypes{} and \datascript{}. Given the broad applicability of
\ddcold{}, we wanted to use it to define the semantics of
\padsml{}. However, the polymorphic types that we wished to include in
\padsml{} can not be formalized with \ddcold{}.  In addition, both
\padsc{} and \padsml{} generate tools from data format descriptions to
{\em print} data in the specified format. For
many applications, printing data correctly can be as important as
parsing it correctly. Yet, our previous work
specified only the type and parsing semantics of \ddcold{}. 

In this work, we address both of these limitations of
\ddcold{}. First, we extend \ddcold{} with abstractions over types to
create \ddc. In the process, we also improve the \ddc\ theory, as
noted in \secref{sec:ddc-sem}. The new \ddc provides basis for
specifying the semantics of \padsml{}. Second, we specify the a
printing semantics for the new \ddc{}.  We used this new
semantics to guide the \padsml{} implementation of printing.
\secref{sec:ddc} presents the extended \ddc{} calculus, focusing on
the semantics of polymorphic types for parsing and the key elements of
the printing semantics.  We show that both parsers and printers in the
\ddc{} are type correct and furthermore that parsers produce pairs of
parsed data and parse descriptors in {\em canonical form}, and that
printers, given data in canonical form, print successfully.

In summary, this work makes the following key contributions:
\begin{itemize}
\item We have defined the formal semantics of both \padsml{} parsers 
and printers. 
\item We have proven our generated code is type safe and
well-behaved as defined by a canonical forms theorem.
\end{itemize}

%%% Local Variables: 
%%% mode: latex
%%% TeX-master: "paper"
%%% End: 

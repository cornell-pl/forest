\section{Introduction}
\label{sec:intro}

System implementers and adminstrators deal with a large variety of system logs on a
daily basis. The common tasks they need to perform on these logs include: querying,
analysis, information extraction, error detection and data transformation.
Because many of these logs are in non-standard, semi-structured formats,
there is often no ready-made tools to help them process or manage these logs.
More than often, system engineers have to resort to writing one-off
Perl or Python parser to parser these data sources, which is tedious,
error-prone and uneconomical. 

To improve the productivity of system engineers, we developed \pads
\cite{fisher+:pads,fisher+:popl06,mandelbaum+:pads-ml,padsweb},
a high level declarative specification language for describing the
physical formats of these system logs. A \pads{} description for a data
source not only documents the data format in a formal way, but also
compiles into a suite of useful processing tools such as an XML-translator,
a query engine and a statistical analyzer, and programmatic
libraries and interfaces. The users can then either manage the logs using
any of these tools, or write their own custom tools using the provided 
library functions. 

However, a significant impediment to using \pads{} is the time and 
expertise required to write a \pads{} description for a new data source.
With clear and unambiguous documentation about a data source, it takes a data 
expert anywhere from a few minutes to several hours to produce a description. 
Unfortunately, many system logs come with very little documentation or none at all.
In such cases, understanding the structure of the data and creating a correct
description can take days or weeks, especially when the logs are complex and in
very large volume. To solve this problem, we developed the \learnpads{} system
\cite{Fisher+:dirttoshovels,fisher+:sigmod08,xi08:tokenization}, which automatically
learns a \pads{} description from sample data, and thus eliminates the need
for the data users to hand-write the description. The \learnpads{} system,
while producing correct descriptions for a range of small to medium size data sources,
cannot handle very large data sources, because at the heart of the system is a
main memory algorithm which analyzes the statistics of the data in its entirety. 
Furthermore, because of the algorithm is completely automatic without any user
intervention, it does not always capture the intuitive structures of the logs
correctly, and as a result, produces overly verbose descriptions at times.

In this paper, we take one step further to improve the \learnpads{} system
such that it incrementally learns a format from a large data source,
while allowing the user to either provide an initial description to begin with,
or to interact with the learning process by modifying the learned description
at various iterations of the incremental learning.

The following is an outline of the remainder of the paper. In section \ref{sec:review},
we will review the \pads{} and \learnpads{} systems with the help of a simple web server
log example. In section \ref{sec:algo}, we will introduce the main incremental
algorithm. This is followed by a discussion on the implementation details and
some experimental results in section \ref{sec:imp}. We then present a few interesting
ideas which might further improve our system in section \ref{sec:future} and conclude
the paper in section \ref{sec:conclusion}.

% - system implementers and admins need tools to help them manage, query, extract info from,
%   detect errors in, and program with a variety of system logs. 
%
% - PADS: high level declarative specifications of physical formats
%
% - generates processing tools and programatic libraries and interfaces
%
% - automatically, incrementally learn PADS specs from sample data

\subsection{Introduction}
\label{ssec:intro}

Complex systems must be {\em monitored} to proactively find problems,
record/archive system health, oversee system operation, detect
malicious processes or security violations and perform a myriad of
other tasks.  Unfortunately, application developers may lack the time, skill
or discipline to create monitoring tools and update them 
as the underlying applications change. As a result, complex systems
are often under-monitored, and can fail in ways that the monitoring
system cannot observe and help diagnose. Failures are therefore harder
to diagnose, and may lead to significant vulnerabilities, because the 
conditions that preceded the failure are often not captured.

To address these problems, we propose to develop a unified {\em
monitoring infrastructure} that can {\bf automatically} perform a
number of tasks that are currently performed manually, if at all.
This unified system should be able to include embedded sensors to
monitor physical processes, machine-level monitors to manage server
infrastructure, intrusion detection systems to monitor network
traffic, and application-level monitors to observe applications
within a system. Monitoring should span multiple such
systems, spread across a facility or multiple locations. It should be
scalable and flexible, allowing the collection of multiple types
and large amounts of data.

%This sentence doesn't add anything over the previous paragraph.
%The goal of our research is to perform whole-system monitoring for
%complex systems, including the development of automated application
%monitoring and anomaly detection, for both small-scale and wide-area
%systems. 

% If there are other key differentiators, they should be mentioned here?
Automation is the key differentiator between our
proposal and existing off-the-shelf monitoring systems. 
As systems grow more complex and data streams increase in size and
diversity, automated monitoring provides the best hope for capturing
and acting on the behaviors of the system. {\em Consequently, our system
will semi- and fully automatically generate monitoring and
anomaly detection tools for complex systems using a novel fusion of
programming language and systems techniques}.
Specifically, our core technology will consist of
the following components: (1) a non-intrusive application traffic
sniffer that performs automated data characterization and analysis,
(2) a high-level, domain-specific programming language, called
\pads{}~\cite{fisher+:pads,fisher+:popl06,mandelbaum+:pads-ml}, that
allows users to specify the data that monitoring systems 
accumulate and present to users, (3) a compiler for \pads{} that will
automatically generate data processing libraries, interfaces and tools
from such specifications, (4) a data format inference engine capable
of automatically generating \pads{} descriptions from
collected data and (5) a scalable monitoring system, CoMon, that can
adaptively present new data sources to operators, allowing them to
analyze and visualize data across multiple domains.

% Our approach consists of three components: (1) a non-intrusive
% application traffic sniffer that performs automated characterization
% and analysis, (2) a high-level language, called PADS, capable of
% specifying the data that monitoring systems accumulate, archive, and
% present to users, and (3) a scalable monitoring system, CoMon, that
% can adaptively present new data sources and allow operators to analyze
% and visualize data across multiple domains.

% Focusing on automation and high-level processing addresses a number of
% shortcomings of modern monitoring systems. Automated format inference
% and parsing of ad-hoc data allows application data to be sniffed on
% the wire, without having to modify applications. The monitoring system
% can react as applications change or as new applications are
% introduced. By working from the live data, rather than a separate
% representation of it, the monitoring system always has access to the
% ``ground truth'' rather than some abstraction that may fail to stay
% synchronized with its associated application.  While this approach may
% sound ambitious, we believe that automated monitoring is feasible and
% practical, and we discuss our approach later in the proposal.

This approach addresses a number of shortcomings of modern monitoring
systems.  First, the \pads{} specification language provides a 
way to create high-level, uniform, easy-to-understand {\em documentation}
of all data processed or transmitted by any system component, 
including legacy components, without having to modify or alter existing
systems or
application in any way.  Second, this documentation is 
{\em executable} --- the
\pads{} compiler will automatically generate a collection of
reliable, secure, and high-performance libraries and interfaces
to perform all low-level data processing tasks such as ingesting
data from multi-source archives, detecting and reporting
errors, gathering statistics, 
transforming data into standard formats and providing
general-purpose programmatic interfaces.  Third, generated
tools are {\em trustworthy} and {\em reliable}. Unlike manually 
implemented tools, there will be no security vulnerabilities caused by
buffer overruns or other such low-level, manual coding errors.
Fourth, generated tools are {\em evolvable}. A simple change to
a specification and recompilation will generate new tools for
an evolving format.  Fifth, new data with unknown formats can be analyzed
using novel format inference algorithms that automatically generate
candidate specifications.  Users can inspect and if necessary modify
such specifications. Once satisfied, they can invoke the \pads{}
compiler to generate new tools {\em with just the press of a button}.
Sixth, such automatically generated data-processing tools and libraries 
can be linked to the CoMon monitoring
system.  This linkage allows users to query
data, view it graphically, and analyze trends, all in real
time and without having to load a database or configure the system manually.



In the following subsections, we explain the key components of
our technical approach (1-5 listed above) and our research agenda in 
more detail.

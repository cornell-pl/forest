%\documentclass[fleqn]{article}
\documentclass[nocopyrightspace]{sigplanconf}

\usepackage{xspace,pads,amsmath,math-cmds,
            math-envs,inference-rules,times,
            verbatim,alltt,multicol,proof,url}

\usepackage{code} 
\usepackage{epsfig}

%\setlength{\oddsidemargin}{0in}
%\setlength{\evensidemargin}{0in}
%\setlength{\textwidth}{6.5in}
%\setlength{\textheight}{8.5in}

\begin{document}
\title{PADS/ML: A Functional Data Description Language}
\authorinfo{Yitzhak Mandelbaum$^\ast$, Kathleen Fisher$^\dagger$, David Walker$^\ast$, Mary
  Fernandez$^\dagger$, Artem Gleyzer$^\ast$}
{$^\ast$Princeton University \ \ \ \ \ \ \ $^\dagger$AT\&T Labs Research}
 {\mono{yitzhakm,dpw,agleyzer@CS.Princeton.EDU} \ \ \ \ \mono{kfisher,mff@research.att.com}}

\newcommand{\cut}[1]{}
\newcommand{\reminder}[1]{{\it #1 }}
\newcommand{\poplversion}[1]{#1}
\newcommand{\trversion}[1]{}

\newcommand{\appref}[1]{Appendix~\ref{#1}}
\newcommand{\secref}[1]{Section~\ref{#1}}
\newcommand{\tblref}[1]{Table~\ref{#1}}
\newcommand{\figref}[1]{Figure~\ref{#1}}
\newcommand{\listingref}[1]{Listing~\ref{#1}}
%\newcommand{\pref}[1]{{page~\pageref{#1}}}

\newcommand{\eg}{{\em e.g.}}
\newcommand{\cf}{{\em cf.}}
\newcommand{\ie}{{\em i.e.}}
\newcommand{\etc}{{\em etc.\/}}
\newcommand{\naive}{na\"{\i}ve}
\newcommand{\role}{r\^{o}le}
\newcommand{\forte}{{fort\'{e}\/}}
\newcommand{\appr}{\~{}}

%\newcommand{\bftt}[1]{{\ttfamily\bfseries{}#1}}
\newcommand{\kw}[1]{\bftt{#1}}
\newcommand{\pads}{\textsc{pads}}
\newcommand{\padsc}{\textsc{pads/c}}
\newcommand{\ipads}{\textsc{ipads}}
\newcommand{\padsl}{\textsc{padsl}}
\newcommand{\blt}{\textsc{blt}}
\newcommand{\ddc}{\textsc{ddc}$^{\alpha}$}
\newcommand{\ddcold}{\textsc{ddc}}
\newcommand{\padsml}{\textsc{pads/ml}}
\newcommand{\padsmlbig}{\textsc{PADS/ML}}
\newcommand{\ddl}{\textsc{ddl}}
\newcommand{\C}{\textsc{c}}
\newcommand{\perl}{\textsc{perl}}
\newcommand{\ml}{\textsc{ml}}
\newcommand{\smlnj}{\textsc{sml/nj}}
\newcommand{\ocaml}{\textsc{o'caml}}
\newcommand{\java}{\textsc{java}}
\newcommand{\xml}{\textsc{xml}}
\newcommand{\xquery}{\textsc{xquery}}
\newcommand{\datascript}{\textsc{datascript}}
\newcommand{\packettypes}{\textsc{packettypes}}
\newcommand{\erlang}{\textsc{Erlang}}

\newcommand{\dibbler}{Sirius}
\newcommand{\ningaui}{Altair}
\newcommand{\darkstar}{Regulus}

%% \newcommand{\IParray}[4]{{\tt Parray} \; #1 \; \[#2, #3, #4\]}

\newcommand{\figHeight}[4]{\begin{figure}[tb]
	\centerline{
	            \epsfig{file=#1,height=#4}}
	\caption{#2}
	\label{#3}
	\end{figure}}


\maketitle{}

\begin{abstract}  

  Massive amounts of useful data are stored and processed in
  \textit{ad hoc} formats for which common tools like parsers and
  pretty printers do not exist.  Traditional data management systems
  provide rich infrastructure for processing well-behaved data, but
  are of little use when dealing with data in ad hoc formats.  To
  address the challenges of ad hoc data, we have designed \padsml{}, a
  declarative data description language for the \ml{} family of
  languages.  \padsml{} is based on the \ml{} type structure and
  features polymorphic, dependent, recursive datatypes for describing
  ad hoc data.  In addition, \padsml{} uses the \ml{} module system as
  a general framework to help users easily develop customized
  data-processing tools for any data format described in \padsml{}.
  We have formalized the semantics of \padsml{}'s datatypes by
  extending our previous work on the Data Description
  Calculus~\cite{fisher+:next700ddl} with type-parameterized types,
  and we have proven the resulting system ``type-correct,'' \ie{},
  generated parsers return data of the expected type.  Finally, we
  have implemented a version of \padsml{} for the \ocaml{} language.
  From a \padsml{} description, our compiler generates an \ocaml{}
  module containing the relevant type declarations, a parser for the
  description, and other tools.

\end{abstract}

\section{Introduction}
\label{sec:intro}

\datascript{}~\cite{gpce02}. \packettypes{}~\cite{sigcomm00}. \padsc{}~\cite{fisher+:pads}
and \padsml{}~\cite{mandelbaum+:padsml}. Bro\cite{paxson:bro}. These
are but a few of the many languages designed for describing data
formats. In his classic paper {\em The Next 700 Programming
  Languages}, 1966~\cite{landin:700}, Landin asserts that principled
programming language design involves thinking in terms of ``families
of languages'' and choosing from a ``well-mapped space.''  However,
when it comes to the domain of processing ad hoc data, there is no
well-mapped space and no systematic understanding of the family of
languages one might be dealing with.

In our previous work, we developed the data description calculus
\ddcold{} to capture the core features of many existing data
description languages~\cite{fisher+:next700ddl}, like \padsc{},
\packettypes{} and \datascript{}. Given the broad applicability of
\ddcold{}, we wanted to use it to define the semantics of
\padsml{}. However, the polymorphic types that we wished to include in
\padsml{} can not be formalized with \ddcold{}.  In addition, both
\padsc{} and \padsml{} generate tools from data format descriptions to
{\em print} data in the specified format. For
many applications, printing data correctly can be as important as
parsing it correctly. Yet, our previous work
specified only the type and parsing semantics of \ddcold{}. 

In this work, we address both of these limitations of
\ddcold{}. First, we extend \ddcold{} with abstractions over types to
create \ddc. In the process, we also improve the \ddc\ theory, as
noted in \secref{sec:ddc-sem}. The new \ddc provides basis for
specifying the semantics of \padsml{}. Second, we specify the a
printing semantics for the new \ddc{}.  We used this new
semantics to guide the \padsml{} implementation of printing.
\secref{sec:ddc} presents the extended \ddc{} calculus, focusing on
the semantics of polymorphic types for parsing and the key elements of
the printing semantics.  We show that both parsers and printers in the
\ddc{} are type correct and furthermore that parsers produce pairs of
parsed data and parse descriptors in {\em canonical form}, and that
printers, given data in canonical form, print successfully.

In summary, this work makes the following key contributions:
\begin{itemize}
\item We have defined the formal semantics of both \padsml{} parsers 
and printers. 
\item We have proven our generated code is type safe and
well-behaved as defined by a canonical forms theorem.
\end{itemize}

%%% Local Variables: 
%%% mode: latex
%%% TeX-master: "paper"
%%% End: 


\section{Describing Data in \padsmlbig{}}
\label{sec:padsml-overview}

A \padsml{} description specifies the physical layout and semantic
properties of an ad hoc data source.  These descriptions are composed
of types: base types describe atomic data, while structured types
describe compound data built from simpler pieces.  Examples of base
types include ASCII-encoded, 8-bit unsigned integers (\cd{Puint8}) and
32-bit signed integers (\cd{Pint32}), binary 32-bit integers (\cd{Pbint32}),
dates (\cd{Pdate}), strings (\cd{Pstring}), zip codes (\cd{Pzip}),
phone numbers (\cd{Pphone}), and IP addresses (\cd{Pip}).  Semantic
conditions for such base types include checking 
that the resulting number fits in the indicated space, \ie, 16-bits
for \cd{Pint16}.

Base types may be parameterized by \ml{} values.  This mechanism
reduces the number of built-in base types and permits base types to
depend on values in the parsed data.  For example, the base type
\cd{Puint16_FW(3)} specifies an unsigned two byte integer physically
represented by exactly three characters, and the base type
\cd{Pstring} takes an argument indicating the \textit{terminator
character}, \ie{}, the character in the source that
follows the string.

To describe more complex data, \padsml{} provides a collection of type
constructors derived from the type structure of functional programming
languages like Haskell and ML.  We explain these structured types in
the following subsections using examples drawn from data sources we
have encountered in practice.

% Readers eager to see the complete syntax
% of types should flip forward to Appendix~\ref{app:syntax-dd}.

\subsection{Simple Structured Types}

\cut{
\begin{figure*}
{\scriptsize
\begin{verbatim}
0|1005022800
9152|9151|1|9735551212|0||9085551212|07988|no_ii152272|EDTF_6|0|APRL1|DUO|10|1000295291
9153|9153|1|0|0|0|0||152268|LOC_6|0|FRDW1|DUO|LOC_CRTE|1001476800|LOC_OS_10|1001649601
\end{verbatim}
}
  \caption{Miniscule example of \dibbler{} data.}
  \label{figure:dibbler-records}
\end{figure*}
}

The bread and butter of a \padsml{} description are the simple
structured types: tuples and records for specifying ordered data,
lists for specifying homogeneous sequences of data, sum types for
specifying alternatives, and singletons for specifying the occurrence
of literal characters in the data.  We describe each of these
constructs as applied to the \dibbler{} data presented in
\figref{fig:sample-data}(b).

\dibbler{} data summarizes orders for phone service placed with AT\&T.
Each \dibbler{} data file starts with a timestamp followed by one
record per phone service order.  Each order consists of a header and a
sequence of events.  The header has 13 pipe separated fields: the
order number, AT\&T's internal order number, the order version, four
different telephone numbers associated with the order, the zip code of
the order, a billing identifier, the order type, a measure of the
complexity of the order, an unused field, and the source of the order
data.  Many of these fields are optional, in which case nothing
appears between the pipe characters.  The billing identifier may not
be available at the time of processing, in which case the system
generates a unique identifier, and prefixes this value with the string
``no\_ii'' to indicate that the number was generated. The event
sequence represents the various states a service order goes through;
it is represented as a new-line terminated, pipe separated list of
state, timestamp pairs.  There are over 400 distinct states that an
order may go through during provisioning.  The sequence is sorted in
order of increasing timestamps.  Clearly
English is a poor language for describing data formats!

\begin{figure}
\begin{code}\scriptsize
\kw{ptype} Semicolon = Pcharlit(';')
\kw{ptype} Vbar = Pcharlit('|')
\mbox{}
\kw{ptype} Pn\_t = Puint64
\kw{ptype} Pzip = Puint32
\mbox{}
\kw{ptype} Summary\_header = "0|" * Puint32 * Peor
\mbox{}
\kw{pdatatype} Dib\_ramp = 
  Ramp of Pint 
| GenRamp of "no\_ii" * Pint
\mbox{}
\kw{ptype} Order\_header = \{ 
    order\_num : Pint;  
'|'; att\_order\_num : [i:Pint | i < order\_num];  
'|'; ord\_version : Pint;  
'|'; service\_tn : Pn\_t Popt;
'|'; billing\_tn : Pn\_t Popt;  
'|'; nlp\_service\_tn : Pn\_t Popt;  
'|'; nlp\_billing\_tn : Pn\_t Popt;  
'|'; zip\_code : Pzip Popt;  
'|'; ramp : Dib\_ramp;  
'|'; order\_sort : Pstring('|');  
'|'; order\_details : Pint;
'|'; unused : Pstring('|');  
'|'; stream : Pstring('|'); 
'|'
\} 
\mbox{}
\kw{ptype} Event  = Pstring('|') * '|' * Puint32
\kw{ptype} Events = (Event,V_bar,Peor) Plist
\mbox{}
\kw{ptype} Order  = Order\_header * Events
\kw{ptype} Orders = (Order, Peor, Peof) Plist
\mbox{}
\kw{ptype} Source = Summary\_header * Orders\end{code}

  \caption{\padsml{} description for \dibbler{} provisioning data.}
  \label{figure:sirius_pml}
\end{figure}

\figref{figure:sirius_pml} contains the \padsml{} description for the
\dibbler{} data format.  The description is a sequence of type
definitions.  Type definitions precede uses, therefore the description
should be read bottom up.
The type \cd{Source} describes a complete \dibbler{} data
file and denotes an ordered tuple containing a
\cd{Summary\_header} value followed by an \cd{Orders} value.

The type \cd{Orders} uses the list type constructor
\cd{Plist} to describe a homogenous sequence of values in a data
source.  The \cd{Plist} constructor takes three parameters: on the
left, the type of elements in the list; on the right, a literal
\emph{separator} that separates elements in the list and a literal
\emph{terminator} that marks the end of the list.  
In this example, the type \cd{Orders} is a list of
\cd{Order} elements, separated by a newline, and terminated by
\cd{peof}, a special literal that describes the \emph{end-of-file
  marker}.  Similarly, the \cd{Events} type denotes a
sequence of \cd{Event} values separated by vertical bars and
terminated by a newline.

Literal characters in type expressions denote singleton types.  For
example, the \cd{Event} type is a string terminated by a
vertical bar, followed by a vertical bar, followed by a timestamp.  The
singleton type \cd{'|'} means that the data source must contain the
character \cd{'|'} at this point in the input stream.  String,
character, and integer literals can be embedded in a description and
are interpreted as singleton types, \eg{}, the singleton type
\cd{"0|"} in the \cd{Summary\_header} type
denotes the string literal \cd{"0|"}.

The type \cd{Order\_header} is a record type, \ie{}, a tuple type in
which each field may have an associated name.  The named field
\cd{att\_order\_num} illustrates two other features of
\padsml: dependencies and constraints.  Here, \cd{att\_order\_num}
depends on the previous field \cd{order\_num} and is constrained to be
less than that value.  In practice, constraints may be complex, have
multiple dependencies, and can specify, for example, the sorted order
of records in a sequence.  Constrained types have the form \cd{[x:T |
e]} where \cd{e} is an arbitrary pure boolean expression.  Data
satisfies this description if it satisfies \cd{T} and boolean \cd{e}
evaluates to true when the parsed representation of the data is
substituted for \cd{x}.  If the boolean expression evaluates to false,
the data contains a \textit{semantic} error.

The datatype \cd{Dib\_ramp} specifies two
alternatives for a data fragment, either one integer or the fixed
string \cd{"no\_ii"} followed by one integer.  The order of
alternatives is significant, that is, the parser attempts to parse the
first alternative and only if it fails, it attempts to parse the
second alternative.  This semantics differs from similar constructs in
regular expressions and context-free grammars, which
non-deterministically choose between alternatives.
\cut{Fortunately, we have yet to come across an ad hoc data
source where we wish we had nondeterministic choice.\footnote{\padsml{}
  can recognize string data based on regular expressions.
  Non-determinism here has been useful, but as it has been confined to
  parsing elements of the \cd{Pstring} base type, it has had no impact
  on the overall parsing algorithm.}
}
\cut{
\begin{figure}
\begin{code}\scriptsize
\kw{ptype} Entry = \{name: Pstring(':'); ':'; dist: Pfloat32\}
\mbox{}
\kw{pdatatype} Tree =
  Interior of '(' * Tree Plist(';',')')  * ')'
| Leaf of Entry\end{code}

%%% Local Variables: 
%%% mode: latex
%%% TeX-master: "paper"
%%% End: 

Tiny fragment of Newick data:

{
\begin{verbatim}
(((erHomoC:0.28006,erCaelC:0.22089):0.40998,
(erHomoA:0.32304,(erpCaelC:0.58815,((erHomoB:
0.5807,erCaelB:0.23569):0.03586,erCaelA:
0.38272):0.06516):0.03492):0.14265):0.63594,
(TRXHomo:0.65866,TRXSacch:0.38791):0.32147,
TRXEcoli:0.57336)
\end{verbatim}
}
  \caption{Simplified tree-shaped Newick data.  Newlines
     inserted to improve legibility.}
  \label{fig:newick}
\end{figure}
}


\subsection{Recursive Types}

\padsml{} can describe data sources with recursive structure.  An
example of such data is the Newick format, a flat representation
of trees used by biologists~\cite{newick}.
Example Newick data provided by Steven Kleinstein appears in
\figref{fig:sample-data}(c). 
The format uses properly nested
parentheses to specify a tree hierarchy.  A leaf node is a string
label followed by a colon and a number.  An interior node contains a
sequence of children nodes, delimited by parentheses, followed by a
colon and a number.  The numbers represent the ``distance'' that
separates a child node from its parent. 
In this example,  the string labels are gene names and the distances denotes the number of mutations that occur in the antibody receptor genes of B lymphocytes. \cut{Kleinstein uses this kind of data to study
the proliferation of B lymphocytes during an immune response.}
The following \padsml{} code 
describes this format:
\begin{code}\scriptsize
\kw{ptype} Entry = \{name: Pstring(':'); ':'; dist: Pfloat32\}
\mbox{}
\kw{pdatatype} Tree =
  Interior of '(' * Tree Plist(';',')')  * ')'
| Leaf of Entry\end{code}

%%% Local Variables: 
%%% mode: latex
%%% TeX-master: "paper"
%%% End: 

 
\cut{
\begin{figure}
  \centering
  \small
\begin{verbatim}
2:3004092508||5001|dns1=abc.com;dns2=xyz.com|
c=slow link;w=lost packets|INTERNATIONAL
3:|3004097201|5074|dns1=bob.com;dns2=alice.com|
src_addr=192.168.0.10;dst_addr=192.168.23.10;
start_time=1234567890;end_time=1234568000;
cycle_time=17412|SPECIAL
\end{verbatim}  
  \caption{Simplified network-monitoring data. Newlines
     inserted to improve legibility.}
  \label{fig:darkstar-records1}
\end{figure}
}

\subsection{Polymorphic Types and Advanced Datatypes}

Polymorphic types enable more concise descriptions and allow
programmers to define convenient libraries of reusable descriptions. The
description in \figref{fig:darkstar-ml} illustrates types
parameterized by both types and values.  It specifies
the format of alarm data recorded by a network-link monitor used in
the \darkstar{} project at AT\&T.  \figref{fig:sample-data}(a) contains corresponding example data. We describe the format in tandem with describing its \padsml{} description.
\begin{figure}
  \centering
  \begin{code}\scriptsize
(* Pstring terminated by ';' or '|'. *)
\kw{ptype} SVString = Pstring\_SE("/;|\\\\|/")
\mbox{}
(* Generic name value pair. Accepts predicate 
   to validate name as argument.� *)
\kw{ptype} (Alpha) Pnvp(p : string -> bool) =
      \{ name : [name : Pstring('=') | p name]; 
            '='; 
       value : Alpha \}
\mbox{}
(* Name value pair with name specified. *)
\kw{ptype} (Alpha) Nvp(name:string) = 
   Alpha Pnvp(fun s -> s = name)
\mbox{}
(* Name value pair with any name. *)
\kw{ptype} Nvp\_a = SVString Pnvp(fun \_ -> true)
\mbox{}
\kw{ptype} Details = \{
      source      : Pip Nvp("src\_addr");
';';  dest        : Pip Nvp("dest\_addr");
';';  start\_time  : Ptimestamp Nvp("start\_time");
';';  end\_time    : Ptimestamp Nvp("end\_time");
';';  cycle\_time  : Puint32 Nvp("cycle\_time")
\}

\kw{pdatatype} Info(alarm\_code : int) =
  match alarm\_code with
    5074 -> Details of Details
  | \_    -> Generic of Nvp\_a Plist(';','|')

\kw{pdatatype} Service = 
    DOMESTIC      of "DOMESTIC" 
  | INTERNATIONAL of "INTERNATIONAL" 
  | SPECIAL       of "SPECIAL"

\kw{ptype} Alarm = \{
       alarm    : [i : Puint32 | i = 2 or i = 3];
 ':';  start    : Ptimestamp Popt;
 '|';  clear    : Ptimestamp Popt;
 '|';  code     : Puint32;
 '|';  src\_dns  : SVString Nvp("dns1");
 ';';  dest\_dns : SVString Nvp("dns2");
 '|';  info     : Info(code);
 '|';  service  : Service
\} 

\kw{ptype} Source = Alarm Plist('\\n',\kw{peof})\end{code}
%
%\kw{let} checkCorr ra = ...
%\kw{ptype} Alarm = [x:Raw\_alarm | checkCorr x]



  \caption{Description of \darkstar{} data.}
  \label{fig:darkstar-ml}
\end{figure}

This data format has several variants of name-value pairs. The
\padsc{} description of this format~\cite{fisher+:popl-sub-long} must
define a different type for each variant. In contrast, the polymorphic
types of \padsml{} allow us to define the type \cd{Pnvp}, which takes
both type and value parameters to encode all the variants. As is
customary in \ml{}, type parameters appear to the left of the type
name, while value parameters and their \ml{} types appear to the
right.  The type\cd{Pnvp} has one type parameter named \cd{Alpha} and
one value parameter named \cd{p}.  Informally, \cd{Alpha Pnvp(p)} is a
name-value pair where the value is described by \cd{Alpha} and the
name must satisfy the predicate \cd{p}.

The \cd{Nvp} type reuses the\cd{Pnvp} type to define a name-value pair
whose name must match the argument string \texttt{name} but whose
value can have any type.  The \cd{Nvp\_a} type also uses the type
\cd{Pnvp}. It defines a name-value pair that permits any name, but
requires the value to have type \cd{SVString} (a string terminated by
a semicolon or vertical bar).  Later in the description, the type
parameter to \cd{Nvp} is instantiated with IP addresses, timestamps,
and integers.

% The source type is an array of \cd{alarm}s, where each alarm is a
% \cd{raw\_alarm}, constrained to ensure that the alarm number is
% properly correlated with the timestamps.  We check this correlation
% with the function \cd{checkCorr}.  The type \cd{raw\_alarm} closely
% follows the description above. We highlight a few important features.
% First, we note that the type of the field \cd{info} depends on the
% alarm code, reflecting the text above. More interestingly, the type
% \cd{info} is implemented with a switched datatype, deciding how to
% parse based on the parameter \cd{alarm\_code}.  Next, we note that the
% description includes five different types of name-value pairs. We take
% advantage of both the type and value parameterization of types to
% encode all of these pair types based on one common description,
% \cd{pnvp}. This type is polymorphic in the type of the value and takes
% an arbitrary constraint \cd{c} as an argument. The type \cd{nvp} is
% polymorphic in the type of the value, but takes the expected name of
% the string as an argument. 

The \darkstar{} description also illustrates the use of
\textit{switched} datatypes.  A switched datatype selects a variant
based on the value of a user-specified \ocaml{} expression, which
typically references parsed data from earlier in the data source.  For
example, the switched datatype \cd{Info} chooses a
variant based on the value of its \cd{alarm\_code} parameter.  More
specifically, if the alarm code is \cd{5074}, the format specification
given by the \cd{Details} constructor will be used to parse the
current data.  Otherwise, the format given by the \cd{Generic}
constructor will be used.

The last construct in the \darkstar{} description is the type
qualifier \cd{omit}.  In the \cd{Service} datatype,
\cd{omit} specifies that the parsed string literal should be omitted
in the internal data representation because the literal can be
determined by the datatype constructor.

\cut{We can do this because we can discern from the
datatype constructor which string was found in the data source.}

%%% Local Variables: 
%%% mode: latex
%%% TeX-master: "../thesis.tex"
%%% End: 


\section{From \padsmlbig{} to \ocamlbig{}}
\label{sec:padsml-impl}

{\em
ToDo: Move detail about Traverse functor to Generic tools section.
}

We have implemented \padsml{} for use with \ocaml{}. The \padsml{}
compiler generates libraries in \ocaml{} source code that can then be
used by any \ocaml{} program. In this section, we describe the
contents of the generated libraries followed by some examples
demonstrating their use.


\subsection{Generated Libraries}
\label{sec:gen-code}

From each \padsml{} description, we generate a collection of types and
functions in \ocaml{}, including:
\begin{itemize}
\item The types of two data structures: one to contain parsed data in
  memory and the other to hold meta-data about the parsing process.
  These data structures are respectively called the
  \emph{representation} and the \emph{parse descriptor}.
\item A parsing function, which parses a data source to produce a
  representation and parse descriptor for the data.
\item A generic tool generator, based on the new tool development
  framework for \padsml{}. This framework is discussed in
  \secref{sec:gen-tool}.
\end{itemize} 

In general, the representation and parse-descriptor type definitions
are designed to closely resemble the original description.
The aim is to minimize the amount of effort a user must invest in
order to understand and use the data structures returned by the
parser.

Furthermore, the type of parse descriptor mimics the type of the
representation so that the parse descriptor can provide a parsing
report for every element of a corresponding representation. Parse
descriptors have two components: a header and a body. The header
reports on the parsing process that produced the representation. It
includes an error count that indicates the number of subcomponents
with errors; an error code that indicates the type of error, if any;
and the location of the data within the original data source. The body
of the parse descriptor contains the parse descriptors (if any) for
subcomponents of corresponding representations. The body for a
base-type parse descriptor is always of type \cd{unit}.

Below is a simple \padsml{} description of a character
and integer separated by a vertical bar.
\begin{code}\scriptsize
  \kw{ptype} Pair = Pchar * '|' * Pint\end{code} 
Here is a partial listing of the elements generated from that description.
\begin{code}\scriptsize
\kw{type} rep = Pchar.rep * Pint.rep
\kw{type} pd_body = Pchar.pd  * Pint.pd
\kw{type} pd = Pads.pd_header * pd_body

\kw{val} parse : Pads.handle -> rep * pd\end{code} 
This sample code and others that follow make use of a module
\cd{Pads} that contains types and functions that commonly occur in
generated and base-type modules. In particular, the above declarations use
\cd{Pads.pd_header}, which is the type of all parse-descriptor
headers, and \cd{Pads.handle}, which is the type of the (abstract)
handles used for data sources.
Note the close correspondence between the structure of the description
and that of the \cd{rep} and \cd{pd_body} types. In addition, we see
that the type of the parse function is defined in terms of the
\cd{rep} and \cd{pd} types.

Given the close relationship between the elements generated from a
description, it is natural to collect them together in a module. For
each named type, therefore, we generate a module with definitions like
those shown in the above example.
% For all generated modules, \cd{rep}, \cd{pd_body},\cd{pd} define the
% types of the data's representation, parse-descriptor body, and parse
% descriptor, respectively. The parsing function is named \cd{parse}.
In general, all types with base kind (i.e. those that are not
parameterized) match the following signature,
\cd{Type.S}:
\begin{code}\scriptsize
\kw{type} rep
\kw{type} pd\_body
\kw{type} pd = Pads.pd_header * pd_body

\kw{val} parse : Pads.handle -> rep * pd\end{code}

Modules, then, become the building blocks of the \padsml{} system.
Base types, too, are implementated with modules. Polymorphic types,
which map types to types, are implemented as functors from (type)
modules to (type) modules. It would even be appropriate to map
recursive types into recursive modules. Unfortunately, this approach
fails due to the limitations of the \ocaml{} implementation of
recursive modules. We would need support for inclusion of functors in
recursive modules in order to take this approach.

Given the signature \cd{Type.S} for types of base kind, we can now
show an example signature for a polymorphic type.
\begin{code}\scriptsize
\kw{ptype} (Alpha,Beta) ABPair = Alpha * '|' * Beta\end{code}
becomes
\begin{code}\scriptsize
\kw{module} ABPair (Alpha : Type.S) (Beta : Type.S) :
\kw{sig}
  \kw{type} rep = Alpha.rep * Beta.rep
  \kw{type} pd\_body = (Pads.pd_header * Alpha.pd\_body) * 
                 (Pads.pd_header * Beta.pd\_body)
  \kw{type} pd = Pads.pd_header * pd\_body

  \kw{val} parse : Pads.handle -> rep * pd
\kw{end}\end{code}

Once a description has been compiled into an \ocaml{} module, that
module can be used like any other.  More specifically, each named type
in a description file is mapped into an \ocaml{} module of the same
name.  The collection of modules is grouped together into a
single file (compilation unit) with a name corresponding to the name
of the original description file. For example, a description file
named ``foo.pml'' with three types inside results in a file ``foo.ml''
with three submodules, each corresponding to one named type.  In the
remainder of this section, we will demonstrate a number of uses of
generated modules, highlighting data processing, transformation, and
filtering.

\subsection{Example: Data Processing}
\label{sec:ex-process}

We begin with a simple example in which we process a triple of
integers. Below is their description:
\begin{code}\scriptsize
\kw{ptype} Source = Pint * '|' * Pint * '|' * Pint\end{code} Next, we
show a complete \ocaml{} program that finds the average of the three
integers. (Note that we assume that the name of the description file
is ``intTriple.pml,'' resulting in an \ocaml{} module \cd{IntTriple}.)
\begin{code}\scriptsize
\kw{open} Pads
\kw{let} ((i1,i2,i3),pd) = 
    parse_source IntTriple.Source.parse "input.txt"
\kw{let} avg = match get_pd_hdr pd with
    \{error_code = Good\} -> (i1 + i2 + i3)/3
  | _ -> 0\end{code}

In this program, we parse the triple, check that it is valid and then
average its elements. The function \cd{parse_source} takes a parsing
function for a data source and a file name in which the data is
stored, and parses the source. In order to ensure that the data is
valid, the program projects the parse descriptor header from the parse
descriptor \cd{pd} and checks that the error code is set to \cd{Good}.
This error code is defined in the \cd{Pads} module, and indicates that
the data is syntactically and semantically valid.

Notice that checking the parse descriptor of the triple is enough to
guarantee that there are no errors in any of the triple's
subcomponents. This property is generally true of all representations
and corresponding parse descriptors. That is, if the header of a parse
descriptor reports no errors, then none of its subcomponents will
report errors. In this way, we support a ``pay-as-you-go'' approach to
application error handling, as the parse descriptor for valid data
need only be consulted once, no matter the size of the corresponding
data. Only if there are errors within the structure does the user then
need to continue consulting the parse descriptor until the error is
located.

\subsection{Example: Filtering}
\label{sec:ex-filter}

An important set of tasks relating to ad hoc data are those
related to errors, including error analysis, repair, and removal.
Programmers might want to clean their data, \ie{}, filter out data
containing errors. In this case, they need only access parse
descriptors to facilitate this task.

\begin{figure}
\begin{code}\scriptsize
\kw{open} Pads
   ...
\kw{let} split_entry (entry,pd) =
   match get\_pd\_hdr pd with
     \{error_code = Good\} -> write_valid entry
   | _ => write_invalid entry\end{code}
\caption{Error filter for \dibbler{} data}
\label{fig:ex-data-clean}
\end{figure}

\figref{fig:ex-data-clean} provides a partial demonstration of
splitting a standard data source into two separate sources, one with
valid records and the other (potentially) invalid records.  The valid
entries may then be further processed or loaded into a database
without corrupting the valuable data therein.  A human might examine
the bad entries off-line to determine the cause of errors or to figure
out how to fix the corrupted entries.

We assume that functions \cd{write_valid} and \cd{write_invalid} are
defined elsewhere to write an entry to a stream of valid and invalid
entries, respectively. The \cd{split_entry} function, then, receives
an entry and its parse descriptor, and, based on the parse descriptor,
writes the entry to the appropriate stream.

\subsection{Example: Transformation}
\label{sec:ex-trans}

\begin{figure}
  \centering
  \begin{code}\scriptsize
...
\kw{ptype} Header = \{
       alarm : [ a : Puint32 | a = 2 or a = 3];
 ':';  start :  Timestamp Popt;
 '|';  clear :  Timestamp Popt;
 '|';  code: Puint32;
 '|';  src\_dns  :  Nvp("dns1");
 ';';  dest\_dns :  Nvp("dns2");
 '|';  service  : service
\}
\mbox{}
\kw{ptype} D\_alarm = \{
       header   : header;
 '|';  details  : details
 \}
\mbox{}
\kw{ptype} G\_alarm = \{
       header   : header;
 '|';  generic  : (Nvp\_a,Semicolon,Vbar) Plist
\}\end{code}
\caption{Normalized format for \darkstar{} data. All named types not
  explicitly included in this figure are unchanged from the original
  \darkstar{} description.}
\label{fig:normal-darkstar}
\end{figure}

\begin{figure}
\begin{code}\scriptsize
\kw{open} Regulus
\kw{open} RegulusNormal
\kw{module} RA = Raw\_alarm
\kw{module} DA = D\_alarm
\kw{module} GA = G\_alarm
\kw{module} Header = H

\kw{let} splitAlarm ra =
    let h = 
       \{H.alarm=ra.RA.alarm; H.start=ra.RA.start; 
         H.clear=ra.RA.clear; H.code=ra.RA.code;
         H.src\_dns=ra.RA.src\_dns; H.dest\_dns=ra.RA.dest\_dns;
         H.service=ra.RA.service\};
    in match ra with
        \{info=Details(d)\} -> 
        (Some \{DA.header = h; DA.details = d\}, None)
      | \{info=Generic(g)\} ->
        (None, Some \{GA.header = h; GA.generic = g\})    
  \end{code}
  \caption{Shredding \darkstar{} data based on the {\tt info} field.}
  \label{fig:ex-no-err-check}
\end{figure}

Once a data source has been parsed and cleaned, a natural desire is to
transform such data to make it more amenable to further analysis.  For
example, analysts often need to convert ad hoc data into a form
suitable for loading into an existing system, such as a relational
database or statistical analysis package. Desired transformations
include removing extraneous literals, inserting delimiters, dropping
or reordering fields, and normalizing the values of fields (\eg{}
converting all times into a specified time zone).

Because relational databases typically cannot store unions directly,
another important transformation is to convert data with variation
(\ie{}, datatypes) into a form that such systems can handle.
Typically, there are two choices for such a transformation.  The first
is to chop the data into a number of relational tables: one table for
each variation.  This approach is called \textit{shredding}. The
second is to create an ``uber'' table, with one ``column'' for each
field in any variation.  If a given field is not in a particular
variation, it is marked as missing. 

The description fragment in \figref{fig:normal-darkstar} and code
fragment in \figref{fig:ex-no-err-check} demonstrate shredding
\darkstar{} data with \padsml{} and \ocaml{}. We shred the data into
two different tables based on the \cd{info} field of \cd{Alarm}
records. In the process, we also reorder the fields, putting the
\texttt{service} field into the common \texttt{header}. Notice that, in
the normalized format, \cd{Alarm} has been replaced with \cd{D\_alarm}
and \cd{G_alarm}, neither of which contain any fields with variable
type.

\begin{figure}
  \centering
  \begin{code}\scriptsize
\kw{let} normalizeTimeToGMT t = 
    match t with
      \{time=t;timezone="GMT"\} => t
    | \{time=t;timezone="EST"\} => t + (5 * 60 * 60)
    | \{time=t;timezone="PST"\} => t + (8 * 60 * 60)
    | ... \end{code}
  \caption{Normalizing timestamps}
  \label{fig:ex-normalize}
\end{figure}

In \figref{fig:ex-normalize}, we show an additional example of data
transformation, where we normalize timestamp-timezone pairs into
simple timestamps in GMT time.

%%% Local Variables: 
%%% mode: latex
%%% TeX-master: "paper"
%%% End: 


\section{The Generic Tool Framework}
\label{sec:gen-tool}

% \begin{itemize}
% \item x show signature of traversal functor.
% \item x show interface - explain that one module/constructor and base types.
% \item x constrast records and one other module.
% \item x Make it 3 tools: accum, XML, debuger.
% \item x Focus on accumulator. describe in more detail.
% \item x show implementations,  including base types to show how they differ.
% \item x then, describe XML, debugger as pretty printers.
% \item xuse ``generic tool'' in consistent way. I like it better than
%   tool generator.
% \end{itemize}

An essential benefit of \padsml{} is that it can provide the users
with a high return-on-investment for describing their data. While the
generated parser alone is enough to justify the user's effort, we aim
to increase the return by enabling users to easily construct data
analysis tools. However, there is a limit, both in resources and
expertise, to the range of tool generators that we can develop.
Indeed, new and interesting data analysis tools are constantly being
developed, and we have no hope of integrating even a fraction of them
into the \padsml{} system ourselves.

Fortunately, we don't have to. Many of the tools that we have
encountered in practice perform their computations in a single pass
over the representation and corresponding parse descriptor, visiting
each value in the data with a pre-,post-,or in-order traversal.

% A large class of data analysis tools
% share a common data processing method while differing in the details
% of how they transform data. These tools traverse the data
% representation and parse descriptor in a depth-first, left-to-right
% manner, often carrying some auxiliary state.  For each element
% visited, they perform some action involving the auxiliary state,
% either before, after or between visiting the element's subcomponents.
% Often, the most interesting computations occur at the leaves, where
% the computation is based on the leaf's type.  

We can, therefore, split such tools into a format-dependent traversal
mechanism, which implements a generalized fold over the representation
and parse descriptor; and a format-independent, \emph{generic tool}
that specifies how to process individual data elements. The \padsml{}
compiler generates such a format-dependent traversal for all
descriptions, while generic tools are developed independently by
users. The traversal mechanism interacts with generic tools through a
signature that every generic tool must match.

The new, generic tool architecture of \padsml{} delivers a number of
benefits over the fixed architecture of \padsc{}. In \padsc{}, all
tools are generated from within the compiler. Therefore, developing a
new tool generator requires understanding and modifying the compiler.
Furthermore, the set of tools to be generated is chosen by the user
when compiling the description.  In \padsml{}, tool generators can be
developed independent of the compiler and they can be developed more
rapidly, as the ``boilerplate'' code to traverse data need not be
replicated for each tool generator. In addition, the user controls
which tools to ``generate'' for a given data format, and the choice
can differ on a program-by-program basis.

\subsection{The Generic-Tool Interface}
\label{sec:gentool-interface}

\begin{figure}
\begin{code}\scriptsize
\kw{module} \kw{type} S = \kw{sig}
  \kw{type} state
  ...
  \kw{module} Record : \kw{sig}
    \kw{type} partial_state
    \kw{val}  init          : (string * state) list -> state
    \kw{val}  start         : state -> Pads.pd_header 
                         -> partial_state
    \kw{val}  project       : state -> string -> state
    \kw{val}  process_field : partial_state -> string
                         -> state -> partial_state
    \kw{val}  finish        : partial_state -> state
  \kw{end}

  \kw{module} Datatype : \kw{sig}
    \kw{type} partial_state
    \kw{val}  init            : unit -> state
    \kw{val}  start           : state -> Pads.pd_header 
                           -> partial_state
    \kw{val}  project         : state -> string -> state option
    \kw{val}  process_variant : partial_state -> string 
                           -> state -> partial_state
    \kw{val}  finish          : partial_state -> state
  \kw{end}
   ...
\kw{end}
\end{code}
\caption{An excerpt of the generic-tool interface \texttt{Generic\_tool.S}.}
\label{fig:gentool-interface}
\end{figure}

% As the elements of data representations and PD's do not have a uniform
% type, the generic tool must provide different functionality for
% different types, which the traversal functor can then apply
% appropriately to each element of the data.  Therefore, the
% generic-tool signature specifies a particular collection of types and
% functions for every construct in \padsml{}.

The generic tool interface specifies an (abstract) type for auxiliary
state that is threaded through the traversal, and, for every type
constructor in \padsml{}, a set of types and functions -- grouped
together in a module -- that a generic tool must implement. Every
module has an \cd{init} function to create an initial state object for
data processed by that module. In addition, a \cd{project} function
retrieves the state of a subcomponent from the state of an element. As
processing an element can occur before, after, or between processing
an element's children, the signature includes functions corresponding
to each of these events. The function \cd{start} begins processing the
element, \cd{process_...} to process a subcomponent, and \cd{finish}
to complete processing the element. For type constructs with only one
subcomponent, the \cd{process} and \cd{finish} functions are combined.

An excerpt of the generic tool interface is shown in
\figref{fig:gentool-interface}. It includes the signatures of the
\cd{Record} and \cd{Datatype} modules. The \cd{Record} module includes
a type \cd{partial_state} that allows tools to represent intermediate
state in a different form than the general state, while processing
subcomponents. The \cd{init} function forms the state of the record
from the state of its fields. The \cd{start} function receives the PD
header for the data element being traversed. Function \cd{project}
takes a record's state and the name of a field and returns that
field's state. Function \cd{process_field} updates the intermediate
state of the record based on the name and state of a field, and
\cd{finish} converts the finished intermediate state into general tool
state.  Note that any of these functions could have side effects.

The \cd{Datatype} module is quite similar to \cd{Record}, however,
there are some important difference. Its \cd{init} function does not
start with the state of all the variants. Instead, variant's state is
added during processing. In this way, only variants that have been
visited will have corresponding state. For this reason, \cd{project}
returns a \cd{state option}, rather than just \cd{state}. This design
is essential for supporting recursive datatypes.

\begin{figure}
\begin{code}\scriptsize
  \kw{module} Traverse (Tool : Generic_tool.S) :
  \kw{sig}
    \kw{val} init : unit -> Tool.state
    \kw{val} traverse : rep -> pd -> Tool.state -> Tool.state
  \kw{end}
\end{code}
\caption{The signature of the Traversal functor within the signature \texttt{Type.S}.}
\label{fig:traversal-interface}
\end{figure}

In~\figref{fig:traversal-interface}, we show the signature for the
traversal functor as it appears in the signature \cd{Type.S} from
\secref{sec:padsml-impl}. The functor takes a generic tool and produce
a specific tool with two functions: \cd{init}, to create the initial
state for the tool, and \cd{traverse}, which traverses the
representation and parse descriptor for the type and updates the given
tool state.

\subsection{Example Tools}
\label{sec:gentool-motivation-ex}

% \begin{figure}
%   \centering
%   \small
% \begin{verbatim}
% 122Joe|Wright|450|95|790
% n/aEd|Wood|10|47|31
% 124Chris|Nolan|80|93|85
% 125Tim|Burton|30|82|71
% 126George|Lucas|32|62|40
% \end{verbatim}  
%   \caption{A fictitious data fragment in the Movie-director Bowling
%     Score (MBS) format. Note that the first record contains semantic
%     errors (in that the minimum is larger than the maximum and the
%     average is larger than both the minimum and the maximum).}
%   \label{fig:gentool-mbs-sample}
% \end{figure}

\begin{figure}
  \centering
  \scriptsize
\begin{verbatim}
<Order>
   <summary>
      <errors>1</errors> <total>2</total>        
   </summary>
   <Order_header>
      <summary>
         <errors>1</errors> <total>2</total>        
      </summary>
      <order_num>
         <errors>0</errors> <total>2</total>        
      </order_num>
      <att_order_num>
         <summary>
            <errors>1</errors> <total>2</total>        
         </summary>
         <val>
            <errors>0</errors> <total>2</total>                
         </val>
      </att_order_num>
      <ord_version>
         <errors>0</errors> <total>2</total>                
      </ord_version>
      ...
   </Order_header>
</Order>
\end{verbatim}  
  \caption{A fragment of the accumulator output for \dibbler{}. The
    output is encoded in XML.}
  \label{fig:gentool-acc-output}
\end{figure}

We have implemented three generic tools that illustrate important
features of the framework: a tool for generating statistical
overviews of the data, a data printer for debugging, and an XML
formatter.

A common desire of a data analyst upon receiving a new data source is
to get a sense of the quality of the data. In particular, they might
be interested in knowing what percentage of the source has errors, or
which fields are the most problematic. For this purpose, an
\emph{accumulator} tool provides a statistical summary of data
sources. For example, it tracks the distribution of the top $n$
distinct legal values. The accumulator is designed for data sources
whose basic structure is a series of records of the same type and
provides the summary for the record type based on viewing many records
of that type in the data source.

We have implemented a generic tool that provides some of the basic
features of accumulators. Our accummulator counts the number of errors
and the total number of values for every element of a description. In
addition, based on past experience, we know of other statistical
algorithms that could be implemented in a similar manner.

In \figref{fig:gentool-acc-output}, we show a sample portion of
accumulator output for the \dibbler{} data
from~\figref{figure:dibbler-records}. The data shows that one out of the
two \cd{Order}s has an error. Investigating further, we notice that the
problem lies in the \cd{Order_header}, in particular within the
\cd{att_order_num} field.  This field has a constraint on it, and one
of the values violates the constraint. A quick glance at the data
fragment reveals that the second order is contains the offending
field. In general, specific information like this could be found in
the parse descriptor.

\begin{figure}
\begin{code}\scriptsize
\kw{type} baseAcc = int * int
\kw{type} acc = ...
| RecordData of baseAcc * acc Table.t

\kw{type} compoundAcc = baseAcc * acc Table.t

\kw{type} state = acc

\kw{module} Record = \kw{struct}
  \kw{type} partial_state = compoundAcc

  \kw{let} init accs = RecordData ((0,0), Table.from_list accs)

  \kw{let} start state header =
    \kw{match} state \kw{with}
      RecordData ((errs, total), accs) ->
	\kw{let} errs' = if header.nerr > 0
                    then errs + 1 else errs
	\kw{in} (((errs', total + 1), accs) : partial_state)
    | _ -> \kw{raise} ...
	  
  \kw{let} project state label = 
     \kw{match} state \kw{with}
      RecordData (_, accs) -> (\kw{try} Table.find accs label
                                 \kw{with} _ -> \kw{raise} ...)
    | _ -> \kw{raise} ...

  \kw{let} process_field (ba, accs) label acc =
    (ba, Table.update accs label acc)
      
  \kw{let} finish (ba, accs) = RecordData (ba, accs)
\kw{end}
...
\end{code}
\caption{Excerpts from the implementation of the accumulator.}
\label{fig:gentool-accum-code}
\end{figure}

In~\figref{fig:gentool-accum-code}, we show a portion of the of the
accumulator implementation, including the \cd{Record} module. We first
define a basic accumulator, which is simply a pair of integers
counting the number of errors seen and the total number of elements
seen. The type \cd{acc} is essentially a universal datatype with
accumulator summaries for every element of the representation type. We
show the \cd{RecordData} variant for illustration.  The \cd{init}
function starts both counts at zero and converts the provided list of
subcomponent accumulators into a \cd{Table.t}.  The \cd{start}
function updates the summary based on the error count \cd{nerr} in the
PD header provided, and increments that total count by one. It returns
a \cd{partial_state}, as indicated by the type annotation. Projection
and processing of fields are simply lookups and insertions in a table
of subcomponent accumulators.

In addition to the accumulator, we have implemented two different
kinds of pretty printers for parsed data.  One formats the data in
XML. The other prints the data in a simple text format that is helpful
for debugging descriptions. Importantly, both tool's output corresponds to the in-memory
representation of the data rather than its original format (which may,
for example, have delimiters that are not present in the
representation).

%%% Local Variables: 
%%% mode: latex
%%% TeX-master: "paper"
%%% End: 


\section{A Data Description Calculus}
\label{sec:ddc}

At the heart of our work is a polymorphic data description calculus (\ddc{}),
designed to capture the core features of \padsml.

\subsection{\ddc{} Syntax}
\begin{figure}
{\small
\begin{bnf}
  \name{Kinds} \meta{\gk} \::= \kty \| \ity \-> \gk 
                               \| \kty \-> \gk \\
  \name{Types} \meta{\ty} \::= 
    \ptrue\| \pfalse \| \pbase{e} \| 
    \plam{\var}{\ity}{\ty} \| \papp{\ty}{e} \nlalt
    \psig x \ty \ty \| \psum \ty e \ty \| \pand \ty \ty \|
    \pset x \ty e \| \pseq \ty \ty {\pterm e \ty} \nlalt
    \ptyvar       \| \pmu{\ptyvar}{\gk}{\ty} \| \pcompute e \ity \| 
     \pabsorb \ty \| \pscan{\ty} 
    \nlalt \ptylam{\ptyvar}{\kty}{\ty} \| \ptyapp{\ty}{\ty}
\end{bnf}
}
\caption{\ddc{} Syntax}
\label{fig:ddc-syntax}
\end{figure}
\figref{fig:ddc-syntax} shows the syntax of \ddc{}. 

\subsection{\Implang{} Language}
\label{sec:host-lang}
\begin{figure}[tp]
\small
\begin{bnf}
%   \name{Variables} \meta{f,x,y} \\
%   \name{Bit}   \meta{b}   \::= 0 \| 1 \\ 
  \name{Bits}  \meta{B}   \::= \cdot \| 0\,B \| 1\,B \\ 
  \name{Constants} \meta{c} \::=
      () \| \itrue \| \ifalse \| 0 \| 1 \| -1 \| \dots \nlalt
      \ierr \| \data \| \off \| \iok \| \iecerr \| \iecpc \| \ldots \\

  \name{Values} \meta{v} \::= 
      \const \| % \ilam{\nrm \var}{\ity}{e} \| 
      \ifun {\nrm f} {\nrm x} e \| \ipair v v \nlalt
      \iinld{\ity}{v} \| \iinrd{\ity}{v} \|
      \iarr{\vec{v}} \\

  \name{Operators} \meta{op} \::= 
      = \; \| \; < \; \| \inotop % \| \isizeofop
      \| \ldots \\

  \name{Expressions} \meta{e} \::= 
      \const \| \var \| \iop{e} \|
%      \ilam {\nrm \var} \ity e \| 
      \ifun {\nrm f} {\nrm x} e \| 
      \iapp e e \nlalt
%      \iletfun {\nrm f} {\nrm x} e \; \iin \; e' \| 
      \ilet {\nrm x} e \; e \|
      \iif e \; \ithen e \; \ielse e \nlalt
      \ipair{e}{e} \| \ipi {\nrm i}{e} \|
      \iinld{\ity}{e} \| \iinrd{\ity}{e} \nlalt
      \icaseg{e}{\nrm x}{e}{\nrm x}{e} \nlalt
      \iarr{\vec e} \| \iappend e e \| \isub e {\nrm e}
      \\
      
  \name{Base Types} \meta{a} \::= 
      \iunitty \| \iboolty \| \iintty  \| 
      \invty \nlalt  \ibitsty \| \ioffty \| \iecty
  \\
  \name{Types} \meta{\ity} \::= 
      \ibasety \| \ityvar \| \iarrow \ity \ity \| \iprod \ity \ity \|
      \isum \ity \ity \nlalt
      \iseq \ity \| \forall \ityvar{:}\kappa.\ity  \|
      \imu \ityvar \ity   
      \| \lambda \alpha{:}\kappa.\ity 
      \| \ity \; \ity
  \\
  \name{Kinds} \meta{\kappa} \::= \kty \| \kappa \rightarrow \kappa
  
\end{bnf}
\caption{\Implang{} Language: F$^\omega$}
\label{fig:implang-syntax}
\end{figure}

In \figref{fig:implang-syntax}, we present the host language of \ddc{}, an
extension of the simple-typed polymorphic lambda calculus. 
We use this host language both to encode the parsing semantics of \ddc{} 
and to write the expressions that can appear within \ddc{} itself.

As the calculus is largely standard, we highlight only its
unusual features. The constants include bitstrings $\data$; offsets $\off$,
representing locations in bitstrings; and error codes $\iok$,
$\iecerr$, and $\iecpc$, indicating success, success with errors and
failure, respectively. We use the constant $\ierr$ to indicate a
failed parse.  
Because of its specific meaning, we forbid its use in user-supplied expressions
appearing in \ddc{} types. 
%We include a special $\isizeofop$ operator, which returns the size in the data
%source of its input. 
Our expressions include arbitrary length
sequences $\iarr{\vec e}$, sequence append $\iappend e
{e'}$, and sequence indexing $\isub e {\nrm i}$.

The type $\invty$ is the singleton type of the constant $\ierr$.
Types $\iecty$ and $\ioffty$ classify error codes and bit string
offsets, respectively. The remaining types have standard
meanings: function types, product types, sum types, sequence types
$\iseqty \ty$; polymorphic types $\forall \ityvar.\ity$ and type
variables $\ityvar$; and recursive types $\imu \ityvar \ity$.

We extend the formal syntax with some syntactic sugar 
for use in the rest of the paper: anonymous functions
$\ilam {\nrm x} \ity e$ for $\ifun {\nrm f} {\nrm x} e$, with $f
\not\in {\rm FV}(e)$; function bindings $\iletfun {\nrm f} {\nrm x} e
\; \iin \; e'$ for $\ilet {\nrm f} {\ifun {\nrm f} {\nrm x} e} \; e'$;
$\ispty$ for $\iprod \ioffty \ioffty$.  We often use
pattern-matching syntax for pairs in place of explicit projections, as
in $\lampair{\codefont e}$ and $\ilet {\itup{\off,r,p}} e\; e'$.  Although
we have no formal records with named fields, we use a dot notation for
commonly occuring projections. For example, for a pair $\mathtt x$ of
rep and PD, we use $\codefont{x.rep}$ and $\codefont{x.pd}$ for the
left and right projections of $\codefont{x}$, respectively. \cut{Generally,
the particular projection intended should be apparent from context,
and will be specified when it is not.} Also, sums and products are
right-associative. 

We use standard judgments for the static semantics
($\stsem[e,{\ctxt},\ity]$) and operational semantics ($e
\stepsto e'$) of the \implang{} language. Details appear in \appref{app:host-lang}.

\section{\ddc{} Semantics}
\label{sec:ddc-sem}

The primitives of \ddc{} are deceptively simple.  Each captures a
simple concept, often familiar from type theory. However, in reality,
each primitive is multi-faceted. Each simultaneously describes a
collection of valid bit strings, two datatypes in the host language --
one for the data representation itself and one for its parse
descriptor -- and a transformation from bit strings, including
invalid ones, into data and corresponding meta-data.
We give semantics to \ddc{} types using three semantic functions, each
of which precisely conveys a particular facet of a type's meaning.
The functions $\itsem[\cdot]$ and $\itpdsem[\cdot]$ describe the {\it
  representation semantics} of \ddc{}, detailing the types of the
data's in-memory representation and parse descriptor. The function
$\trans[\cdot,,]$ describes the {\it parsing semantics} of \ddc{},
defining a \implang{} language function for each type that parses bit
strings to produce a representation and parse descriptor. We define
the set of valid bit strings for each type to be those strings for
which the PD indicates no errors when parsed.

We begin with a kinding judgment that checks if
a type is well formed. We then formalize
the three-fold semantics of \ddc{} types.

\begin{table}
  \begin{center}
    \renewcommand{\arraystretch}{1.35}
    \begin{tabular}{l l}
      $\ddck[\ty,{\pctxt;\rctxt;\ctxt},\kind,\mcon]$ & {\it \ddc{}-type
        kinding}\\
      $\itsem[\ty] = \ity$ & {\it representation types of \ddc{} types}\\
      $\itpdsem[\ty] = \ity$ & {\it pd types of \ddc{} types}\\
      $\trans[\ty,\ctxt,\gk] = e$   & {\it \ddc{}-type semantics} \\
      $\kTrans[\gk,\ty] = \ity$     & {\it parser type} \\
      $\ptyc \rctxt = \ctxt$     & {\it context parser type}\\
      $\stsem[e,{\pctxt;\ctxt},\ity]$ & {\it \implang expression typing} \\
      $e \stepsto e'$ & {\it \implang expression evaluation}
    \end{tabular}
    \caption{Translations and Judgments}
    \label{tab:judg-list}
  \end{center}
\end{table}
For reference, we provide in
\tblref{tab:judg-list} a listing of all the functions and judgments
defined in this section and a brief description of each.  


\subsection{\ddc{} Kinding}
\label{sec:ddc-kinding}

\begin{figure*}[t]
\small
\fbox{$\ddck[\ty,\pctxt;\rctxt;\ctxt,\kind,\mcon]$}\\[-2ex]
\[
\infer[\text{Unit}]{
    \ddck[\ptrue,\pctxt;\rctxt;\ctxt,\kty,\con]
  }{\wfd {} \ctxt}
\quad 
\infer[\text{Bottom}]{
    \ddck[\pfalse,\pctxt;\rctxt;\ctxt,\kty,\con]
  }{\wfd {} \ctxt}
\quad 
\infer[\text{Const}]{
    \ddck[\pbase{e},\pctxt;\rctxt;\ctxt,\kty,\con]
  }{
    \begin{semcond}
      \stsem[e,\ctxt,\ity] &
      (\vlet {\ity \iarrowi \kty} {\Ikind(C)})
    \end{semcond}
  }
\]

\[
\infer[\text{Abs}]{
    \ddck[\plam{\var}{\ity}{\ty},
         \pctxt;\rctxt;\ctxt,\ity \iarrowi \kind,\mcon]
  }{
    \ddck[\ty,\pctxt;\rctxt;\ectxt{\var{:}\ity},\kind,\mcon]
  }
\quad
\infer[\text{App}]{
  \ddck[\papp{\ty}{e},\pctxt;\rctxt;\ctxt,\gk,\mcon]
}{
  \ddck[\ty,\pctxt;\rctxt;\ctxt,\ity \iarrowi \gk,\mcon] &
  \stsem[e,\ctxt,\ity]
}
\]

\[
\infer[\text{Prod}]{
    \ddck[\psig{x}{\ty}{\ty'},\pctxt;\rctxt;\ctxt,\kty,\con]
  }{       
    \ddck[\ty,\pctxt;\rctxt;\ctxt,\kty,\mcon] &
    \ddck[\ty',\pctxt;\rctxt;
          \ectxt {x{:}\iprod {\itsem[\asub \rctxt \ty]} 
              {\itpdsem[\asub \rctxt \ty]}},
          \kty,\mcon']
  }
\]

\[
\infer[\text{Sum}]{
    \ddck[\psum{\ty}{e}{\ty'},\pctxt;\rctxt;\ctxt,\kty,\con]
  }{
    \ddck[\ty,\pctxt;\rctxt;\ctxt,\kty,\mcon] & \ddck[\ty',\pctxt;\rctxt;\ctxt,\kty,\mcon'] 
  }
\quad
  \infer[\text{Intersection}]{
    \ddck[\pand \ty {\ty'},\pctxt;\rctxt;\ctxt,\kty,\con]
  }{
    \ddck[\ty,\pctxt;\rctxt;\ctxt,\kty,\mcon] & \ddck[\ty',\pctxt;\rctxt;\ctxt,\kty,\mcon'] 
  }
\]

\[
  \infer[\text{Con}]{
    \ddck[\pset x \ty e,\pctxt;\rctxt;\ctxt,\kty,\con]
  }{ 
    \ddck[\ty,\pctxt;\rctxt;\ctxt,\kty,\mcon] & 
    \stsem[e,
    \ectxt{x{:}\iprod{\itsem[\asub \rctxt \ty]} 
      {\itpdsem[\asub \rctxt \ty]}},\iboolty]
  }
\]

\[\infer[\text{Seq}]{
    \ddck[\pseq \ty {\ty_s} {\pterm e {\ty_t}},\pctxt;\rctxt;\ctxt,\kty,\con]
  }{
    \begin{array}{c}
    \ddck[\ty,\pctxt;\rctxt;\ctxt,\kty,\mcon] \qquad
    \ddck[{\ty_s},\pctxt;\rctxt;\ctxt,\kty,\mcon_s] \qquad
    \ddck[{\ty_t},\pctxt;\rctxt;\ctxt,\kty,\mcon_t] \\
    \stsem[e,\ctxt,
    \iprod {\itsem[{\ty_m}]}      
    {\itpdsem[{\ty_m}]}
    \iarrowi \iboolty]
    \quad (\ty_m = \asub \rctxt {\pseq \ty {\ty_s} {\pterm e {\ty_t}}})
    \end{array}
  }
\]

\[
  \infer[\text{RecVar}]{
    \ddck[\ptyvar,{\pctxt;\rctxt;\ctxt},\kty,\ncon]
  }{\wfd {} \ctxt \quad \ptyvar {=} \pmu \ptyvar \kty \ty \in \dom \rctxt}
\quad
  \infer[\text{Rec}]{
    \ddck[\pmu \ptyvar \kty \ty,\pctxt;\rctxt;\ctxt,\kty,\con]
  }{
    \ddck[\ty,{\rctxt,\ptyvar {=} \pmu \ptyvar \kty \ty;\ctxt},\kty,\con]
  }
\]

\[
  \infer[\text{Compute}]{       
    \ddck[\pcompute{e}{\ity},\pctxt;\rctxt;\ctxt,\kty,\con]
  }{
    \stsem[e,\ctxt,\ity]
  }      
\quad
\infer[\text{Absorb}]{
    \ddck[\pabsorb{\ty},\pctxt;\rctxt;\ctxt,\kty,\con]
  }{
    \ddck[\ty,\pctxt;\rctxt;\ctxt,\kty,\mcon]
  }
\quad
  \infer[\text{Scan}]{
    \ddck[\pscan{\ty},\pctxt;\rctxt;\ctxt,\kty,\con]
  }{
    \ddck[\ty,\pctxt;\rctxt;\ctxt,\kty,\mcon]
  }
\]

\[
  \infer[\text{TyVar}]{
    \ddck[\ptyvar,{\pctxt;\rctxt;\ctxt},\kty,\ncon]
  }{\wfd {} \ctxt \quad \tyvar{:}\kty \in \pctxt}
\quad
\infer[\text{TyAbs}]{
    \ddck[\ptylam{\tyvar}{\kty}{\ty},
         \pctxt;\rctxt;\ctxt,\kty \iarrowi \kind,\mcon]
  }{
    \ddck[\ty,{\pctxt,\tyvar{:}\kty;\rctxt;\ectxt},\kind,\mcon]
  }
\quad
\infer[\text{TyApp}]{
  \ddck[\ptyapp{\ty_1}{\ty_2},\pctxt;\rctxt;\ctxt,\gk,\mcon]
}{
  \ddck[\ty_1,\pctxt;\rctxt;\ctxt,\kty \iarrowi \gk,\mcon] &
  \ddck[\ty_2,\pctxt;\rctxt;\ctxt,\kty,\mcon]
}
\]

\caption{\ddc{} Kinding Rules}
\label{fig:ddc-kinding}
\end{figure*}

The kinding judgment defined in \figref{fig:ddc-kinding} determines
well-formed \ddc{} types, assigning kind $\kty$ to basic types and
kind $\ity \iarrowi \kind$ to type abstractions.  We use two contexts to express our kinding judgment:
\[
\begin{array}{ll}
\ctxt  & \mathrel{::=} \cdot \bnfalt \ctxt,{\var{:}\ity}\\
\rctxt & \mathrel{::=} \cdot \bnfalt \rctxt,\ptyvar{=}\pmu \ptyvar \gk \ty\\
\pctxt  & \mathrel{::=} \cdot \bnfalt \pctxt,\tyvar{:}\kty
\end{array}
\]
Context $\Gamma$ is a finite partial map that binds expression
variables to their types.
Context $\rctxt$ is an ordered list of mappings between type variables and recursive types.
This context serves two purposes: first, to ensure the well-formedness
of types with free type variables; and second, to provide mappings
between recursive type variables and their associated types. 
This second purpose leads us to consider a context $\rctxt$ to be a
substitution from type variables to types. Application of such a
substitution has the form $\asub \rctxt \ty$.
Context $\rctxt$ is a list of type variables with kind $\kty$.

To ensure that recursive types have properly-shaped parse descriptors
with a valid PD header (a condition necessary for the type safety of
generated parsers), we disallow types such as $\pmu \ptyvar
\ptyvar$. More generally, we ensure that
recursive type variables are separated from their binder by at least
one basic primitive, such as a product or sum, a condition called {\it contractiveness}. To this end, we annotate every judgment with a contractiveness
indicator, one of $\con$, $\ncon$, or $\mcon$. A
$\con$ indicates the type is contractive, an $\ncon$
indicates it is not, and a $\mcon$ indicates it may be either. 
We consider $\ncon < \con$. 

As the rules are otherwise mostly straightforward, we highlight
just two of them. We use the function $\Ikind$ to assign kinds to base types.
While their kind does not differentiate them from type
abstractions, base types are not well formed when not applied.  
Once applied, all base types have kind $\kty$. The product rule
shows that the name of the first component is bound to a pair of a representation and corresponding PD.
The semantic functions defined in the next section determine the type of this pair.
Note that we apply $\rctxt$ to the type of the first component before
translation, thereby closing it,
 as open \ddc{}
types do not translate into well-formed \implang{} types.

\subsection{Representation Semantics}
\label{sec:intty-sem}

\begin{figure}
\fbox{$\itsem[\ty] = \ity$}
\[
\begin{array}{lcl} 
\itsem[\ptrue] & = & \iunitty \\
\itsem[\pfalse] & = & \invty \\
\itsem[\pbase{e}] & = & \isum {\Irty(C)} \invty   \\
\itsem[\plam{\var}{\ity}{\ty}] & = & \itsem[\ty] \\
\itsem[\papp \ty e] & = & \itsem[\ty] \\
\itsem[\psig \var {\ty_1} {\ty_2}]  & = & \iprod {\itsem[\ty_1]} {\itsem[\ty_2]}    \\
\itsem[\psum {\ty_1} e {\ty_2}]     & = & \isum {\itsem[\ty_1]} {\itsem[\ty_2]} \\
\itsem[\pand {\ty_1} {\ty_2}]  & = & \iprod {\itsem[\ty_1]}{\itsem[\ty_2]}\\
\itsem[\pset x \ty e] & = & \isum {\itsem[\ty]}{\itsem[\ty]}\\
% field names: length, elts
\itsem[\pseq \ty {\ty_{\text{sep}}} {\pterm e {\ty_{\text{term}}}}] & = & 
    \iprod \iintty {(\iseq{\itsem[\ty]})}             \\
%% \itsem[\pcase e c {\ty_1} {\ty_2}]       & = & \isum {\itsem[\ty_1]} {\itsem[\ty_2]}\\
\itsem[\ptyvar] & = & \ptyvar_\repname \\
\itsem[\pmu{\ptyvar}{\gk}{\ty}] & = & \imu{\ptyvar_\repname}{\itsem[\ty]} \\
\itsem[\pcompute e \ity]                 & = & \ity \\
\itsem[\pabsorb \ty]                     & = & \isum \iunitty \invty \\
\itsem[\pscan \ty] & = & \isum {\itsem[\ty]} \invty \\
%% \pext{
%% \itsem[\ptransform e e \ty]              & = & \itsem[\ty]\\
%% }
\itsem[\lambda \ptyvar.\ty]       & = & \lambda \ptyvar_\repname.\itsem[\ty] \\
\itsem[\ty_1 \ty_2]              & = & \itsem[\ty_1] \itsem[\ty_2] \\
\end{array}
\]
\caption{Representation Types}
\label{fig:rep-tys}
\end{figure}

In Figure~\ref{fig:rep-tys}, we present the representation type
of each \ddc{} primitive. While the primitives are
dependent types, the mapping to the \implang{} language erases the dependency because the \implang{} language does not have dependent types. For \ddc{} types in which expressions appear,
the translation drops the expressions to remove the dependency.
With these expressions gone, variables become useless, so we drop 
variable bindings as well, as in product and constrained types.
Similarly, as type abstraction and application are only relevant for
dependency, we translate them according to their underlying types.

In more detail,
the \ddc{} type $\ptrue$ consumes no input and produces only
the $\iunitty$ value.  Correspondingly, $\pfalse$ consumes no input,
but uniformly fails, producing the value $\invty$. The
function $\Irty$ maps each base type to a representation for
successfully parsed data. Note that this representation does not depend
on the argument expression. As base type parsers can fail, we sum this type
with $\invty$ to produce the actual representation type.
Intersection types produce a pair of values, one for each sub-type,
because the representations of the subtypes need not be identical nor
even compatible. 
Constrained types produce sums, where a left branch indicates the data
satisfies the constraint and the right indicates it does not. In
the latter case, the parser returns the offending data rather than
$\ierr$ because the error is semantic rather than syntactic.
Sequences produce a \implang{} language sequence paired with its
length.  Recursive types generate recursive representations. Note that the \implang{} type uses the same variable name
as the \ddc{} type, and so the type corresponding to the type variable
$\ptyvar$ is exactly $\ityvar$.
The output of a $\pcomputen$ is exactly the computed value, and
therefore shares its type.  The output of $\pabsorbn$ is a sum
indicating whether parsing the underlying type succeeded or failed.
The type of $\pscann$ is similar, but also returns an element of the
underlying type in case of success.

\begin{figure}
\fbox{$\itpdsem[\ty] = \ity$}
\[ 
\begin{array}{lcl} 
%% %% example: \ua.(int * a) + None
%% %%          pd = \ua.pd_hdr  * ((pd_hdr * ([int]_pd * [a]_pd)) + [None]_pd)
%% %%             = \ua.pd_hdr  * ((pd_hdr * ([int]_pd * a)) + [None]_pd)
\itpdsem[\ptrue] & = & \ipty \iunitty \\                                                  
\itpdsem[\pfalse] & = & \ipty \iunitty \\                                                  
\itpdsem[\pbase{e}] & = & \ipty \iunitty\\
\itpdsem[\plam \var \ity \ty] & = & \itpdsem[\ty] \\
\itpdsem[\papp \ty e] & = & \itpdsem[\ty] \\
\itpdsem[\psig \var {\ty_1} {\ty_2}] & = & 
               \ipty {\iprod {\itpdsem[\ty_1]} {\itpdsem[\ty_2]}} \\
\itpdsem[\psum {\ty_1} e {\ty_2}] & = & 
               \ipty {(\isum {\itpdsem[\ty_1]} {\itpdsem[\ty_2]})} \\
\itpdsem[\pand {\ty_1} {\ty_2}] & = & \ipty {\iprod {\itpdsem[\ty_1]} {\itpdsem[\ty_2]}}    \\
\itpdsem[\pset x \ty e] & = & \ipty {\itpdsem[\ty]} \\
\itpdsem[\pseq \ty {\ty_{\text{sep}}} {\pterm e {\ty_{\text{term}}}}] & = & 
  \iapty {\itpdsem[\ty]} \\
\itpdsem[\ptyvar] & = & \ipty{\ptyvar_\pdname} \\
\itpdsem[\pmu \ptyvar \kty \ty] & = & 
  \ipty{\imu{\ptyvar_\pdname}{\itpdsem[\ty]}} \\
\itpdsem[\pcompute e \ity]            & = & \ipty \iunitty \\
\itpdsem[\pabsorb \ty]                & = & \ipty \iunitty \\
\itpdsem[\pscan{\ty}] & = & \ipty {(\isum {(\iprod \iintty
    {\itpdsem[\ty]})} \iunitty)} \\
\itpdsem[\lambda \ptyvar.\ty]      
     & = & \lambda \ptyvar_\pdname.\itpdsem[\ty] \\
\itpdsem[\ty_1 \ty_2]            & = & \itpdsem[\ty_1] \itpdsemstrip[\ty_2] \\
\end{array}
\]

\fbox{$\itpdsemstrip[\ty] = \ity$}

\[
\begin{array}{lcl} 
\itpdsemstrip[\ty] & = & \ity \ \ \mbox{where}\ \itpdsem[\ty] = \ipty{\ity}
\end{array}
\]
\caption{Parse Descriptor Types}
\label{fig:pd-tys}
\end{figure}

In \figref{fig:pd-tys}, we give the parse descriptor
type for each \ddc{} type. Each PD type has a header and body.
This common shape allows us to define functions that polymorphically
process PDs based on their headers. Each header stores the number of
errors encountered during parsing, an error code indicating the degree
of success of the parse -- success, success with errors, or failure --
and the span of data described by the descriptor.  Formally, the type
of the header  ($\tyface{pd\_hdr}$) is $\iintty \iprodi \iecty \iprodi
\ispty$.  Each body consists of subdescriptors corresponding to the
subcomponents of the representation and any type-specific meta-data. For types with neither subcomponents nor special meta-data, we
use $\iunitty$ as the body type.

We discuss a few of the more complicated parse descriptors in detail.
The parse descriptor body for sequences contains the parse descriptors of its elements,
the number of element errors, and the sequence length. Note that the
number of element errors is distinct from the number of sequence
errors, as sequences can have errors that are not related to their
elements (such as errors reading separators).  We introduce an
abbreviation for array PD body types, $\iaptyname \; \ity =
\iintty \iprodi \iintty \iprodi (\iseq \ity)$.
\trversion{The $\pcomputen$ parse descriptors have no subelements because the
data they describe is not parsed from the data source.}
The $\pabsorbn$ PD
type is $\iunitty$ as with its representation. We assume that just as
the user does not want the represenation to be kept, so too the parse
descriptor.  The $\pscann{}$ parse descriptor is either $\iunitty$, in case
no match was found, or records the number of bits skipped before the
type was matched along with the type's corresponding parse descriptor.


\begin{figure}
\small
\fbox{$\kTrans[\gk,\ty] = \ity$} 
    
\begin{align*}
  &\kTrans[\kty,\ty] = \extdom * \offdom \iarrowi \offdom * \itsem[\ty] * \itpdsem[\ty]
   \\
   &\kTrans[\ity \iarrowi \gk,\ty] = \ity \iarrowi \kTrans[\gk,\ty]
   \\
   &\kTrans[\kty \iarrowi \gk,\ty] = 
      \forall\tyvar_\repname.\forall\tyvar_\pdname.
         \kTrans[\kty,\tyvar] \iarrowi \kTrans[\gk,\ty \tyvar] 
\end{align*}  
  \caption{\Implang{} Language Types for Parsing Functions}
  \label{fig:parser-types}
\end{figure}

\subsection{Parsing Semantics of the \ddc{}}
\label{sec:parse-sem}

\begin{figure*}
\small
\fbox{$\trans[\ty,\ctxt,\gk] = e$} 

\[
\begin{array}{l}
  %% None 
\trans[\ptrue,\ctxt,\kty] =
  \lampair{\spair<\off,\newrep{unit}{},\newpd{unit}{\off}>}
\\[3pt] %\\
%% False 
\trans[\pfalse,,] =
  \lampair{\spair<\off,\newrep {bottom}{},\newpd {bottom}{\off}>}
\\[3pt] %\\ 
%% Const 
\trans[\pbase{e},\ctxt,\kty] =
  \lampair{\iapp {\iapp {\Iimp(C)} (e)} {\itup {\idata,\off}}}
\\[3pt] %\\
%% Abs 
\trans[\plam{\var}{\ity}{\ty},,] =
   \sfn{\nrm\var}{\ity}{\trans[\ty,\ectxt{\var{:}\ity},\kind]}
\\[3pt] %\\
%% App 
\trans[\papp{\ty}{e},\ctxt,\gk] =
  \trans[\ty,,] \sapp e  
\\[3pt]
%% Prod 
%\begin{array}{l}
\trans[\psig{x}{\ty}{\ty'},\ctxt,\kty] = \\
  \begin{array}{l}  
    \lampair{} \\
    \quad  \ilet {\spair<\off',r,p>} 
    {{\trans[\ty,,]} \sapp \spair<\idata,\off>} \\
    \quad  \ilet x {\ictup{r,p}}\\
    \quad  \ilet {\spair<\off'',r',p'>} 
    {{\trans[\ty',,]} \sapp \spair<\idata,\off'>} \\
    \quad \spair<\off'',\newrep {\gS}{r,r'},\newpd {\gS}{p,p'}>
  \end{array}  
%\end{array}
\\
%% Sum 
%\begin{array}{l}
  \trans[\psum{\ty}{e}{\ty'},,] = \\
  \begin{array}{l}  
  \lampair{} \\
  \quad \ilet {\itup{\off',r,p}}{\trans[\ty,,] \sapp \spair<\idata,\off>} \\
  \quad \iif {\pdok p} \; \ithen {
    \def \r {\newrep {+left}{r}}
    \def \p {\newpd {+left}{p}}
    \spair<\off',\r,\p>} \\
  \quad \ielse {\ilet {\itup{\off',r,p}}{\trans[\ty',,] \sapp \spair<\idata,\off>}} \\
  \quad 
  {  % begin scope
    \def \r {\newrep {+right}{r}}
    \def \p {\newpd {+right}{p}}
    %% 
    \spair<\off',\r,\p>
  }\\ % end scope
  \end{array}
\\
%% Intersection 
  \trans[\pand{\ty}{\ty'},,] = \\
  \begin{array}{l}  
     \lampair{} \\
     \quad \ilet {\itup{\off',r,p}} {\trans[\ty,,] \sapp \spair<\idata,\off>} \\
     \quad \ilet {\itup{\off'',r',p'}} {\trans[\ty',,] \sapp \spair<\idata,\off>} \\
     \quad {\spair<\codefont{max}(\off',\off''),\newrep {\&}{r,r'},\newpd {\&}{p,p'}>}
   \end{array}
\\
%\quad
%% Set 
  \trans[\pset{x}{\ty}{e},\ctxt,\kty] = \\
  \begin{array}{l}  
    \lampair{} \\
    \quad \ilet {\itup{\off',r,p}}{\trans[\ty,,] \sapp \spair<\idata,\off>} \\
    \quad \ilet x {\ictup{r,p}}\\
    \quad \ilet c e \\
    \quad \spair<\off',\newrep {con} {c,r},\newpd {con} {c,p}>
  \end{array}
\\
\end{array}
\begin{array}{l}
%% Array 
\trans[\pseq{\ty}{\ty_s}{\pterm e {\ty_t}},,] = \\
  \begin{array}{l}  
    \lampair{}\\
      \quad \iletfun {isDone}{\itup{\off,r,p}}{\\
        \qquad \ior {\eofpred {\idata,\off}} {e\codefont {\sapp
          \spair<r,p>}} \iori \\
        \qquad \ilet {\itup{\off',r',p'}}{\trans[\ty_t,,] \spair<\idata,\off>}\\
        \qquad \pdok{p'}
      }\\
      \quad \iin \\
      \quad \iletfun {continue} {\itup{\off,\off',r,p}} {\\
        \qquad \iif  {\off = \off' \iori \isdone {\off',r,p}} \; \ithen {\itup{\off',\codefont{r,p}}} \\
        \qquad \ielse {
          \ilet {\itup{\off_s,r_s,p_s}}{\trans[\ty_s,,] \sapp \spair<\idata,\off'>}}\\
        \qquad \ilet {\itup{\off_e,r_e,p_e}}{\trans[\ty,,] \sapp \ictup{\idata,\off_s}}\\
        \qquad \mathtt{continue} \sapp \ictup{
            \off,\off_e,\newrep {seq} {r,r_e}, \newpd {seq} {p, p_s, p_e}
        }}\\
      \quad \iin
   \end{array}\\
  \begin{array}{l}  
      \quad \ilet {\mathtt{r}} {\newrep {seq\_init}{}}\\
      \quad \ilet {\mathtt{p}} {\newpd {seq\_init}{\off}}\\
      \quad \iif {\isdone{\off,r,p}} \; \ithen {\itup{\off,\codefont{r,p}}}\\
      \quad \ielse {\ilet {\itup{\off_e,r_e,p_e}}{\trans[\ty,,] \sapp
          \spair<\idata,\off>}} \\
      \quad \mathtt{continue} \sapp \ictup{\off,\off_e,
        \newrep {seq} {r,r_e}, \newpd {seq} {p, \newpd {unit} \off, p_e}}      
  \end{array}  
\\
%\end{array}
%\quad
%\begin{array}{l}
%% Var
\trans[\ptyvar,,] = \codefont{f_\ptyvar}
\\[3pt]
%% Mu
\trans[\pmu \ptyvar \gk \ty,,] = \\
  \begin{array}{l}
  \ifun {f_\ptyvar} {\itup{\data,\off}} {}\\
  \quad \ilet {\itup{\off',r,p}} 
  {\trans[\ty,,] \iappi \ictup{\data,\off}} \\ 
  \qquad \ictup{\off',r,\newpd{mu}{p}}
  \end{array}  
\\[3pt]
%% Compute
\trans[\pcompute e \ity,,] = \\
  \quad \lampair{\itup{\off,\newrep {compute} {\nrm e},\newpd {compute} \off}}
\\[3pt]
%% Absorb
\trans[\pabsorb \ty,,] = \\
  \begin{array}{l}  
    \lampair{}\\
    \quad \ilet {\itup {\off',r,p}} {\trans[\ty,,] \sapp \spair<\idata,\off>}\\
    \quad \itup{\off',\newrep {absorb} p,\newpd {absorb} p}   
  \end{array}  
\\
%% Scan
\trans[\pscan \ty,,] = \\
  \begin{array}{l}  
    \lampair{}\\
    \quad \iletfun {try} {i} {\\
      \qquad \ilet {\itup{\off',r,p}} {\trans[\ty,,] \sapp
        \codefont{\spair<\data,\off + i>}} \\
      \qquad \iif {\pdok p}\; \ithen \\
      \qquad {\ictup{\off',\newrep {scan} r,
        \newpd {scan} {i,p}}}\; \ielse {}\\
      \qquad \iif {\codefont{i = scanMax}}\; \ithen \\
      \qquad {\ictup{\off,\newrep {scan\_err} {},
        \newpd {scan\_err} {\off}}}\; \ielse {}\\
      \qquad \codefont {try \sapp (i+1)}
   }\\
   \quad \iin \sapp \codefont{try \sapp 0} \\
  \end{array}  
\\
%% lambda \alpha
\trans[\lambda\tyvar . \ty,,] = %\\
%  \begin{array}{l}
    \Lambda \tyvar_\repname. 
    \Lambda \tyvar_\pdname. \lambda \codefont{f_\ptyvar}. \trans[\ty,,]
%  \end{array}  
\\
%% t1 t2
\trans[\ty_1 \ty_2,,] = 
    \trans[\ty_1,,]\; [\itsem[\ty_2]]\; [\itpdsemstrip[\ty_2]]\; \trans[\ty_2,,]

\\
\end{array}
\]
%\caption{\ddc{} Semantics (cont.)}
\caption{\ddc{} Semantics}
\label{fig:ddc-sem}
\end{figure*}

\begin{figure}
\small
\begin{itemize}
\renewcommand{\labelitemi}{}

%\item %[Unit:]
\item $\ifun {R_{unit}} \iuval \iuval$
\item $\ifun {P_{unit}} \off {\itup{\itup{0,\iok,\ipair \off \off},\iuval}}$

%\item %[Bottom:]
\item $\ifun {R_{bottom}} \iuval \ierr$
\item $\ifun {P_{bottom}} \off ((1,\iecpc,\ipair \off \off),())$

\item %[Pair:]
\item $\ifun {R_{\gS}} {\ipair {r_1} {r_2}} {\itup {\codefont{r_1,r_2}}}$
\item $\ifun{H_{\gS}} {\ictup{h_1,h_2}}{}$ \\
  $\begin{array}{l}
    \ilet {nerr} {\codefont{pos \itup{{h_1}.{nerr}} + pos \itup{{h_2}.{nerr}}}}\\
    \ilet {ec} {\iif {\codefont{h_2.ec} = \iecpc}\; \ithen {\iecpc}\\
    \quad \ielse {\codefont{max\_ec} \iappi \codefont{h_1.ec} \iappi \codefont{h_2.ec}}} \\
    \ilet {sp} {\ictup{h_1.sp.begin, h_2.sp.end}} \\
    \quad \ictup {nerr,ec,sp}
  \end{array}$

\item $\ifun {P_{\gS}} {\ictup{p_1, p_2}} {\ictup {H_{\gS} \itup{p_1.h,p_2.h},\itup{p_1,p_2}}}$

\end{itemize}
\caption{Selected Constructor Functions. 
The type of PD headers is $\iintty
  \iprodi \iecty \iprodi \ispty$. 
  We refer to the projections using
  dot notation as $\codefont{nerr}$, $\codefont{ec}$ and
  $\codefont{sp}$, respectively. A span is a pair of offsets, referred
  to as $\codefont{begin}$ and $\codefont{end}$, respectively.  The full collection of such constructor functions appears in \appref{app:asst-functions}.}
\label{fig:cons-funs}
%\caption{Constructor Functions (cont.)}
\end{figure}


The parsing semantics of a type $\tau$ is a function that transforms some amount of input into a pair of a representation and a parse descriptor, the types of which are determined by $\tau$.
\figref{fig:parser-types} specifies the \implang{} language types of the parsers generated from well-kinded \ddc{} types.  Note that parameterized \ddc{} types require their arguments before they can parse any input.

\figref{fig:ddc-sem} shows the parsing semantics function.  For each
type, the input to the corresponding parser is a bit string and an
offset which indicates the point in the bit string at which parsing
should commence.  The output is a new offset, a representation of the
parsed data, and a parse descriptor. As the bit string input is
never modified, it is not returned as an output.  In addition
to specifying how to handle correct data, each function describes how
to transform corrupted bit strings, marking detected errors in
a parse descriptor. The semantics function is partial, applying only
to well-formed \ddc{} types.

For any type, there are three steps to parsing: parse the
subcomponents of the type (if any), assemble the resultant representation, and
tabulate meta-data based on subcomponent meta-data
(if any). For the sake of clarity, we have factored the latter two
steps into separate representation and PD constructor functions which we define for
each type. For some types, we additionally factor the PD header
construction into a separate function. For example, the representation 
and PD constructors for $\ptrue$ are $\newrepf {unit}$ and $\newpdf
{unit}$, respectively, and the header constructor for products is
${\codefont{H_{\gS}}}$. Selected constructors are shown in
\figref{fig:cons-funs}. We have also factored out some commonly
occuring code into ``built-in'' functions, explained as needed and
defined formally in \appref{app:asst-functions}.

The PD constructors determine the error code and
calculate the error count.  There are three possible error codes:
$\iok$, $\iecerr$, and $\iecpc$, corresponding to the three possible results of a parse: 
it can succeed, parsing the data without errors; it can succeed,
but discover errors in the process; or, it can find an
unrecoverable error and fail.
\trversion{
Note that the the purpose of the $\iecpc$ code is to indicate to any
higher level elements that some form of error recovery is required.
Hence, the whole parse is marked as failed exactly when the parse ends
in failure.}
The error count is determined by subcomponent error counts and any errors associated directly with the type
itself.  
\trversion{
If a subcomponent has errors then the error count is
increased by one; otherwise its not increased at all. We use the
function $\codefont {pos}$, which maps all positive numbers to 1
(leaving zero as is), to assist in calculating the contribution of
subcomponents to the total error count.  Errors at the level of the
element itself - such as constraint violation in constrained types - are
generally counted individually.}

With this background, we can now discuss selected portions of the semantics.
%With this background, we can now understand the semantics. 
The semantics of $\ptrue$ and $\pfalse$ show that they do not consume any input, \ie{}, they do not change the offset. 
A look at their constructors shows that the parse
descriptor for $\ptrue$ always indicates no errors and a corresponding
$\iok$ code, while that of $\pfalse$ always indicates failure with an
error count of one and the $\iecpc$ error code. The semantics of base
types applies the implementation of the base type's parser, provided
by the function $\Iimp$, to the appropriate arguments.  Abstraction
and application are defined directly in terms of \implang language
abstraction and application.  Dependent pairs read the first element
at $\off$ and then the second at $\off'$, the offset returned from
parsing the first element.  Notice that we bind the pair of the
returned representation and parse descriptor to the variable $\codefont{x}$
before parsing the second element, implicitly mapping the 
\ddc{} variable $x$ to the \implang{} language variable $\codefont{x}$ in the process.
Finally, we combine the results
using the constructor functions, returning $\off''$ as the final
offset of the parse.

\trversion{
Sums first attempt to parse according to the left type, returning the resulting
value if it parses without errors. Otherwise, it parses according to
the right type. Intersections read both types starting at the same
point. They advance the stream to the maximum of the two offsets
returned by the component parsers. The construction of the parse
descriptor is similar to that of products. For constrained types, we call the
parser for the underlying type $\ty$, bind $\var$ to the resulting rep
and PD, and check whether constraint is satisfied. The result
indicates whether the data has a semantic error and is used in
constructing the representation and PD. For example, the PD constructor will add
one to the error count if the constraint is not satisfied. Notice that
we advance the stream independent of whether the constraint was
satisfied.
}
Sequences have the most complicated semantics because the number of subcomponents depends upon a combination of the data, the termination predicate, and the terminator type. Consequently, the sequence parser uses mutually
recursive functions $\codefont{isDone}$ and $\codefont{continue}$ to implement this open-ended semantics. 
Function $\codefont{isDone}$ determines if the parser
should terminate by checking whether the end of the source has been
reached, the termination condition $e$ has been satisfied, or the
terminator type can be read from the stream without errors at
$\off$.
Function $\codefont{continue}$ takes four
arguments: two offsets, a sequence representation, and a sequence PD.  The two
offsets are the starting and ending offset of the previous round of
parsing. They are compared to determine whether the parser is
progressing in the source, a check that is critical to ensuring that
the parser terminates. Next, the parser checks whether the sequence is
finished, and if so, terminates. Otherwise, it attempts to read a
separator followed by an element and then continues parsing the
sequence with a call to $\codefont{continue}$.

\trversion{
Finally, the body of the parser creates an initial sequence representation and PD and
then checks whether the sequence described is empty. If not, it reads
an element and creates a new rep and PD for the sequence.  Note that
it passes the PD for $\ptrue$ in place of a separator PD, as no
separator is read before the first element.  Finally, it continues
reading the sequence with a call to $\codefont{continue}$.

Because of  the iterative nature of sequence parsing, the representation and PD are constructed incrementally. The parser first creates an empty representation and PD
and then adds elements to them with each call to
$\codefont{continue}$. The error count for an array is the sum of the
number of separators with errors plus one if there were any element
errors. Therefore, in function ${\codefont{H_{seq}}}$ we first check
if the element is the first with an error, setting $\codefont{eerr}$
to one if so. Then, the new error count is a sum of the old,
potentially one for a separator error, and $\codefont{eerr}$. In
$\newpdf{seq}$ we calculate the element error count by unconditionally
adding one if the element had an error.
}
We translate recursive types into
recursive functions with a special
function name corresponding to the name of the 
bound type   variable.
Recursive type variables translate to these special names.
\trversion{We note that the body of the recursive function is somewhat
  redundant. However, the simpler encoding of $\ifun {f_\ptyvar}
  {\itup{\data,\off}} {\trans[\ty,,] \; \itup{\data,\off}}$ would have
  complicated the meta-theory.}

The $\pscann$ type attempts to parse the underlying type from the
stream at an increasing scan-offset, $i$, from the original offset
$\off$, until success is achieved or a predefined maximum scan-offset
(\cd{scanMax}) is reached.  In the semantics we give here, offsets are
incremented one bit at a time --- a practical implementation would choose
some larger increment ({\it e.g.,} 32 bits at a time).

\trversion{
The definition of $\pcomputen$ just calls the compute constructors.
The representation constructor returns the value computed by $e$, while the PD
records no errors and reports a span of length 0, as no data is
consumed by the computation. The $\pabsorbn$ parser first parses the
underlying type and then calls the absorb constructors, passing only
the PD, which is needed by the rep constructor to determine whether an
error occured while parsing the underlying type. If so, the value
returned is a $\ierr$. Otherwise, it is $\iunitty$.  The absorb parse
descriptor duplicates the error information of its underlying type.


The $\pscann$ type attempts to parse the underlying type from the
stream at an increasing scan-offset, $i$, from the original offset
$\off$, until success is achieved or a predefined maximum scan-offset
(\cd{scanMax}) is reached. Note that, upon success, $i$ is passed to
the PD constructor function, which both records it in the PD and sets
the error code based on it. It is considered a semantic error for the
value to be found at a positive $i$, whereas it is a syntactic error
for it not to be found at all.  }

\section{Meta-theory}
\label{sec:meta-theory}

\cut{This paragraph belongs somewhere in the TR:
A few things to note regarding variable names. First, all variable
names introduced in translation are by definition not equal to the
target of the substitution, nor present in the free variables of the
term being substituted. Second, for those types with bound variables,
we note that the potential alpha-conversion when performing a
substitution on the type exactly parallels any alpha-conversion of the
same variable where it appears in the translation of the type. Last,
all constructors, support functions and base-type parsers are closed
with respect to user-defined variable names.}

One of the most difficult, and perhaps most interesting, challenges of our
work on \ddc{} was determining what
properties we wanted to hold. What are the ``correct''
invariants of data description languages? While there are many
well-known desirable invariants for programming languages, the
meta-theory of data description languages has been
uncharted.
%
We present the following two properties as critical invariants of
our theory. We feel that they should hold, in some form, for any data
description language.
\begin{itemize}
\trversion{
\item {\bf Translation Completeness}: The semantic function
  $\trans[\cdot,,]$ is a total function of well-formed \ddc{} types.
}
\item {\bf Parser Type Correctness}: For a \ddc{} type $\ty$, the
  representation and PD output by the parsing function of $\ty$ will
  have the types specified by $\itsem[\ty]$ and
  $\itpdsem[\ty]$, respectively.
  
\item {\bf Parser Error Correlation}: For any representation and PD output by a
  parsing function, the errors reported in the PD
  will be correlated with the errors present in the representation.
\end{itemize}
%
\trversion{
Before presenting the formal statement of these properties, we specify
some assumptions that we make about \ddc{} base types, their kinds,
types, and implementations, that are necessary for the satisfaction of
these properties. In essence, we assume that the properties we desire of
the rest of the calculus hold for the base types.
}
% In formalizing these properties, we assume that \ddc{} base types, their kinds, representation and parse descriptor types, and parsers satisfy the properties we desire to hold of the rest of the calculus.  
% \appref{app:meta-theory} contains a formal statement of these assumptions.

\trversion{
\begin{theorem}[Translation Completeness]
\label{thm:translation-completeness}
  If \ $\ddck[\ty,,\kind,\mcon]$ then $\trans[\ty,,] = e$.
\end{theorem}

\begin{proof}
  By induction on kinding derivations. For constant case, use the
  first item of Condition~\ref{cond:base-types}.
\end{proof}
}
\trversion{
We continue by stating and proving that parsers are type correct.
However, to do so, we must first establish some typing properties of the Rep
and PD constructors, as at least one of them appears in most
parsing functions. 

With this lemma we now formally state that each constructor
produces a value whose type corresponds to its namesake \ddc{} type.
Note that all universally quantified \ddc{} types $\ty$ are assumed
to be well formed.

\begin{lemma}[Types of Constructors]
\label{lem:nice-types-of-constructors}
\begin{itemize}
\item $\newrepf {true} : \iarrow \iunitty \itsem[\ptrue]$
\item $\newpdf  {true} : \iarrow \ioffty {\itpdsem[\ptrue]}$
\item $\newrepf {false} : \iarrow \iunitty {\itsem[\pfalse]} $
\item $\newpdf  {false} : \iarrow \ioffty {\itpdsem[\pfalse]}$
\item $\forall \ty_1,\ty_2,x.\newrepf {\gS} : \iarrow {\iprod
    {\itsem[\ty_1]}{\itsem[\ty_2]}} {\itsem[\psig x {\ty_1}{\ty_2}]}$
\item $\forall \ty_1,\ty_2,x.\newpdf {\gS} : 
  \iarrow {\iprod {\itpdsem[\ty_1]}{\itpdsem[\ty_2]}}
  {\itpdsem[\psig x {\ty_1}{\ty_2}]}
$
\item $\forall \ty_1,\ty_2.\newrepf {+left} : \iarrow {\itsem[\ty_1]} 
                            {\itsem[\psum {\ty_1}{}{\ty_2}]}$
\item $\forall \ty_1,\ty_2.\newrepf {+right} : \iarrow {\itsem[\ty_2]} 
                            {\itsem[\psum {\ty_1}{}{\ty_2}]}$
\item $\forall \ty_1,\ty_2.\newpdf {+left} : \iarrow {\itpdsem[\ty_1]} 
                            {\itpdsem[\psum {\ty_1}{}{\ty_2}]}$
\item $\forall \ty_1,\ty_2.\newpdf {+right} : \iarrow {\itpdsem[\ty_2]} 
                            {\itpdsem[\psum {\ty_1}{}{\ty_2}]}$
\item $\forall \ty_1,\ty_2.\newrepf {\&} : \iarrow {\iprod
    {\itsem[\ty_1]}{\itsem[\ty_2]}} {\itsem[\pand {\ty_1}{\ty_2}]}$
\item $\forall \hdtvs {\ty_1}{\ty_2}.\newpdf {\&} : 
  \iarrow {\iprod {\itpdsem[\ty_1]}{\itpdsem[\ty_2]}}
  {\itpdsem[\pand {\ty_1}{\ty_2}]}$.
\item $\forall \ty,x,e.\newrepf {set} : \iprod \iboolty {\itsem[\ty]} 
  \iarrowi \itsem[\pset x \ty e]$
\item $\forall \ty,x,e.\newpdf {set} : \iprod \iboolty {\itpdsem[\ty]} 
  \iarrowi \itpdsem[\pset x \ty e]$
\item $\forall \ty,\ty_1.
  \newrepf {seq\_init} : \iarrow \iunitty {\itsem[\pseq \ty
      {\ty_1} {\_}]}$
\item $\forall \ty,\ty_1.
  \newpdf {seq\_init} : \iarrow \ioffty {\itpdsem[\pseq \ty
      {\ty_1} {\_}]}$
\item $\forall \ty,\ty_1.
  \newrepf {seq} : \iarrow {\itsem[\pseq \ty
      {\ty_1} {\_}] \iprodi \itsem[\ty]} 
  {\itsem[\pseq \ty
      {\ty_1} {\_}]}$
\item $\forall \ty,\ty_1.
  \newpdf {seq} : 
  \itpdsem[\pseq \ty {\ty_1} {\_}] \iprodi
   \itpdsem[\ty_1] \iprodi \itpdsem[\ty] \iarrowi \\
  \itpdsem[\pseq \ty {\ty_1} {\_}]$
% \item $\forall \ty,\ty_1,\ty_2,e.
%   \newrepf {seq\_init} : \iarrow \iunitty {\itsem[\pseq \ty
%       {\ty_1} {\pterm e {\ty_2}}]}$
% \item $\forall \ty,\ty_1,\ty_2,e.
%   \newpdf {seq\_init} : \iarrow \ioffty {\itpdsem[\pseq \ty
%       {\ty_1} {\pterm e {\ty_2}}]}$
% \item $\forall \ty,\ty_1,\ty_2,e.
%   \newrepf {seq} : \iarrow {\itsem[\pseq \ty
%       {\ty_1} {\pterm e {\ty_2}}] \iprodi \itsem[\ty]} 
%   {\itsem[\pseq \ty
%       {\ty_1} {\pterm e {\ty_2}}]}$
% \item $\forall \ty,\ty_1,\ty_2,e.
%   \newpdf {seq} : 
%   \itpdsem[\pseq \ty {\ty_1} {\pterm e {\ty_2}}] \iprodi
%    \itpdsem[\ty_1] \iprodi \itpdsem[\ty] \iarrowi \\
%   \itpdsem[\pseq \ty {\ty_1} {\pterm e {\ty_2}}]$
\item $\forall e.\newrepf {compute} : \forall \ga.\iarrow \ga {\itsem[\pcompute
    e \ga]}$
\item $\forall e.\newpdf {compute} : \forall \ga.\iarrow \ioffty
  {\itpdsem[\pcompute e \ga]}$
\item $\forall \ty.\newrepf {absorb} : \iarrow {\itpdsem[\ty]}
  {\itsem[\pabsorb \ty]}$
\item $\forall \ty.\newpdf {absorb} : \iarrow {\itpdsem[\ty]}
  {\itpdsem[\pabsorb \ty]}$
\item $\forall \ty.\newrepf {scan} : \itsem[\ty] \iarrowi
  \itsem[\pscan \ty]$
\item $\forall \ty.\newpdf {scan} : \iintty \iprodi \itpdsem[\ty] \iarrowi
  \itpdsem[\pscan \ty]$
\item $\forall \ty.\newpdf {scan} : \iintty \iprodi \itpdsem[\ty] \iarrowi
  \itpdsem[\pscan \ty]$
\item $\forall \ty.\newrepf {scan\_err} : \iunitty \iarrowi
  \itsem[\pscan \ty]$
\item $\forall \ty.\newpdf {scan\_err} : \ioffty \iarrowi
  \itpdsem[\pscan \ty]$
\end{itemize}  
\end{lemma}

\begin{proofsketch}
  By inspection of code. First we infer (by hand) the (polymorphic)
  type of each function. Then we verify that it matches the types
  specified above.
\end{proofsketch}
}

\trversion{
\begin{figure}
{\small
\fbox{$\ptyc{\rctxt} = \ctxt$} 
    
\[
  \ptyc{\cdot} = \cdot \qquad\qquad
  \ptyc{\rctxt,\ptyvar{=}\pmu \ptyvar \ty} = \ptyc \rctxt,\codefont{f_\ptyvar}{:}
  \kTrans[\kty,\asub \rctxt {\pmu \ptyvar \ty}]
\]}
  \caption{Recursive Parser \Implang{} Types}
  \label{fig:rec-parser-types}
\end{figure}
}

To prove our type correctness theorem by induction, we must account
for the fact that any free recursive type variables in
a \ddc{} type $\ty$ will become free function variables
in $\trans[\ty,,]$.  
\trversion{
To that end, we define in \figref{fig:rec-parser-types}
the function $\ptyc \rctxt$, 
which maps recursive variable contexts
$\rctxt$ to typing contexts $\ctxt$. 
}
\poplversion{
To that end, we define the function $\ptyc \rctxt$, 
which maps recursive variable contexts
$\rctxt$ to typing contexts $\ctxt$:
\vskip -1.5ex
{\small
\[
\begin{array}{l}
  \ptyc{\cdot} = \cdot \\[1ex]
  \ptyc{\rctxt,\ptyvar{=}\pmu \ptyvar \ty} = \ptyc \rctxt,\codefont{f_\ptyvar}{:}
  \kTrans[\kty,\asub \rctxt {\pmu \ptyvar \ty}]
\end{array}
\]%
}%
}%
\noindent
We also apply $\rctxt$ to $\ty$
to close any open references to recursive types before
determining the corresponding parser type. 
\trversion{  We do not explicitly require that $\ctxt$
be well formed in the premises, as it follows from the fact that $\ty$
is well formed in $\ctxt$.
}

\begin{theorem}[Type Correctness]
\label{thm:type-correctness}
  If  $\wfd \ctxt \rctxt$ and
   $\ddck[\ty,{\pctxt;\rctxt;\ctxt},\gk,\mcon]$ then
  $\stsem[{\trans[\ty,,]},{\ctxt,\ptyc \rctxt},
            {\kTrans[\kind,\asub \rctxt \ty]}]$.
\end{theorem}

\begin{proof}
  By induction on the height of the second derivation.
\end{proof}

\begin{corollary}[Type Correctness of Closed Types]
  If $\ddck[\ty,,\gk,\con]$ then
  $\stsem[{\trans[\ty,,]},,\kTrans[\kind,\ty]]$.  
\end{corollary}

We start our formalization of the error-correlation property by
defining representation and PD correlation.
Informally, a representation and a PD are correlated when the number
of errors recorded in the PD is at least as many as the number of
errors in the representation and semantic errors, \ie{},
constraint violations, are properly reported.  Formally, we define
correlation using two mutually recursive definitions.  The first,
$\corrkl \ty r p$, defines error correlation between a representation
$r$ and a parse descriptor $p$ at type $\ty$.  It does so by
computing a weak-head normal form $\tyval$ for $\ty$ and then
using the subsidiary relation $\corr \tyval r p$, which is defined for
all weak-head normal types $\tyval$ with base kind $\kty$.
Types with higher kind such as abstractions are excluded from this definition
as they cannot directly produce representations and PDs.
\figref{fig:revised-ddc-syntax} defines the weak-head
normal types $\tyval$ and give normalization rules while the
following definitions specify error correlation.   
Below, we abbreviate $p.h.{nerr}$ as $p.{nerr}$.
and use $\mathtt{pos}$ to denote the function which returns zero when
passed zero and one otherwise.



\begin{figure}
\small
\begin{bnf}
%   \name{Kinds} \meta{\gk} \::= \kty \| \ity \-> \gk 
%                                \pext{\| \gk \-> \gk} \\
  \mname{Normalized\\ Types}{2} \meta{\tyval} \::= 
    \ptrue\| \pfalse \| \pbase{e} \| \plam{\var}{\ity}{\ty} \|
%  \nlnalt{Types}
%%        \|
%%       \pext{\plam{\ptyvar}{\gk}{\ty} \| \papp{\ty}{\ty} \nlalt}
%%       \pxpd{\ty}{e}
%%       \pext{\nlalt
%%         \ptransform{e}{e}{\ty} \| 
%%       }
    \psig x \ty \ty  \nlalt
    \psum \ty e \ty  \| \pand \ty \ty \|
    \pset x \ty e \|
    \pseq \ty \ty {\pterm e \ty} \nlalt
    \pcompute e \ity \| \pabsorb \ty \| \pscan{\ty} 
    \\
  \name{Types} \meta{\ty} \::= \tyval \| \papp{\ty}{e} \|
    \ptyvar \| \pmu{\ptyvar}{\ty} 
\end{bnf}  
\[
  \infer{
    \papp {\ty} {e} \stepsto \papp {\ty'} {e}
  }{
    \ty \stepsto \ty'
  }
\quad
  \infer{
    \papp {\tyval} {e} \stepsto \papp {\tyval} {e'}
  }{
    e \stepsto e'
  }
\quad
  \infer{
    \papp {(\plam x {} \ty)} {v} \stepsto \ty[v/x]
  }{}
\quad
  \infer{
    \pmu \ptyvar \ty \stepsto \ty[\pmu \ptyvar \ty/\ptyvar]
  }{}
\]
  \caption{\ddc{} Weak-Head Normal Types and Normalization}
  \label{fig:ddc-reduction-rules}
  \label{fig:revised-ddc-syntax}
\end{figure}

\begin{definition}
$\corrkl \ty r p$ iff if $\ty \kstepsto \tyval$ then $\corr \tyval r p$.
\end{definition}

\begin{definition}[Representation and PD Correlation Relation]
$\corr \tyval r p$ iff exactly one of the following is true:
  \begin{itemize}
  \item $\tyval = \ptrue$ and $r = \iuval$ and $p.{nerr} = 0$.
  \item $\tyval = \pfalse$ and $r = \ierr$ and $p.{nerr} = 1$.
  \item $\tyval = \pbase{e}$ and $r = \iinld \ity \const$ and $p.{nerr} = 0$.
  \item $\tyval = \pbase{e}$ and $r = \iinrd \ity \ierr$ and $p.{nerr} = 1$.
  \item $\tyval = \psig x {\ty_1} {\ty_2}$ and $r =\ipair {r_1} {r_2}$ and $p =
    \ipair h {\ipair {p_1} {p_2}}$ 
    and $h.{nerr} = \mathtt{pos}(p_1.{nerr}) + \mathtt{pos}(p_2.{nerr})$, $\corrkl
    {\ty_1} {r_1} {p_1}$ and $\corrkl {\ty_2[(r,p)/x]} {r_2} {p_2}$.
  \item $\tyval = \psum {\ty_1} e {\ty_2}$ and $r =\iinld {\ity}{r'}$
    and $p = \ipair h {\iinld {\ity}{p'}}$
    and $h.{nerr} = \mathtt{pos}(p'.{nerr})$ and $\corrkl
    {\ty_1} {r'} {p'}$.
  \item $\tyval = \psum {\ty_1} e {\ty_2}$ and $r =\iinr {r'}$
    and $p = \ipair h {\iinr {p'}}$
    and $h.{nerr} = \mathtt{pos}(p'.{nerr})$ and $\corrkl
    {\ty_2} {r'} {p'}$.
  \item $\tyval = \pand {\ty_1} {\ty_2}$, $r = \ipair {r_1} {r_2}$ and $p =
    \ipair h {\ipair {p_1}{p_2}}$, 
    and $h.{nerr} = \mathtt{pos}(p_1.{nerr}) + \mathtt{pos}(p_2.{nerr})$, 
    $\corrkl {\ty_1} {r_1} {p_1}$ and $\corrkl {\ty_2} {r_2} {p_2}$.
  \item $\tyval = \pset x {\ty'} e$, $r = \iinld \ity {r'}$ and $p =
    \ipair h {p'}$, 
    and $h.{nerr} = \mathtt{pos}(p'.{nerr})$, $\corrkl {\ty'}{r'}{p'}$
    and $e[(r',p')/x] \kstepsto\itrue$.
  \item $\tyval = \pset x {\ty'} e$, $r = \iinrd \ity {r'}$
    and $p = \ipair h {p'}$,
    and $h.{nerr} = 1 + \mathtt{pos}(p'.{nerr})$,
    $\corrkl {\ty'}{r'}{p'}$ and $e[(r',p')/x] \kstepsto \ifalse$.
  \item $\tyval = \pseq {\ty_e}{\ty_s}{\pterm {e,\ty_t}}$, 
    $r = \ipair {len} {\iarr{\vec {r_i}}}$, $p = \itup{h,\itup{{neerr},{len}',\iarr {\vec {p_i}}}}$,
    ${len} = {len}'$, ${neerr} = \sum_{i=1}^{len}
    \mathtt{pos}(p_i.{nerr})$, $\corrkl {\ty_e}
    {r_i} {p_i}$, (for $i=1 \ldots {len}$), and
    $h.{nerr} \geq \mathtt{pos}({neerr})$.
%   \item $\tyval = \pmu \ptyvar {\ty'}$ and 
%     $\corrkl {\ty'[\pmu \ptyvar {\ty'}/\ptyvar]} r p$.
  \item $\tyval = \pcompute e \ity$ and $p.{nerr} = 0$.
  \item $\tyval = \pabsorb {\ty'}$, $r = \iinl \iuval$, and $p.nerr = 0$.
  \item $\tyval = \pabsorb {\ty'}$, $r = \iinr \ierr$, and $p.nerr > 0$.
  \item $\tyval = \pscan {\ty'}$, $r =\iinl {r'}$,
      $p = \ipair h {\iinl {\ipair i {p'}}}$,
      $h.nerr = \mathtt{pos}(i) + \mathtt{pos}(p'.nerr)$, and
      $\corrkl {\ty'}{r'}{p'}$.
  \item $\tyval = \pscan {\ty'}$,
      $r =\iinr \ierr$,
      $p = \ipair h {\iinr \iuval}$, and
      $h.{nerr} = 1$.
  \end{itemize}
\end{definition}

\trversion{
Once again, we first need to prove error correlation properties of
the Rep and PD constructors, as expressed in the following lemma.

\begin{lemma}[Correlation Properties of Constructors]
  \label{lem:cons-props}
  \begin{itemize}
  \item $\corr \ptrue {\newrep {true} {}} {\newpd {true} \off}$.
  \item $\corr \pfalse {\newrep {false} {}} {\newpd {false} \off}$.
  \item If $\corrkl {\ty_1} {r_1} {p_1}$ and $\corrkl {\ty_2[(r,p)/x]} {r_2} {p_2}$
    then\\ $\corr {\psig x {\ty_1} {\ty_2}}
    {\newrep {\gS} {r_1,r_2}}{\newpd {\gS} {p_1,p_2}}$.
  \item If $\corrkl \ty r p$ then $\corr {\psum \ty e {\ty'}} 
      {\newrep {+left} r} {\newpd {+left} p}$.
  \item If $\corrkl \ty r p$ then $\corr {\psum {\ty'} e \ty} 
      {\newrep {+right} r} {\newpd {+right} p}$.
  \item If $\corrkl {\ty_1} {r_1} {p_1}$ and $\corrkl {\ty_2} {r_2} {p_2}$
    then\\ $\corr {\pand {\ty_1} {\ty_2}}
    {\newrep {\&} {r_1,r_2}}{\newpd {\&} {p_1,p_2}}$.
  \item If $\corrkl \ty r p$ and $e[(r,p)/x] \kstepsto c$ then\\ $\corr {\pset x \ty e} 
    {\newrep {set} {c,r}} {\newpd {set} {c,p}}$
  \item $\corr {\pseq \ty {\ty_s}{\pterm e {\ty_t}}} 
    {\newrep {seq\_init} {}} {\newpd {seq\_init} \off}$.
  \item If $\corr {\pseq \ty {\ty_s} {\pterm e {\ty_t}}} r p$ and
    $\corrkl \ty {r'} {p'}$ then for any $p''$, $\corr {\pseq \ty {\ty_s}{\pterm e {\ty_t}}}
    {\newrep {seq} {r,r'}} {\newpd {seq} {p,p'',p'}}$.    
  \item $\corr {\pcompute e \ity} {\newrep {compute} {e}} {\newpd {compute} \off}$.
  \item $\corr {\pabsorb \ty} {\newrep {absorb} p} {\newpd {absorb} p}$.
  \item If $\corrkl {\ty} r p$ then $\corr {\pscan \ty} 
      {\newrep {scan} r} {\newpd {scan} {i,p}}$.
  \item $\corr {\pscan \ty} 
      {\newrep {scan\_err} {}} {\newpd {scan\_err} \off}$.
  \end{itemize}
\end{lemma}

\begin{proof}
  By inspection of code. 
\cut{
  \reminder{Fix the following or drop it: }
  Array case is most complicated, in particular proving the clause
  $h.{nerr} \geq \mathtt{pos}({neerr})$. To do so, you must prove that
  $H_{seq}$ maintains this invariant. The first case of the match is the
  hard one, as ${nerr}$ is 0 (if its $1$, then it must be greater than
  or equal to $\mathtt{pos}(n)$, for any $n$).  First, as $h.{nerr} =
  0$, so too must $neerr$. Next, note that in this first case,
  $h1.{nerr} = 0$. Now, the new value of ${neerr}$ is just the sum of
  the original ${neerr}$ and $\mathtt{pos}(h1.{nerr})$, that is, $0 +
  0$.}
\end{proof}
}

Definition~\ref{def:err-corr} specifies the exact property we require
of parsing functions. At base kind, we require that any representation and PD
returned by the parser must be correlated. At higher kind, we require
that the function preserve the property of error correlation. Hence,
the definition is a simple form of logical relation.
Lemma~\ref{lem:err-corr-at-T} states that any well-formed type of base
kind is error-correlated.

\begin{definition}[Error Correlation Relation]
\label{def:err-corr}
$\ecpred \ty \kind$ iff exactly one of the following is true:
  \begin{itemize}
  \item $\kind = \kty$ and if $\trans[\ty,,] \sapp \spair<B,\off> \kstepsto
  \spair<\off',r,p>$ then $\corrkl \ty r p$
  \item $\kind = \ity \iarrowi \kind'$ and if $\stsem[v,,\ity]$
    then $\ecpred {\ty \sapp v} {\kind'}$
  \end{itemize}
\end{definition}

\begin{lemma}[Error Correlation at Base Kind]
\label{lem:err-corr-at-T}
If $\ddck[\ty,,\kty,\con]$ and $\trans[\ty,,] \sapp \spair<B,\off> \kstepsto
  \spair<\off',r,p>$ then $\corrkl \ty r p$.
\end{lemma}

\begin{proof}
  By induction on the height of the second derivation.
\end{proof}

\begin{theorem}[Error Correlation]
\label{thm:err-corr}
If $\ddck[\ty,,\kind,\con]$ then $\ecpred \ty \kind$.
\end{theorem}

\begin{proof}
  By induction on the size of the kind $\kind$. For $\kind = \kty$, we
  use Lemma~\ref{lem:err-corr-at-T}.
\end{proof}

\trversion{
We conclude this section with a useful property of correlated representations
and PDs. If the PD reports no errors, then there are no syntactic
errors in the representation data structure {\it at any level}. 
We formally define
{\it clean} (error-free) values next, followed by the statement of the
property itself.

\begin{definition}[Clean Value]
$\noerr v$ iff exactly one of the following is true:
\begin{itemize}
\item $v = c$ and $c \neq \ierr$.
\item $v = \ifun f x e$.
\item $v = \ipair {v_1} {v_2}$ and $\noerr {v_1}$ and $\noerr {v_2}$.
\item $v = \iinl {v'}$ and $\noerr {v'}$.
\item $v = \iinr {v'}$ and $\noerr {v'}$.
\item $v = \iarr{v_1 \cdots v_n}$ and $\noerr {v_i}$ for $i=1\ldots n$.
\end{itemize}
\end{definition}

\begin{lemma}
  If $\corrkl \ty r p$ and $p.h.nerr = 0$ then $\noerr r$.
\end{lemma}

\begin{proof}
  By induction on the structure of r.
\end{proof}
}

\begin{corollary}
  If $\corrkl \ty r p$ and $p.h.nerr = 0$ then there are no syntactic
  or semantic errors in the representation data structure $r$.
\end{corollary}


%%% Local Variables: 
%%% mode: latex
%%% TeX-master: "semantics"
%%% End: 


\section{Related Work}
\label{sec:related}

There are many useful tools designed to help programmers generate
parsers.  Examples include compiler technology such as the many
variants of Lex and Yacc as well as interpreter technology such the
parser combinator libraries found in functional programming languages
(Haskell~\cite{hutton+:parser-combinators}, for example).  Likewise,
there are tools to help programmers generate printers.  Each of these
technologies are very useful in their own domain, but none of them
could possibly be substituted for \padsml{}.  A key difference is that
a single \padsml{} description is sufficient to generate a large
collection of useful data processing tools including a parser, a
printer, a statistical error analysis, a format debugger, an \xml{}
translator, and in the future, a query engine~\cite{fernandez+:padx},
a content-based search engine~\cite{lv+:cbs,oh:siw}, more statistical
analysis, etc.  Though we chose a compiled solution, a parser or
printer combinator library might have served as an alternative
implementation strategy for certain individual components of our tool
suite.  However, by themselves, parser or printer combinator
libraries, \lex{} or \yacc, are not properly architected to give the
same, simple and powerful user experience as \padsml{}, which can be
summed up through the motto ``one declarative description; many
reliable tools.''

% A second way to characterize some of the differences between parser
% technologies is to recognize that \padsml{} is higher level (leaving
% out semantic actions, thereby making descriptions relatively simpler
% and enabling post facto tool implementation) and more domain specific
% (generating tools useful for processing ad hoc data, such as the
% \xml{} translator and statistical analysis, but not particularly
% useful for programming language implementation, for instance).  There
% are also many technical differences between the parsing technologies
% but we believe these are less important than the high-level vision.

Generic
programming~\cite{jeuring+:polytypic-programming,hinze+:generic-programming,lammel+:syb}
and design patterns such as the visitor are technologies that can
facilitate the implementation of type-directed data structure
traversals.  Lammel and Peyton Jones' original ``scrap your
boilerplate'' article~\cite{lammel+:syb} provides a detailed summary
of the trade-offs between different techniques.  We investigated using
one of these techniques before implementing the generator for
\padsml{} traversal functors from scratch.  However, we found most
advanced techniques for functional programming languages required
features such as type classes that are only present in variants of
Haskell.  The generated \padsml{} traversal functors are less flexible
than some of these traversal techniques, but they suffice for helping
us program the tools we have implemented, and for many more tools for
ad hoc data we are considering implementing.  On the other hand, we
have not seen type directed programming techniques developed for
systems of dependent types similar to the \padsml{} dependent type
structure.  
% However, to summarize the central differences between
% generic programming and \padsml{}, generic programming is a powerful
% {\em implementation technique} which can be used to implement many
% programs, including some of the programs generated from a \padsml{}
% description.  \padsml{}, however, is a higher-level system that
% directly provides a simple and powerful user experience for analysts
% who need to parse, print, process or transform ad hoc data.


%Again, the phrase ``one declarative description; many 
%reliable tools'' is a concise summary of \padsml{} goals.

% As with parser and printer
% combinator libraries, eview generic programming techniques as an
% implementation technique


% are two other closely-related 
% technologies that, like parser or printer generators, might have
% been used as implementation components for parts of the \padsml{} system.
% Indeed, before beginning our implementation we investigated these technologies
% to see whether or not they would apply in our context.  Unfortunately,
% all the work we are aware of is quite specific to Haskell.  For instance,
% Hinze's "Generics for the Masses"~\cite{hinze+: improves upon a number of
% previous efforts because he "only" requires Haskell 98, not Haskell 98
% with extensions.
  
% There are also a number of technical differences in the parsing
% technology that \padsml{} uses (particularly 
% the dependent types, critical for many
% ad hoc data formats but unavailable in Yacc)
% By generating
% parsers and printers together \padsml{} also provides a very convenient
% framework for programmers to implement their own format translators




% It is possible to generate any number of tools from
% a single \padsml{} type because the type is 
% limited to describing properties of the data.
% The descriptions themselves do not contain an 
% algorithmic component similar to the semantic actions found in Yacc
% or the associated functional parameters 
% a The \padsml{} architecture is designed

% A parser or printer combinator library might have been helped
% us implement the specific components of the

% Some of the oldest tools for describing data formats are parser
% generators for compiler construction such as
% Lex and Yacc.  While excellent for parsing programming languages, Lex and Yacc
% are too heavyweight for parsing many of the simpler ad hoc data formats that
% arise in networking, the computational sciences, finance, \etc{}   
% In addition, Lex and Yacc do not support data-dependent parsing, 
% do not generate internal representations automatically, 
% and do not supply a collection of value-added tools such as
% \padsml's \xml{} translator.

% More modern compiler construction tools alleviate
% several of the problems of Lex and Yacc by providing more
% built-in programming support.  For instance,
% the ANTLR parser generator~\cite{antlr} allows the user to add
% annotations to a grammar to direct construction of a parse tree.
% Demeter's class dictionaries~\cite{lieberherr+:class-dictionaries}
% can generate parsers that construct internal parse trees
% as well as traversal functions, much like the traversal functions
% generated by \padsml.  However, these tools do not have dependent
% and polymorphic data descriptions or a formal semantics.  Moreover,
% they were designed for imperative and object-oriented languages 
% as opposed to strongly typed functional languages like \ml{}.

The networking community has developed a number of domain-specific
languages, including PacketTypes~\cite{sigcomm00},
DataScript~\cite{gpce02} and Bro's~\cite{paxson:bro} packet processing
language for parsing and printing binary data.  Like \padsml{}, these
languages use a type-directed approach to describing ad hoc data and
permit the user to define semantic constraints.  In contrast to our
work, these systems handle only binary data and assume the data is
error-free or halt parsing if an error is detected. DFDL is a data
format specification language with an XML-based syntax and type
structure~\cite{dfdl-proposal,dfdl-primer}. At this stage in its
development, it appears DFDL remains a language specification; a tool
generation architecture has not yet been developed. We believe that
DFDL is similar in its expressiveness to \padsc{}.  However, because
the specification is still under development, we cannot give a more
detailed comparison at this point.

A somewhat different class of languages includes
\textsc{ASN.1}~\cite{asn} and \textsc{ASDL}~\cite{asdl}.  Both of
these systems specify the {\em logical\/} in-memory representation of
data and then automatically generate a {\em physical\/} on-disk
representation.  This technology does not help process data that
arrives in predetermined, ad hoc formats.

There are a number of tools designed to deal with converting ad hoc
data formats into \xml{} and various related tasks, including
XSugar~\cite{brabrand+:xsugar2005} and the Binary Format Description
language (BFD)~\cite{bfd}. The scope of both of these projects is
limited to conversion to-and-from \xml{}, and neither seek to provide
programmers with access to the raw data, nor produce a broad suite of
data processing tools.

% There are probably hundreds of tools that one might use if their data were
% in \xml.  However, the point of PADS is to allow scientists whose data is {\em not}
% already in \xml to get work done, particularly when that data contains errors,
% as ad hoc data often does.  Since many processes, machines, programs and other devices
% currently output data and a whole most of

Commercial database products provide support for
parsing data in external formats so the data can be imported into
their database systems, but they typically support a limited number of
formats.  Also, no declarative description of the
original format is exposed to the user for their own use, and they
have fixed methods for coping with erroneous data.  For these reasons,
\padsml{} is complementary to database systems.  

On the theoretical front, the scientific community's understanding of
type-based languages for data description is much less mature.  To the
best of our knowledge, our previous work on the
DDC~\cite{fisher+:next700ddl} was the first to provide a formal
interpretation of dependent types as parsers and to study the
properties of these parsers including error correctness and type
safety.  The current paper extends and improves our earlier work by
simplifying the basic theory in a number of subtle but important ways
and by adding polymorphic types for the purpose of code reuse.
Regular expressions and context-free grammars, the basis for Lex and
Yacc have been well-studied, but they do not have dependency, a key
feature necessary for expressing constraints and parsing ad hoc
scientific data.  {\em Parsing Expression Grammars} (PEGs), studied in
the early seventies~\cite{birman+:parsing}, revitalized more recently
by Ford~\cite{ford:pegs} and implemented using ``packrat parsing''
techniques~\cite{ford:packrat,grimm:packrat}, are somewhat more
similar to \padsml{}'s recursive descent parsers. However, our
multiple interpretations of types in the DDC makes our theory
substantially different from the theory of PEGs.

% {\em PEG difference: not just dependency, but error handling?}


% To our knowledge, we are the first to attempt to specify a semantics for
% data description languages based on types such as \packettypes{},
% \datascript{} or \pads.  
% %Prior to our work, this family of languages 
% %was described informally and by example.  There was no precise
% %connection to formal dependent type theory.

% Of course, there are other formalisms for
% defining parsers, most famously, regular expressions and
% contex-free grammars.  In terms of recognition power,
% these formalisms differ from our type theory
% in that they have nondeterministic choice, but do not have
% dependency or constraints.  We have found that 
% dependency and constraints are absolutely essential for
% describing most of the ad hoc data sources we have studied.
% Perhaps more importantly though, unlike standard theories of
% context-free grammars,
% we do not treat our type theory merely as a recognizer for
% a collection of strings.  Our type-based descriptions 
% define {\em both} external data formats {\em and} 
% rich invariants on %(\ie{} types for)
% the internal parsed data structures.  This dual interpretation
% of types lies at the heart of tools such as \pads, \datascript{} and
% \packettypes{}.  
% %\pads{} programmers, for instance, demand that
% %representations produced by their \pads{} parsers have the expected type and 
% %count on the fact that the associated PD is accurately correlated
% %with the representation.  
% %Existing formalisms simply do not address
% %this elements of data description languages.

% {\em Parsing Expression Grammars} (PEGs),
% studied in the early 70s~\cite{birman+:parsing} and revitalized more 
% recently by Ford~\cite{ford:pegs}, 
% evolved from context-free grammars but
% have deterministic, prioritized choice like \ddc{} as opposed to
% nondeterministic choice.  Though PEGs have syntactic lookahead operators,
% they may be parsed in linear time through the use of
% ``packrat parsing'' techniques~\cite{ford:packrat,grimm:packrat}.
% Once again, the dual interpretation of types in \ddc{} both as
% data descriptions and as classifiers for internal representations
% make our theory substantially different from the theory of PEGs.
% %In practice, PEGs has not been used to parse ad hoc data.

% {\sc antlr}~\cite{antlr}, a popular programming language parsing tool, 
% uses top-down recursive descent
% parsing and appears roughly similar in recognition power to PEGs and \ddc.
% {\sc antlr} also allows programmers to place annotations
% in the grammar definitions to guide construction of an abstract syntax
% tree. However, all nodes in the abstract syntax tree have a 
% single type, hence the guidance is rather crude when compared with
% the richly-typed structures that can be constructed using
% \ddc.


% Practical experience indicates that
% tools based on these formalisms, such as the many variations of
% Lex and Yacc, are highly effective for processing
% programming language syntax.  However, there is also ample evidence
% that these tools are a poor fit for processing
% ad hoc data --- simply put, {\em no one ever uses Lex or Yacc for 
% these tasks}.
% Unfortunately, the nondeterminism and lack of
% dependency in these formalisms limit their suitability to formalizing
% data description languages. While the parsing expression grammars
% (PEG) formalism~\cite{ford:parsing-expression-grammars} is
% significantly closer to the \ddc{}, it too lacks the necessary
% dependency.

% Less related, but still relevant, are Haskell's parsing
% combinators. While these are not a formalism, they do provide an
% elegant manner in which to express parsers. Hence, while we chose to
% define our parsing semantics in the polymorphic lambda calculus,they
% potentialy provide a more elegant alternative.

% % {\em Mention dependent type theory work?}


\section{Conclusions}
\label{sec:future}

\cut{
\padsml{} is already an effective, working system for data description
and processing.  However, there are a number of ways we plan to make it 
even better.

First, there are a number of properties of data descriptions
a programmer might want to infer or verify.  For example, it is not hard to
write a non-terminating data description by accident.  It
is also possible to write a description with completely redundant
subparts (dead parser code).  While these problems might be caught 
through testing,
we would prefer to catch them at compile time.  
% It is also often useful
% for programmers to know the ``size'' of any data that matches 
% a description.  For instance, programmers describing network packets can use
% such properties to check
% their work. 
Consequently,
we plan to explore development a \padsml{} ``type checker'' 
to infer description properties and catch obvious errors.

A second long-term goal is to build a collection of
higher-level, format-independent data analysis tools.  By
``higher-level'' tools, we mean tools that perform semantic data
analysis as opposed to simpler, low-level syntactic transformation
(such as \xml\ conversion) and analysis.  
Tools in this category include tools for
content-based search, clustering, statistical data modeling, data 
generation and machine learning.  We believe that if we can automatically
generate stand-alone,
end-to-end tools that perform these functions over arbitrary data, 
we can have a substantial impact on the
productivity of many researchers in fields ranging from computational
in biology and networking.  We hope to provide access to these tools
through LaunchPADS, our data visualization 
environment~\cite{launchpads:planx,launchpads:sigmod}, which currently
only interfaces with \padsc{}.

Third, as mentioned in Section~\ref{sec:intro}, ad hoc data sources are often
very large scale.  Large data volumes often require that the data be
processed without loading it into main memory all at once.  The
\padsc{} language accommodates efficient processing of very
large-scale data~\cite{fisher+:pads} by supporting multiple-entry
point parsing, which permits a user to write tools that have fixed
memory requirements and that can yield a result in one scan of the
data source.  We plan to explore similar techniques in \padsml{}. 
}
% \section{Conclusions}
% \label{sec:conc}

Vast quantities of important information exist only in ad hoc formats.  
Data analysts desperately need reliable, high-level tools to 
help them document, parse, analyze, transform, query, and visualize such data.  
\padsml{} is a high-level domain-specific language and system
designed for this purpose.  
Inspired by the type structure of functional
programming languages, \padsml{} uses dependent
polymorphic recursive data types to describe the syntax and the semantic properties of ad hoc data sources.  The language is compact and expressive, capable of describing data from diverse domains including networking, computational biology, finance, and cosmology. 
The \padsml{} compiler uses a ``types as modules'' compilation strategy
in which every \padsml{} type definition is compiled into
an \ocaml{} module containing types for data representations
and functions for data processing.  Functional programmers
can use the generated modules to write clear and concise {\em format-dependent}
data processing programs.  Furthermore, our system design
allows external tool developers to write new {\em format-independent} tools
simply by supplying a module that matches the appropriate generic
signature.  
To give \padsml{} a precise semantics, we have simplified and extended the Data Description Calculus (\ddc)~\cite{fisher+:next700ddl} to account for parametric polymorphism.


We hope the \padsml{} system can serve as a 
challenge problem for researchers studying functional programming language
design and implementation.  In particular, our ``types as modules'' 
compilation strategy pushes against
the limits of modern module system design --- \ocaml{}'s experimental
recursive modules do not allow us to implement recursive types as
recursive modules in the natural way we envision.  
In addition, future \padsml{} programs
might be phrased extremely elegantly
as (dependently) type-directed programs, but mainstream
languages lack either dependent types or type-directed programming
features, or  both.  Lastly, rather than erasing
dependent typing information upon translation of \padsml{} into \ocaml{},
it would be ideal to preserve the dependency and to verify
that data processors preserve necessary data invariants.
Unfortunately,
sufficiently practical and powerful dependent type systems 
do not currently exist.  So while functional languages are clearly the
``programming tools of choice for discriminating hackers,''
many challenges remain in the domain of ad hoc data processing.

% Unfortunately again, while there are 
% ``generics for the masses,''~\cite{ it seems the ``the masses'' must
% program in Haskell 98.

% We challenge 

% The poses
% the number of challenges

% The \padsml{} compilation strategy also provides a stimulating and
% practical test case for researchers studying functional language
% design.  Since recursive types are compiled into recursive modules and
% parameterized types are compiled into functors, \padsml{} pushes the
% limits of the most expressive modern module systems.  It also suggests
% a collection of problems for researchers studying type-directed
% programming.  We encountered these limits in our work implementing
% \padsml{} in
% \ocaml, but rather than changing
% our basic compilation strategy, which we feel is very natural and elegant,
% we have left certain combinations of features (recursion and polymorphism)
% unimplemented.
% We challenge functional language designers to extend their favourite
% language to meet the demands of our application.




\section*{Acknowledgments}

We would like to thank Derek Dreyer for discussions and advice on
advanced module systems.

\bibliographystyle{abbrv}
\bibliography{pads}
%\bibliography{../common/pads-long,../common/pads}

\appendix
\section{Language Syntax}
{\allowdisplaybreaks
\noindent
{\bf Syntax of data descriptions and other types}
\label{app:syntax-dd}
\begin{bnf}
\name{Constants} \meta{k} \::= \mcd{true} \| \mcd{false} \| \mcd{()} \| ...
\\
\name{Type Variables} \meta{\alpha}
\\
\name{Type Names} \meta{t}
\\
\name{\Core{} Types} \meta{T} \::= 
  \alpha 
\| {Pbase} 
\| M 
\nlalt \ppair x {T_1} {T_2} 
\| \precord {\nont{ffts}} 
\| \nont{tas}\;t(M) 
\nlalt \pset x T M 
\nlalt \parray T {M_{sep}} {M_{term}} 
\\
\name{\Core{} Datatypes} \meta{D} \::= 
  \mcd{datatype}\; \nont{tps}\; t(x{:}F) = \nont{b} \nlalt
  \mcd{type}\; \nont{tps}\; t(x{:}F) = T
\\
\name{Type Parameters} \meta{tps} \::= \cdot \| \alpha \| (\nont{tvs})
\\
\name{} \meta{tvs} \::= \alpha \| \alpha,\, \nont{tvs}
\\
\name{Type Arguments} \meta{tas} \::= \cdot \| T \| (\nont{ts})
\\
\name{} \meta{ts} \::= T \| T,\, \nont{tss}
\\
\name{} \meta{b} \::= \nont{cs} \| \mcd{case}\; M\; \mcd{of}\; \nont{ccs}
\\
\name{} \meta{cs} \::= c\;\mcd{of}\;T \| c\;\mcd{of}\;T \cvb \nont{cs}
\\
\name{} \meta{ccs} \::= 
  \nont{pat} \Rightarrow c\;\mcd{of}\;T \nlalt
  \nont{pat} \Rightarrow c\;\mcd{of}\;T \cvb \nont{ccs}
\\
\name{\Core{} Field Types} \meta{ffts} \::= \nont{fft} \| \nont{fft};\;\nont{ffts}
\\
\name{\Core{} Field Type} \meta{fft} \::= T \| x = T
\end{bnf}
%\begin{bnf}
\name{Constants} \meta{k} \::= \mcd{true} \| \mcd{false} \| \mcd{()} \| ...
\\
\name{Type Variables} \meta{\alpha}
\\
\name{Type Names} \meta{t}
\\
\name{\Core{} Types} \meta{T} \::= 
  \alpha 
\| {Pbase} 
\| M 
\nlalt \ppair x {T_1} {T_2} 
\| \precord {\nont{ffts}} 
\| \nont{tas}\;t(M) 
\nlalt \pset x T M 
\nlalt \parray T {M_{sep}} {M_{term}} 
\\
\name{\Core{} Datatypes} \meta{D} \::= 
  \mcd{datatype}\; \nont{tps}\; t(x{:}F) = \nont{b} \nlalt
  \mcd{type}\; \nont{tps}\; t(x{:}F) = T
\\
\name{Type Parameters} \meta{tps} \::= \cdot \| \alpha \| (\nont{tvs})
\\
\name{} \meta{tvs} \::= \alpha \| \alpha,\, \nont{tvs}
\\
\name{Type Arguments} \meta{tas} \::= \cdot \| T \| (\nont{ts})
\\
\name{} \meta{ts} \::= T \| T,\, \nont{tss}
\\
\name{} \meta{b} \::= \nont{cs} \| \mcd{case}\; M\; \mcd{of}\; \nont{ccs}
\\
\name{} \meta{cs} \::= c\;\mcd{of}\;T \| c\;\mcd{of}\;T \cvb \nont{cs}
\\
\name{} \meta{ccs} \::= 
  \nont{pat} \Rightarrow c\;\mcd{of}\;T \nlalt
  \nont{pat} \Rightarrow c\;\mcd{of}\;T \cvb \nont{ccs}
\\
\name{\Core{} Field Types} \meta{ffts} \::= \nont{fft} \| \nont{fft};\;\nont{ffts}
\\
\name{\Core{} Field Type} \meta{fft} \::= T \| x = T
\end{bnf}
%%% Local Variables: 
%%% mode: latex
%%% TeX-master: "paper"
%%% End: 


\noindent
{\bf Syntax of terms}
\label{app:syntax-terms}
\begin{bnf}
\name{Types} \meta{F} \::= 
  T           \descr{type of \pvalue{}} 
\nlalt \nont{base} \descr{values of ordinary base types} 
\nlalt \mcd{PD}    \descr{PD type}
\nlalt F * F       \descr{ordinary pairs} 
\nlalt \{\nont{fts}\}     \descr{ordinary records} 
\nlalt F \-> F     \descr{functions}
\nlalt \pstream F  \descr{streams}
\\
\name{Field Types} \meta{fts} \::= x = F \| x=F,\;\nont{fts}
\\
%\end{bnf}
%\begin{bnf}
\name{Parse Descriptors} \meta{pd} \::=   
  G \| B \| N \| S \| U
\\
\name{\Core{} Terms} \meta{N} \::=  
       Pbase[M_1](M_2)                \descr{base type constructor}
\nlalt \langle M \rangle              \descr{unit value (with singleton type M)}
\nlalt (x{=}{M_1} \mathrel{**} {M_2}) \descr{pair}
\nlalt \lcr \nont{fs} \rcr            \descr{record}
\nlalt c[M_1](M_2)                    \descr{data type constructor}
\nlalt \{x = {M_1} \cvb {M_2}\}       \descr{constrained type, with
  $M_2$ the constraint}
\nlalt \mcd{Parray}(M, M_{sep}, M_{term})   \descr{array; first element is stream}
\end{bnf}

\newpage

\begin{bnf}
\name{Terms} \meta{M} \::= 
       x                        \descr{variable}
\nlalt N                        \descr{\core{} terms}
\nlalt k                        \descr{constants}
\nlalt \nont{pd}                      \descr{pd value}
\nlalt (M_1 * M_2)              \descr{ordinary pair}
\nlalt \{\nont{fs}\}        \descr{ordinary record}
\nlalt \tfun {x_1}{x_2}{F_1}{F_2}{M}     \descr{recursive function x1 with arg x2}
\nlalt \mcd{nil}                \descr{empty stream}
\nlalt M_1 \mathrel{::} M_2     \descr{cons}
\nlalt \mcd{case}\;M\;\mcd{of}\;\nont{ms} \descr{deconstructors}
\nlalt M_1\;(M_2)               \descr{function application}
\nlalt \mcd{op}\;M                     \descr{additional uninteresting operations}
\nlalt \mcd{let}\;x = M_1\;\mcd{in}\;M_2           \descr{computation in host language}
\nlalt \mcd{cast}\;(M : T)             \descr{type annot/dependent cast?}
\\
\name{Fields} \meta{fs} \::= x = M \| x = M; \nont{fs}
\\
\name{Matches}\meta{ms} \::= 
  \nont{pat} \Rightarrow M \| \nont{pat} \Rightarrow M \cvb \nont{ms}
\end{bnf}
%{\small
\begin{verbatim}
M ::=  x                        // variable
     | Pbase[M1](M2)            // base type constructor; M1 is
                                   an argument to the type; M2 computes the rep)
     | c(M)                     // data type constructor with parameter M
     | (x:M1 ** M2)             // pair
     | {fields}                 // record
     | {x = M1 | M2}            // set-type; M2 is the predicate
     | Parray(M, Msep, Mterm)   // array; first element is stream
     | k                        // constants
     | let x = M in M           // computation in host language
     | <M>                      // unit value given singleton type M
     | (M1 * M2)                // ordinary pair
     | nil                      // empty list
     | M1 :: M2                 // cons
     | case M of MS             // deconstructors
     | fun x1(x2:F1):F2 = M     // recursive function x1 with arg x2
     | M1 (M2)                  // function application
     | cast (M : T)             // type annot/dependent cast?
     | op M                     // additional uninteresting operations
     
fields ::= x = M | x = M; fields

pd ::=   G    // good
     |   B    // bad
     |   N    // nested error
     |   S    // semantic error
     |   U    // unknown
\end{verbatim}
}

  
\newpage

\noindent
{\bf Syntax of patterns}
\label{app:syntax-pat}
\begin{bnf}
% \name{Parse Descriptors} \meta{pd} \::=   
%          G    \descr{good}
% \nlalt   B    \descr{bad}
% \nlalt   N    \descr{nested error}
% \nlalt   S    \descr{semantic error}
% \nlalt   U    \descr{unknown}
\name{\Core{} Patterns} \meta{fpat} \::=
x \| \nont{Pbase}(\nont{pat})
\nlalt \langle \nont{pat} \rangle             \descr{singleton}
\nlalt (\nont{fpat} \mathrel{**} \nont{fpat})   \descr{\core{} pair}
\nlalt \lcr \nont{ffps} \rcr                  \descr{\core{} record}
\nlalt c(\nont{fpat})                          \descr{constructor}
\nlalt \{\nont{fpat} \cvb \nont{cpat}\}        \descr{type constaint}
\nlalt \mcd{Parray}(\nont{pat}, x_{sep}, x_{term}) \descr{array with stream, sep and term.}
\nlalt \nont{fpat}\langle\langle\nont{pdpat}\rangle\rangle
\\
\name{Patterns}\meta{pat} \::= 
       \nont{fpat} \descr{\core{} pattern}
\nlalt k \| \nont{pdpat}                      \descr{constants and parse descriptors}
\nlalt (\nont{pat} * \nont{pat})              \descr{normal pair}
\nlalt \{\nont{fps}\}                        \descr{record}
\nlalt \mcd{nil} \| \nont{pat}_1 \mathrel{::} \nont{pat}_2 \descr{stream}
\\
\name{\Core{} Field Pattern} \meta{ffps} \::= x = \nont{fpat} \| x = \nont{fpat};\;\nont{ffps}
\\
\name{Constraint Pattern} \meta{cpat} \::= x \| \mcd{true} \| \mcd{false}
\\
\name{PD Pattern} \meta{pdpat}\::= x \| \nont{pd}
\\
\name{Field Pattern} \meta{fps} \::= x = \nont{pat} \| x = \nont{pat};\;\nont{fps}
\end{bnf}
%{\small
\begin{bnf}
% \name{Parse Descriptors} \meta{pd} \::=   
%          G    \descr{good}
% \nlalt   B    \descr{bad}
% \nlalt   N    \descr{nested error}
% \nlalt   S    \descr{semantic error}
% \nlalt   U    \descr{unknown}
\name{\Core{} Patterns} \meta{fpat} \::=
x \| \nont{Pbase}(\nont{pat})
\nlalt \langle \nont{pat} \rangle             \descr{singleton}
\nlalt (\nont{fpat} \mathrel{**} \nont{fpat})   \descr{\core{} pair}
\nlalt \lcr \nont{ffps} \rcr                  \descr{\core{} record}
\nlalt c(\nont{fpat})                          \descr{constructor}
\nlalt \{\nont{fpat} \cvb \nont{cpat}\}        \descr{type constaint}
\nlalt \mcd{Parray}(\nont{pat}, x_{sep}, x_{term}) \descr{array with stream, sep and term.}
\nlalt \nont{fpat}\langle\langle\nont{pdpat}\rangle\rangle
\\
\name{Patterns}\meta{pat} \::= 
       \nont{fpat} \descr{\core{} pattern}
\nlalt k \| \nont{pdpat}                      \descr{constants and parse descriptors}
\nlalt (\nont{pat} * \nont{pat})              \descr{normal pair}
\nlalt \{\nont{fps}\}                        \descr{record}
\nlalt \mcd{nil} \| \nont{pat}_1 \mathrel{::} \nont{pat}_2 \descr{stream}
\\
\name{\Core{} Field Pattern} \meta{ffps} \::= x = \nont{fpat} \| x = \nont{fpat};\;\nont{ffps}
\\
\name{Constraint Pattern} \meta{cpat} \::= x \| \mcd{true} \| \mcd{false}
\\
\name{PD Pattern} \meta{pdpat}\::= x \| \nont{pd}
\\
\name{Field Pattern} \meta{fps} \::= x = \nont{pat} \| x = \nont{pat};\;\nont{fps}
\end{bnf}
}

%%% Local Variables: 
%%% mode: latex
%%% TeX-master: "paper"
%%% End: 


\noindent
{\bf Syntax of programs}
\label{app:syntax-prog}
\begin{bnf}
\name{Program} \meta{prog} \::= 
  M              
 \nlalt D\; \nont{prog}          \descr{type declaration}
 \nlalt \mcd{val}\;x = M\;\mcd{prog} \descr{value declaration}
\end{bnf}
%{\small
\begin{verbatim}
prog ::= M              
       | D prog          // type declaration
       | val x = M prog  // value declaration
\end{verbatim}
}

}
%%% Local Variables: 
%%% mode: latex
%%% TeX-master: "paper"
%%% End: 


\end{document}

%%% Local Variables:
%%% mode: outline-minor
%%% End:


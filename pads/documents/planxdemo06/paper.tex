\documentclass[nocopyrightspace]{sigplanconf}

\usepackage{xspace,pads,amsmath,math-cmds,
            math-envs,times,
            verbatim,multicol,url}

\usepackage{graphicx}

\usepackage{code} 
%\usepackage{epsfig}

\newcommand{\kw}[1]{\tt{#1}}
\renewcommand{\floatpagefraction}{0.9}

%\newcommand{\cut}[1]{}

\newcommand{\appref}[1]{Appendix~\ref{#1}}
\newcommand{\secref}[1]{Section~\ref{#1}}
\newcommand{\tblref}[1]{Table~\ref{#1}}
\newcommand{\figref}[1]{Figure~\ref{#1}}
\newcommand{\listingref}[1]{Listing~\ref{#1}}
%\newcommand{\pref}[1]{{page~\pageref{#1}}}

\newcommand{\eg}{{\em e.g.}}
\newcommand{\cf}{{\em cf.}}
\newcommand{\ie}{{\em i.e.}}
\newcommand{\etc}{{\em etc.\/}}
\newcommand{\naive}{na\"{\i}ve}
\newcommand{\role}{r\^{o}le}
\newcommand{\forte}{{fort\'{e}\/}}

\newcommand{\bftt}[1]{{\ttfamily\bfseries{}#1}}
\newcommand{\kw}[1]{{\tt #1 }}
\newcommand{\pads}{\textsc{pads}}
\newcommand{\padsc}{\textsc{pads/c}}
\newcommand{\padsl}{\textsc{padsl}}
\newcommand{\C}{\textsc{C}}
\newcommand{\ocaml}{\textsc{O'Caml}}
\newcommand{\java}{\textsc{Java}}


\title{\textsc{LaunchPads}: A System for Processing Ad Hoc Data}

\authorinfo{Mark Daly}{
	   Princeton University}
       {\mono{mdaly@Princeton.EDU}}

\authorinfo{Mary Fern\'andez \\
            Kathleen Fisher}{
	   AT\&T Labs Research}
       {\mono{mff,kfisher@research.att.com}}

\authorinfo{Yitzhak Mandelbaum \\  
            David Walker}{
	   Princeton University}
       {\mono{yitzhakm,dpw@CS.Princeton.EDU}}

% \authorinfo{}
%        {AT\& Labs Research}
%        {\mono{\{mff,kfisher\}@research.att.com}}

% \authorinfo{ Robert Gruber\titlenote{Work carried out while at AT\&T
%                                      Labs Research.}}
%        {Google}
%        {\mono{gruber@google.com}}

% \authorinfo{Yitzhak Mandelbaum}
%        {Princeton University}
%        {\mono{yitzhakm@cs.princeton.edu}}

\date{\today}


\begin{document}

\maketitle
%\begin{abstract}
%....
%\end{abstract}


\paragraph*{An Introduction to PADS.}
Ideally, any data we ever encounter will be presented to us
in standardized formats, such as \textsc{XML}.  Why?  Because
for formats like \textsc{XML}, there are a whole host of software libraries,
query engines, visualization tools and even programming languages
specially designed to help users process their data.
However, we do not live in an ideal world, and in reality, 
vast amounts of data is produced and communicated 
in \textit{ad hoc formats}, those formats for which
no data processing tools are readily available.  Figure~\ref{figure:data-sources} presents a small selection of ad hoc data sources.
As one can see, ad hoc data exists in a very wide variety of fields
and the users range from network administrators to computational 
biologists and genomics researchers to physicists, financial 
analysts and everyday programmers.

Programmers often deal with this data by whipping up one-time Perl scripts
or C programs.  Unfortunately, this
task is slow, tedious, and often produces code that is insufficiently robust,
lacks error-checking, is
difficult to understand, and is hard to evolve over time as formats change.  
To expedite and improve this process, we developed
the \textsc{PADS} data description language and system~\cite{fisher+:pldi05,fisher+:popl06}.  Using the
\textsc{PADS} language, one may write a declarative
description of the structure of almost any ad hoc data source.  
The descriptions take the form of types,
drawn from a dependent type theory.  For instance, \textsc{PADS} 
base types describe simple objects including
strings, integers, floating-point numbers, dates, times, and ip addresses.
Records and arrays specify sequences of elements in a data source, 
and unions, switched unions and enums specify
alternatives.  Any of these structured types may be 
parameterized and
users may write arbitrary semantic constraints over their data
as well.
% Figure~\ref{figure:wsl} presents part of a description
% of a web server log to give the reader a feel for the kind of
% code one finds in a PADS description.  Readers curious about the details
% may download and examine the \textsc{PADS} manual~\cite{padsmanual} for
% more information.


Once a programmer has written a description in the \textsc{PADS} language,
the \textsc{PADS} compiler can generate a collection of format-specific
libraries in C, including a parser, printer, and verifier.
In addition, the compiler can compose these libraries with
generic templates to create value-added tools such as an 
ad hoc-to-\textsc{XML} format conversion tool, a histogram generator, and
a statistical analysis and error summary tool.  Finally, \textsc{PADS}
has been composed with the \textsc{Galax} query engine~\cite{galax:edbt2004,galaxmanual,galax:icde2006} 
for \text{XQuery} to create \textsc{PADX}~\cite{fernandez+:padx}, a new system
that allows users to query and transform any ad hoc data source as if it 
was \textsc{XML}, without incurring the performance penalty that usually
results when one converts ad hoc data into a much more verbose
\textsc{XML} representation.

\begin{figure}
\begin{center}
\begin{tabular}{|l|l|}
\hline
\textbf{Name} : Use   &  Representation               \\ \hline
\textbf{Web server logs (CLF)}:  &  Fixed-column ASCII records \\ 
Measure web workloads &                             \\ \hline
\textbf{CoMon data}: & ASCII records  \\ 
Monitor PlanetLab Machines &                              \\ \hline
\textbf{Call detail}: Fraud detection  &  Fixed-width binary records \\  \hline 
\textbf{AT\&T billing data}: & Various Cobol data formats  \\ 
Monitor billing process   &                             \\ \hline
%IP backbone data:  & ASCII   \\
%Monitor network performance  &        \\ \hline
\textbf{Netflow}:                        & Data-dependent number of     \\ 
Monitor network performance  & fixed-width binary records  \\ \hline
\textbf{Newick}:   Immune                 & Fixed-width ASCII records \\ 
system response simulation & in tree-shaped hierarchy\\ \hline                                
\textbf{Gene Ontology}:             & Variable-width ASCII records \\
Gene-gene correlations     & in DAG-shaped hierarchy \\ \hline
\textbf{CPT codes}:  & Floating point numbers \\ 
Medical diagnoses & \\ \hline
\end{tabular}

\caption{Selected ad hoc data sources.}
\label{figure:data-sources}
\end{center}
\end{figure}


% \begin{figure}
% \begin{code}
% \kw{Punion} client\_t \{
%   Pip       ip;      /- 135.207.23.32
%   Phostname host;    /- www.research.att.com
% \};
% \mbox{}
% \kw{Punion} auth\_id\_t \{
%   '-';
%   Pstring(:' ':) id;
% \};
% \mbox{}
% ...
% \mbox{}
% \kw{Ptypedef} Puint16\_FW(:3:) response\_t :
%          response\_t x => \{ 100 <= x && x < 600\};
% \mbox{}
% \kw{Precord} \kw{Pstruct} entry\_t \{
%          client\_t       client;
%    ' ';  auth\_id\_t      remoteID;
%    ' ';  auth\_id\_t      auth;
%    " ["; Pdate(:']':)   date;
%    "] "; request\_t      request;
%    ' ';  response\_t     response;
%    ' ';  Puint32        length;
% \};
% \mbox{}
% \kw{Psource} \kw{Parray} clt\_t \{
%   entry\_t [];
% \}
% \end{code}
% \caption{Part of the description of a web server log format.
% Types are written in a C syntax. Records and arrays indicate sequences
% of elements in a format.  Literal characters such as {\tt ' '} 
% indicate the occurrence of that character at that point in the format.
% Typedefs may be used to add user-defined constraints to a type.}
% \label{figure:wsl}
% \end{figure}


\begin{figure*}
  \begin{center}
    \includegraphics*{launch.pdf}
  \end{center}

  \caption{\small LaunchPads Interface.}
  \label{figure:launchpads}
\end{figure*}

While the \textsc{PADS} language provides an extremely versatile means
of creating tools for processing ad hoc data, it is nevertheless a
\textit{new} language and learning a new language is time-consuming
for anyone, especially for computational biologists or 
other scientists for whom programming is not their primary
area of expertise.  To ease the way for novice 
\textsc{PADS} users, we developed have \textsc{LaunchPADS},
a new tool that provides access to the PADS system
without requiring foreknowledge of the \textsc{PADS} language itself.
Hence, \textsc{LaunchPADS} graphic interface will also help more
experienced \textsc{PADS} users to shorten their development cycle
and provides a convenient way for experts to quickly create
any of the data processing tools they need.

\paragraph*{LaunchPads.}  \textsc{LaunchPADS} combines mechanisms 
for graphically defining structure and semantic properties of ad hoc data,
for translation of this definition into
\textsc{PADS} code, and for compilation/execution of the generic tools
that operate over ad hoc data.  
More specifically, \textsc{LaunchPADS} breaks definition of an 
ad hoc data format and generation of data processing tools
into the following parts.  Figure~\ref{figure:launchpads}
presents a screenshot of \textsc{LaunchPADS} being used to construct
a data description for a web server log format.

\begin{enumerate}
\item {\bf Selection of sample data from which to build the description.} 
Creation of a definition within \textsc{LaunchPADS} begins
when a user loads sample data into the graphical interface.
In Figure~\ref{figure:launchpads}, web log data (beginning with
the IP address {\tt 207.136.97.49 ...}) appears
in the top right hand corner of the picture.  A user then
selects a row of data to work on in the \textsc{LaunchPADS} {\it gridview}
immediately below.

\item {\bf Iterative refinement in the gridview.}
Once in the gridview, users may specify descriptions 
for regions of text using a highlighting scheme.  The 
color assigned to a region
represents the description class (base or composite) and region boundaries.  
Structure within a definition is represented through a series of
refinement steps: composite regions are broken down and level after 
level, thereby allowing for nested elements (Figure~\ref{figure:launchpads}
shows four nesting levels).  The refinement process bottoms out
when one reaches an atomic description such as a character string,
IP address or date.  Once all regions have
been given a base type in the gridview,  \textsc{LaunchPADS} will generate a
{\it treeview} of the definition for further processing.

\item {\bf Customization in the treeview.}  The treeview is
a graphical representation of the abstract syntax of a PADS description.
In this view, programmers can manipulate definitions with a high degree
of precision: definition elements may be created, destroyed, and
renamed; type associations for existing elements may be changed
(within limitations); element ordering may be altered; user defined
types may be added to the definition and applied to elements;
content-aware error constraints may be imposed.  Indeed, from within
the tree view it is possible to access the ``expert'' functions of
\textsc{PADS} directly if one so chooses, or to completely avoid them
in lieu of a simpler definition and/or faster development time.

\item {\bf \textsc{PADS} code generation, tool compilation and use.}  When 
the user is satisfied with their \textsc{PADS} definition in the treeview,
they may generate \textsc{PADS} code.  Any such generated code is guaranteed 
to be syntactically correct so the user need not worry about fussing with
concrete \textsc{PADS} syntax if they do not want to. 
Figure~\ref{figure:launchpads}
shows the generated code in the window at the bottom of the
interface.  By using the pulldown menus at the top and a set of
``wizards,'' the user may now
issue commands to compile the generated code and create data processing
tools including the XML converter and statistical analyzer.  
As development of \textsc{LaunchPADS} continues, we will add further
tools and corresponding wizards to the interface.
\end{enumerate}.

% Creation of a definition within \textsc{LaunchPADS} begins with the
% selection of a prototype key, ideally a single record in a particular
% format containing as many full data fields as possible, and loading of
% that key into the \textsc{LaunchPADS} grid layout control.  Once in
% the grid, users may specify types for regions within the prototype key
% using a ``highlighting'' scheme, where the color assigned to a region
% represents the type class (base or composite) and region boundaries
% (so as to distinguish between consecutive elements of similar type) of
% what will eventually become a variable in the definition.  Structure
% within a definition is represented through a series of
% ``fragmentation'' steps: regions to which irreducible base types are
% applied are bound to their particular type, while composite regions
% continue on to the next level of region isolation, thereby allowing
% for nested elements.  Once all regions within the prototype key have
% been given a base type at their final ``depth'' of fragmentation, a
% tree representation of the definition can be generated and the
% definition can be further refined.

% The tree view component of \textsc{LaunchPADS} is without doubt its
% most complex component.  Using the tree, it is possible to extend a
% rapidly created (and perhaps overly general) proto-definition into one
% that employs the full specificity of \textsc{PADS}.  The tree
% representation of a definition can be manipulated with a high degree
% of precision: definition elements may be created, destroyed, and
% renamed; type associations for existing elements may be changed
% (within limitations); element ordering may be altered; user defined
% types may be added to the definition and applied to elements;
% content-aware error constraints may be imposed.  Indeed, from within
% the tree view it is possible to access the ``expert'' functions of
% \textsc{PADS} directly if one so chooses, or to completely avoid them
% in lieu of a simpler definition and/or faster development time.

% Once the definition has been suitable refined, users may generate from
% it \textsc{PADS} code.  This can be immediately compiled from within
% the GUI without need for user input (save for the specification of
% certain necessary files).  It is also possible to use
% \textsc{LaunchPADS} to compile preexisting \textsc{PADS} definitions,
% though currently no facility exists for importing existing definitions
% into the \textsc{LaunchPADS} development environment (this feature is
% planned for a future version).
 
% Additional features enclude:
% \begin{itemize}
% \item Graphical delineation of type and structure of portions 
% of prototypical data keys
% \item Tree-based visualization of definitions created within 
% \textsc{LaunchPADS}
% \item Automated conversion of graph into \textsc{PADS} description 
% language code
% \item Integrated \textsc{PADS} compiler, generic tool compilation, 
% and execution functionality (build robust format-to-XML converters 
% in C without ever having to touch a piece of code)
% \item Cross platform compatibility
% \item Built entirely using open source libraries
% \end{itemize}


\paragraph*{Conclusions} 
In summary, in this demonstration, we will explain the many challenges
that ad hoc data pose and how the \textsc{PADS} language is
structured to meet these challenges.  In addition, we will explain how
\textsc{LaunchPADS} provides further support for processing ad hoc
data by demonstrating both features for helping users construct data
descriptions and features for creating and invoking tools that operate
over data.  We believe that both expert programmers and novices alike
can benefit from this simple system for manipulating ad hoc data.

% by creating software which distills the full (and ever growing)
% functionality of \textsc{PADS} into a purely graphical environment, we
% hope to make the power of \textsc{PADS} available to all who wrestle
% with ad hoc data.  commonly found in data mining, but may be too esoteric
% or unfamiliar for many of those facing such problems.
% \textsc{LaunchPADS} seeks to make employing \textsc{PADS} intuitive
% and convenient, and thus accessible to non-programmers.  While it
% currently faces certain limitations and assumes some knowledge of the
% \textsc{PADS} description language (the type system in particular),
% future versions will incorporate even more automation, shortening
% further the path from data to definition to conversion and analysis.

\bibliographystyle{abbrv}
\small
\bibliography{pads,galax} 
\end{document}


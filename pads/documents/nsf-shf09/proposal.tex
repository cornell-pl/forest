\documentclass[11pt]{article}
\usepackage{times,url}
\usepackage{code} 
%\usepackage{proof}
%\usepackage{newcode}
%\usepackage{epsfig}
\usepackage{xcolor}
\usepackage{graphicx}
\usepackage{epstopdf} 

\newcommand{\cut}[1]{}

\newcommand{\appref}[1]{Appendix~\ref{#1}}
\newcommand{\secref}[1]{Section~\ref{#1}}
\newcommand{\tblref}[1]{Table~\ref{#1}}
\newcommand{\figref}[1]{Figure~\ref{#1}}
\newcommand{\listingref}[1]{Listing~\ref{#1}}
%\newcommand{\pref}[1]{{page~\pageref{#1}}}

\newcommand{\eg}{{\em e.g.}}
\newcommand{\cf}{{\em cf.}}
\newcommand{\ie}{{\em i.e.}}
\newcommand{\etc}{{\em etc.\/}}
\newcommand{\naive}{na\"{\i}ve}
\newcommand{\role}{r\^{o}le}
\newcommand{\forte}{{fort\'{e}\/}}

\newcommand{\bftt}[1]{{\ttfamily\bfseries{}#1}}
\newcommand{\kw}[1]{{\tt #1 }}
\newcommand{\pads}{\textsc{pads}}
\newcommand{\padsc}{\textsc{pads/c}}
\newcommand{\padsl}{\textsc{padsl}}
\newcommand{\C}{\textsc{C}}
\newcommand{\ocaml}{\textsc{O'Caml}}
\newcommand{\java}{\textsc{Java}}


%%%%%%%%%%%%%%%%%%%%%%%%%%%%%%%%%%%%%%%%%%%%%%%%%%%%%%%%%%%%%%%%%%%%%%%%%%%%
\voffset             0in    %  top vertical offset
\hoffset             0in    %  left horizontal offset
\oddsidemargin       0pt    %  Left margin on odd-numbered pages.
\evensidemargin      0pt    %  Left margin on even-numbered pages.
\topmargin           0pt    %  Nominal distance from top of page to top of
\headheight          0pt    %  Height of box containing running head.
\headsep             0pt    %  Space between running head and text.
\textwidth         6.5in    %  Width of text on page
\textheight          9in    %  Height of text on page
\setlength{\parskip}{.05in}
\renewcommand{\floatpagefraction}{.9}
\renewcommand{\textfraction}{0.1}
\renewcommand{\baselinestretch}{1.01}
%\input{macro}  %%%  other macro definitions

%%%%%%%%%%%%%%%%%%%%%%%%%%%%%%%%%%%%%%%%%%%%%%%%%%%%%%%%%%%%%%%%%%%%%%%%%%%%
\begin{document}
\setcounter{page}{1}
\pagenumbering{roman}
\appendix
%%%%%%%%%%%%%%%%%%%%%%%%%%%%%%%%%%%%%%%%%%%%%%%%%%%%%%%%%%%%%%%%%%%%%%%%%%%%
\section{SHF:Small:Language Support for Ad Hoc Data Processing}
%%%%%%%
%%%%%%%%%%%%%%%%%%%%%%%%%%%%%%
% \centerline{
% \begin{tabular}{cc}
%           \\[-1ex]
% \multicolumn{2}{c}{Automatic Tool Generation for Ad Hoc Scientific Data} \\
%           & \\[-1ex]
% David Walker (PI)\\
% Princeton University\\[2ex]
% \end{tabular}}
%%%%%%%%%%%%%%%%%%%%%%%%%%%%%%

\paragraph*{Intellectual Merits:} 
In every scientific discipline, researchers are digitizing their knowledge and
beginning to use computational methods to categorize,
query, filter, search, diagnose, and visualize their data.  
While this effort is leading to remarkable advances,
it is also generating enormous amounts of {\em ad hoc data}.
Ad hoc data is any data for which standard data processing tools
such as query engines, statistical packages, graphing tools, parsers, printers,
transformers or programming libraries are not readily available.
This data, which is often unpredictable, poorly documented,
filled with errors, high volume and unwieldy,
poses tremendous challenges to its users and the software
that manipulates it.  We cannot maximize the productivity of top 
computational scientists unless we can maximize the efficiency and 
accuracy with which they deal with this data.

Our overall goal is to alleviate the burden, risk and confusion
associated with ad hoc data.  Our overall strategy is (1) to develop a
specification language capable of precisely describing any ad hoc data
format at an easy-to-understand, high level of abstraction and (2) to
automatically generate useful data processing tools including
programming libraries, a query engine, format converters, a
statistical summarizer, a histogram generator and others.
 
We will accomplish our task by building upon our preliminary data
description and processing system, \pads{}, created by the PI and his
collaborators.  This preliminary system demonstrates the basic
feasibility of the \pads{} approach, but there is much research to do.
While the work to date demonstrates the feasibility of the \pads{}
approach, the \pads{} design and implementation are still in their
infancy: We have discovered many common, real-world data formats that
the current PADS infrastructure is incapable of describing, parsing or
analyzing.  To address these deficiencies, we propose to explore
innovative ways of extending the basic \pads{} specification language
and related infrastructure.  The second thrust of our research
involves improving the automatic tool generation infrastructure in the
\pads{} system.  The current tool generation system is brittle,
inflexible and suffers from subpar performance.  We will develop a
novel extension to \pads{} that allows users to augment \pads{}
descriptions with application-specific attributes and customizations.
The attributes convey semantic information to the tool generators and
the customizations improve error-handling and performance.  Our new
architecture has the promise of being substantially more robust,
flexible and efficient.  Finally, in order to improve the reliability
of the infrastructure, we will formalize \pads{} and prove important
correctness properties of our formal model.  Overall, this proposal
involves challenging research in language design, efficient systems
implementation and semantic analysis, all aimed at solving real-world
data processing problems.

\paragraph*{Broader Impacts:}  We are collaborating with Kathleen Fisher
and Mary Fernandez at
AT\&T who will be able to use our tools to address real problems such
as telephone fraud detection.  In addition, our tools will be freely
available to researchers and scientists over the web.  Moreover, part
of our mission will be to work with specific biologists and
physicists at Princeton and the broader community to help them with
their data processing needs.  We have already been meeting with Olga
Troyanskaya, who works in Princeton's Lewis-Sigler Institute for
Integrative Genomics on pathway modeling and analysis of
protein-protein interactions, and with Rachel Mandelbaum, Ph.D. candidate
in physics who analyzes cosmology data.  Many of our proposed extensions to
\pads{} were inspired directly by their data processing needs.  

The collaboration between Computer Science and Genomics will also be an
excellent platform for developing interdisciplinary undergraduate
research projects.  The PI has a proven track-record 
for following through with undergraduate and graduate
educational plans:  Last year, his undergraduate student advisee, Rob Simmons,
won the Princeton Computer Science Senior Thesis Award;
last year and the year before he organized two NSF-sponsored summer schools
on secure and reliable computing.


\newpage
% %%%%%%%%%%%%%%%%%%%%%%%%%%%%%%%%%%%%%%%%%%%%%%%%%%%%%%%%%%%%%%%%%%%%%%%%%%%%
% \section{Table of Contents}
% \tableofcontents\
% \vspace*{.1in} 
% \begin{center}
% {\Large\bf\em{}This is a place-holder. Fastlane generates this page automatically.}
% \end{center} 
% \newpage
%%%%%%%%%%%%%%%%%%%%%%%%%%%%%%%%%%%%%%%%%%%%%%%%%%%%%%%%%%%%%%%%%%%%%%%%%%%%
\pagenumbering{arabic}
\setcounter{page}{1}
%%%%%%%%%%%%%%%%%%%%%%%%%%%%%%%%%%%%%%%%%%%%%%%%%%%%%%%%%%%%%%%%%%%%%%%%%%%%
\section{Project Description}

\subsection{Introduction}
\label{ssec:intro}

An {\em ad hoc data format} is any nonstandard data format for which
parsing, querying, analysis or transformation tools are not readily
available.  \xml{} is not an ad hoc data format --- there are hundreds
of tools for reading, querying, and transforming \xml{}.  However,
network administrators,
data analysts, computer scientists,
biologists, chemists, and physicists deal with ad hoc
data in a myriad of complex formats on a daily basis.
This data, which is often unpredictable, poorly documented,
and filled with errors
poses tremendous challenges to its users and the software
that manipulates it.  
Our goal is to alleviate the burden, risk and confusion associated
with ad hoc data by developing a universal data processing system
capable of 

\begin{enumerate}
\item concisely and accurately describing any ad hoc data source at an 
easy-to-understand, high-level of abstraction, and
\item automatically generating a suite of highly reliable and
efficient data processing tools that make understanding, manipulating 
and transforming ad hoc data an effortless task.
\end{enumerate}


Together with Kathleen Fisher (Senior Personnel),
his collaborators and students, the PI has developed 
a system called \pads{} (Processing Ad hoc Data Sources)~\cite{padsweb,fisher+:pads,fisher+:popl06,mandelbaum+:pads-ml,fisher+:dirttoshovels} that 
helps to address some of these concerns.
It includes a high-level, declarative language for describing
ad hoc data formats and it is possible to generate tools for
parsing, printing, querying, generating histograms and 
statistical summaries of data, and transforming 
data into \xml.  An online demo~\cite{padsdemo} illustrates
our progress to date.  However, the PI's work date has generated
important new scientific questions and shown the need for new language
support and tool generation technology.  In particular, if funded, the 
PI will pursue the following critical new research agenda:

\begin{enumerate}
\item {\bf Language support for multi-level directory systems.}
Many ad hoc data sets are physically represented as collections of 
data files, organized in a multi-level directory system.
The PI will develop powerful, new extensions to PADS that allow programmers
to specify the structure of these multi-level directory systems
and to develop new tool-generation infrastructure capable of
automatically producing useful tools for monitoring, querying and programming
against these complex data sources.
\item {\bf Semi-automatic data description generation.}
While using the PADS data description language speeds the generation
and maintainance of data processing tools, a user must invest some effort
to learn the language syntax and semantics.  The PI will develop 
an entirely new kind of {\em text mark-up language} that allows users to add simple
labels to raw data sources, and from such labels, generate full descriptions
that can in turn serve as lasting documentation or be used to generate useful
programming tools.
\item {\bf Grammatical foundations and efficient implementation.}  
Many complex, modern text and binary data formats require a very rich collection
of features in order to describe them declaratively:  
{\em full context-free grammars}, the ability to bind 
parts of a data source to {\em variables} for later use in
{\em constraints}, {\em parameters} on non-terminals and the
ability to include pre-defined libraries as {\em black boxes} 
within a description.  The PI will develop a new grammatical 
theory that explains how to implement this rich collection of features
efficiently.  In addition, the PI will use the new theory to build
important new tools including a random data generator for using
in testing data processing infrastructure.
\end{enumerate}

Overall, our research combines novel language design, high-performance
systems engineering and theoretical analysis.  It will substantially
increase the overall productivity of data analysts, researchers and
software architects who deal with ad hoc data regularly, and it will
improve the security and reliability of the software they produce.
Finally, our research will also have a broad impact on research in the
natural sciences, where ad hoc data is pervasive, and on
interdisciplinary computer science.

The PI and his team are uniquely qualified to carry out this research agenda
and have an unsurpassed track record of top research production
in this important area.  In particular, this research team has produced papers
in this area in 4 of the last 5 ACM SIGPLAN conferences on Principles
of Programming Languages (POPL 06~\cite{fisher+:popl06}, 
POPL 07~\cite{mandelbaum+:pads-ml}, POPL 08~\cite{fisher+:dirttoshovels}
and POPL 10~\cite{jim+:data-dependent-grammars}) 
--- POPL being widely considered the top conference
in programming language theory and design. Few, if any, other research
projects can boast such a claim.  The PI's have also recently produced an 
article for the Journal of the ACM~\cite{fisher+:700journal}, the top journal in computer
science, as well as a variety of other papers~\cite{fisher+:pads,fernandez+:padx,pads-sigmod06,burke+:cagi07,mandelbaum+:dual-semantics,mandelbaum+:generic-padsml,pads-sigmod08,token-ambiguity,pads:distributed,pads:wasl}.  The
research team have also 
delivered software
free to download on the 
web~\cite{padscdistributions,padsmldistributions,padslearndistributions} 
and developed on-line
demos~\cite{padsdemo} 
so students and researchers can learn about their results and
experiment with their research ideas.

In the rest of this introduction, we will explain the immense
challenges and risks posed by ad hoc data in more detail.  We will
explain how our proposed system as a whole can deal with them.  We
will then describe some of the features of \pads{} we have already
implemented.  After this introductory section, we will describe the
key components of our new research agenda and explain how our research
will make a broad impact on disciplines well outside the bounds of
traditional computer science and on interdisciplinary education.

\subsubsection{The Challenges of Ad Hoc Data}

There are vast amounts of useful data stored in
traditional databases and \xml{} formats, but there is just as much in
ad hoc formats.  \figref{figure:data-sources} provides some information
on ad hoc data formats from several different domains ranging from genomics
to cosmology to networking to finance to internal corporate billing information.  
They include ASCII, binary, and Cobol data formats, with
both fixed and variable-width records, ranging in size from
relatively small files through network applications which process over
a gigabyte per second.  Common errors include undocumented data,
corrupted data, missing data, and multiple missing-value
representations.


\begin{figure*}
\begin{center}
\begin{tabular}{@{}|l|l|l|l|l|}
\hline
Name: Use                           & Representation    & Processing Problems \\ \hline\hline
Gene Ontology (GO)~\cite{geneontology}:                  & Variable-width    & White-space ambiguities \\
Gene Product Information 	      & ASCII records &  \\ \hline
Web server logs (CLF):                & Fixed-column      & Race conditions on log entry\\ 
Measuring web workloads               & ASCII records     & Unexpected values\\ \hline
AT\&T Call detail data:                          & Fixed-width       & Undocumented data\\
Phone call fraud detection            & binary records  & \\ \hline 
AT\&T billing data:                 & Cobol             &  Unexpected values\\ 
Monitoring billing process          &                   & Corrupted data feeds \\ \hline
Newick:   Immune                    & Fixed-width ASCII records & None \\ 
system response simulation          & in tree-shaped hierarchy &\\ \hline                                
OPRA:                               & Mixed binary \& ASCII records 
                                                       & 100-page informal \\
Options-market transactions         & with data-dependent unions & documentation \\ \hline
Palm PDA:                           & Mixed binary \& character & No high-level  \\
Device synchronization              & with data-dependent constraints & documentation available\\ \hline
\end{tabular}
\caption{Selected ad hoc data sources.}
\label{figure:data-sources}
\end{center}
\end{figure*}


Processing ad hoc data is challenging for a variety of further reasons. 
First, ad hoc data typically arrives ``as is'': the analysts
who receive it can only say ``thank you,'' not request a more
convenient format.  Second, documentation for the format may not exist
at all, or it may be out of date.  A common phenomenon is for a field
in a data source to fall into disuse.  After a while, a new piece of
information becomes interesting, but compatibility issues prevent data
suppliers from modifying the shape of their data, so instead they
hijack the unused field, often failing to update the documentation in
the process.

Third, such data frequently contain errors, for a variety of reasons:
malfunctioning equipment, race conditions on log entry~\cite{wpp},
non-standard values to indicate ``no data available,'' human error in
entering data, unexpected data values, \etc{} The appropriate response
to such errors depends on the application.  Some applications require
the data to be error free: if an error is detected, processing needs
to stop immediately and a human must be alerted.  Other applications
can repair the data, while still others can simply discard erroneous
or unexpected values.  For some applications, errors in the data can
be the most interesting part because they can signal where two systems
are failing to communicate.

% A fourth challenge is that ad hoc data sources can be high volume:
% AT\&T's call-detail stream contains roughly 300~million calls per day
% requiring approximately 7GBs of storage space. Although this data is
% eventually archived in a database, analysts mine it profitably before
% such archiving~\cite{kdd98,kdd99}. More challenging, the \ningaui{}
% project at AT\&T accumulates billing data at a rate of 250-300GB/day,
% with occasional spurts of 750GBs/day. Netflow data arrives from Cisco
% routers at rates over a gigabyte per second~\cite{gigascope}! Such
% volumes mean it must be possible to process the data without loading
% it all into memory at once.

Finally, before anything can be done with an ad hoc data source,
someone has to produce a suitable parser for it.  Today, people tend
to use \C{} or \perl{} for this task.  Unfortunately, writing parsers
this way is tedious and error-prone, complicated by the lack of
documentation, convoluted encodings designed to save space, the need
to produce efficient code, and the need to handle errors robustly to
avoid corrupting down-stream data.  Moreover, the parser writers'
hard-won understanding of the data ends up embedded in parsing code,
making long-term maintenance difficult for the original writers and
sharing the knowledge with others nearly impossible.

\subsubsection{\pads{}:  Taking on the Challenge of Ad Hoc Data}

The \pads{} system makes life easier for data analysts by addressing
each of the concerns outlined in the previous section.
It provides a declarative data description
language that permits analysts to describe the physical layout of
their data, \textit{as it is}.  The language also permits analysts to
describe expected semantic properties of their data so that deviations can
be flagged as errors. The intent is to allow analysts to capture in a
\pads{} description all that they know about a given data source
and to provide the analysts with a library of useful routines in exchange. 


\pads{} descriptions are intended to be concise enough to serve as
documentation and flexible enough to describe many ASCII, binary,
Cobol, and mixed data formats.  In addition, useful software tools
can be generated from the descriptions and this feature provides
strong incentive for keeping the descriptions current, allowing them
to serve as living documentation.  

Given a \pads{} description, the \pads{} compiler produces
customizable \C{} libraries and tools for parsing, manipulating, and
summarizing the data.  The core \C{} library includes functions for
reading the data, writing it back out in its original form, writing it
into a canonical \xml{} form, pretty printing it in forms suitable for
loading into a relational database, and accumulating statistical
properties.  An auxiliary library provides an instance of the data API
for Galax~\cite{galax,galaxmanual}, an implementation of XQuery.  This
library allows users to query data with a \pads{} description as if
the data were in \xml{} without having to convert to \xml, which often
can avoid an 8-10 times blow-up in space requirements.  

The declarative nature of \pads{} descriptions facilitates the
insertion of error handling code.  The generated parsing code checks
all possible error cases: system errors related to the input file,
buffer, or socket; syntax errors related to deviations in the physical
format; and semantic errors in which the data violates user
constraints.  Because these checks appear only in generated code, they
do not clutter the high-level declarative description of the data
source.  Moreover, since tools are generated
automatically by a compiler rather than written by hand, 
they are far more likely to be robust
and far less likely to have dangerous vulnerabilities such as
buffer overflows.  Moreover, as formats change and are updated,
one can make a small change in the high-level description
and the compiler will takes care of the rest.  It is extremely unlikely
new vulnerabilities or buffer overflows will be added to the code
base over time as formats change.

The result of a parse is a pair consisting of a canonical
in-memory {\em representation} of the data and a {\em parse descriptor}. 
The parse
descriptor precisely characterizes both the syntactic and the semantic
errors that occurred during parsing.  This structure allows analysts
to choose how to respond to errors in application-specific ways.

% With such huge datasets, performance is critical. The \pads{} system
% addresses performance in a number of ways.  First, we compile the
% \pads{} description rather than simply interpret it to reduce run-time
% overhead.  Second, the generated parser provides multiple entry
% points, so the data consumer can choose the appropriate level of
% granularity for reading the data into memory to accommodate very large
% data sources.  Finally, we parameterize library functions by
% \textit{masks}, which allow data analysts to choose which semantic
% conditions to check at run-time, permitting them to specify all known
% properties in the source description without forcing all users of that
% description to pay the run-time cost of checking them.

% Given the importance of the problem, it is perhaps surprising that
% more tools do not exist to solve it.  \xml{} and relational databases
% only help with data already in well-behaved formats.  Lex and Yacc are
% both over- and under- kill.  Overkill because the division into a
% lexer and a context free grammar is not necessary for many ad hoc data
% sources, and under-kill in that such systems require the user to build
% in-memory representations manually, support only ASCII sources, and
% don't provide extra tools.  ASN.1~\cite{asn} and related
% systems~\cite{asdl} allow the user to specify an in-memory
% representation and generate an on-disk format, but this doesn't help
% when given a particular on-disk format.  Existing ad hoc description
% languages~\cite{gpce02,sigcomm00,erlang} are steps in the right
% direction, but they focus on binary, error-free data and they do not
% provide auxiliary tools.

% We propose to extend and improve the current \pads{}\ prototype by engaging
% in three intertwined research efforts:

% \begin{enumerate}
% \item Our experience with real-world data has revealed that
% the current \pads{}\ system is unable to process many common
% data formats.  We propose a number of extensions to \pads{} 
% that will greatly improve its expressive power and enable
% us to handle many more important ad hoc formats.
% \item While \pads{}\ can automatically generate tools for providing
% statistical data summaries, querying ad hoc data and generating
% \xml, the tool generation process is extremely brittle, inflexible
% and results in low-performance software.  We propose to redesign
% the automatic tool generation process using a principled methodology
% based upon a novel \pads{}-attribute system that can communicate
% semantic information across tools.  Our new methodology will
% improve the flexibility and reliability of the \pads{} system.
% \item There is no formal description of the \pads{} language
% and no semantics for the compiler or tools.  We propose to
% formalize the language and analyze the semantics of the
% compiler.  Our formal analysis may find bugs or flaws in the
% \pads{} system and will improve our confidence in the implementation.
% \end{enumerate}

% \noindent
% In addition, part of our mission will be to use \pads{} to provide
% data processing tools for researchers in the natural sciences including
% biology, chemistry and physics.  

% In building the \pads{} system, we faced challenges on two fronts:
% designing the data description language and
% crafting the generated libraries and tools.
% In the rest of the paper, 
% we describe each of these designs.
% We also give examples to show how useful they have been at AT\&T.


\subsubsection{The Current \pads{} Language Infrastructure}

% \figref{figure:data-sources} summarizes some of the sources we have
% worked with.  They include ASCII, binary, and Cobol data formats, with
% both fixed and variable-width records, ranging in size from
% relatively small files through network applications which process over
% a gigabyte per second.  Common errors include undocumented data,
% corrupted data, missing data, and multiple missing-value
% representations.


% \subsection{Common Log Format}
% Web servers use the Common Log Format (CLF) to log client
% requests~\cite{wpp}.  Researchers use such logs to measure
% properties of web workloads and to evaluate protocol changes
% by "replaying" the user activity recorded in the log.
% This ASCII format consists of a sequence of
% records, each of which has seven fields: the host name or IP address
% of the client making the request, the account associated with the
% request on the client side, the name the user provided for
% authentication, the time of the request, the actual request, the
% \textsc{http} response code, and the number of bytes returned as a
% result of the request.  The actual request has three parts: the
% request method (\eg, \texttt{GET}, \texttt{PUT}), the requested
% \textsc{uri}, and the protocol version.  In addition, the second and
% third fields are often recorded only as a '-' character to indicate
% the server did not record the actual data.  \figref{figure:clf-records}
% shows a couple of typical records.


In order to understand further how \pads{} can be used,
will take a look at an example of ad hoc data:
a tiny fragment of data from the Gene Ontology Project~\cite{geneontology}.  
Data in this format is widely used by molecular biologists to
analyze gene products produced by various organisms.
\figref{figure:dibbler-records} shows a tiny sample of
the format.  Cursory examination of the example data reveals
first of all that this is an ASCII format with two 
logical parts, a header portion (that portion beginning with
\texttt{format-version} and ending with the first blank line)
and a main portion, which includes a sequence of 
information blocks separated by blank lines (only one such block 
is shown in the example).  The format
designers call these information blocks {\em stanzas}.
Each stanza begins with a stanza tag, which may be either
\texttt{[Term]}, as shown in the example data, or
\texttt{[Typeref]}.  On the following line,
the first element of each stanza, is
the string \texttt{id}, followed by a colon, followed by a
GO identifier.  A GO identifier is the word GO, followed
by another colon, followed by a string.  
After the first mandatory id field, the rest of the stanza includes
any number of identifier, colon, field entries.
% While this format is actually remarkably simple, and
% we have left out several details in this coarse description,
% it should be apparent that English is a poor
% language for describing data formats!

\begin{figure*}
\begin{small}
%\begin{center}
\begin{verbatim}
format-version: 1.0
date: 11:11:2005 14:24
saved-by: midori
auto-generated-by: DAG-Edit 1.419 rev 3
default-namespace: gene_ontology
remark: cvs version: $Revision: 1.2 $
subsetdef: goslim_goa "GOA and proteome slim"
subsetdef: goslim_yeast "Yeast GO slim"
subsetdef: goslim_plant "Plant GO slim"
subsetdef: goslim_generic "Generic GO slim"
subsetdef: gosubset_prok "Prokaryotic GO subset"

[Term]
id: GO:0000001
name: mitochondrion inheritance
namespace: biological_process
def: "The distribution of mitochondria\, including the mitochondrial 
genome\, into daughter cells after mitosis or meiosis\, mediated by 
interactions between mitochondria and the cytoskeleton." [PMID:10873824, 
PMID:11389764, SGD:mcc]
is_a: GO:0048308 ! organelle inheritance
is_a: GO:0048311 ! mitochondrion distribution

\end{verbatim}
\caption{Tiny example of Gene Ontology data. (Some lines broken to
present data on this page.)}
\label{figure:dibbler-records}
%\end{center}
\end{small}
\end{figure*}

Now, we can examine how to use \pads{} to describe 
the {\em physical layout} and 
{\em semantic properties} of our ad hoc data source. 
The language provides a type-based model:
basic types describe atomic data such as integers, characters, 
strings, dates, urls, \etc, while
structured types describe compound data built from simpler pieces.
\suppressfloats

In a bit more detail,
the \pads{} library provides a collection of broadly useful base
types.  Examples include 8-bit unsigned integers (\cd{Puint8}), 32-bit
integers (\cd{Pint32}), dates (\cd{Pdate}), strings (\cd{Pstring}),
and IP addresses (\cd{Pip}).  Semantic conditions for such base types
include checking that the resulting number fits in the indicated
space, \ie, 16-bits for \cd{Pint16}.  By themselves, these base types
do not provide sufficient information to allow parsing because they do
not specify how the data is coded, \ie{}, in ASCII, EBCDIC, or binary.
To resolve this ambiguity, \pads{} uses the \textit{ambient} coding,
which the programmer can set.  By default, \pads{} uses ASCII.  
% To
% specify a particular coding, the description writer can select base
% types which indicate the coding to use.  Examples of such types
% include ASCII 32-bit integers (\cd{Pa_int32}), binary bytes
% (\cd{Pb_int8}), and EBCDIC characters (\cd{Pe_char}).  In addition to
% these types, users can define their own base types to specify more
% specialized forms of atomic data.

To describe more complex data, \pads{} provides a collection of
structured types loosely based on \C{}'s type structure.  In
particular, \pads{} has \mytt{Pstruct}s, \mytt{Punion}s, and \mytt{Parray}s
to describe record-like structures, alternatives, and sequences,
respectively.  \mytt{Penum}s describe a fixed collection of literals,
while \mytt{Popt}s provide convenient syntax for optional data.  Each of
these types can have an associated predicate that indicates whether a
value calculated from the physical specification is indeed a legal
value for the type.  For example, a predicate might require that two
fields of a \mytt{Pstruct} are related or that the elements of a
sequence are in increasing order.  Programmers can specify such
predicates using \pads{} expressions and functions, written using a
\C{}-like syntax.  Finally, \pads{} \mytt{Ptypedef}s can be used to
define new types that add further constraints to existing types.

\pads{} types can be parameterized by values.  This mechanism serves
both to reduce the number of base types and to permit the format and
properties of later portions of the data to depend upon earlier
portions.  For example, the base type \cd{Puint_FW(:x:)} specifies
an unsigned integer physically represented by exactly \cd{x}
characters, where \cd{x} is a value that has been read earlier in the
parse.  The type \cd{Pstring(:COLON:)} describes a string
terminated by a colon (when \texttt{COLON} is defined to be \texttt{':'}).  
Parameters can be used with compound types like arrays and unions to
specify the size of an array or which branch of a union should be
taken.  This parameterization is what makes PADS a {\em dependently-typed}
language and substantially different from languages based on
context-free grammars or regular expressions.

\figref{figure:dibbler} gives an abbreviated \pads{} description 
for the Gene Ontology
data.  
We will use this example to illustrate some of the basic
features of the current \pads{} language.  
In \pads{} descriptions, types are declared before they are used, 
so the type that describes the totality of the data source appears 
at the bottom of the description.  In this case,
the type \texttt{OBO\_file} describes the the entirety of the
GO data source (the \texttt{Psource} type qualifier indicates
this fact explicitly).  \texttt{OBO\_file} is a \mytt{Pstruct} type with
two fields, a \texttt{hdr} field, with type  \texttt{OBO\_header}
and a \texttt{stanzas} field with type \texttt{OBO\_stanza[]},
an array of \texttt{OBO\_stanza}s of arbitrary length.
In general, \mytt{Pstruct}s describe fixed sequences of data with 
unrelated types.  A little further up the description, there is
a slightly more complex \mytt{Pstruct}:  \texttt{OBO\_stanza\_tag}.
This struct contains two kinds of fields, named fields, like
before, and unnamed fields \texttt{'['} and \texttt{']'}.
These unnamed fields match literal characters in the data, in this
case square brackets.  In 
\texttt{OBO\_tag\_value\_pair}, another unnamed field involving 
a regular expression (introduced by \texttt{Pre}) matches a colon
followed by any number of spaces.  Named fields are included in
the internal representation of the data, while unnamed fields are
not.  

Both \texttt{OBO\_tag\_value\_pair} and
\texttt{OBO\_stanza\_tag} mentioned above are preceded by the
\texttt{Precord} type qualifier.  
The notion of a record varies depending upon the data encoding.  
ASCII data typically uses new-line characters to delimit 
records, binary sources tend to have fixed-width records, while 
COBOL sources usually store the length of each record before the actual data.
\pads{} supports each of these encodings of records and allows users to define
their own encodings.  By default, \pads{} assumes records are new-line terminated.  The constant \texttt{Peor} is also set to this
\texttt{e}nd-\texttt{o}f-\texttt{r}ecord marker, and may be used anywhere in a
description, as we did in the
type \texttt{OBO\_tag\_value\_pair}.
\texttt{Precords} have error-recovery semantics -- if errors 
in the data cause the parser to become seriously confused,
it will attempt to recover to a record boundary.
In practice, we have found this to be a very robust recovery mechanism
for ad hoc data.  

The \texttt{OBO\_stanza} structure 
illustrates the use of some simple semantic constraints.
For instance, following the \texttt{id} field is the constraint
\texttt{hastag(id,"id")}, which is defined earlier in the file
as a function in defined in C.  In general, such constraints should
be pure (have no externally visible effects), but otherwise may be
arbitrary C expressions.  Notice also that the first argument to the constraint
is \texttt{id}, the name of the current field.  More generally,
constraints may refer to any type parameters in scope and any previous field.
The type parameters in particular, allow information concerning data
just parsed to be passed arbitrarily far forward in a description.
The second semantic constraint in \texttt{OBO\_stanza} is
\texttt{Pterm(Peor)}, which is a built-in constraint for 
controlling array termination.  In this case, it states that
the array terminates when it sees an end-of-record marker 
\texttt{Peor} following any array element.  
In general, \pads{} provides a rich collection
of array-termination conditions: reaching a maximum size, finding 
a terminating literal 
or satisfying a user-supplied predicate over the already-parsed portion of 
the \mytt{Parray}.  \pads{} also has convenient syntax for 
specifying separator symbols that
appear between elements of an array and declaring inter-element
constraints including sorting.

One last feature of note in this description is the \texttt{Penum}
\texttt{OBO\_stanza\_type}.  \texttt{Penum}s specify that
the data must be one of several fixed strings, in this case, 
either \texttt{Term} or \texttt{Typeref}.  \pads{} also admits
\texttt{Punion}s of several forms for more general sorts of alternatives
in data formats.

Other examples of \pads{} descriptions and an online demo may be
found at \url{www.padsproj.org}.

\begin{figure}[t]

\begin{code}
/* www.geneontology.org OBO flat file format. */
\mbox{}
#define COLON ':'
\mbox{}
Precord Pstruct OBO\_tag\_value\_pair\{
  Pstring(:COLON:) tag;  
  Pre "/: */"; 
  Pstring\_ME(:Peor:) val;
\};
\mbox{}
int hasTag(OBO\_tag\_value\_pair p,char *tag)\{
  return Pstring\_eq\_cstr(&p.tag,tag); 
\};
\mbox{}
Precord Pstruct OBO\_header\{
  ...
\};
\mbox{}
Penum OBO\_stanza\_type\{ Term, Typeref \};
\mbox{}
Precord Pstruct OBO\_stanza\_tag \{
    '['; OBO\_stanza\_type s\_type; ']';
\};
\mbox{}
Precord Pstruct OBO\_stanza\{
  OBO\_stanza\_tag       tag;
  OBO\_tag\_value\_pair   id      : hasTag(id,"id");
  OBO\_tag\_value\_pair[] tvpairs : Pterm(Peor);
\};
\mbox{}
Psource Pstruct OBO\_file\{
  OBO\_header hdr;
  OBO\_stanza[] stanzas;
\};
\end{code}

%\input{dibbler_new}

\caption{Abbreviated \pads{} description for Gene Ontology data.}
\label{figure:dibbler}
\end{figure}


% Returning to the CLF description in \figref{figure:wsl}, the \mytt{Penum} type \cd{method_t} describes
% a collection of data literals.  During parsing, \pads{} interprets these
% constants using the ambient character encoding.  The \mytt{Ptypedef} 
% \cd{response_t} describes possible server response codes in CLF data by adding
% the constraint that the three-digit integer must be between 100 and 600.

% \setlength{\floatsep}{0pt}
% \setlength{\dblfloatsep}{0pt}
% \setcounter{totalnumber}{4}
% \setcounter{dbltopnumber}{2}
% \begin{figure*}[t!]
% \begin{small}
% \begin{center}
% \begin{code}
% \chapter{Common log format example}
\section{pads description}
\section{c header file}
% \end{code}
% \vskip -2ex
% \caption{Tiny example of web server log data.}
% \label{figure:clf-records}
% \end{center}
% \end{small}
% \end{figure*}


% \begin{figure}
% \begin{code}
\kw{Punion} client\_t \{
  Pip       ip;      /- 135.207.23.32
  Phostname host;    /- www.research.att.com
\};
\mbox{}
\kw{Punion} auth\_id\_t \{
  Pchar unauthorized : unauthorized == '-';
  Pstring(:' ':) id;
\};
\mbox{}
\kw{Pstruct} version\_t \{
  "HTTP/";
  Puint8 major; '.';
  Puint8 minor;
\};
\mbox{}
\kw{Penum} method\_t \{
    GET,    PUT,  POST,  HEAD,
    DELETE, LINK, UNLINK
\};
\mbox{}
bool chkVersion(version\_t v, method\_t m) \{
  \kw{if} ((v.major == 1) && (v.minor == 1)) \kw{return} true;
  \kw{if} ((m == LINK) || (m == UNLINK)) \kw{return} false;
  \kw{return} true;
\};
\mbox{}
\kw{Pstruct} request\_t \{
  '\\"';   method\_t       meth;
  ' ';    Pstring(:' ':) req\_uri;
  ' ';    version\_t      version :
                  chkVersion(version, meth);
  '\\"';
\};
\mbox{}
\kw{Ptypedef} Puint16\_FW(:3:) response\_t :
         response\_t x => \{ 100 <= x && x < 600\};
\mbox{}
\kw{Precord} \kw{Pstruct} entry\_t \{
         client\_t       client;
   ' ';  auth\_id\_t      remoteID;
   ' ';  auth\_id\_t      auth;
   " ["; Pdate(:']':)   date;
   "] "; request\_t      request;
   ' ';  response\_t     response;
   ' ';  Puint32        length;
\};
\mbox{}
\kw{Psource} \kw{Parray} clt\_t \{
  entry\_t [];
\}
\end{code}

% \caption{\pads{} description for web server log data.}
% \label{figure:wsl}
% \end{figure}

\subsection{Overview of Planned Research}
\label{ssec:sow}

Our central research agenda is divided into three broad subsections,
which we will discuss here; our broader impacts will be discussed in the next 
section.  First, our experience with real-world data has revealed that
it is common to find a single logical repository of ad hoc data
physically represented as a collection of files, organized in a structured
multi-level directory system.  We propose to extend the \pads{}\ 
language and tool generation system so that it can describe and
process such rich repositories.  Second, while
\pads{} lifts the level abstraction for generation of data processing
tools, \pads{} takes some time to learn and descriptions take time to
write.  We propose to develop an entirely new way to generate \pads{}
descriptions, by using of a novel kind of markup language.  This new
paradigm holds the promise of making it possible to describe
rich, complex data sources in just minutes and then immediately
generate data processing tools, all with very little 
learning overhead.  Third, we propose to develop new algorithms for
parsing and format analysis based on developing a new theory
of data-dependent grammars.
Overall, our research combines novel language
design, high-performance systems engineering and theoretical analysis,
all aimed at solving crucial data processing problems.

\subsubsection{Hierarchical, Multi-source Data Descriptions}

Often, a single logical data source is represented
as several distinct, concrete repositories.   This is the case
in the GO data source we have examined, where data is split into four disjoint
files: a molecular function file, a biological process file,
a cellular component file and a term definitions file.
However, the size and number of repositories that make up a single source
may vary widely.  At the other end of the spectrum are systems involving
continuous monitoring of widescale phenomena that automatically
produce new data at phenominal rates.  

\begin{figure}
\begin{center}
\includegraphics[width=100mm]{coral-pic.jpg}
\end{center}
\vspace{-1.8cm}

\caption{Directory Structure of the Coral Monitoring and Logging Subsystem}
\label{fig:coral-pic}
\end{figure}

The Coral
content distribution system~\cite{freedman:coral} is an example of one of the latter classes
of system.  This system collects several different kinds of log files
at regular intervals from a set of planetlab nodes and stores them in
a set of directories organized as shown in Figure~\ref{fig:coral-pic}.
The picture shows that a top-level root directory contains a set of subdirectories,
one per host in the Coral system.  Each of those subdirectories contains
contains a further set of subdirectories, one per date.  Finally, each
of those directories contains 4 different sorts of log files needed for
monitoring the health and security of the Coral system.

Unfortunately, the current \pads{} 
implementation is limited to processing a single data source.
We propose to extend our specification language to enable automatic generation of tools
that process multiple data sources, either on one local machine or distributed across a wide area network.  Doing so will require investigation of how to specify rich directory structures
and to generate effective libraries and stand-alone tools
for parsing, querying and monitoring their contents.

Figure~\ref{fig:example-multi-source} illustrates how we might extend
\pads{} to incorporate description of multi-level directories.  Starting at the
bottom of the description, the reviewer will see a {\tt Pdirectory} declaration,
which states that {\tt coral\_d} describes a directory (as opposed to a single
file).  Such directory descriptions may include a series of clauses with the form
\begin{code}
<name> is <path-description> :: <object-description>
\end{code}
In this case, {\tt <path-description>} describes a path to set of objects (either
files or further directories) and {\tt <object-description>} describes the
objects at the end of the path. The {\tt <name>} field 
provides a name for the internal, in-memory data structure
that represents this element of the directory.  Such names will be useful for
generating interfaces for accessing, querying or monitoring the data.
In the path description on the second last line of this example 
({\tt (host :: phostname)/(time :: pdate)}), 
we use \pads{} types to describe
the syntax of the file names we are interested in.  In particular, {\tt phostname}
describes the top-level directory names and {\tt pdate} describes the next level
directory names.  Moreover, we bind the actual directory names we find to variables
{\tt host} and {\tt time}.  Every specified path is expected to lead to an
object described by {\tt host\_info\_d(host,time)}.  If we investigate the
definition of the type {\tt host\_info\_d}, we find that it has 
two sorts of fields: (1) the path-based fields just mentioned and 
(2) a computed field that associates each object with a host name {\tt h}
and time {\tt t}, both of which are passed to the description as a parameter.
One other feature that shows up in this simple example is a meta-data constraint.
In this case, the timestamp on the files is constrained.  More generally, one
might specify constraints on other bits of meta data such as the owner,
access controls or size.

Though we have some solid ideas in terms of the overall direction of the language design,
there is much research left to do in terms of (1) fleshing out the language design,
(2) developing infrastructure for generating tools for accessing, querying, monitoring,
parsing and validating data that meets the description, (3) implementing compiler
and language infrastructure, (4) experimenting on examples
and (5) analyzing the semantics of the language.  On the latter point, 
directories can be viewed naturally as trees and 
we anticipate understanding the language in terms of tree logics~\cite{calcagno:tree-logic},
though there is much research yet to be done on this topic.

\begin{figure}
\begin{code}
Ptypedef pdate = pstring_ME ("/[1-2][0-9]-[0-9][0-9]-2[0-9][0-9][0-9]/")
Ptypedef phostname = ...  // pads description 
\mbox{}
Psource corald_ty   = ...  // pads description 
Psource coraldns_ty = ...  // pads description 
Psource coralweb_ty = ...  // pads description 
Psource probe_ty    = ...  // pads description
\mbox{}
Pdirectory host_info_d (h,t) \{
  host :: phostname = h;
  time :: ptimeformat = t;
  corald   is  "corald.log.head"      :: corald_ty   <| (timestamp >= t) |>;
  coraldns is  "coraldnssrv.log.head" :: coraldns_ty <| (timestamp >= t) |>;
  coralweb is  "coralwebsrv.log.head" :: coralweb_ty <| (timestamp >= t) |>; 
  probe    is  "probed.log.head"      :: probe_ty    <| (timestamp >= t) |>;
 \}
\mbox{}
Pdirectory coral_d \{
   hosts is (host :: phostname)/(time :: ptimeformat) :: host_info_d(host,time);
 \}
\end{code}
\caption{Example Hierarchical, Multi-source Data Description}
\label{fig:example-multi-source}
\end{figure}


\subsubsection{A Markup Language for Ad Hoc Text Data}

\input{raw-log}

In order to make programmers who work with ad hoc data more productive,
we propose to develop a new kind of {\em markup language} designed
to help users generate \pads{} descriptions, and from there,
an entire suite of data processing tools for ad hoc text data. 
More specifically, given a new ad hoc
data source, a programmer will edit the document to add
a number of simple annotations, which serve to specify its syntactic
structure. Annotations will include elements that specify constants,
optional data, alternatives, enumerations, sequences, tabular data,
and recursive patterns. Our proposed system will use a combination of
user annotations and the raw data itself to extract a 
\pads{} description from the document. This description can then
be used to parse the data and transform it into an XML parse tree,
which may be viewed through a browser for analysis or debugging
purposes. 

When the user is satisfied with the generated  description, 
he or she may save it as lasting documentation of
the data format or compile it into a host of useful data processing
tools ranging from parsers, printers and traversal libraries to format
translators and query engines. Overall, the proposed {\em markup language}
represents an entirely new way to generate data processing tools
and holds the promise of improving the
productivity of programmers who work with ad hoc data substantially.

To illustrate our ideas, consider the web server log presented in 
Figure~\ref{fig:raw-server-log}.  To begin the process of 
constructing a description through the mark-up process, a programmer
encloses key text fragments in braces~\footnote{We assume braces do not
otherwise appear in the file.  The delimiter syntax would be customizable
in any system we build.}  For example, the first step might be to
identify the key top-level unit of repetition in the log and call it
``{\tt Record}''.\footnote{In the following, we highlight additions to
the raw text file using a grey boxes.}
{
\begin{code}\scriptsize
\kw{\{Record:}207.136.97.49 - - \verb+\+
  [15/Oct/1997:18:46:51 -0700] \verb+\+
  "GET /turkey/amnty1.gif HTTP/1.0" 200 3013\kw{\}}\end{code}
}
\noindent
Intuitively, the portion of 
a grammar ({a.k.a.,} description)
so-defined involves a single non-terminal named {\tt Record}:
\begin{code}\scriptsize
Record ::= ...\end{code}
Moreover, since there are no other annotations to guide grammar
generation, the system uses a simple default rule to generate the 
right-hand side -- it assumes the desired right-hand side is
a simple concatenation of basic tokens derived by running a default
lexer over the data enclosed in braces.  
\begin{code}\scriptsize
Record ::= Num '.' Num WS '-' WS '-' WS '[' ...\end{code}
In order to maintain predictability and ease-of-use, and avoid
ambiguities, the set of default
tokens will be kept to the barest minimum.  However, we will develop
various mechanisms for overriding defaults.  For instance, programmers
will be able to draw upon a rich collection of predefined atomic tokens
such as IP addresses, dates, times, email address, file paths and others.
For instance:
{
\begin{code}\scriptsize
\{Record:\kw{\{IP<:}207.136.97.49\kw{\}} - - \verb+\+
  [\kw{\{Date<:}15/Oct/1997\kw{\}}:\kw{\{Time<:}18:46:51 -0700\kw{\}}] \verb+\+
  "GET /turkey/amnty1.gif HTTP/1.0" 200 3013\}\end{code}
}

We also need techniques for introducing alternatives into the description.
The simplest way is merely to use a particular non-terminal name
repeatedly.  We illustrate this technique below by
using the non-terminal {\tt Size} twice, once around an integer
(which represents the normal case -- the number of bytes returned
by the server is reported properly) and
once around {\tt "-"} (which represents the non-standard case of no
data available).
%To express the possibility of alternatives, we simply use the same name 
%more than once in different places:
{
\begin{code}\scriptsize
\{Record:\{IP<:207.136.97.49\} - - \verb+\+
  [\{Date<:15/Oct/1997\}:\{Time<:18:46:51 -0700\}] \verb+\+
  "\{Message>:GET /turkey/amnty1.gif HTTP/1.0\}" 200 \verb+\+
  \kw{\{Size:}3013\kw{\}}\}
...
152.163.207.138 - - \verb+\+
  [15/Oct/1997:19:06:03 -0700] \verb+\+
  "GET /images/spot5.gif HTTP/1.0" 304 \kw{\{Size:}-\kw{\}}\end{code}
}
\noindent
Such annotations extend the grammar with a union of two or more options:
\begin{code}\scriptsize\noindent
Size ::= Num + '-'
Record ::= IP WS '-' WS '-' WS ... Size
\end{code}

So far, the message in quotations has been treated
as an uninterpreted string rather than a semi-structured subdocument.
To begin to break the string down, one may want to specify
that it always begins with the keyword {\tt GET}.
To generate the description that specifies this constraint,
as opposed to the more liberal grammar that allows any word in that position,
one could use an equality annotation {\tt \{Name=...\}} or it's
unnamed variant {\tt \{=...\}} as in the following
example.
\begin{code}\scriptsize
\{Record:\{IP<:207.136.97.49\} - - \verb+\+
  [\{Date<:15/Oct/1997\}:\{Time<:18:46:51 -0700\}] \verb+\+
  "\kw{\{=}GET\kw{\}} \{Message>:/turkey/amnty1.gif HTTP/1.0\}" \verb+\+
  200 \{Size/S:3013\}\}\end{code}
On the other hand, however, one might 
notice that not all such strings begin with {\tt GET} --- there
are a small number of other keywords such strings can begin with:  {\tt PUT}, 
{\tt POST}, {\tt HEAD}, {\tt DELETE}, {\tt LINK},  and {\tt UNLINK}.
To generate a grammar involving the list of keywords that actually
appears in this file, it appears that we need an enumeration annotation.  
We might write enumeration using an annotation with the
form {\tt \{Name//enum>:...\}}.  
For instance, we might annotate our example as follows.
\begin{code}\scriptsize
\{Record:\{IP<:207.136.97.49\} - - \verb+\+
  [\{Date<:15/Oct/1997\}:\{Time<:18:46:51 -0700\}] \verb+\+
  "\kw{\{Method//enum:}GET\kw{\}} \verb+\+
  \{Message>:/turkey/amnty1.gif HTTP/1.0\}" 200 \verb+\+
  \{Size/S:3013\}\}\end{code}
If the web log contains examples of {\tt PUT} and {\tt POST} in addition
to {\tt GET}, the following grammar fragment would be generated.
\begin{code}\scriptsize
Method ::= 'GET' + 'PUT' + 'POST'
Record ::= IP ... '\verb+\"+' Method WS ...
\end{code}

In addition to these annotations, we plan to research to develop many 
more including:

\begin{itemize}
\item Repetitions akin to many variations on the Kleene Star from regular
expressions or \pads{} array declarations.
\item Support for recursion and definition of full context-free grammars.
\item Support for optional data and tagged alternatives.
\item Direct support for tablular data with formatted rows and columns.
\item Support for data dependent constraints such as descriptions that
involve reading a number $N$ and then reading an iteration of items in
which there are $N$ items in the iteration. 
\item Support data transformations by specifying computations that produce
new data from the data in the document or that substitute one piece of data
for another.
\item Support for multiple files and multi-level directories.
\end{itemize}

Using these annotations, we believe users with very little training
will be able to generate data descriptions (and therefore data processing
libraries and tools) in just a minute or two instead of hours.  
Moreover, by adding support for data transformations and computations,
we will vastly increase the power of the current \pads{} infrastructure
and tool generation chain.  Overall, we anticipate these new ideas will
lead to a substantial improvement
in programmer productivity.

Interestingly, our initial
research also indicates that there are deep theoretical connections to 
the field of {\em relevance logic}~\cite{relevance-logic}.  In particular,
it appears as though it is possible to generate a grammar from a document
using a markup language when each alternative of the grammar is {\em used}
in parsing the document.  This is a very similar condition under which a
theorem has a proof in relevance logic:  In relevance logic, a theorem
has a proof when each hypothesis in the context is {\em used}.
We plan further research to fully uncover the extent of these 
theoretical connections.

\subsubsection{Grammatical Foundations and Efficient Implementation}

Modern data formats pose an enormous challenge to automatic 
parser generators as they may contain elements best described by
some combination of full context-free grammars and, further 
{\em non-context free} features such as 
variable binding, data-dependent constraints, and
parameterized non-terminals.  In addition, it is useful to be able
to include pre-defined libraries (for dates or urls or email 
addresses) as {\em black boxes} within a description. 

The current implementation of \pads{} handles a subset of these features
through the use of a simple recursive-descent, back-tracking parser.  
As a recursive descent parser, \pads{} is unable to handle arbitrary
context-free grammars, particularly those involving left recursion. 
In addition,
PADS descriptions, like PEGS~\cite{birman+:parsing,ford:pegs,ford:packrat,grimm:packrat}, have a non-standard 
semantics for unions. In particular, if, when
parsing the union $A + B$, PADS succeeds in parsing an $A$, it will
commit to that choice and never backtrack, even when downstream
errors arise that could be avoided if the input was interpreted as a
$B$. 
Unfortunately, the PADS and PEGS semantics both have undesirable consequences
in certain applications because it causes closure under
union and concatenation to fail:
\begin{itemize}
\item  $L(A + B) != L(A) \cup L(B)$
\item  $L(A \cdot B) != L(A) \cdot L(B)$
\end{itemize}
Both of these properties are extremely useful when processing grammars,
and, for instance, are required for the correct functioning
of divide-and-conquer grammar induction algorithms, including
algorithms designed to infer PADS descriptions~\cite{fisher+:dirttoshovels}. 
The PADS
grammar induction algorithms attempt to avoid learning incorrect
grammars by using various heuristics, but the heuristics are not
always successful and the algorithms do fail occasionally on real
data sources.

In order to develop a solid and general framework for implementing all
of the features required by modern applications, the PI and colleagues
have recently developed a new grammatical foundation for implementing
languages like \pads{}~\cite{jim+:data-dependent-grammars}.
  This new work defines the notion of 
a data-dependent
grammar and a data-dependent automaton.  These theories
generalize context-free grammars and push-down automata.  

Figure~\ref{fig:ddg-pic} gives a simple example of a data-dependent
automaton.    Such automata process inputs relative to environment
(in this case the environment involves a single variable $n$).
The automaton may follow edges between states marked
with a predicate $\{P\}$ when P is true (edge 1-2 is an example of this). 
Call edges (such as edge 2-5 or 2-4 or 4-1) push the current state and
may be parameterized (as in the edge 4-1, which instantiates the
parameter $n$ with $n-1$).  Final states are denoted using double-circles.
These final states may also be marked by one or more grammar
non-terminal symbols ({\em e.g.,} state 6 is marked by non-terminal ``dig'').  
When control reaches a final state for non-terminal $A$, the execution
engine returns to the state on the top of the stack and follows
an edge marked by $A$.  For instance, upon arrival at state 6, 
control will return to state 2 (because that will have been placed
on the top of the stack by the call from 2 to 5) and then from there,
control will proceed along the edge marked ``dig'' to state 4.  Overall,
given an initial (dynamically defined) parameter $n$, this automaton 
will recognize $n$ digits.

While we have developed a non-deterministic execution semantics 
for these automata as well as an algorithm based on Earley's 
parsing algorithm~\cite{Earley}.  However, there remain many theoretical
and practical questions left to answer about them:

\begin{itemize}
\item How do we map \pads{} descriptions to data-dependent automata?
\item What data structures do we use to ensure efficient parse tree 
generation?
\item Data dependent grammars may be ambiguous.  How do we specify and
resolve ambiguities during parsing?  How do we detect ambiguities in the
presence of constraints?
\item What algorithms do we use to determinize data-dependent automata? 
More generally, how do we optimize data-dependent automata?
\item What is the theoretical complexity of parsing in this new model?
Empirically, how do they compare to backtracking algorithms on grammars
of interest?
\item Can we develop algorithms to answer (or approximate) 
questions of emptiness or
inclusion of grammars or automata?
\item Can we use data-dependent grammars and automata to generate data
matching giving descriptions?  Can we use these algorithms to
develop effective test generation infrastructure
for data processing tools?
\end{itemize}

\begin{figure}
\begin{center}
\includegraphics[width=100mm]{ddg-pic.jpg}
\end{center}
\vspace{-2cm}

\caption{Data Dependent Automaton for Parsing a Fixed-Width Integer 
with Width $n$.
Final states}
\label{fig:ddg-pic}
\end{figure}


\subsection{Broader Impacts}
\label{ssec:impact}

%If funded, this project will have two major broad impacts.

\paragraph*{Industrial and Governmental Impact}
Today�s most important industries and governmental services run on
information. As different companies, NGOs, and governmental groups
implement buy, sell and exchange data, they are invariably frustrated
by the time and expense it takes to extract information from the
low-level formats in which their data is represented. 
The new technology proposed here will help improve the productivity
of companies and industries that face such data processing problems.
More directly,
through the PI's ongoing collaboration with Kathleen Fisher,
Yitzhak Mandelbaum and Trevor Jim at AT\&T Research, he has access
to industrial data sets and information on concrete industrial data
processing problems.  Working with AT\&T Research, he will have the tools
and resources required to help solve real problems in server log processing, 
network system monitoring and billing fraud detection.

\paragraph*{Supporting Research in the Natural Sciences}
Part of our mission will be
to provide support for the data processing needs of biologists,
chemists and physicists at Princeton University.
In particular, we will meet with Princeton computational biologists
Mona Singh, Hillary Coller and Olga Troyanskaya and their students
to determine their data processing needs.  In particular, Troyanskaya
has several projects that involve assembling and integrating various
ad hoc data sources (such as the GO data format described earlier in
this proposal) so that other computational biologists may query
and analyze them.  With the insights we learn from these colleagues, 
we will produce PADS descriptions for important biological data
sources used by them and others and use the PADS compiler to 
generate data processing tools for them.
In addition to working directly with scientists at Princeton, we
will make all of our software tools 
available on the web so academics anywhere can use them.

\paragraph*{Undergraduate Research}
Princeton University has a very active undergraduate research program.
Princeton students usually do at least one semester of
Junior Independent Work and one semester of Senior Independent Work. 
Many students also do a year-long
Junior and Senior Theses.  
%For the past two years, PI Walker has
%lead the Princeton Computer Science Department Independent Work Program.
We believe that PADS is an excellent
platform for undergraduate independent work and we plan to link
undergraduate independent work with our interdisciplinary effort to
produce tools for biologists.  
%% We have already recruited Mark Daly,
%% a Princeton Senior who is doing his undergraduate senior thesis
%% on a PADS user interface and automatic format inference.
%% We hope encourage other undergraduates to help us build
%% specific data processing tools for the biological data formats
%% used in Princeton's Genomics Institute.
This will introduce students both to interesting and novel
computer science research and also allow them to build useful
tools for biological applications. 
%% The PI has a proven track record
%% when it comes to advancing graduate and undergraduate education
%% as he has organized two summer schools (2004, 2005) on technology for
%% secure and reliable programming and mentored several undergraduates,
%% the latest of which, Rob Simmons, won the Princeton Computer Science Department
%% Senior Thesis Award.

\subsection{Comparison with Other Research}
\label{ssec:related}

The oldest tools for describing data formats are parser generators such as
Lex and Yacc.  While excellent for parsing programming languages, Lex and Yacc
are too heavyweight for parsing the simpler ad hoc data formats one
runs into in the sciences.   
Unlike PADS, whose syntax is based on types from the well-known C language,
the syntax of Lex and Yacc is somewhat foreign.  Perhaps more importantly,
users must write a lexer, write a
grammar, and construct the in-memory representations by hand.  In
addition, they only work for ASCII data, they do not easily
accommodate data-dependent parsing, and they do not provide auxiliary
services such as automatic XML conversion, stastical analysis and
others.  Some more modern parser generators such as ANTLR~\cite{antlr} alleviate
a few of these problems, but they still do not automatically generate auxiliary tools
useful in processing ad hoc data nor do they provide good support for generating
rich, well-typed in-memory representations (ANTLR's in-memory representations
are very limited when compared with PADS and the extensions we propose).

There are parallels between PADS types and some of the elements of parser
combinator libraries found in languages like
Haskell~\cite{burge:parser-combinators,hutton+:parser-combinators}. 
However, as with most other general-purpose parsing tools, one cannot
simply put together a collection of Haskell's parser combinators and
automatically generate domain-specific programs such as 
an XML converter or a histogram generator, for instance.  

A somewhat different class of languages includes
\textsc{ASN.1}~\cite{asn} and \textsc{ASDL}~\cite{asdl}. 
Both of these systems specify the {\em logical\/} in-memory representation of data
and then automatically generate a {\em physical\/} on-disk representation.
Although useful for many purposes, this technology does not help
process data that arrives in predetermined, ad hoc formats.
Another language in this category is the Hierarchical Data Format 5 
(HDF5)~\cite{hdf5}.  This file format allows users to store scientific
data, but it does not help users deal with legacy ad hoc formats like PADS does.

The networking community has developed a number of domain-specific
languages~\cite{sigcomm00,bfd,gpce02} for parsing and printing binary data, particularly packets
from common networking protocols such as \textsc{TCP/IP} and also
\java{} jar-files.  Like \pads{}, these languages have a type-directed
approach to describing ad hoc data and permit the user to define
semantic constraints.  In contrast to our work, these systems handle
only binary data and assume the data is error-free or halt parsing if
an error is detected.  Parsing non-binary data poses additional
challenges because of the need to handle delimiter values and to
express richer termination conditions on sequences of data. These
systems also focus exclusively on the parsing/printing problem,
whereas we have leveraged the declarative nature of our data
descriptions to build additional useful tools.  

Currently, the Global Grid Forum is working on a standard
data-format description language for describing ad hoc data formats,
called DFDL~\cite{dfdl-proposal,dfdl-primer}.  Like \pads{},
DFDL{} has a rich collection of base types and supports a variety of
ambient codings.  Unlike \pads{}, DFDL{} does not support semantic
constraints on types nor dependent types, \eg{}, it is not possible to
specify that the length of an array is determined by some previously parsed field in the
data.  Our practical experience indicates that many ad hoc formats,
particularly binary formats, absolutely require dependent types in their
specifications.  DFDL{} is an annotated subset of XML{} Schema, which means
that the XML{} view of the ad hoc data is implicit in a DFDL{}
description.  DFDL{} is still being specified, so no DFDL-aware
parsers or data analyzers exist yet.  

% There are probably hundreds of tools that one might use if their data were
% in \xml.  However, the point of PADS is to allow scientists whose data is {\em not}
% already in \xml to get work done, particularly when that data contains errors,
% as ad hoc data often does.  Since many processes, machines, programs and other devices
% currently output data and a whole most of

XSugar~\cite{brabrand+:xsugar2005} allows user to specify an alternative non-XML
syntax for XML languages using a context-free grammar.  This tool
automatically generates conversion between XML and non-XML 
syntax. It also guarantees that
conversion will be invertable.  However, it does not use theory of Galois
connections, but instead introduces a notion of ``ignorable'' grammar
components (inferred based on properties of grammar) and proves
bijection modulo these ignorable elements.  More importantly,
since the basis of interconversion is a context free grammar, unlike PADS,
formats that required dependency may not be expressed.

XDTM~\cite{zhao+:sigmod05,xdtm} uses XML Schema to describe the locations of a collection
of sources spread across a local file system or distributed
across a network of computers.
However, XDTM has no means of specifying the contents of files,
so XDTM and PADS solve complementary problems.  Nevertheless, the XDTM design 
may provide ideas to us as we extend PADS from a single-source system to a
multi-source system. The METS schema~\cite{mets} is similar to XDTM as it describes 
metadata for objects in a digital library,
including a hierarchy such objects. 

Commercial database products provide support for
parsing data in external formats so the data can be imported into
their database systems, but they typically support a limited number of
formats, \eg{}, COBOL copybooks.  Also, no declarative description of the
original format is exposed to the user for their own use, and they
have fixed methods for coping with erroneous data.  For these reasons,
PADS is complementary to database systems.  We strongly believe that
in the future, commercial database systems could and should support a 
PADS-like description language that allows users to import information from
almost any format.  We hope that our research on PADS will make a broad
impact in this area.

On the theoretical front, the scientific community's understanding of type-based languages for data description
is much less mature.  To the best of our knowledge, our work on
the DDC is the first to provide a formal interpretation of dependent 
types as parsers and to study the properties of these parsers including error correctness and
type safety.  Regular expressions and context-free grammars, the basis for Lex and Yacc
have been well-studied, but they do not have dependency, a key feature necessary for expressing
constraints and parsing ad hoc scientific data.
{\em Parsing Expression Grammars} (PEGs),
studied in the early seventies~\cite{birman+:parsing}, revitalized more 
recently by Ford~\cite{ford:pegs} and implemented using
``packrat parsing'' techniques~\cite{ford:packrat,grimm:packrat}, 
are somewhat more similar to PADS recursive descent parsers. However, PADS does
not use packrat parsing techniques as the space overhead is too high for
large scientific data sets.  Moreover, our multiple interpretations of types in the DDC
makes our theory substantially different from the theory of PEGs.

%\subsection{Old Related}
%\section{Related Work}
\label{sec:related}

There are many useful tools designed to help programmers generate
parsers.  Examples include compiler technology such as the many
variants of Lex and Yacc as well as interpreter technology such the
parser combinator libraries found in functional programming languages
(Haskell~\cite{hutton+:parser-combinators}, for example).  Likewise,
there are tools to help programmers generate printers.  Each of these
technologies are very useful in their own domain, but none of them
could possibly be substituted for \padsml{}.  A key difference is that
a single \padsml{} description is sufficient to generate a large
collection of useful data processing tools including a parser, a
printer, a statistical error analysis, a format debugger, an \xml{}
translator, and in the future, a query engine~\cite{fernandez+:padx},
a content-based search engine~\cite{lv+:cbs,oh:siw}, more statistical
analysis, etc.  Though we chose a compiled solution, a parser or
printer combinator library might have served as an alternative
implementation strategy for certain individual components of our tool
suite.  However, by themselves, parser or printer combinator
libraries, \lex{} or \yacc, are not properly architected to give the
same, simple and powerful user experience as \padsml{}, which can be
summed up through the motto ``one declarative description; many
reliable tools.''

% A second way to characterize some of the differences between parser
% technologies is to recognize that \padsml{} is higher level (leaving
% out semantic actions, thereby making descriptions relatively simpler
% and enabling post facto tool implementation) and more domain specific
% (generating tools useful for processing ad hoc data, such as the
% \xml{} translator and statistical analysis, but not particularly
% useful for programming language implementation, for instance).  There
% are also many technical differences between the parsing technologies
% but we believe these are less important than the high-level vision.

Generic
programming~\cite{jeuring+:polytypic-programming,hinze+:generic-programming,lammel+:syb}
and design patterns such as the visitor are technologies that can
facilitate the implementation of type-directed data structure
traversals.  Lammel and Peyton Jones' original ``scrap your
boilerplate'' article~\cite{lammel+:syb} provides a detailed summary
of the trade-offs between different techniques.  We investigated using
one of these techniques before implementing the generator for
\padsml{} traversal functors from scratch.  However, we found most
advanced techniques for functional programming languages required
features such as type classes that are only present in variants of
Haskell.  The generated \padsml{} traversal functors are less flexible
than some of these traversal techniques, but they suffice for helping
us program the tools we have implemented, and for many more tools for
ad hoc data we are considering implementing.  On the other hand, we
have not seen type directed programming techniques developed for
systems of dependent types similar to the \padsml{} dependent type
structure.  
% However, to summarize the central differences between
% generic programming and \padsml{}, generic programming is a powerful
% {\em implementation technique} which can be used to implement many
% programs, including some of the programs generated from a \padsml{}
% description.  \padsml{}, however, is a higher-level system that
% directly provides a simple and powerful user experience for analysts
% who need to parse, print, process or transform ad hoc data.


%Again, the phrase ``one declarative description; many 
%reliable tools'' is a concise summary of \padsml{} goals.

% As with parser and printer
% combinator libraries, eview generic programming techniques as an
% implementation technique


% are two other closely-related 
% technologies that, like parser or printer generators, might have
% been used as implementation components for parts of the \padsml{} system.
% Indeed, before beginning our implementation we investigated these technologies
% to see whether or not they would apply in our context.  Unfortunately,
% all the work we are aware of is quite specific to Haskell.  For instance,
% Hinze's "Generics for the Masses"~\cite{hinze+: improves upon a number of
% previous efforts because he "only" requires Haskell 98, not Haskell 98
% with extensions.
  
% There are also a number of technical differences in the parsing
% technology that \padsml{} uses (particularly 
% the dependent types, critical for many
% ad hoc data formats but unavailable in Yacc)
% By generating
% parsers and printers together \padsml{} also provides a very convenient
% framework for programmers to implement their own format translators




% It is possible to generate any number of tools from
% a single \padsml{} type because the type is 
% limited to describing properties of the data.
% The descriptions themselves do not contain an 
% algorithmic component similar to the semantic actions found in Yacc
% or the associated functional parameters 
% a The \padsml{} architecture is designed

% A parser or printer combinator library might have been helped
% us implement the specific components of the

% Some of the oldest tools for describing data formats are parser
% generators for compiler construction such as
% Lex and Yacc.  While excellent for parsing programming languages, Lex and Yacc
% are too heavyweight for parsing many of the simpler ad hoc data formats that
% arise in networking, the computational sciences, finance, \etc{}   
% In addition, Lex and Yacc do not support data-dependent parsing, 
% do not generate internal representations automatically, 
% and do not supply a collection of value-added tools such as
% \padsml's \xml{} translator.

% More modern compiler construction tools alleviate
% several of the problems of Lex and Yacc by providing more
% built-in programming support.  For instance,
% the ANTLR parser generator~\cite{antlr} allows the user to add
% annotations to a grammar to direct construction of a parse tree.
% Demeter's class dictionaries~\cite{lieberherr+:class-dictionaries}
% can generate parsers that construct internal parse trees
% as well as traversal functions, much like the traversal functions
% generated by \padsml.  However, these tools do not have dependent
% and polymorphic data descriptions or a formal semantics.  Moreover,
% they were designed for imperative and object-oriented languages 
% as opposed to strongly typed functional languages like \ml{}.

The networking community has developed a number of domain-specific
languages, including PacketTypes~\cite{sigcomm00},
DataScript~\cite{gpce02} and Bro's~\cite{paxson:bro} packet processing
language for parsing and printing binary data.  Like \padsml{}, these
languages use a type-directed approach to describing ad hoc data and
permit the user to define semantic constraints.  In contrast to our
work, these systems handle only binary data and assume the data is
error-free or halt parsing if an error is detected. DFDL is a data
format specification language with an XML-based syntax and type
structure~\cite{dfdl-proposal,dfdl-primer}. At this stage in its
development, it appears DFDL remains a language specification; a tool
generation architecture has not yet been developed. We believe that
DFDL is similar in its expressiveness to \padsc{}.  However, because
the specification is still under development, we cannot give a more
detailed comparison at this point.

A somewhat different class of languages includes
\textsc{ASN.1}~\cite{asn} and \textsc{ASDL}~\cite{asdl}.  Both of
these systems specify the {\em logical\/} in-memory representation of
data and then automatically generate a {\em physical\/} on-disk
representation.  This technology does not help process data that
arrives in predetermined, ad hoc formats.

There are a number of tools designed to deal with converting ad hoc
data formats into \xml{} and various related tasks, including
XSugar~\cite{brabrand+:xsugar2005} and the Binary Format Description
language (BFD)~\cite{bfd}. The scope of both of these projects is
limited to conversion to-and-from \xml{}, and neither seek to provide
programmers with access to the raw data, nor produce a broad suite of
data processing tools.

% There are probably hundreds of tools that one might use if their data were
% in \xml.  However, the point of PADS is to allow scientists whose data is {\em not}
% already in \xml to get work done, particularly when that data contains errors,
% as ad hoc data often does.  Since many processes, machines, programs and other devices
% currently output data and a whole most of

Commercial database products provide support for
parsing data in external formats so the data can be imported into
their database systems, but they typically support a limited number of
formats.  Also, no declarative description of the
original format is exposed to the user for their own use, and they
have fixed methods for coping with erroneous data.  For these reasons,
\padsml{} is complementary to database systems.  

On the theoretical front, the scientific community's understanding of
type-based languages for data description is much less mature.  To the
best of our knowledge, our previous work on the
DDC~\cite{fisher+:next700ddl} was the first to provide a formal
interpretation of dependent types as parsers and to study the
properties of these parsers including error correctness and type
safety.  The current paper extends and improves our earlier work by
simplifying the basic theory in a number of subtle but important ways
and by adding polymorphic types for the purpose of code reuse.
Regular expressions and context-free grammars, the basis for Lex and
Yacc have been well-studied, but they do not have dependency, a key
feature necessary for expressing constraints and parsing ad hoc
scientific data.  {\em Parsing Expression Grammars} (PEGs), studied in
the early seventies~\cite{birman+:parsing}, revitalized more recently
by Ford~\cite{ford:pegs} and implemented using ``packrat parsing''
techniques~\cite{ford:packrat,grimm:packrat}, are somewhat more
similar to \padsml{}'s recursive descent parsers. However, our
multiple interpretations of types in the DDC makes our theory
substantially different from the theory of PEGs.

% {\em PEG difference: not just dependency, but error handling?}


% To our knowledge, we are the first to attempt to specify a semantics for
% data description languages based on types such as \packettypes{},
% \datascript{} or \pads.  
% %Prior to our work, this family of languages 
% %was described informally and by example.  There was no precise
% %connection to formal dependent type theory.

% Of course, there are other formalisms for
% defining parsers, most famously, regular expressions and
% contex-free grammars.  In terms of recognition power,
% these formalisms differ from our type theory
% in that they have nondeterministic choice, but do not have
% dependency or constraints.  We have found that 
% dependency and constraints are absolutely essential for
% describing most of the ad hoc data sources we have studied.
% Perhaps more importantly though, unlike standard theories of
% context-free grammars,
% we do not treat our type theory merely as a recognizer for
% a collection of strings.  Our type-based descriptions 
% define {\em both} external data formats {\em and} 
% rich invariants on %(\ie{} types for)
% the internal parsed data structures.  This dual interpretation
% of types lies at the heart of tools such as \pads, \datascript{} and
% \packettypes{}.  
% %\pads{} programmers, for instance, demand that
% %representations produced by their \pads{} parsers have the expected type and 
% %count on the fact that the associated PD is accurately correlated
% %with the representation.  
% %Existing formalisms simply do not address
% %this elements of data description languages.

% {\em Parsing Expression Grammars} (PEGs),
% studied in the early 70s~\cite{birman+:parsing} and revitalized more 
% recently by Ford~\cite{ford:pegs}, 
% evolved from context-free grammars but
% have deterministic, prioritized choice like \ddc{} as opposed to
% nondeterministic choice.  Though PEGs have syntactic lookahead operators,
% they may be parsed in linear time through the use of
% ``packrat parsing'' techniques~\cite{ford:packrat,grimm:packrat}.
% Once again, the dual interpretation of types in \ddc{} both as
% data descriptions and as classifiers for internal representations
% make our theory substantially different from the theory of PEGs.
% %In practice, PEGs has not been used to parse ad hoc data.

% {\sc antlr}~\cite{antlr}, a popular programming language parsing tool, 
% uses top-down recursive descent
% parsing and appears roughly similar in recognition power to PEGs and \ddc.
% {\sc antlr} also allows programmers to place annotations
% in the grammar definitions to guide construction of an abstract syntax
% tree. However, all nodes in the abstract syntax tree have a 
% single type, hence the guidance is rather crude when compared with
% the richly-typed structures that can be constructed using
% \ddc.


% Practical experience indicates that
% tools based on these formalisms, such as the many variations of
% Lex and Yacc, are highly effective for processing
% programming language syntax.  However, there is also ample evidence
% that these tools are a poor fit for processing
% ad hoc data --- simply put, {\em no one ever uses Lex or Yacc for 
% these tasks}.
% Unfortunately, the nondeterminism and lack of
% dependency in these formalisms limit their suitability to formalizing
% data description languages. While the parsing expression grammars
% (PEG) formalism~\cite{ford:parsing-expression-grammars} is
% significantly closer to the \ddc{}, it too lacks the necessary
% dependency.

% Less related, but still relevant, are Haskell's parsing
% combinators. While these are not a formalism, they do provide an
% elegant manner in which to express parsers. Hence, while we chose to
% define our parsing semantics in the polymorphic lambda calculus,they
% potentialy provide a more elegant alternative.

% % {\em Mention dependent type theory work?}


\subsection{Results from Prior NSF Support}
\label{ssec:results}

\paragraph*{David Walker, PI} (NSF CCR-0238328 CAREER: Programming Languages for Secure and Reliable Component Software
Systems, PI)
The goal of Walker's career award is to develop new programming language
technology that will improve the security and reliability of component software systems.
More specifically, Walker and his students have begun to develop new techniques for defining,
implementing and reasoning about run-time enforcement of program properties.
For example, he developed a rich new theory of security monitors as formal
automata that transform untrusted program behavior as they 
execute~\cite{ligatti+:edit-automata}.
This theory allows security
architects to model a variety of different sorts of run-time
enforcement mechanisms, to prove that certain mechanisms can or cannot
enforce various security properties, and to compare the power of
different classes of security monitors.    Recently, Walker 
has used the theory to prove the surprising new result that powerful run-time
program monitors can enforce certain kinds of liveness properties~\cite{ligatti+:renewal}
and build an implementation for Java~\cite{bauer+:polymer}.

Walker's security monitoring language is a form of aspect-oriented programming language.
In order to better understand aspect-oriented technologies and their
potential impact on security, Walker formalized and proved
safe the {\em first} higher-order, strongly-typed calculus of 
aspects~\cite{walker+:aspects}, built an implementation and extended it with facilities for polymorphic
and type-directed programming~\cite{dantas+:polyaml}.  This calculus defines
both static typing rules and the execution behavior of aspect-oriented
programs.  Consequently, it may
serve as a starting point for analysis of deeper properties of programs.
Recently, he has used the calculus to study the design of a
program analysis that determines the effect of security monitors on
the code they monitor~\cite{dantas+:harmless-advice,dantas+:harmless-popl}.   The analysis
can guarantee that a security monitor does not interfere with the 
original application, which greatly increases a user's incentive
to apply security patches.

To complement his work on run-time monitoring programs, Walker has also
developed several type systems to ensure basic type and memory safety conditions
for low-level programs.  Basic type- and memory-safety guarantees provide a foundation on which
richer security mechanisms can be implemented.  More specifically, he
has extended his earlier work on typed assembly language (TAL)~\cite{morrisett+:tal,morrisett+:journal-stal} with
logic-based type systems that can detect memory errors involving
stack-allocated data~\cite{ahmed+:stack,jia+:stack} and heap or region-allocated
data~\cite{ahmed+:hierarchical-storage}.  In addition to studying memory safety
properties, Walker has shown how to use related type-theoretic and logical techniques
to verify programs~\cite{jia+:ilc} and enforce general software
protocols~\cite{mandelbaum+:refinements}.  

Finally, Walker's career grant allowed him to be a leader in
programming languages and security education. In 2004 and 2005 he organized
a 10-day summer school on software security and reliable computing,
attended by over 100 participants combined~\cite{summerschool04,summerschool05}.  He has also written a
chapter of a new textbook on type systems~\cite{walker:attapl}.  Also in 2005, his
undergraduate research advisee, Rob Simmons, won the Princeton Computer Science Department
Senior Thesis Award.



\newpage
%%%%%%%%%%%%%%%%%%%%%%%%%%%%%%%%%%%%%%%%%%%%%%%%%%%%%%%%%%%%%%%%%%%%%%%%%%%%
\section{References Cited}

{\bibliographystyle{abbrv}
\bibliography{pads-long,pads,galax,padsdave}
}
%{\bibliographystyle{abbrv}
% \small\bibliography{pads}
%} 
\newpage
%%%%%%%%%%%%%%%%%%%%%%%%%%%%%%%%%%%%%%%%%%%%%%%%%%%%%%%%%%%%%%%%%%%%%%%%%%%%
%\section{Biographical Sketch}
%\documentclass[11pt]{article}
        % Use font 11pt Roman.
%\documentstyle{article}

%\def\references{\newpage\section*{References}}
%\newcounter{refcount}
%\def\lit#1{~[\ref{#1}]}
%\def\cit#1#2#3#4{#1\quad#2\quad{\it #3}\quad #4\vskip.15in }
%\def\citb#1#2#3{#1\quad{\it #2}.\quad#3.}
%\def\citu#1#2{#1\quad#2.}


\newcommand{\comment}[1]{}


% No header or foot; top 2.0in, left 0.6in, right 0.9in, bottom 0.5in

%\pagestyle{empty}
%\topmargin=-.5in
%\textheight=8.8in
%\leftmargin=-1.0in
%\oddsidemargin=.25in
%\oddsidemargin=.25in
%\textwidth=6.2in

\setlength{\textheight}{22.8true cm}
%\setlength{\textwidth}{15.4true cm}
\setlength{\textwidth}{15.9true cm}
\setlength{\oddsidemargin}{.55true cm}
\setlength{\evensidemargin}{.55true cm}
\setlength{\topmargin}{-1.0cm}
\pagestyle{empty}

\newlength{\oldparindent}
\setlength{\oldparindent}{\parindent}
\setlength{\parindent}{0pt}

% No indentation at the beginning of paragraphs.

% Don't hyphenate.
%\hyphenpenalty=10000


% Rsection: a resume section environment.
\newenvironment{Rsection}[1]{
  \noindent{\bf #1}
  \begin{list}{}{\topsep=-\parskip
                 \leftmargin=-0.0in}
    \item[]
}{
  \end{list}
%  \vspace \baselineskip
}
                           
% Section: a resume section environment.
\newenvironment{Section}[1]{
  \centerline{\bf #1}
  \begin{list}{}{\topsep=-\parskip
                 \leftmargin=-0.5in}
    \item[]
}{
  \end{list}
%  \vspace \baselineskip
}

% edlist: a resume education list environment.
\newenvironment{edlist}{
  \begin{list}{}{\topsep=-\parskip
                 \itemsep=0.5\baselineskip
                 \leftmargin=0.25in}
}{
  \end{list}
}


% edlist2: a resume education list environment.
\newenvironment{edlist2}{
  \begin{list}{}{\topsep=-\parskip
                 \itemsep=0.5\baselineskip
                 \leftmargin=0.25in
                 \setlength{\rightmargin}{\leftmargin}
                 \textwidth=4in}
}{
  \end{list}
}
                           

% honorlist: a resume honor list environment.
\newenvironment{honor2list}{
  \begin{list}{}{\topsep=-\parskip
                 \itemindent=-0.in
                 \itemsep=0.0\baselineskip
                 \leftmargin=.25in}
}{
  \end{list}
}  

\newenvironment{honor2enum}{
  \begin{enumerate}{}{\topsep=-\parskip
                 \itemindent=-0.in
                 \itemsep=0.0\baselineskip
                 \leftmargin=.25in}
}{
  \end{enumerate}
}  

\newenvironment{honor3enum}{
  \begin{enumerate}{}{\topsep=-\parskip
                 \itemindent=-0in
                 \itemsep=0.0\baselineskip
                 \leftmargin=.25in}
}{
  \end{enumerate}
}  



% honorlist: a resume honor list environment.
\newenvironment{honorlist}{
  \begin{list}{}{\topsep=-\parskip
                 \itemindent=-0.in
                 \itemsep=0.2\baselineskip
                 \leftmargin=.25in}
}{
  \end{list}
}  


% joblist: a resume job list environment.
\newenvironment{joblist}{
  \begin{list}{}{\topsep=-\parskip
                 \itemsep=0.5\baselineskip
                 \leftmargin=0.5in}
}{
  \end{list}
}

                                      
% job: a job entry for a joblist environment
\newcommand{\job}[3]{
  \item[]
  \begin{tabbing}
    \hspace{-0.5in}\=\hspace{1.5in}\=\kill
    \>    {\it #1:}       \> #2   \\
    \>                    \> #3   \\
  \end{tabbing}
  \vspace{-\baselineskip}
}

              
%%%%%%%%%%%%%%%%%%%%%%%%%%%%%%%%%%%%%%%%%%%%%%%%%%%%%%%%%%%%%%%%%%%%%%%%%%%%%%%




%%%%%%%%%%%%%%%%%%%%%%%%%%%%%%%%%%%%%%%%%%%%%%%%%%%%%%%%%%%%%%%%%

%\leftmargin=-1in
\begin{document}


\centerline{\Large \bf {\LARGE \bf D}AVID {\LARGE \bf W}ALKER}
\vskip .3in

\normalsize
\it
\begin{tabular}{@{\hspace{0in}}l@{\hspace{1.4in}}l}
35 Olden Street   & Phone: (609) 258--7654  \\
Princeton University   & Email: {\rm dpw@cs.princeton.edu} \\
Princeton, NJ 08544

\end{tabular}
\rm
\vskip.3in

\begin{Rsection}{\Large \bf {Professional Preparation}}
\vskip.1in
\begin{edlist}

\item {\bf Queens University, Kingston, Ontario, Canada }
  \begin{honor2list}
  \item {Computer Science, B.Sc. Honors, 1995.}  

  \end{honor2list}
\item {\bf Cornell University}
\begin{honor2list}
\item {Computer Science, M.S. 1999.}
\item {Computer Science, Ph.D. 2001.}
\end{honor2list}
\end{edlist}
\end{Rsection}


\vskip.3in
\begin{Rsection}{\Large \bf {Appointments}}
\vskip.1in
\begin{edlist}
\item {\bf Princeton University}\\
{\bf  Department of Computer Science} \\
%\vskip .05in
\vspace{-1ex}
\begin{honor2list}
\item Assistant Professor, Feb. 2002--present.
\end{honor2list}

\item {\bf Carnegie Mellon University}\\
{\bf  Department of Computer Science} \\ 
%\vskip .05in
\vspace{-1ex}
\begin{honor2list}
\item Postdoctoral Fellow, Oct. 2000--Oct. 2001
\end{honor2list}


%\newpage
\end{edlist}
\end{Rsection}


\vskip.3in
\begin{Rsection}{{\Large \bf {Selected Publications}} 
}
\vskip.1in
\begin{edlist}

\item {\bf Selected papers of greatest relevance (in chronological order)}
\begin{honor2enum}
\vskip .1in

\item Harmless advice.  Daniel S. Dantas and David Walker.
To appear in Workshop on
Foundations of Object-oriented
programming.  January 2005.

\item Edit Automata: Enforcement Mechanisms for Run-time Security Policies.  Jay Ligatti, Lujo Bauer and David Walker.  
International Journal of Information Security. 2004

\item A Theory of Aspects.  David Walker, Steve Zdancewic and Jay Ligatti.  In the ACM SIGPLAN International Conference on Functional Programming, August 2003. 

\item Types and Effects for Non-interfering Program Monitors.  
Lujo Bauer, Jarred Ligatti and David Walker.  
International Symposium on Software Security.  Tokyo, November, 2002.  

\item A Type System for Expressive Security Policies.  David Walker. 
Twenty-Seventh ACM SIGPLAN Symposium on Principles of Programming Languages. 
pages 254-267, Boston, January 2000

%


\end{honor2enum}
\item {\bf Selected other publications (in chronological order):}
\begin{honor2enum}

\item Substructural type systems.  David Walker.
Chapter 1 of Benjamin Pierce, ed.,
Advanced Topics in Type Systems for Programming Language.  
To appear January 2005.

\item An Effective Theory of Type Refinements.  Yitzhak Mandelbaum, David Walker and Robert Harper.  In the ACM SIGPLAN International Conference on Functional Programming, August 2003. 

\item 
Greg Morrisett, Karl Crary, Neal Glew, and David Walker. Stack-based
Typed Assembly Language.  Journal on Functional Programming,
12(1):43-88, January 2002.

\item 
David Walker, Karl Crary, and Greg Morrisett. Typed Memory Management
via Static Capabilities. ACM Transactions on Programming Languages and
Systems, 22(4):701-771, July 2000.   A previous version of this paper
appeared in the 26th ACM Symposium on Principles of Programming Languages.

\item 
Greg Morrisett, David Walker, Karl Crary, and Neal Glew. From System F
to Typed Assembly Language. ACM Transactions on Programming Languages
and Systems, 21(3):528-569, May 1999.  A previous version of this paper
appeared in the 25th ACM Symposium on Principles of Programming Languages.


\end{honor2enum}

\end{edlist}
\end{Rsection}



\vskip.3in
\begin{Rsection}{\Large \bf {Synergistic Activities}}
\begin{honor2list}
\vskip .1in
\item Program Chair, Workshop on Foundations of Aspect-Oriented
programming languages.  March 2005.
\item Alfred P. Sloan Fellow.  Sept 2004-Sept 2006. 
\item Program Committee,  ACM SIGPLAN International Conference
on Functional Programming.  September 2004.
\item Program Committee,  Workshop on Foundations of Aspect-Oriented
programming languages.  March 2004.
\item ACM SIGPLAN Summer School on Software Security.  Chair.  June 2004.
\item 
CAREER:  Programming Languages for Secure and Reliable Component Software Systems.  NSF Career Grant, CCR-0238328,  PI.  July 2003-June 2008.  
\item Publicity chair and executive committee for ACM Symposium on
Principles of Programming Languages, 2003--2006.




\end{honor2list}
\end{Rsection}
\vskip.3in
\begin{Rsection}{\Large \bf {Collaborators and Other Affiliations}}
\begin{itemize}
\vskip .1in
\item Collaborators: 
Amal Ahmed (Harvard University),
%Andrew Appel (Princeton University),
Zena Ariola (University of Oregon),
Lujo Bauer (Carnegie Mellon University),
Iliano Cervesato,
Karl Crary (Carnegie Mellon University),
%Daniel Dantas (Princeton University),
%Ed Felten (Princeton University),
Kathleen Fisher (AT\&T),
Neal Glew (Intel),
Daniel Grossman (University of Washington),
Robert Harper (Carnegie Mellon University),
%Limin Jia (Princeton University),
%Jarred Ligatti (Princeton University),
%Yitzhak Mandelbaum (Princeton University),
Greg Morrisett (Harvard University),
Benjamin Pierce (University of Pennsylvania),
Frank Pfenning (Carnegie Mellon University),
Fred Smith (MathWorks),
%Frances Spalding (Princeton University),
%Daniel Wang (Princeton University),
Geoff Washburn (University of Pennsylvania),
Kevin Watkins (Carnegie Mellon University),
Stephanie Weirich (University of Pennsylvania),
Steve Zdancewic (University of Pennsylvania)

\item Graduate (Ph.D.) Advisor: 
Greg Morrisett (Cornell University, now Harvard University)
\item Thesis Advisees (graduated): none
\item Thesis Advisees (current): Dan Dantas, Limin Jia, Jarred Ligatti, 
Yitzhak Mandelbaum, Frances Spalding
\end{itemize}
\end{Rsection}



% \iffalse
% \vskip.3in
% \begin{Rsection}{\Large \bf {Selected Awards and Fellowships}}
% \begin{itemize}
% \vskip .1in
% \end{itemize}
% \end{Rsection}
% \fi




%\newpage

%\newpage





%%%%%$$^{\ref{fn:second}}$







%\vskip.3in




\end{document}



%They are in separate files now (see cv/lastname-cv.tex).

%%%%%%%%%%%%%%%%%%%%%%%%%%%%%%%%%%%%%%%%%%%%%%%%%%%%%%%%%%%%%%%%%%%%%%%%%%%%
%\section{Budget}

%The budget pages (no longer necessary because of fastlane). 

%%%%%%%%%%%%%%%%%%%%%%%%%%%%%%%%%%%%%%%%%%%%%%%%%%%%%%%%%%%%%%%%%%%%%%%%%%%%
%\section{Current and Pending Support (NSF Form 1239)}

%The current-and-pending pages (no longer necessary because of fastlane). 

%%%%%%%%%%%%%%%%%%%%%%%%%%%%%%%%%%%%%%%%%%%%%%%%%%%%%%%%%%%%%%%%%%%%%%%%%%%%
%\section{Facilities, Equipment and Other Resources (NSF Form 1363)}

%The facility page (no longer necessary because of fastlane).

%%%%%%%%%%%%%%%%%%%%%%%%%%%%%%%%%%%%%%%%%%%%%%%%%%%%%%%%%%%%%%%%%%%%%%%%%%%%
%\section{Special Information and Supplementary Documentation}

%We may consider asking AT\&{}T to provide a support letter.

\end{document}



\documentclass{letter}
\usepackage{suresh-attletter-belanger}
\usepackage{times}
\input{moby-defs.tex}
\signature{David Belanger\\
           Vice President-Information, Software, and Systems Research\\
           Chief Scientist}
\begin{document}
\begin{letter}{
}

\opening{Dear NSF Program Officer,} 
AT\&T Labs-Research is committed to participating in the research initiative set forth in David Walker�s (Princeton University) CyberTrust proposal (NSF solicitation 05-518) on �Tools for Processing Ad Hoc Data Sources.�

One of the biggest challenges in the telecommunications industry is effectively manipulating the vast quan- tities of data associated with running the various networks. Properly understanding this data is of critical importance, both to detect fraud and to monitor the performance of these networks for outages, denial of service attacks, provisioning errors, etc. These tasks are complicated by legacy and ad hoc data formats, the volume of data to be processed, poor documentation, and errors in the data. The research described in this proposal addresses each of these problems by providing a declarative description language for describing ad hoc data sources as they are and a system that converts such descriptions into an efficient suite of tools for understanding and transforming the associated data in a principled way.

The work outlined is quite challenging in terms of design, implementation, and theory. At the design level, the question is how to provide simply and concisely sufficient expressiveness to permit analysts to describe the many unruly formats they encounter in practice. Implementation-wise, the challenge is to produce powerful but efficient tools that scale to processing gigabytes worth of data. Theoretically, the goal is to establish precise semantics for the data description language and the associated tools so the system can guarantee that the output of the tools is consistent with the semantic properties asserted in the data description. The semantically sound, efficient tools produced by this project will provide a strong platform upon which to build novel applications for processing ad hoc data. The research described in this proposal is a wonderful example of using an innovative language design to solve a critical industry problem.

AT\&T Labs-Research commits to supporting this activity should NSF fund the proposal. Kathleen Fisher, a senior researcher, will fully and actively participate in the research at no cost to NSF, including project meetings and graduate student supervision. AT\&T Labs-Research will where appropriate also make available experimental data to be used to evaluate the project. Finally, we will explore supporting summer internship opportunities at AT\&T Labs-Research for graduate students involved in the project. All these add up to a substantial commitment of AT\&T resources, and indicate our enthusiasm for the work outlined in this proposal.

\closing{Yours Sincerely,}

\end{letter}

\end{document}



%%%%%%%%%%%%%%%%%%%%%%%%%%%%%%
% \centerline{
% \begin{tabular}{cc}
%           \\[-1ex]
% \multicolumn{2}{c}{Automatic Tool Generation for Ad Hoc Scientific Data} \\
%           & \\[-1ex]
% David Walker (PI)\\
% Princeton University\\[2ex]
% \end{tabular}}
%%%%%%%%%%%%%%%%%%%%%%%%%%%%%%

\paragraph*{Intellectual Merits:} 
In every scientific discipline, researchers are digitizing their knowledge and
beginning to use computational methods to categorize,
query, filter, search, diagnose, and visualize their data.  
While this effort is leading to remarkable advances,
it is also generating enormous amounts of {\em ad hoc data}.
Ad hoc data is any data for which standard data processing tools
such as query engines, statistical packages, graphing tools, parsers, printers,
transformers or programming libraries are not readily available.
This data, which is often unpredictable, poorly documented,
filled with errors, high volume and unwieldy,
poses tremendous challenges to its users and the software
that manipulates it.  We cannot maximize the productivity of top 
computational scientists unless we can maximize the efficiency and 
accuracy with which they deal with this data.  Hence, the overall goal of
our research is to alleviate the burden, risk and confusion
associated with ad hoc data.  Our overall strategy is (1) to develop a
specification language capable of precisely describing any ad hoc data
format at an easy-to-understand, high level of abstraction and (2) to
automatically generate useful data processing tools including
programming libraries, a query engine, format converters, a
statistical summarizer, a histogram generator and others.
We will accomplish our task by building upon our preliminary data
description and processing system, \pads{}, created by the PI and his
collaborators.  This preliminary system has demonstrated
there is much research to do in this area.  In particular, if funded we 
will to pursue research on three fronts:

\begin{enumerate}
\item Extend the preliminary system with new features we have discovered are indispensible to
scientists who work with ad hoc data.  In particular, the preliminary system only operates
on single files.  We will extend it so it may operate over and integrate collections of files,
either local or distributed across a network.  We will also augment PADS so that
programmers may write simple, high level and easy-to-understand PADS transforms that can fix data errors
and filter, standardize, coerce, compress, or santize data fields.
\item  Improve automatic tool generation.  We will research new tool-generation architectures
that allow anyone to add new tools to the tool-set, which is currently is currently fixed.
We will also provide mechanisms that allow data descriptions to communicate useful semantic information
to downstream tools.
\item Develop the semantic theory of PADS.  We have a partial PADS parsing semantics based on dependent type
theory but we will extend this semantics to cover our new features and generalize it so it may
describe other components of the systems including printers and other tools.
\end{enumerate}

\paragraph*{Broader Impacts:}  We are collaborating with Kathleen Fisher
and Mary Fernandez at
AT\&T who will be able to use our tools to address real problems such
as telephone fraud detection.  In addition, our tools will be freely
available to researchers and scientists over the web.  Moreover, part
of our mission will be to work with specific biologists and
physicists at Princeton and the broader community to help them with
their data processing needs.  We have already been meeting with Olga
Troyanskaya, who works in Princeton's Lewis-Sigler Institute for
Integrative Genomics on pathway modeling and analysis of
protein-protein interactions, and with Rachel Mandelbaum, Ph.D. candidate
in physics who analyzes cosmology data.  Many of our proposed extensions to
\pads{} were inspired directly by their data processing needs.  
The collaboration between Computer Science and Genomics will also be an
excellent platform for developing interdisciplinary undergraduate
research projects.  The PI has a proven track-record 
for following through with undergraduate and graduate
educational plans.  Last year, for instance, his undergraduate student advisee, Rob Simmons,
won the Princeton Computer Science Senior Thesis Award.


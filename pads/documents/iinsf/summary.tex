%%%%%%%%%%%%%%%%%%%%%%%%%%%%%%
% \centerline{
% \begin{tabular}{cc}
%           \\[-1ex]
% \multicolumn{2}{c}{Automatic Tool Generation for Ad Hoc Scientific Data} \\
%           & \\[-1ex]
% David Walker (PI)\\
% Princeton University\\[2ex]
% \end{tabular}}
%%%%%%%%%%%%%%%%%%%%%%%%%%%%%%

\paragraph*{Intellectual Merits:} 
In every scientific discipline, researchers are digitizing their knowledge and
beginning to use computational methods to categorize,
query, filter, search, diagnose, and visualize their data.  
While this effort is leading to remarkable advances,
it is also generating enormous amounts of {\em ad hoc data}.
Ad hoc data is any data for which standard data processing tools
such as query engines, statistical packages, graphing tools, parsers, printers,
transformers or programming libraries are not readily available.
This data, which is often unpredictable, poorly documented,
filled with errors, high volume and unwieldy,
poses tremendous challenges to its users and the software
that manipulates it.  We cannot maximize the productivity of top 
computational scientists unless we can maximize the efficiency and 
accuracy with which they deal with this data.

Our overall goal is to alleviate the burden, risk and confusion
associated with ad hoc data.  Our overall strategy is (1) to develop a
specification language capable of precisely describing any ad hoc data
format at an easy-to-understand, high level of abstraction and (2) to
automatically generate useful data processing tools including
programming libraries, a query engine, format converters, a
statistical summarizer, a histogram generator and others.
 
We will accomplish our task by building upon our preliminary data
description and processing system, \pads{}, created by the PI and his
collaborators.  This preliminary system demonstrates the basic
feasibility of the \pads{} approach, but there is much research to do.
While the work to date demonstrates the feasibility of the \pads{}
approach, the \pads{} design and implementation are still in their
infancy: We have discovered many common, real-world data formats that
the current PADS infrastructure is incapable of describing, parsing or
analyzing.  To address these deficiencies, we propose to explore
innovative ways of extending the basic \pads{} specification language
and related infrastructure.  The second thrust of our research
involves improving the automatic tool generation infrastructure in the
\pads{} system.  The current tool generation system is brittle,
inflexible and suffers from subpar performance.  We will develop a
novel extension to \pads{} that allows users to augment \pads{}
descriptions with application-specific attributes and customizations.
The attributes convey semantic information to the tool generators and
the customizations improve error-handling and performance.  Our new
architecture has the promise of being substantially more robust,
flexible and efficient.  Finally, in order to improve the reliability
of the infrastructure, we will formalize \pads{} and prove important
correctness properties of our formal model.  Overall, this proposal
involves challenging research in language design, efficient systems
implementation and semantic analysis, all aimed at solving real-world
data processing problems.

\paragraph*{Broader Impacts:}  We are collaborating with Kathleen Fisher
and Mary Fernandez at
AT\&T who will be able to use our tools to address real problems such
as telephone fraud detection.  In addition, our tools will be freely
available to researchers and scientists over the web.  Moreover, part
of our mission will be to work with specific biologists and
physicists at Princeton and the broader community to help them with
their data processing needs.  We have already been meeting with Olga
Troyanskaya, who works in Princeton's Lewis-Sigler Institute for
Integrative Genomics on pathway modeling and analysis of
protein-protein interactions, and with Rachel Mandelbaum, Ph.D. candidate
in physics who analyzes cosmology data.  Many of our proposed extensions to
\pads{} were inspired directly by their data processing needs.  

The collaboration between Computer Science and Genomics will also be an
excellent platform for developing interdisciplinary undergraduate
research projects.  The PI has a proven track-record 
for following through with undergraduate and graduate
educational plans:  Last year, his undergraduate student advisee, Rob Simmons,
won the Princeton Computer Science Senior Thesis Award;
last year and the year before he organized two NSF-sponsored summer schools
on secure and reliable computing.


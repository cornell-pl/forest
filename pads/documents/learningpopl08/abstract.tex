An {\em ad hoc data source} is any semistructured data source for
which useful data analysis and transformation tools have not yet been
defined.  Such data must be queried, transformed and displayed by
systems administrators, computational biologists, financial analysts
and hosts of others on a regular basis.  The job of doing so is
normally very tedious and quite time-consuming.  Consequently, we have
developed a new framework to vastly improve the productivity of data
analysts and thereby cut the cost of ubiquitous data processing tasks.
More specifically, we demonstrate it is possible to generate a suite
of useful data processing tools, including a semi-structured query
engine, an xml-converter, a statistical analyzer and data
visualization routines directly from the ad hoc data itself, often
without any human intervention.  

The key technical contribution of the work is a multi-phase algorithm
that automatically infers the structure of an ad hoc data source and
produces a format specification in the \pads{} data description
language.  These descriptions can then be used to generate printing
and parser libraries by a programmer wishing to implement custom data
transformations, or, alternatively, our software infrastructure will
push these descriptions through the \pads{} compiler and automatically
generate fully functional tools that are easily invoked on the
commandline.

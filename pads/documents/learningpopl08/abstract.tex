An {\em ad hoc data source} is any semistructured data source for
which useful data analysis and transformation tools are not 
readily available.  Such data must be queried, transformed and displayed by
systems administrators, computational biologists, financial analysts
and hosts of others on a regular basis.  These tasks are
normally tedious and time-consuming.  To improve this situation, we have
developed a new framework to vastly improve the productivity of data
analysts and thereby cut the cost of ubiquitous data processing tasks.
More specifically, we demonstrate that it is possible to generate a suite
of useful data processing tools, including a semi-structured query
engine, an xml-converter, a statistical analyzer and data
visualization routines directly from the ad hoc data itself, often
without any human intervention.  

The key technical contribution of the work is a multi-phase algorithm
that automatically infers the structure of an ad hoc data source and
produces a format specification in the \pads{} data description
language.  Programmers wishing to implement custom data analysis tools
can use such descriptions to generate printing and parsing libraries
for the data.  Alternatively,  our software infrastructure will
push these descriptions through the \pads{} compiler and automatically
generate fully functional tools that are easily invoked on the
commandline.

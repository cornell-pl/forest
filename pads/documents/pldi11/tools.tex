\section{Tools}
\label{sec:tools}
Third-party developers can use type-directed programming techniques~\cite{??} to
generate tools that will work for any file system structure that has a
\forest{} description.  As a proof of concept, we have written a
number of such tools, which we describe in this section.  We simulated
being third-party users by not changing the code of the \forest{}
compiler to build any of these tools.

\subsection{Graph View}
The first tool, which we call \fg{}, generates a graphical
representation of any directory structure that matches a 
\forest{} specification.  As an example of \fg{}'s
output, we generated the graph in \figref{fig:coral-pic} using 
the tool.  This graph uses boxes to denote directories and ovals to
denote files, with borders of varying thickness to distinguish between
ASCII and binary files.  The borders of file components that are
symbolic links are drawn as dashed lines.  Components that have errors
are drawn in red. 

The core functionality of \fg{} lies in a Haskell function
called \texttt{mdToPDF}:

\begin{code}
mdToPDF :: ForestMD md => 
     md -> FilePath -> IO (Either String FilePath)
\end{code}
This type indicates the function takes a value having any \forest{}
meta-data type

In this type, type variable \texttt{md} denotes any type belonging to
the \texttt{ForestMD} type class, which means that the type
\texttt{md} is a generated \forest{} meta-data type. 

This function takes as input any value belonging to the
\texttt{ForestMD} type class, (\ie, any value of type \texttt{md}
where \texttt{md} belongs 
which intuitively means \texttt{}

\begin{itemize}
\item pretty printer
\item Graph representation giving status
\item Permissions checker
\item Shell tools: 
\end{itemize}

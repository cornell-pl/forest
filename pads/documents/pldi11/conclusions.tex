\section{Conclusions}
\label{sec:conclusion}
In this paper, we present the design of \forest{}, an embedded
domain-specific language for describing \filestores{}.  A \forest{}
description concisely specifies a collection of files, directories,
and symbolic links as well as expected file system attributes such as
owners and permissions.  From a description, the \forest{} compiler
generates code to lazily load the on-disk data into an isomorphic
in-memory representation, lowering the divide between on-disk and
in-memory data.  
%The generated metadata identifies errors in the
%\filestore{}.  
\forest{} also generates type class instances that make it 
easy for third-party tool developers to use
Haskell's generic programming infrastructure.  We have used this
infrastructure ourselves to define a number of useful tools. 
In addition, the language has a formal semantics based on classical tree logics and is
fully implemented.  On the latter point, our work 
serves as extensive case study in domain-specific
language design, and, as such, has inspired changes in the design of
\template{}.  We anticipate releasing the 
source code for \forest{} shortly.

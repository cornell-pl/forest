Many users and many applications use part of the file system as a
low-overhead database or persistent store.  The correct behavior of
such applications relies on the collection of files having the
expected hierarchical organization and meta-properties such as file
ownership, permissions, and timestamps.  Although expedient, this
approach is tedious and error prone.  Currently, there is little
support for documenting or checking such assumptions, and loading
information from disk robustly requires a lot of boring code.
\forest{} is a new \haskell{}-based embedded domain-specific language
designed to address this problem.  \forest{} descriptions allow users
to declaratively specify the portion of the file system relevant to a
particular use using a type-based metaphor.  Given such a
specification, the \forest{} compiler can automatically generate tools
to detect whether the relevant portion of the file system conforms to
the specification and to load the corresponding data into \haskell{}
data structures.  This design greatly lowers the divide between
on-disk and in-memory representations of data.  \haskell{}'s support
for generic programming makes it is easy to write third-party tools
for manipulating these application-specific file system fragments.  We
demonstrate this ability by defining tools to query meta-data, detect
errors, visualize on-disk data, and mimic a variety of shell tools
with application-specific versions.  We demonstrate that the language
design is both general and useful by applying it to a number of real
data sets and programming tasks.  Finally, we formalize the core
elements of the language as a simple calculus inspired by classical
tree logics. \forest{} is fully implemented. 



One of the most pressing modern programming language problems involves
designing mechanisms that provide programmers with immediate,
low-cost, easy-to-understand data access and data management tools.
This paper tackles one crucial aspect of this problem: access to
structured file systems.  It does so through the conception, design
and implementation of a new programming language called \forest.
\forest{} is an embedded domain-specific language in \haskell{} that
allows programmers to specify the structure of file system fragments
using a form of extended type declaration.  These specifications
include complete information about hierarchical and data-dependent
directory structure, directory unfolding (untarring), file attributes,
file compression, and file contents.  Once programmers have defined a
file system specification, they may apply it to a file system fragment
and read or analyze the file system contents as if it were any other
strongly-typed data structure within their program.  In addition to
providing ordinary programming support, the \forest{} system also
generates a range of useful end-to-end tools for managing file system
fragments including tools for error-detection, for visualization and
debugging, for permission verification and for scripting.  For the
latter application, \forest{} generates a suite of
specification-directed analogues of conventional Unix-style commands
such as {\tt ls}, {\tt grep}, {\tt tar}, and {\tt cp}.  \forest{} is
also tightly connected to generic programming support available in
\haskell{} and automatically generates the type class instance
declarations necessary to perform generic traversals, maps, folds or
queries.  Consequently, users may design new, domain-specific generic
tools quickly and easily.  We demonstrate that the language design is
both general and useful by applying it to a number of real data sets
and programming tasks.  We formalize the core elements of the
language as a simple calculus inspired by classical tree logics.


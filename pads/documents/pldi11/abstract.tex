Many users and many applications use part of the file system as a
low-overhead database or persistent store.  The correct behavior of
such applications relies on the collection of files having the
expected hierarchical organization and meta-data such as file
ownership, permissions, and timestamps.  Although expedient, this
approach is tedious and error prone.  Currently, there is little
support for documenting or checking such assumptions, and loading
information from disk robustly requires a lot of boring code.
\forest{} is a new \haskell{}-based embedded domain-specific language
designed to address this problem.  \forest{} descriptions use a
type-based metaphor to declaratively specify the portion of the file
system relevant to a particular use.  By specifying the root of a
given application's data, programmers can use \forest{} to seamlessly
connect their on-disk data to an isomorphic in-memory representation
that they can manipulate as if it were any other strongly-typed data
structure in their program.  \forest{} generates meta-data that
makes it easy to detect to what extent the relevant portion of the
file system conforms to the specification.  This design greatly lowers
the divide between on-disk and in-memory representations of data.
\haskell{}'s support for generic programming makes it easy to write
third-party tools for manipulating these application-specific file
system fragments.  We demonstrate this ability by defining tools to
query meta-data, visualize on-disk data, detect errors, and mimic a
variety of shell tools with application-specific versions.  We
demonstrate that the language design is both general and useful by
applying it to a number of real data sets and programming tasks.
Finally, we formalize the core elements of the language as a simple
calculus inspired by classical tree logics. \forest{} is fully
implemented.

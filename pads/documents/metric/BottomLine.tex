\newpage
\section{Bottom Line}
\label{sec:BottomLine}

\subsection{Version 1 complexity metric}

The PADS types that will be supported by the complexity metric in
version 1 are listed in table \ref{tab:v1Complexity}. The table
specifies the information that is needed for some of the PADS types.

\begin{longtable}{||l|l|l|}
\caption[Version 1 complexity metric]{Version 1 complexity metric}
\label{tab:v1Complexity}
\\\hline
\hline
PADS type & Length distribution & Value distribution \\\hline\hline

Base type &
\multicolumn{2}{l}{
\parbox[t]{11cm}{
The list of values of the base type is stored in the abstract
syntax tree resulting from the parse of the data. From this
list of values we can derive a length distribution and/or
a value distribution.
\vspace{0.5mm}}} \vline\\\hline

\textbf{Punion} &
\parbox[t]{5cm}{
A histogram of frequencies versus variant in the \textbf{Punion}.
\vspace{0.5mm}} &
\parbox[t]{6cm}{
Each subtype of the union will have its own distribution information
associated with it.
\vspace{0.5mm}} \\\hline

\textbf{Pstruct} &
\parbox[t]{5cm}{
No additional information at this time.
\vspace{0.5mm}} &
\parbox[t]{6cm}{
No additional information at this time.
\vspace{0.5mm}} \\\hline

\textbf{Pre} &
\parbox[t]{5cm}{
Not supported at this time
\vspace{0.5mm}} &
\parbox[t]{6cm}{
Not supported at this time
\vspace{0.5mm}} \\\hline

\textbf{Parray} &
\parbox[t]{5cm}{
A histogram of length of \textbf{Parray} versus frequency.
\vspace{0.5mm}} &
\parbox[t]{6cm}{
The element type will have its own distribution information
associated with it.
\vspace{0.5mm}}

\\\hline
\end{longtable}

\subsection{Version 2 complexity metric}

This section describes the complexity metric as coded during February,
2007. The metric is a function called \textsf{measure} having type
\textsf{Ty} $\rightarrow$ \textsf{Ty}. The function fills in complexity
information in the \textsf{AuxInfo} associated with each node of the
\textsf{Ty} structure. Two fields have been added to the \textsf{AuxInfo} type
to support the complexity metric:

\begin{verbatim}
    type AuxInfo = { coverage : int
                   , label    : Id option
                   , typeComp : Complexity
                   , dataComp : Complexity
                   }
\end{verbatim}

The \textsf{typeComp} field records the complexity of the type itself,
regardless of the data scanned. The \textsf{dataComp} field records
the complexity of the data as viewed through the type.

\newpage
\begin{longtable}{||l|l|l|l|}
\caption[Base types]{Base types}
\label{tab:v2Base}
\\\hline
\hline
Sub type & Type complexity & Data complexity & Notes \\\hline\hline

\textsf{Ptime} &
      $\left(\frac{1}{60}\right) \cdot \left(\frac{1}{60}\right) \cdot \left(\frac{1}{24}\right)$ &
      $\left(\frac{1}{60}\right) \cdot \left(\frac{1}{60}\right) \cdot \left(\frac{1}{24}\right)$ &
\parbox[t]{5cm}{
This is a much oversimplified case. In order to do better, we need to
have a regular expression identified with the time, and analyze the
complexity of the regular expression. See section
\ref{sec:RegularExpressions}.
\vspace{0.5mm}} \\\hline

\textsf{Pmonth} & $\left(\frac{1}{12}\right)$ & $\left(\frac{1}{12}\right)$ &
\parbox[t]{5cm}{
Again an oversimplification of month, need also regular expressions here
eventually.
\vspace{0.5mm}} \\\hline

\textsf{Pip} &
    $\left(\frac{1}{255}\right)$ &
    $\left(\frac{1}{255}\right)^\mathrm{maxlen}$
\footnote{$\mathrm{maxlen}$ is the maximum length of the token address seen} &
\parbox[t]{5cm}{
Perhaps we should use average length? Maximum length gives us a worst
case complexity. On the other hand, IP address are often all of the same
length, namely four number in the range 0 through 255.
\vspace{0.5mm}} \\\hline

\textsf{Pint} &
    $\left(\frac{1}{10}\right)$ &
    $\left(\frac{1}{10}\right)^\mathrm{maxlen}$ &
\parbox[t]{5cm}{
This assumes integers are a sequence of digits 0 \ldots 9.
\vspace{0.5mm}} \\\hline

\textsf{Pstring} &
    $\left(\frac{1}{68}\right)$ &
    $\left(\frac{1}{68}\right)^\mathrm{maxlen}$ &
\parbox[t]{5cm}{
There are 68 characters that make up strings.
\vspace{0.5mm}} \\\hline

\textsf{Pgroup} & 0 & 0 &
\parbox[t]{5cm}{
Don't know how to handle this one yet.
\vspace{0.5mm}} \\\hline

\textsf{Pwhite} & $\frac{1}{2}$ & $\left(\frac{1}{2}\right)^\mathrm{maxlen}$ &
\parbox[t]{5cm}{
White space can have either blank or tab characters (2 possibilities).
\vspace{0.5mm}} \\\hline

\textsf{Other} & $\frac{1}{256}$ & $\frac{1}{256}$ &
\parbox[t]{5cm}{
A single ASCII character.
\vspace{0.5mm}} \\\hline

\textsf{Pempty} & 0 & 0 &
\parbox[t]{5cm}{
Not sure about this one.
\vspace{0.5mm}} \\\hline

\textsf{Error} & \textsf{impossible} & \textsf{impossible} &
\parbox[t]{5cm}{
Not sure about this one.
\vspace{0.5mm}} \\\hline

\textsf{PbXML} &
      $\left(\frac{1}{52}\right)$ &
      $\left(\frac{1}{52}\right)^\mathrm{maxlen} $ &
\parbox[t]{5cm}{
XML seems to handled as a fancy string type. There are 52 possible XML
characters.  Should consider sum of two string lengths?
\vspace{0.5mm}} \\\hline

\textsf{PeXML} &
      $\left(\frac{1}{52}\right)$ &
      $\left(\frac{1}{52}\right)^\mathrm{maxlen}$ &
\parbox[t]{5cm}{
XML seems to handled as a fancy string type. There are 52 possible XML
characters.  Should consider sum of two string lengths?
\vspace{0.5mm}} \\\hline

\end{longtable}

\begin{longtable}{||l|l|l|l|}
\caption[Refined base types]{Refined base types}
\label{tab:v2RefinedBase}
\\\hline
\hline
Type & Type complexity & Data complexity & Notes \\\hline\hline

\textsf{StringME} &
    $\left(\frac{1}{68}\right)$ &
    $\left(\frac{1}{68}\right)^\mathrm{maxlen}$ &
\parbox[t]{5cm}{
There are 68 characters that make up strings.
\vspace{0.5mm}} \\\hline

\textsf{Int} &
    $\log (\mathrm{max} - \mathrm{min} + 1)$ &
    $\log (\mathrm{max} - \mathrm{min} + 1)^\mathrm{maxlen}$ &
\parbox[t]{5cm}{
\vspace{0.5mm}} \\\hline

\end{longtable}

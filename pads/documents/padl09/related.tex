\section{Related Work}\label{sec:related}
Statistical methods have been used in many grammar induction problems,
including
\xml{} schema inference~\cite{bex+:dtd-inference},
information extraction from the web~\cite{hong:thesis,arasu+:sigmod03long}
and
natural language understanding~\cite{Chen95bayesiangrammar}.
These areas do not typically suffer from the
token ambiguity problem that we see in ad hoc data, however:
tags cleanly divide
\xml{} and web-based data, while spaces and known punctuation symbols
separate natural language words.
In contrast,
the separators and token types found in ad hoc data sources such as
web logs and financial records are far more variable and
ambiguous.  We contribute to the literature on statistical
data processing by analyzing the effectiveness of statistical models
in a new application area, that of ad hoc data, which contains
markedly different characteristics from the most frequently studied
data processing domains.

\begin{itemize}
\item Grammar induction \& structure discovery without token ambiguity problem
Arasu \& Garcia-Molina '03 \cite{arasu+:sigmod03},
Garofalakis et al. '00 \cite{garofalakis+:xtract},
Kushmerick et al. '97 \cite{kushmerick-phd1997}.

\item Detect row table components by Hidden Markov Model \& Conditional Random Fields:
Pinto et al. '03 \cite{Pinto+:texttables}.

\item  Extract certain fields in records from text:
Borkar et al.'01 \cite{borkar+:text-segmentation}.

\item Predict exons and introns in DNA sequences using generalized HMM:
Kulp '96 \cite{kulp96generalized}

\item  Part-of-speech tagging in natural language processing:
Heeman '99 (Decision Tree) \cite{Heeman99:speech}

\item Speech Recognition: Rabiner '89 \cite{rabiner89:hmm}

\end{itemize}

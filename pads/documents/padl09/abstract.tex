\begin{abstract}
This paper attacks the problem of automatically generating
data processing tools such as parsers, printers and xml-transformers
for ad hoc text data.  It does so by showing how to
infer a representation of the data's format 
in \pads{}, a declarative format description language, and to run this
description through the \pads{} compiler to generate
the useful tools.  A key subproblem in this process
is the {\em token ambiguity problem} --- the
problem of determining which substrings in the example data
correspond to complex tokens such as dates, urls, or comments.
In order to solve the token ambiguity problem, the paper studies the
relative effectiveness of three different statistical
models, Hidden Markov Models, Hierarchical Maximum Entropy Models,
and Support Vector Machines, in tokenizing ad hoc data.  It also shows 
how to incorporate probabilistic parsing information into 
the LearnPADS system developed in previous research.  Finally,
it describes an effective new heuristic for simplifying formats inferred
by the system.
\end{abstract}

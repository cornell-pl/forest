\section{Semantics Rewrite -- dpw -- Feb 18, 2009}

A {\em dependency set} is a set of time-location pairs.
Let type $\dsty$ be $\setty{\timety * \locty}$. 
Let $\ds$ range over objects with type $\dsty$.
A {\em schedule} is a set of times.  Let
$\schedulety$ be $\setty{\timety}$.  Let
$\schedule$ range over objects with type $\schedulety$.
({\em Previously, $\schedule$ was an atomic, uninterpreted set of constants in the language.
Previously, sets were not included as a type.}).
Figure~\ref{fig:syntax-revised} contains the revised language 
syntax.

\begin{figure}[t]
\[
\begin{array}{lll}
\multicolumn{3}{l}{\mbox{(host-language base types)}}\\ 
\multicolumn{3}{l}{\basety \ ::= \boolty \bnfalt \stringty \bnfalt \locty \bnfalt \timety} \\
\\
\multicolumn{3}{l}{\mbox{(host-language types)}}\\ 
\multicolumn{3}{l}{\ty \ ::=\ \basety
\bnfalt \optionty{\ty}
\bnfalt \ty_1 * \ty_2
\bnfalt \ty_1 + \ty_2
\bnfalt \listty{\ty}
\bnfalt \setty{\ty}
\bnfalt \ty_1 \arrow \ty_2
} \\
\\
\multicolumn{3}{l}{\mbox{(host-language values)}}\\ 
\multicolumn{3}{l}{\data \ ::=} \\
& \boolf \bnfalt \boolt & \mbox{booleans} \\
\bnfalt & \astring \bnfalt \loc \bnfalt \atime &
 \mbox{strings, locations, times} \\
\bnfalt & \none \bnfalt 
                           \some{\data} & \mbox{optional values}\\
\bnfalt & (\data_1,\data_2) & \mbox{pairs} \\
\bnfalt & \inl{\data} \bnfalt 
                           \inr{\data} & \mbox{sum values} \\
\bnfalt & 
%\nillist \bnfalt 
                           \conslist{\data_1}{\data_n} & \mbox{list values} \\
% & \bnfalt & \nilstream \bnfalt 
%                           \consstream{\data_1}{\data_2} & \mbox{stream values} \\

\bnfalt & \lambda x{:}\ty.\expression & \mbox{function values} \\
\\
\multicolumn{3}{l}{\mbox{(host-language expressions)}}\\ 
\multicolumn{3}{l}{\expression \ ::=}\\ 
& \generalvar & \mbox{variables} \\
\bnfalt & \data & \mbox{data values} \\
\bnfalt & \none \bnfalt 
              \some{\expression} & \mbox{option expressions}\\
%\bnfalt & (\expression_1,\expression_2) \bnfalt e.1 \bnfalt e.2 
%    & \mbox{pair expressions} \\
% & \bnfalt & \inl{\expression} \bnfalt 
%             \inr{\expression} & \mbox{sum expressions} \\
% & \bnfalt & \expression_1 \; \expression_2 & \mbox{application expression} \\
\bnfalt & ... & \mbox{more typed lambda expressions} \\
%\\
%\multicolumn{4}{l}{\mbox{(feed meta-data:  a subset of host language values)}}\\ 
%\multicolumn{4}{l}{\mbox{(a special location (\generatedloc) is used when data is created artificially)}}\\ 
\end{array}
\]
\caption{Host Language Syntax.}
\label{fig:host-language}
\end{figure}


\begin{figure}[t]
\[
\begin{array}{lll}
\multicolumn{3}{l}{\mbox{(feed types)}}\\ 
\multicolumn{3}{l}{\sigma \ ::= \tau \bnfalt \optionty{\tau} 
  \bnfalt \sigma_1 * \sigma_2
  \bnfalt \sigma_1 + \sigma_2
  \bnfalt \listty{\sigma}
}   \\  
\\
\multicolumn{3}{l}{\mbox{(core feed spec)}}\\ 
\multicolumn{3}{l}{\corefeed \ ::= }\\
& \{ \ \mathtt{src=}\    e_1;    & \mbox{source specification} \\
& \ \ \ \mathtt{sched=}\  e_2;    & \mbox{schedule specification}\\
& \ \ \ \mathtt{win=}\    e_3;    & \mbox{time-out window specification} \\
& \ \ \ \mathtt{pp=}\     e_4;    & \mbox{pre-processor} \\
& \ \ \ \mathtt{format=}\ e_5; \} & \mbox{format specification}\\ 
\\
\multicolumn{3}{l}{\mbox{(feed specs)}}\\ 
\multicolumn{3}{l}{\feed \ ::=}   \\  
% & x &  \mbox{feed variable} \\ %% no feed variables now
% & \bnfalt 
         & \mathtt{all}\ \corefeed & \mbox{all sources}\\ 
 \bnfalt & \mathtt{any}\ \corefeed & \mbox{one of multiple sources}\\ 
 \bnfalt & \emptyfeed & \mbox{empty feed} \\
 \bnfalt & \onefeed{e_1}{e_2} & \mbox{singleton feed} \\
 \bnfalt & \sfeed{e} & \mbox{schedule to feed} \\
% \bnfalt & \lfeed{e} & \mbox{list to feed} \\
 \bnfalt & \feed_1 \unionfeed \feed_2 & \mbox{union feed} \\
 \bnfalt & \feed_1 \sumfeed \feed_2 & \mbox{sum feed} \\
 \bnfalt & (\feed_1, \feed_2) & \mbox{pair feed} \\
 \bnfalt & [\feed \bnfalt x \leftarrow e ] & \mbox{list comprehension feed} \\
 \bnfalt & \comprehensionfeed{\feed_2}{x}{\feed_1} & \mbox{feed comprehension} \\
 \bnfalt & \filterfeed{\feed}{e} & \mbox{filter feed} \\
 \bnfalt & \letfeed{x}{e_1}{\feed_2} & \mbox{let feed} \\
% & \bnfalt & \feed_1 cartesian \feed_2 & \mbox{cartesian pair -- use a symbol different from *} \\
% & \bnfalt & \feed_1 * \feed_2 & \mbox{continuous pair} \\
% & \bnfalt & \feed_1 {*}{*} \feed_2 & \mbox{local pair} \\
% \bnfalt & x{:}\feed_1 * \feed_2 & \mbox{dependent continuous pair} \\
% \bnfalt & x{:}\feed_1\, {*}{*} \, \feed_2 & \mbox{dependent local pair} \\
% \bnfalt &     \mathtt{foreach{*}}\; x \; 
%    \mathtt{in}\; \feed_1 & \mbox{for each $x$ create continuous $\feed_2$} \\
% &   \quad \mathtt{create}\; \feed_2 \\
% \bnfalt &     \mathtt{foreach{*}{*}}\; x \; 
%    \mathtt{in}\; \feed_1 & \mbox{for each $x$ update local $\feed_2$}\\
% &   \quad \mathtt{update}\; \feed_2 \\
%\foreachcreate{x}{\feed_1}{\feed_2} & \mbox{for each $x$ create continuous $F_2$} \\
% \bnfalt & \foreachupdate{x}{\feed_1}{\feed_2} & \mbox{for each $x$ create local $F_2$} \\
% & \bnfalt & \ppfeed{\feed}{e} & \mbox{preprocess (eg, unzip) data} \\
% & \bnfalt & \remap{\feed}{e} & \mbox{direct feed to different locations/times} \\
% & \bnfalt & \refeed{\feed}{e} & \mbox{adapt feed to new schedule; 
%                                               fill missing entries with ``None''} \\
% & \bnfalt & \stutterfeed{\feed}{e} & \mbox{stutter on new schedule} \\
\end{array}
\]
\caption{Revised Feed Language Syntax.}
\label{fig:syntax-revised}
\end{figure}

Having talked to Amal on
the phone, I believe the theorem I intend to prove 
is called dependency correctness and
is very, very similar to what James Cheny, Amal Ahmed and Umut Acar
proved in their paper ``Provenance as Dependency Analysis.''
It requires a rewrite of the semantics to fully track the set of time-location
pairs upon which the values in a feed depend. Here is the modified metadata 
definition.  Let $\nested$ range over ``nested'' meta-data and $\meta$ range over
a complete meta-data data structure.

\[
\begin{array}{lcll} 
\nested & ::=     
          & (\atime,\loc,\mathtt{None}) & \mbox{base metadata (timeout)} \\
& \bnfalt & (\atime,\loc,\mathtt{Some}\; \atime) & \mbox{base metadata (success)} \\
& \bnfalt & (\nested_1,\nested_2) & \mbox{pair metadata} \\
& \bnfalt & \inl{\nested} & \mbox{sum metadata} \\
& \bnfalt & \inr{\nested} & \mbox{sum metadata} \\
& \bnfalt & [\nested_1,\ldots,\nested_k] & \mbox{list metadata} \\
\\
\meta & ::= & (\atime,\ds,\nested) & \mbox{complete metadata} \\  
\end{array}
\] 

Given meta-data $\meta$, we write $\mytime{\meta}$, $\myds{\meta}$ and
$\myval{\meta}$ for the first, second and third projections (respectively) of $\meta$.
Note the base metadata is a triple of scheduled time, location of origin 
and optional arrival time where {\tt None} indicates the data did not arrive
in a timely manner.

If the feed typing rules give feed $\feed$ type $\feedty{\sigma}$, 
then its data has type $\sigma$ and
its meta data has type $\metatype{\sigma}$ where $\metatype{\sigma}$ is
defined as follows.

\[
\begin {array} {lcl}
\nestedtype{\ty} & = & \timety * \locty * (\optionty{\timety}) \\
\nestedtype{\optionty{\ty}} & = & \timety * \locty * (\optionty{\timety}) \\
\nestedtype{\sigma_1 * \sigma_2} & = & \nestedtype{\sigma_1} * \nestedtype{\sigma_2} \\
\nestedtype{\sigma_1 + \sigma_2} & = & \nestedtype{\sigma_1} + \nestedtype{\sigma_2} \\
\nestedtype{\listty{\sigma}} & = & \listty{\nestedtype{\sigma}} \\
\\
\metatype{\sigma} & = & \timety * \dsty * \nestedtype{\sigma} \\
\end{array}
\]


The theorem to prove goes as follows.

\begin{definition}[Equal Universes Relative to a Dependency Set]
$\ueq{\universe_1}{\ds}{\universe_2}$ if and only if for all
$(\atime,\loc) \in \ds$, $\universe_1(\atime,\loc) = \universe_2(\atime,\loc)$.
\end{definition}

Let $S_1$, $S_2$ range over denotations of feeds.

\begin{definition}[Feed Subset Relative to a Dependency Set]
$\fsubset{S_1}{\ds}{S_2}$ if and only if for all
$(\meta,v) \in S_1$ such that
$\myds{\meta} \subseteq \ds$, $(\meta,v) \in S_2$.
\end{definition}

\begin{definition}[Feed Equality Relative to a Dependency Set]
$\feq{S_1}{\ds}{S_2}$ if and only if 
$\fsubset{S_1}{\ds}{S_2}$ and
$\fsubset{S_2}{\ds}{S_1}$
\end{definition}

\begin{theorem}[Dependency Correctness]
If $\ueq{\universe_1}{\ds}{\universe_2}$ then
$\feq{\semantics{\feed}{\environment}{\universe_1}}{\ds}{\semantics{\feed}{\environment}{\universe_2}}$
\end{theorem}

The proof of dependency correctness is by induction on the structure of feeds.  I
checked out a couple of cases to see how the proof goes and it seemed plausible to
me that I would be able to prove it completely.   
Then I fleshed out all the rules.  I need to proof-read the rules and then go back
and do the full proof start-to-finish.  However, the proof could definitely
fail for the cases of pairs and lists -- I think those cases will need to be revised,
but as of writing this note (1pm feb 19) I haven't had a chance to go back.  

%Figure~\ref{fig:typing-revised} contains 
We'll naturally also want to keep our type safety theorem, hence we need a slight rewrite of
the typing rules.  Here follows the revised definition of feed typing.

%\begin{figure}[t]

% \[
% \infer[(\textit{t-var})]
% {\Gamma \turn x : \Gamma(x)}
% {}
% \]

\[
\infer[(\textit{t-core})]
{ \begin{array}{l}
  \Gamma \turn 
   \{
      \mathtt{src=} e_1;\
      \mathtt{sched=} e_2; \
      \mathtt{win=} e_3;\\ \qquad \ \ 
      \mathtt{pp=} e_4;\
      \mathtt{ format=} e_5; 
   \} 
   : \schedulety * (\feedty{\optionty{\ty}})
 \end{array}
}
{
 \begin{array}{c}
  \Gamma \turn e_1 : \listty{\locty} \quad \
  \Gamma \turn e_2 : \schedulety \quad \
  \Gamma \turn e_3 : \timety\\
  \Gamma \turn e_4 : \optionty{\stringty} \arrow \optionty{\stringty}  \\
  \Gamma \turn e_5 : \optionty{\stringty} \arrow \optionty{\ty} \\
 \end{array}
}
\]

\[
\infer[(\textit{t-all})]
{ \begin{array}{l}
  \Gamma \turn \mathtt{all}\ \corefeed{} : \feedty{\sigma}
 \end{array}
}
{
 \begin{array}{c}
  \Gamma \turn \corefeed{} : \schedulety * (\feedty{\sigma})
 \end{array}
}
\]

\[
\infer[(\textit{t-any})]
{ \begin{array}{l}
  \Gamma \turn \mathtt{any}\ \corefeed{} : \feedty{\sigma}
 \end{array}
}
{
 \begin{array}{c}
  \Gamma \turn \corefeed{} : \schedulety * (\feedty{\sigma})
 \end{array}
}
\]

\[
\infer[(\textit{t-empty})]
{\Gamma \turn \emptyfeed : \feedty{\sigma}}
{}
\]

\[
\infer[(\textit{t-one})]
{\Gamma \turn \onefeed{e_1}{e_2} : \feedty{\tau}}
{\Gamma \turn e_1 : \tau
 \qquad
 \Gamma \turn e_2 : \timety
}
\]

\[
\infer[(\textit{t-schedule})]
{\Gamma \turn \sfeed{e} : \feedty{\timety}}
{\Gamma \turn e : \schedulety
}
\]


%% \[
%% \infer[(\textit{t-list})]
%% {\Gamma \turn \lfeed{e} : \feedty{\tau}}
%% {\Gamma \turn e : \listty{\tau}
%% }
%% \]

\[
\infer[(\textit{t-union})]
{\Gamma \turn \feed_1 \unionfeed \feed_2  : \feedty{\sigma}}
{
  \Gamma \turn \feed_1 : \feedty{\sigma} &
  \Gamma \turn \feed_2 : \feedty{\sigma}
}
\]

\[
\infer[(\textit{t-sum})]
{\Gamma \turn \feed_1 \sumfeed \feed_2  : \feedty{\sigma_1 + \sigma_2}}
{
  \Gamma \turn \feed_1 : \feedty{\sigma_1} &
  \Gamma \turn \feed_2 : \feedty{\sigma_2}
}
\]

\[
\infer[(\textit{t-pair})]
{\Gamma \turn (\feed_1, \feed_2)  : \feedty{\sigma_1 * \sigma_2}}
{
  \Gamma \turn \feed_1 : \feedty{\sigma_1} &
  \Gamma \turn \feed_2 : \feedty{\sigma_2}
}
\]

\[
\infer[(\textit{t-list})]
{\Gamma \turn [\feed \bnfalt x \leftarrow e ]  : \feedty{\listty{\sigma}}}
{
  \Gamma \turn e : \listty{\tau} &
  \Gamma,x{:}\tau \turn \feed : \feedty{\sigma} 
}
\]

\[
\infer[(\textit{t-comph})]
{\Gamma \turn \comprehensionfeed{\feed_2}{x}{\feed_1} : \feedty{\sigma}}
{
  \Gamma \turn \feed_1 :  \feedty{\sigma} &
  \Gamma,x{:}\metatype{\sigma} * \sigma \turn \feed_2 : \feedty{\sigma} 
}
\]

\[
\infer[(\textit{t-filter})]
{\Gamma \turn \filterfeed{\feed}{e} : \feedty{\sigma}}
{
  \Gamma \turn \feed : \feedty{\sigma} &
  \Gamma \turn e : (\metatype{\sigma} * \sigma) \arrow \boolty
}
\]

\[
\infer[(\textit{t-let})]
{\Gamma \turn \letfeed{x}{e_1}{\feed_2} : \feedty{\sigma_2}}
{
  \Gamma \turn e_1 : \ty_1 & 
  \Gamma,x{:}\ty_1 \turn \feed_2 : \feedty{\sigma_2} 
}
\]

%\caption{Feed Language Typing.}
%\label{fig:typing-revised}
%\end{figure}


Figure~\ref{fig:semantics-revised} contains revised definitions of feeds.


\begin{figure*}[t]
\[
\begin{array}{lcl}

    {\cal C}\lsem\mathtt{\{ src=} e_{src}; 
 &=& \{( \mathtt{meta} (\atime,\loc), \mathtt{val} (\atime,\loc))
          \setalt \atime \in S
          \;\mbox{and}\; \loc \in  L
     \}
\\
 \quad\ \   \mathtt{sched=} e_{sched};
&&\quad\mbox{where} \\
 \quad\ \  \mathtt{win=} e_{win};
&& \qquad S = \esemantics{e_{sched}}{\environment} \\
 \quad\ \  \mathtt{pp=} e_{pp};
&& \qquad W = \esemantics{e_{win}}{\environment}\\
 \quad\ \  \mathtt{format=} e_{f}; \}\rsem_{{\environment} \,
   {\universe}}
&& \qquad L = \esemantics{e_{src}}{\environment}
\\
&& \qquad \mathtt{timeout} =  
     \lambda (x_t,(x_{at},x_s)).
        \mathtt{if}\, x_{at} \leq x_t + W \,
        \mathtt{then}\,  x_s \, \mathtt{else} \, \mathtt{None} 
\\
&& 
\qquad \mathtt{arrival} =  
     \lambda (x_t,(x_{at},x_s)).
        \mathtt{if}\, x_{at} \leq x_t + W \,
        \mathtt{then}\,  \mathtt{Some}\, x_{at} \, \mathtt{else} \, \mathtt{None} 
 \\
&& 
\qquad \mathtt{meta} =  
      \lambda (\atime,\loc).
       (\atime,\{(\atime,\loc)\}, (\atime,\loc,\mathtt{arrival}(\atime,\universe(\atime,\loc))))
\\
 && \qquad \mathtt{val} =
      \lambda (\atime, \loc).
             \esemantics{e_f\; (\universe'(\atime, \loc))}{\environment}
 \\
&& \qquad \universe' =
     \lambda (\atime, \loc). 
           \esemantics{e_{pp}}{\environment}\, 
                 (\mathtt{timeout}\, (\atime,\universe (\atime,\loc))) 

\\



\\\\

\semantics{\mathtt{all}\ \corefeed}{\environment}{\universe} 
&=& 
\csemantics{C}{\environment}{\universe}
\\\\


%New version: any rule
\semantics{\mathtt{any}\ \corefeed}{\environment}{\universe}
& = & \{ i_t\ |\ \atime \in S\}\\
&&
\begin{array}{l}
 \begin{array}{ll@{\hspace{1ex}}c@{\hspace{1ex}}l}
 \mbox{where} & A        & = &\csemantics{C}{\environment}{\universe}\\
              & S        & = & \{\mytime{\meta}\ |\ (\meta, v) \in A\}\\
              & A_\atime & = & \{(\meta,\some{v})\ | \ (\meta, \some{v}) \in A\ \mbox{and} \ \mytime{\meta} = \atime\}\\
              & i_\atime & = & \left\{ \begin{array}{lll}
                                           \mbox{\selectOne}(A_\atime) & \mbox{if} & |A_\atime| > 0\\
                                           ((\atime,\generatedloc), \none) & \mbox{if} & |A_\atime| = 0 \\
                                           \end{array} \right.\\
 \end{array}
\end{array} 
%%End New version: any rule
\\\\

\semantics{\emptyfeed}{\environment}{\universe} 
 &=& \{\;\}
\\\\
\semantics{\onefeed{e_1}{e_2}}{\environment}{\universe} 
 &=& \{
   ((\esemantics{e_2}{\environment},
    \{\;\}, 
       (\esemantics{e_2}{\environment},\generatedloc,
         \mathtt{Some}\;\esemantics{e_2}{\environment})),
    \mathtt{Some}(\esemantics{e_1}{\environment}))\}
\\\\
\semantics{\sfeed{e}}{\environment}{\universe} 
 &=& \{((\atime,\{\;\},(\atime,\generatedloc,\mathtt{Some}\; \atime)), \atime) 
          \setalt \atime \in  \esemantics{e}{\environment} 
     \}
\\\\
%% \semantics{\lfeed{e}}{\environment}{\universe} 
%%  &=& \{((\atime,\{\},\generatedloc), \atime) 
%%           \setalt \atime \in  \esemantics{e}{\environment} 
%%      \}
%% \\\\
\semantics{\feed_1 \unionfeed \feed_2}{\environment}{\universe} 
 &=& \semantics{\feed_1}{\environment}{\universe} 
     \bigcup
     \semantics{\feed_2}{\environment}{\universe} 
\\\\
\semantics{\feed_1 \sumfeed \feed_2}{\environment}{\universe} 
 &=& \{
      ((\mytime{\meta},\myds{\meta},\inl{\myval{\meta}}),\inl{v}) \setalt 
        (\meta,v) \in \semantics{\feed_1}{\environment}{\universe} 
     \} \bigcup
 \\
&&
     \{
      ((\mytime{\meta},\myds{\meta},\inr{\myval{\meta}}),\inr{v}) \setalt 
        (\meta,v) \in \semantics{\feed_2}{\environment}{\universe}
     \}
\\\\
\semantics{(\feed_1, \feed_2)}{\environment}{\universe} 
 &=&
 \{((\mytime{\meta_1},\myds{\meta_1}\cup\myds{\meta_2}, (\myval{\meta_1},\myval{\meta_2})),(v_1,v_2)) 
  \setalt 
\\
&& \quad
     (\meta_1,v_1) \in \semantics{\feed_1}{\environment}{\universe} 
     \; \mbox{and} \; 
     (\meta_2,v_2) \in \semantics{\feed_2}{\environment}{\universe}
     \; \mbox{and} \; 
     \mytime{\meta_1} = \mytime{\meta_2}
  \}
\\\\
\semantics{[\feed \bnfalt x \leftarrow e]}{\environment}{\universe} 
 &=&
 \{((\atime,\bigcup_{i=1\ldots{}k} \myds{\meta_i}, [\myval{\meta_1},\ldots,\myval{\meta_k}]),[v_1,\ldots,v_k]) \setalt 
\\ && \quad
    \forall i:1\ldots k.
     (\meta_i,v_i) \in \semantics{\feed}{\environment[x\mapsto z_i]}{\universe} 
     \; \mbox{and} \; 
     \mytime{\meta_i} =\atime
  \} \\
&&\quad\mbox{where} \quad\mbox{$[z_1,\ldots,z_k] = \esemantics{e}{\environment}$}
\\\\
\semantics{\comprehensionfeed{\feed_2}{x}{\feed_1}}{\environment}{\universe} 
 &=& \{((\mytime{\meta_2},\myds{\meta_1} \cup \myds{\meta_2}, \myval{\meta_2}),v_2) 
          \setalt (\meta_1, v_1) \in  \semantics{\feed_1}{\environment}{\universe} \; \mbox{and} \;
          (\meta_2,v_2) \in \semantics{\feed_2}{(\environment,x\mapsto(\meta_1,v_1))}{\universe}  
     \} 
\\\\
\semantics{\filterfeed{\feed}{e}}{\environment}{\universe} 
 &=&
\{(\meta,v) \setalt (\meta,v) \in \semantics{\feed}{\environment}{\universe} \; \mbox{and} \;
            \esemantics{e \; (\meta,v)}{\environment} = \mathtt{true}
\}
\\\\
\semantics{\letfeed{x}{e_1}{\feed_2}}{\environment}{\universe} 
 &=& \semantics{\feed_2}{(\environment,x\mapsto\esemantics{e_1}{\environment})}{\universe} 
\\
\end{array}
\]
\caption{Feed Language Semantics.}
\label{fig:semantics-revised}
\end{figure*}

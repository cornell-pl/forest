%The term {\em ad hoc data} refers to the billions of bytes of
%non-standard and continuously evolving data spread across all computer
%systems.  Such data includes server logs, distributed system
%performance and debugging data, telephone call records, financial data
%and online repositories of scientific data.
% including genetic information, cosmology data and
%global weather systems.

This paper presents the theory, design and implementation of
\padsd{}, a novel domain-specific language that allows users to
specify the provenance, syntax and semantic properties of 
collections of distributed 
data sources.  In particular, \padsd{} specifications indicate
{\em where} the given data is located, {\em how} to obtain it, {\em
when} to get it or to give up trying, and {\em what}
format it will be in on arrival or what preprocessing 
needs to be done.  Any such specification can be compiled into
a power suite of data-processing tools including 
an archiver, a provenance tracking system, a database loading tool, 
an alert system, an RSS feed generator, and a
debugging tool.  In addition, the system generates libraries for
application developers, including modules for parsing, printing, error
management, data traversal and transformation, which developers can use
to create their own application-specific tools.  Advanced users can
build new generic, description-directed tools applicable to any 
collection of data sources.

To elucidate the finer points of our design, we show how \padsd{} may
be used to specify properties of the data used in 
CoMon, a monitoring system for PlanetLab, and in Arrakis, a monitoring
system for a web hosting service provided by AT\&T.  
To demonstrate the practicality of our implementation, we show our system is
capable of scaling to distributed systems the size of CoMon,
the current monitor for Planetlab's 800+ nodes and we evaluate its
performance on such workloads.  To clarify the
meaning and soundness of our language, we provide a
denotational semantics for it and use this semantics to prove two important
theorems.  The first theorem shows that our denotational semantics respects the
typing rules for the language, while the second theorem demonstrates that our 
system correctly maintains provenance metadata.




% CoMon 
% the Safari web
% cache to SDSS star charts {\em [dpw: correct me ... cites]} to
% PlanetLab's CoMon network monitoring system to the log files for the
% Coral content distribution network.  We also provide a denotational
% semantics for the language, which is helpful in understanding the
% language's many features.

% Descriptions written in \padsd{} are compiled into O'Caml libraries
% with clean interfaces that support both polymorphic and monomorphic
% programming.  We illustrate how to use the polymorphic interfaces to
% program description-independent applications, including a selector
% tool that can extract key elements of a data source as well as an
% accumulator tool that collects simple statistical information about
% the data.  The monomorphic interfaces allow users to develop
% ``one-of'' data source-specific applications.


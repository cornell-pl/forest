\begin{figure*}[t]
\[
\begin{array}{lrll}
\multicolumn{4}{l}{\mbox{(host-language base types)}}\\ 
\basety & ::= & \multicolumn{2}{l}{\boolty \bnfalt \stringty \bnfalt \locty \bnfalt \timety \bnfalt \schedulety} \\
\\
\multicolumn{4}{l}{\mbox{(host-language types)}}\\ 
\ty & ::= & \basety & \mbox{base types} \\
 & \bnfalt & \ty_1 * \ty_2 & \mbox{pair types} \\
 & \bnfalt & \optionty{\ty} & \mbox{option types}\\
 & \bnfalt & \ty_1 + \ty_2 & \mbox{union types} \\
 & \bnfalt & \listty{\ty} & \mbox{list types}\\
 & \bnfalt & \feedty{\ty} & \mbox{stream types}\\
 & \bnfalt & \ty_1 \arrow \ty_2 & \mbox{function types} \\
\\
\multicolumn{4}{l}{\mbox{(host-language values)}}\\ 
\data & ::=     & \constant & \mbox{generic constant} \\
 & \bnfalt & \parser & \mbox{parser (generated from PADS/ML)} \\
 & \bnfalt & \loc & \mbox{locations} \\
 & \bnfalt & \atime & \mbox{times} \\
 & \bnfalt & \schedule & \mbox{schedules} \\
 & \bnfalt & \none \bnfalt 
                           \some{\data} & \mbox{optional values}\\
 & \bnfalt & (\data_1,\data_2) & \mbox{pairs} \\
 & \bnfalt & \inl{\data} \bnfalt 
                           \inr{\data} & \mbox{union values} \\
 & \bnfalt & \nillist \bnfalt 
                           \conslist{\data_1}{\data_2} & \mbox{list values} \\
 & \bnfalt & \nilstream \bnfalt 
                           \consstream{\data_1}{\data_2} & \mbox{stream values} \\

& \bnfalt & \lambda x{:}\ty.\expression & \mbox{function values} \\
\\
\multicolumn{4}{l}{\mbox{(host-language expressions)}}\\ 
\expression & ::= & \generalvar & \mbox{variables} \\
 & \bnfalt & \data & \mbox{data values} \\
 & \bnfalt & \none \bnfalt 
              \some{\expression} & \mbox{option expressions}\\
 & \bnfalt & (\expression_1,\expression_2) \bnfalt e.1 \bnfalt e.2 
    & \mbox{pair expressions} \\
% & \bnfalt & \inl{\expression} \bnfalt 
%             \inr{\expression} & \mbox{union expressions} \\
% & \bnfalt & \expression_1 \; \expression_2 & \mbox{application expression} \\
 & \bnfalt & ... & \mbox{more typed lambda expressions} \\
\multicolumn{4}{l}{\mbox{(a subset of host language values are used as meta-data)}}\\ 
\multicolumn{4}{l}{\mbox{(a special location (\generatedloc) is used when data is created artificially)}}\\ 
\meta & ::=     
& (\atime,\loc) & \mbox{base metadata} \\
& \bnfalt & (\atime,(\meta,\meta)) & \mbox{pair metadata} \\
& \bnfalt & (\atime,\inl{\meta}) & \mbox{sum metadata} \\
& \bnfalt & (\atime,\inr{\meta}) & \mbox{sum metadata} \\
\end{array}
\]
\caption{Host Language Syntax.}
\label{fig:host-language}
\end{figure*}

\begin{figure*}[t]
\[
\begin{array}{lrll}
\multicolumn{4}{l}{\mbox{(feed specs)}}\\ 
\feed & ::=     
% & x &  \mbox{feed variable} \\ %% no feed variables now
% & \bnfalt 
& \emptyfeed & \mbox{empty feed} \\
 & \bnfalt & \computed{e_1}{x}{e_2} & \mbox{computed feed} \\
 & \bnfalt & \letfeed{x}{e_1}{\feed_2} & \mbox{let feed} \\
 & \bnfalt & \allfeed{e_1}{e_2}{e_3}{e_4}{e_5} & \mbox{all locations} \\
 & \bnfalt & \existsfeed{e_1}{e_2}{e_3}{e_4}{e_5} & \mbox{one location} \\
 & \bnfalt & \feed_1 \unionfeed \feed_2 & \mbox{union feed} \\
 & \bnfalt & \feed_1 \sumfeed \feed_2 & \mbox{sum feed} \\
 & \bnfalt & \feed_1 \spairfeed \feed_2 & \mbox{synchronous pair} \\
% & \bnfalt & \feed_1 cartesian \feed_2 & \mbox{cartesian pair -- use a symbol different from *} \\
% & \bnfalt & \feed_1 * \feed_2 & \mbox{continuous pair} \\
% & \bnfalt & \feed_1 {*}{*} \feed_2 & \mbox{local pair} \\
 & \bnfalt & x{:}\feed_1 * \feed_2 & \mbox{dependent continuous pair} \\
 & \bnfalt & x{:}\feed_1\, {*}{*} \, \feed_2 & \mbox{dependent local pair} \\
 & \bnfalt & \foreachcreate{x}{\feed_1}{\feed_2} & \mbox{for each $x$ create continuous $F_2$} \\
 & \bnfalt & \foreachupdate{x}{\feed_1}{\feed_2} & \mbox{for each $x$ create local $F_2$} \\
 & \bnfalt & \filterfeed{\feed}{e} & \mbox{filter out some locations/times} \\
% & \bnfalt & \ppfeed{\feed}{e} & \mbox{preprocess (eg, unzip) data} \\
% & \bnfalt & \remap{\feed}{e} & \mbox{direct feed to different locations/times} \\
 & \bnfalt & \refeed{\feed}{e} & \mbox{adapt feed to new schedule; 
                                               fill missing entries with ``None''} \\
 & \bnfalt & \stutterfeed{\feed}{e} & \mbox{stutter on new schedule} \\

\end{array}
\]
\caption{Feed Language Syntax.}
\label{fig:syntax}
\end{figure*}


\begin{figure*}[t]

% \[
% \infer[(\textit{t-var})]
% {\Gamma \turn x : \Gamma(x)}
% {}
% \]

\[
\infer[(\textit{t-empty})]
{\Gamma \turn \emptyfeed : \feedty{\ty}}
{}
\]

\[
\infer[(\textit{t-compute})]
{\Gamma \turn \computed{e_1}{x}{e_2} : \feedty{\optionty{\ty}}}
{
  \Gamma \turn e_2 : \schedulety &
  \Gamma,x{:}\timety \turn e_1 : \optionty{\ty} 
}
\]

\[
\infer[(\textit{t-let})]
{\Gamma \turn \letfeed{x}{e_1}{\feed_2} : \feedty{\ty_2}}
{
  \Gamma \turn e_1 : \ty_1 & 
  \Gamma,x{:}\ty_1 \turn \feed_2 : \feedty{\ty_2} 
}
\]

\[
\infer[(\textit{t-all})]
{\Gamma \turn \allfeed{e_1}{e_2}{e_3}{e_4}{e_5} : \feedty{\optionty{\ty}}}
{
 \begin{array}{c}
  \Gamma \turn e_1 : \optionty{\stringty} \arrow \optionty{\ty} \qquad
  \Gamma \turn e_2 : \listty{\locty} \\
  \Gamma \turn e_3 : \schedulety \qquad
  \Gamma \turn e_4 : \optionty{\stringty} \arrow \optionty{\stringty}  \qquad
  \Gamma \turn e_5 : \timety
 \end{array}
}
\]

\[
\infer[(\textit{t-any})]
{\Gamma \turn \existsfeed{e_1}{e_2}{e_3}{e_4}{e_5} : \feedty{\optionty{\ty}}}
{
 \begin{array}{c}
  \Gamma \turn e_1 : \optionty{\stringty} \arrow \optionty{\ty} \qquad
  \Gamma \turn e_2 : \listty{\locty} \\
  \Gamma \turn e_3 : \schedulety \qquad
  \Gamma \turn e_4 : \optionty{\stringty} \arrow \optionty{\stringty}  \qquad
  \Gamma \turn e_5 : \timety
 \end{array}
}
\]

\[
\infer[(\textit{t-union})]
{\Gamma \turn \feed_1 \unionfeed \feed_2  : \feedty{\ty}}
{
  \Gamma \turn \feed_1 : \feedty{\ty} &
  \Gamma \turn \feed_2 : \feedty{\ty}
}
\]

\[
\infer[(\textit{t-sum})]
{\Gamma \turn \feed_1 \sumfeed \feed_2  : \feedty{\ty_1 + \ty_2}}
{
  \Gamma \turn \feed_1 : \feedty{\ty_1} &
  \Gamma \turn \feed_2 : \feedty{\ty_2}
}
\]

\[
\infer[(\textit{t-synch-pair})]
{\Gamma \turn \feed_1 \spairfeed \feed_2  : \feedty{\ty_1 * \ty_2}}
{
  \Gamma \turn \feed_1 : \feedty{\ty_1} &
  \Gamma \turn \feed_2 : \feedty{\ty_2}
}
\]

% \[
% \infer[(\textit{t-local-pair})]
% {\Gamma \turn \feed_1 * \feed_2  : \feedty{\ty_1 * \ty_2}}
% {
%   \Gamma \turn \feed_1 : \feedty{\ty_1} &
%   \Gamma \turn \feed_2 : \feedty{\ty_2}
% }
% \]

\[
\infer[(\textit{t-cont-pair})]
{\Gamma \turn x{:}\feed_1 * \feed_2  : \feedty{\ty_1 * \ty_2}}
{
  \Gamma \turn \feed_1 : \feedty{\ty_1} &
  \Gamma,x{:}\ty_1 \turn \feed_2 : \feedty{\ty_2}
}
\]

\[
\infer[(\textit{t-local-pair})]
 {\Gamma \turn x{:}\feed_1 \allpairfeed \feed_2  : \feedty{\ty_1 * \ty_2}}
 {
   \Gamma \turn \feed_1 : \feedty{\ty_1} &
   \Gamma,x{:}\ty_1 \turn \feed_2 : \feedty{\ty_2}
 }
\]

\[
\infer[(\textit{t-foreachcont})]
{\Gamma \turn \foreachcreate{x}{\feed_1}{\feed_2}  : \feedty{\ty_2}}
{
  \Gamma \turn \feed_1 : \feedty{\ty_1} &
  \Gamma,x{:}\ty_1 \turn \feed_2 : \feedty{\ty_2}
}
\]

\[
\infer[(\textit{t-foreachlocal})]
{\Gamma \turn \foreachupdate{x}{\feed_1}{\feed_2}  : \feedty{\ty_2}}
{
  \Gamma \turn \feed_1 : \feedty{\ty_1} &
  \Gamma,x{:}\ty_1 \turn \feed_2 : \feedty{\ty_2}
}
\]


\[
\infer[(\textit{t-filter})]
{\Gamma \turn \filterfeed{\feed}{e} : \feedty{\ty}}
{
  \Gamma \turn \feed : \feedty{\ty} &
  \Gamma \turn e : \ty \arrow \boolty
}
\]

% \[
% \infer[(\textit{t-pp})]
% {\Gamma \turn \ppfeed{\feed}{e} : \feedty{\ty}}
% {
%   \Gamma \turn \feed : \feedty{\ty} &
%   \Gamma \turn e : ((\locty * \timety) * \stringty) \arrow \stringty
% }
% \]

% \[
% \infer[(\textit{t-redirect})]
% {\Gamma \turn \remapfeed{\feed}{e} : \feedty{\ty}}
% {
%   \Gamma \turn \feed : \feedty{\ty} &
%   \Gamma \turn e : \locty * \timety \arrow \locty * \timety
% }
% \]

\[
\infer[(\textit{t-reschedule})]
{\Gamma \turn \refeed{\feed}{e} : \feedty{\optionty{\ty}}}
{
  \Gamma \turn \feed : \feedty{\ty} &
  \Gamma \turn e : \schedulety
}
\]

\[
\infer[(\textit{t-stutter})]
{\Gamma \turn \stutterfeed{\feed}{e} : \feedty{\ty}}
{
  \Gamma \turn \feed : \feedty{\ty} &
  \Gamma \turn e : \schedulety
}
\]

\caption{Feed Language Typing.}
\label{fig:typing}
\end{figure*}

%
\begin{figure*}[t]
\[
\begin{array}{lcl}
\esemantics{e}{\environment} &=& \mbox{denotation of simply-typed term $e$ in environment $\environment$ mapping variables to values.}
\\
\\
\semantics{\feed}{\environment}{\universe} &=& \mbox{denotation of 
$\feed$ in environment $\environment$ mapping variables to values.} \\
 && \mbox{and ``universe'' $\universe$ mapping schedule time and location to 
          arrival time and string data}
\\
% \\
% \semantics{x}{\environment}{\universe} 
%  &=& (\environment(x_t), \environment(x))
% \\
\\
\semantics{\emptyfeed}{\environment}{\universe} 
 &=& \{\;\}
\\\\
\semantics{\computed{e_1}{x}{e_2}}{\environment}{\universe} 
 &=& \{(\atime, \esemantics{(\lambda x.e_1)}{\environment}\; \atime) 
          \setalt \atime \in  \esemantics{e_2}{\environment} 
     \} 
\\\\

%  \begin{array}{l}
    {\cal F}\lsem\mathtt{all \{ format=} e_1; 
%    \allfeed{e_1}{e_2}{e_3}{e_4}
%  \end{array} 
%}{\environment}{\universe} 
 &=& \{(\atime, (\loc, \esemantics{e_1}{\environment} \; (\universe'(\loc,\atime))))
          \setalt \atime \in  \esemantics{e_3}{\environment} 
          \;\mbox{and}\; \loc \in  \esemantics{e_2}{\environment}
     \} 
\\
 \qquad\quad\ \,   \mathtt{locs=} e_2;
&&\quad\mbox{where} \\
 \qquad\quad\ \,    \mathtt{sched=} e_3;
&& \qquad \mathtt{timeout} =  
     \lambda (x_t,(x_{at},s)).
        \mathtt{if}\, x_{at} \leq x_t + \esemantics{e_5}{\environment} \,
        \mathtt{then}\,  s \, \mathtt{else} \, \mathtt{None} \\
 \qquad\quad\ \,    \mathtt{pp=} e_4;
&& \qquad \universe' =
     \lambda (x_{\ell}, x_t). 
           \esemantics{e_4}{\environment}\, 
                 (\mathtt{timeout}\, (x_t,\universe (x_\ell,x_t))) 
 \\
 \qquad\quad\ \,    \mathtt{win=} e_5; \}\rsem_{{\environment} \, {\universe}}
&& \\
% \\
% {\cal F}\lsem\mathtt{exists \{ format=} e_1;
% % \semantics{\existsfeed{e_1}{e_2}{e_3}{e_4}}{\environment}{\universe} 
%  &=& \bigcup_{\atime \in \esemantics{e_2}{\environment}} f\; \atime \\
%  \qquad\qquad\ \ \ \,   \mathtt{locs=} e_2;
% &&\quad\mbox{where} \\
% %\begin{array}{l}
%  \qquad\qquad\ \ \ \,    \mathtt{sched=} e_3;
%   && \qquad \universe' \, = \lambda x{:}\locty * \timety. \esemantics{e_4}{\environment} (x,\universe(x)) \\
%  \qquad\qquad\ \ \ \,    \mathtt{pp=} e_4;
% &&\qquad
%   f\; \atime = \{(\atime,(\loc,\esemantics{e_1}{\environment}\; (\some{v})\} \\
%  \qquad\qquad\ \ \ \,   \mathtt{win=} e_5;
%  \}\rsem_{{\environment} \, {\universe}}
% &&\qquad\qquad \qquad\qquad 
%   \mbox{for some $\loc \in \esemantics{e_2}{\environment}$ such that
%     $\universe'(\atime,\loc) = \some{v}$} \\
% &&\qquad
%   f \; \atime = \{(\atime,(\loc,\esemantics{e_1}{\environment}\; \none\} \\
% &&\qquad\qquad \qquad\qquad 
%   \mbox{if no such $\loc$ exists} \\
% %\end{array}
% \\
\\
{\cal F}\lsem\mathtt{exists \{ format=} e_1;
% \semantics{\existsfeed{e_1}{e_2}{e_3}{e_4}}{\environment}{\universe} 
 &=& \bigcup_{\atime \in \esemantics{e_2}{\environment}} f\; \atime \\
 \qquad\qquad\ \ \ \,   \mathtt{locs=} e_2;
&&\quad\mbox{where} \\
%\begin{array}{l}
 \qquad\qquad\ \ \ \,    \mathtt{sched=} e_3;
  && 
\qquad f = \lambda \atime. \{(\atime,(\ell,\mathtt{Some} \; s))\}\ \mbox{for some $\ell$ and $s$ such that:}\\
 \qquad\qquad\ \ \ \,    \mathtt{pp=} e_4;
&&\qquad\quad
 \mbox{$(\atime,(\ell,\mathtt{Some} \; s)) \in \semantics{\allfeed{e_1}{e_2}{e_3}{e_4}{e_5}}{\environment}{\universe}$}
  \\
 \qquad\qquad\ \ \ \,   \mathtt{win=} e_5;
 \}\rsem_{{\environment} \, {\universe}}
&&\qquad
  f = \lambda \atime. \{(\atime,(\loc, \none))\}\  
  \mbox{for some $\loc \in \esemantics{e2}{\environment}$
if there exists no $\ell$ and $s$ such that:} \\
&&\qquad\quad 
 \mbox{$(\atime,(\ell,\mathtt{Some} \; s)) \in \semantics{\allfeed{e_1}{e_2}{e_3}{e_4}{e_5}}{\environment}{\universe}$}
\\

%\end{array}
\\
\\
\semantics{\feed_1 \unionfeed \feed_2}{\environment}{\universe} 
 &=& \semantics{\feed_1}{\environment}{\universe} 
     \bigcup
     \semantics{\feed_2}{\environment}{\universe} 
\\\\
\semantics{\feed_1 \sumfeed \feed_2}{\environment}{\universe} 
 &=& \{
      (\atime,\inl{v}) \setalt 
        (\atime,v) \in \semantics{\feed_1}{\environment}{\universe} 
     \}
     \bigcup
     \{
      (\atime,\inr{v}) \setalt 
        (\atime,v) \in \semantics{\feed_2}{\environment}{\universe}
     \}
\\\\
\semantics{\feed_1 \spairfeed \feed_2}{\environment}{\universe} 
 &=&
 \{(\atime,(v_1,v_2)) \setalt 
     (\atime,v_1) \in \semantics{\feed_1}{\environment}{\universe} 
     \; \mbox{and} \; 
     (\atime,v_2) \in \semantics{\feed_2}{\environment}{\universe}
  \}
\\\\
% \semantics{\feed_1 * \feed_2}{\environment}{\universe} 
%  &=&
%  \{(\atime_2,(v_1,v_2)) \setalt 
%      (\atime_1,v_1) \in \semantics{\feed_1}{\environment}{\universe} 
%      \; \mbox{and} \; 
% \\&&\qquad\qquad\qquad\ \ \,
%      (\atime_2,v_2) \in \semantics{\feed_2}{\environment}{\universe}
%      \; \mbox{and} \;
% \\&&\qquad\qquad\qquad\ \ \,
%      ((\atime_1',v_1') \in \semantics{\feed_1}{\environment}{\universe} 
%       \; \mbox{implies} \; (t_1' \leq t_1 \; \mbox{or} \; t_1' > t_2)) 
%   \}
% \\\\
\semantics{x{:}\feed_1 * \feed_2}{\environment}{\universe} 
 &=&
 \{(\atime_2,(v_1,v_2)) \setalt 
     (\atime_1,v_1) \in \semantics{\feed_1}{\environment}{\universe} 
     \; \mbox{and} \; 
\\&&\qquad\qquad\qquad\ \ \,
     (\atime_2,v_2) \in \semantics{\feed_2}{(\environment,x\mapsto{}v_1)}{\universe}
     \; \mbox{and} \; \atime_2 > \atime_1
  \}
\\\\
\semantics{x{:}\feed_1 \, {*}{*} \, \feed_2}{\environment}{\universe} 
 &=&
 \{(\atime_2,(v_1,v_2)) \setalt 
     (\atime_1,v_1) \in \semantics{\feed_1}{\environment}{\universe} 
     \; \mbox{and} \; 
\\&&\qquad\qquad\qquad\ \ \,
     (\atime_2,v_2) \in \semantics{\feed_2}{(\environment,x\mapsto{}v_1)}{\universe}
     \; \mbox{and} \; \atime_2 > \atime_1
\\&&\qquad\qquad\qquad\ \ \,
     ((\atime_1',v_1') \in \semantics{\feed_1}{\environment}{\universe} 
      \; \mbox{implies} \; (t_1' \leq t_1 \; \mbox{or} \; t_1' > t_2)) 
  \}
\\\\
{\cal F}\lsem
\mathtt{foreach{*}}\; x \; \mathtt{in}\; \feed_1 
%\semantics{\foreachupdate{x}{\feed_1}{\feed_2}}{\environment}{\universe} 
 &=&
 \{(\atime_2,v_2) \setalt 
     (\atime_1,v_1) \in \semantics{\feed_1}{\environment}{\universe} 
     \; \mbox{and} \; 
\\
\qquad\qquad\quad\ \, \mathtt{create}\; \feed_2 \rsem_{{\environment} \, {\universe}}
&&\qquad\qquad\ \,
     (\atime_2,v_2) \in \semantics{\feed_2}{(\environment,x\mapsto{}v_1)}{\universe}
     \; \mbox{and} \;
%\\&&\qquad\qquad\ \,
     \atime_2 > \atime_1 
  \}
\\\\
{\cal F}\lsem
\mathtt{foreach{*}{*}}\; x \; \mathtt{in}\; \feed_1 
%\semantics{\foreachcreate{x}{\feed_1}{\feed_2}}{\environment}{\universe} 
 &=&
 \{(\atime_2,v_2) \setalt 
     (\atime_1,v_1) \in \semantics{\feed_1}{\environment}{\universe} 
     \; \mbox{and} \; 
\\
\qquad\qquad\quad\ \, \mathtt{extend}\; \feed_2 \rsem_{{\environment} \, {\universe}}
&&\qquad\qquad\ \,
     (\atime_2,v_2) \in \semantics{\feed_2}{(\environment,x\mapsto{}v_1)}{\universe}
     \; \mbox{and} \; t_2 > t_1 \; \mbox{and} \;
\\&&\qquad\qquad\ \,
     ((\atime_1',v_1') \in \semantics{\feed_1}{\environment}{\universe} 
      \; \mbox{implies} \; (t_1' \leq t_1 \; \mbox{or} \; t_1' > t_2))      
  \}
\\\\
\semantics{\filterfeed{\feed}{e}}{\environment}{\universe} 
 &=&
\{(t,v) \setalt (t,v) \in \semantics{\feed}{\environment}{\universe} \; \mbox{and} \;
            \esemantics{e}{\environment}\; v = \mathtt{true}
\}
\\\\
% \semantics{\ppfeed{\feed}{e}}{\environment}{\universe} 
%  &=&
% \semantics{\feed}{\environment}{
%   (\lambda x{:}\locty * \timety. \esemantics{e}{\environment} (x,\universe(x)))} 
% \\\\
% \semantics{\remapfeed{\feed}{e}}{\environment}{\universe} 
%  &=&
% \semantics{\feed}{\environment}{(\universe \circ \esemantics{e}{\environment})}
% \\\\
\semantics{\refeed{\feed}{e}}{\environment}{\universe} 
 &=&
\{(\atime,\some{v}) \setalt 
   (\atime,v) \in \semantics{\feed}{\environment}{\universe} \; \mbox{and} \;
   \atime \in \esemantics{e}{\environment}
\} \bigcup
\\&&
\{(\atime,\none) \setalt
   (\atime,\_) \not\in \semantics{\feed}{\environment}{\universe} \; \mbox{and} \;
   \atime \in \esemantics{e}{\environment}
\}
\\\\
\semantics{\stutterfeed{\feed}{e}}{\environment}{\universe} 
 &=&
\{(\atime,v) \setalt 
   (\atime,v) \in \semantics{\feed}{\environment}{\universe} \; \mbox{and} \;
   \atime \in \esemantics{e}{\environment}
\} \bigcup
\\&&
\{(\atime,v) \setalt 
   (\atime',v) \in \semantics{\feed}{\environment}{\universe} \; \mbox{and} \;
   \atime \in \esemantics{e}{\environment}  \; \mbox{and} \;
\\&&\qquad\qquad\qquad\ \ \,
    \mbox{for all $\atime''$ such that $\atime' < \atime'' \leq \atime$,} \;
   (\atime'',\_) \not\in \semantics{\feed}{\environment}{\universe} \; 
\}

\end{array}
\]
\caption{Feed Language Semantics.}
\label{fig:semantics}
\end{figure*}



\begin{figure*}[t]
\[
\begin{array}{lcl}
%\esemantics{e}{\environment} &=& \mbox{denotation of simply-typed term $e$ in environment $\environment$ mapping variables to values.}
%\\
%\\
%\semantics{\feed}{\environment}{\universe} &=& \mbox{denotation of 
%$\feed$ in environment $\environment$ mapping variables to values.} \\
% && \mbox{and ``universe'' $\universe$ mapping schedule time and location to 
%          arrival time and string data}
%\\
% \\
% \semantics{x}{\environment}{\universe} 
%  &=& (\environment(x_t), \environment(x))
% \\
\\
\semantics{\emptyfeed}{\environment}{\universe} 
 &=& \{\;\}
\\\\
\semantics{\computed{e_1}{x}{e_2}}{\environment}{\universe} 
 &=& \{((\atime,\generatedloc), \esemantics{(\lambda x.e_1)}{\environment}\; \atime) 
          \setalt \atime \in  \esemantics{e_2}{\environment} 
     \} 
\\\\
\semantics{\filterfeed{\feed}{e}}{\environment}{\universe} 
 &=&
\{(\meta,v) \setalt (\meta,v) \in \semantics{\feed}{\environment}{\universe} \; \mbox{and} \;
            \esemantics{e}{\environment}\; v = \mathtt{true}
\}
\\\\
\semantics{\letfeed{x}{e_1}{\feed_2}}{\environment}{\universe} 
 &=& \semantics{\feed_2}{(\environment,x\mapsto\esemantics{e_1}{\environment})}{\universe} 
\\\\

%  \begin{array}{l}
    {\cal F}\lsem\mathtt{all \{ format=} e_1; 
%    \allfeed{e_1}{e_2}{e_3}{e_4}
%  \end{array} 
%}{\environment}{\universe} 
 &=& \{((\atime,\loc), \esemantics{e_1}{\environment} \; (\universe'(\loc,\atime))))
          \setalt \atime \in  \esemantics{e_3}{\environment} 
          \;\mbox{and}\; \loc \in  \esemantics{e_2}{\environment}
     \} 
\\
 \qquad\quad\ \,   \mathtt{locs=} e_2;
&&\quad\mbox{where} \\
 \qquad\quad\ \,    \mathtt{sched=} e_3;
&& \qquad \mathtt{timeout} =  
     \lambda (x_t,(x_{at},x_s)).
        \mathtt{if}\, x_{at} \leq x_t + \esemantics{e_5}{\environment} \,
        \mathtt{then}\,  x_s \, \mathtt{else} \, \mathtt{None} \\
 \qquad\quad\ \,    \mathtt{pp=} e_4;
&& \qquad \universe' =
     \lambda (x_{\ell}, x_t). 
           \esemantics{e_4}{\environment}\, 
                 (\mathtt{timeout}\, (x_t,\universe (x_\ell,x_t))) 
 \\
 \qquad\quad\ \,    \mathtt{win=} e_5; \}\rsem_{{\environment} \, {\universe}}
&& \\
% \\
% {\cal F}\lsem\mathtt{any \{ format=} e_1;
% % \semantics{\existsfeed{e_1}{e_2}{e_3}{e_4}}{\environment}{\universe} 
%  &=& \bigcup_{\atime \in \esemantics{e_2}{\environment}} f\; \atime \\
%  \qquad\qquad\ \ \ \,   \mathtt{locs=} e_2;
% &&\quad\mbox{where} \\
% %\begin{array}{l}
%  \qquad\qquad\ \ \ \,    \mathtt{sched=} e_3;
%   && \qquad \universe' \, = \lambda x{:}\locty * \timety. \esemantics{e_4}{\environment} (x,\universe(x)) \\
%  \qquad\qquad\ \ \ \,    \mathtt{pp=} e_4;
% &&\qquad
%   f\; \atime = \{(\atime,(\loc,\esemantics{e_1}{\environment}\; (\some{v})\} \\
%  \qquad\qquad\ \ \ \,   \mathtt{win=} e_5;
%  \}\rsem_{{\environment} \, {\universe}}
% &&\qquad\qquad \qquad\qquad 
%   \mbox{for some $\loc \in \esemantics{e_2}{\environment}$ such that
%     $\universe'(\atime,\loc) = \some{v}$} \\
% &&\qquad
%   f \; \atime = \{(\atime,(\loc,\esemantics{e_1}{\environment}\; \none\} \\
% &&\qquad\qquad \qquad\qquad 
%   \mbox{if no such $\loc$ exists} \\
% %\end{array}
% \\
\\
{\cal F}\lsem\mathtt{any \{ format=} e_1;
% \semantics{\existsfeed{e_1}{e_2}{e_3}{e_4}}{\environment}{\universe} 
 &=& \bigcup_{\atime \in \esemantics{e_2}{\environment}} f\; \atime \\
 \qquad\qquad\ \ \ \,   \mathtt{locs=} e_2;
&&\quad\mbox{where} \\
%\begin{array}{l}
 \qquad\qquad\ \ \ \,    \mathtt{sched=} e_3;
  && 
\qquad f = \lambda \atime. \{((\atime,\ell),\mathtt{Some} \; \astring)\}\ \mbox{for some $\ell$ and $\astring$ such that:}\\
 \qquad\qquad\ \ \ \,    \mathtt{pp=} e_4;
&&\qquad\quad
 \mbox{$((\atime,\ell),\mathtt{Some} \; \astring) \in \semantics{\allfeed{e_1}{e_2}{e_3}{e_4}{e_5}}{\environment}{\universe}$}
  \\
 \qquad\qquad\ \ \ \,   \mathtt{win=} e_5;
 \}\rsem_{{\environment} \, {\universe}}
&&\qquad
  f = \lambda \atime. \{((\atime,\generatedloc), \none)\}\  
  \mbox{%for some $\loc \in \esemantics{e2}{\environment}$
if there exists no $\ell$ and $\astring$ such that:} \\
&&\qquad\quad 
 \mbox{$((\atime,\ell),\mathtt{Some} \; \astring) \in \semantics{\allfeed{e_1}{e_2}{e_3}{e_4}{e_5}}{\environment}{\universe}$}
\\

%\end{array}
\\
\\
\semantics{\feed_1 \unionfeed \feed_2}{\environment}{\universe} 
 &=& \semantics{\feed_1}{\environment}{\universe} 
     \bigcup
     \semantics{\feed_2}{\environment}{\universe} 
\\\\
\semantics{\feed_1 \sumfeed \feed_2}{\environment}{\universe} 
 &=& \{
      ((\mytime{\meta},\inl{\meta}),\inl{v}) \setalt 
        (\meta,v) \in \semantics{\feed_1}{\environment}{\universe} 
     \}
     \bigcup
     \{
      ((\mytime{\meta},\inr{\meta}),\inr{v}) \setalt 
        (\meta,v) \in \semantics{\feed_2}{\environment}{\universe}
     \}
\\\\
\semantics{(\feed_1, \feed_2)}{\environment}{\universe} 
 &=&
 \{((\mytime{\meta_1},(\meta_1,\meta_2)),(v_1,v_2)) \setalt 
     (\meta_1,v_1) \in \semantics{\feed_1}{\environment}{\universe} 
     \; \mbox{and} \; 
     (\meta_2,v_2) \in \semantics{\feed_2}{\environment}{\universe}
     \; \mbox{and} \; 
     \mytime{\meta_1} = \mytime{\meta_2}
  \}
\\\\
% \semantics{\feed_1 * \feed_2}{\environment}{\universe} 
%  &=&
%  \{(\atime_2,(v_1,v_2)) \setalt 
%      (\atime_1,v_1) \in \semantics{\feed_1}{\environment}{\universe} 
%      \; \mbox{and} \; 
% \\&&\qquad\qquad\qquad\ \ \,
%      (\atime_2,v_2) \in \semantics{\feed_2}{\environment}{\universe}
%      \; \mbox{and} \;
% \\&&\qquad\qquad\qquad\ \ \,
%      ((\atime_1',v_1') \in \semantics{\feed_1}{\environment}{\universe} 
%       \; \mbox{implies} \; (t_1' \leq t_1 \; \mbox{or} \; t_1' > t_2)) 
%   \}
% \\\\
\semantics{x{:}\feed_1 * \feed_2}{\environment}{\universe} 
 &=&
 \{(\mytime{\meta_2},(\meta_1,\meta_2)),(v_1,v_2)) \setalt 
     (\meta_1,v_1) \in \semantics{\feed_1}{\environment}{\universe} 
     \; \mbox{and} \; 
\\&&\qquad\qquad\qquad\ \ \,
     (\meta_2,v_2) \in \semantics{\feed_2}{(\environment,x\mapsto{}v_1)}{\universe}
     \; \mbox{and} \; \mytime{\meta_2} > \mytime{\meta_1}
  \}
\\\\
\semantics{x{:}\feed_1 \, {*}{*} \, \feed_2}{\environment}{\universe} 
 &=&
 \{(\mytime{\meta_2},(\meta_1,\meta_2)),(v_1,v_2)) \setalt 
     (\meta_1,v_1) \in \semantics{\feed_1}{\environment}{\universe} 
     \; \mbox{and} \; 
\\&&\qquad\qquad\qquad\ \ \,
     (\meta_2,v_2) \in \semantics{\feed_2}{(\environment,x\mapsto{}v_1)}{\universe}
     \; \mbox{and} \; \mytime{\meta_2} > \mytime{\meta_1}
\\&&\qquad\qquad\qquad\ \ \,
     ((\meta_1',v_1') \in \semantics{\feed_1}{\environment}{\universe} 
      \; \mbox{implies} \; (\mytime{\meta_1'} \leq \mytime{\meta_1} 
            \; \mbox{or} \; \mytime{\meta_1'} > \mytime{\meta_2})) 
  \}
\\\\
{\cal F}\lsem
\mathtt{foreach{*}}\; x \; \mathtt{in}\; \feed_1 
%\semantics{\foreachupdate{x}{\feed_1}{\feed_2}}{\environment}{\universe} 
 &=&
 \{(\meta_2,v_2) \setalt 
     (\meta_1,v_1) \in \semantics{\feed_1}{\environment}{\universe} 
     \; \mbox{and} \; 
\\
\qquad\qquad\quad\ \, \mathtt{create}\; \feed_2 \rsem_{{\environment} \, {\universe}}
&&\qquad\qquad\ \,
     (\meta_2,v_2) \in \semantics{\feed_2}{(\environment,x\mapsto{}v_1)}{\universe}
     \; \mbox{and} \;
%\\&&\qquad\qquad\ \,
     \mytime{\meta_2} > \mytime{\meta_1} 
  \}
\\\\
{\cal F}\lsem
\mathtt{foreach{*}{*}}\; x \; \mathtt{in}\; \feed_1 
%\semantics{\foreachcreate{x}{\feed_1}{\feed_2}}{\environment}{\universe} 
 &=&
 \{(\meta_2,v_2) \setalt 
     (\meta_1,v_1) \in \semantics{\feed_1}{\environment}{\universe} 
     \; \mbox{and} \; 
\\
\qquad\qquad\quad\ \, \mathtt{extend}\; \feed_2 \rsem_{{\environment} \, {\universe}}
&&\qquad\qquad\ \,
     (\meta_2,v_2) \in \semantics{\feed_2}{(\environment,x\mapsto{}v_1)}{\universe}
     \; \mbox{and} \; \mytime{\meta_2} > \mytime{\meta_1} \; \mbox{and} \;
\\&&\qquad\qquad\ \,
     ((\meta_1',v_1') \in \semantics{\feed_1}{\environment}{\universe} 
      \; \mbox{implies} \; (\mytime{\meta_1'} \leq \mytime{\meta_1} 
           \; \mbox{or} \; \mytime{\meta_1'} > \mytime{\meta_2}))      
  \}
\\\\
% \semantics{\ppfeed{\feed}{e}}{\environment}{\universe} 
%  &=&
% \semantics{\feed}{\environment}{
%   (\lambda x{:}\locty * \timety. \esemantics{e}{\environment} (x,\universe(x)))} 
% \\\\
% \semantics{\remapfeed{\feed}{e}}{\environment}{\universe} 
%  &=&
% \semantics{\feed}{\environment}{(\universe \circ \esemantics{e}{\environment})}
% \\\\
%% \semantics{\refeed{\feed}{e}}{\environment}{\universe} 
%%  &=&
%% \{(\atime,\some{v}) \setalt 
%%    (\atime,v) \in \semantics{\feed}{\environment}{\universe} \; \mbox{and} \;
%%    \atime \in \esemantics{e}{\environment}
%% \} \bigcup
%% \\&&
%% \{(\atime,\none) \setalt
%%    (\atime,\_) \not\in \semantics{\feed}{\environment}{\universe} \; \mbox{and} \;
%%    \atime \in \esemantics{e}{\environment}
%% \}
%% \\\\
%% \semantics{\stutterfeed{\feed}{e}}{\environment}{\universe} 
%%  &=&
%% \{(\atime,v) \setalt 
%%    (\atime,v) \in \semantics{\feed}{\environment}{\universe} \; \mbox{and} \;
%%    \atime \in \esemantics{e}{\environment}
%% \} \bigcup
%% \\&&
%% \{(\atime,v) \setalt 
%%    (\atime',v) \in \semantics{\feed}{\environment}{\universe} \; \mbox{and} \;
%%    \atime \in \esemantics{e}{\environment}  \; \mbox{and} \;
%% \\&&\qquad\qquad\qquad\ \ \,
%%     \mbox{for all $\atime''$ such that $\atime' < \atime'' \leq \atime$,} \;
%%    (\atime'',\_) \not\in \semantics{\feed}{\environment}{\universe} \; 
%% \}

\end{array}
\]
\caption{Feed Language Semantics.}
\label{fig:semantics}
\end{figure*}


\subsection{Programming Interface}
{\small 
\begin{verbatim}
(* these are interfaces for clean feeds *)

type 'a feed = 'a Feed.ifeed
type 'a prefeed = 'a Combinators.CreatorTys.feed_rep

type path = string list

type sep = string option

(* create a new feed from a prefeed spec *)
val to_feed : 'a prefeed -> 'a feed

(* get the next item from the feed *)
val next : 'a feed -> ('a * 'a feed) option

(* filter a feed *)
val filter : ('a -> bool) -> 'a feed -> 'a feed

(* iterate a function over a feed *)
val iter : ('a -> unit) -> 'a feed -> unit

(** Map a function on the feed *)
val map : ('a -> 'b) -> 'a feed -> 'b feed

(** fold over a feed *)
val fold : ('a -> 'b -> 'a) -> 'a -> 'b feed -> 'a

(* take the first n elements of a feed *)
val take : int -> 'a feed  -> 'a feed

(* take elements from the front until f x is true *)
val takeuntil : ('a -> bool) -> 'a feed -> 'a feed

(* drop the first n elements from a feed *)
val drop : int -> 'a feed -> 'a feed

(* drop the front elements till f is true *)
val dropuntil : ('a -> bool) -> 'a feed -> 'a feed 

(* slice a feed at a position *)
val sliceat : int -> 'a feed -> 'a feed * 'a feed

(* slice the stream when some predicate is true *)
val sliceuntil : ('a -> bool)  -> 'a feed -> 
	'a feed * 'a feed

(** Flatten a feed of list into a feed*)
val flatten : ('a list) feed -> 'a feed

(** Create a feed of one element *)
val singleton: 'a -> 'a feed


(*** The following are some functions for efeed: 
feeds with network errors ***)

type 'a efeed = 'a Feed.efeed
type 'a edata = 'a EData.t

val to_efeed : 'a prefeed -> 'a efeed

val next_e : 'a efeed -> ('a edata * 'a efeed) option

val filter_e: ('a edata -> bool) -> 'a efeed -> 'a efeed

val iter_e : ('a edata -> unit) -> 'a efeed -> unit

val map_e: ('a edata -> 'b edata) -> 'a efeed -> 'b efeed

val fold_e: ('a -> 'b edata ->'a) -> 'a -> 
		'b efeed -> 'a
\end{verbatim}
}

\subsection{Tools Interfaces}
\begin{verbatim}

(* Archiver Tool *)

type arch_spec =
{
 arch_dir: string;
 log_file_name: string;
(* max num of files before doing compression *)
 max_file_count: int; 
 compress_files: bool;
}

val to_archive : arch_spec -> 
	'r Feedmain.prefeed -> unit

(* Feed2RRD Tool *)

type 'a rrd_spec =
{
  transform : 'a -> float;
  (** Name of RRD database. *)
  rrd_db : string;
  (** Name of data source that will be fed into rrd DB.*)
  ds_name : string;
  ds_type : string;
  (** Number of samples averaged to create each data entry. *)
  num_samples : int;
  (** Number of entries (averages) saved in database. *)
  num_slots : int;
  create_db : bool;
  rrd_exec : string
}
val default_rrd_exec : string

val to_rrd : 'a rrd_spec -> 'a Feedmain.feed -> unit

(* Accumulator Tool *)

val to_accum : 'r Feedmain.prefeed -> 
	'r Feedmain.feed -> (bool * bool * bool * bool) -> 
	('r -> bool) -> Accumulator.acc

val to_string : Accumulator.acc -> string

val to_file : Accumulator.acc -> string -> unit

(* Selector Tool *)

val select : path -> sep -> 'r prefeed -> 
	'r feed -> string feed

(* Printer Tool *)

(* this returns a feed of pairs, 
  where the first string in the pair is 
  dot separated path and second string 
  is the string rep of the data *)

val to_print_feed : sep -> 'a prefeed -> 
	'a feed -> ((path * string) list) feed
val to_file : sep -> 'a prefeed -> 'a feed -> 
	string -> unit  (* string is the file name *)
val to_files : sep -> 'a prefeed -> 'a feed 
	-> unit (* output to files  auto generated from paths *)
val to_string : sep -> 'a prefeed -> 
	'a feed -> string (* print whole feed to string *)

(* Feed2RSS Tool *)

type rss_spec = {
  title : string;
  link : string;
  desc : string;
  schedule : Time.t option;
  path : Feedmain.path
}

(* auguments are rss spec and the feed *)
val to_rss : rss_spec -> 'r Feedmain.prefeed -> 
	'r Feedmain.feed -> unit

\end{verbatim}

{\em Note: the following was taken from a grant proposal written with Vivek
Pai.  The actual text below was likely largely written by Vivek and hence
we should rewrite it and/or include Vivek as an author.}

One of the oldest and most widely-used protocols for general monitoring
is SNMP, the simple network management protocol~\cite{snmprfc1157},
which is supported by commercial tools such as HP's
OpenView~\cite{openview} and free tools such as MRTG~\cite{mrtg}. It
provides an open protocol format that can be used to monitor a variety
of different types of equipment, using a vendor-supplied management
information base (MIB) that provides the specifics of the kinds of
monitoring provided by each piece of hardware. SNMP's hierarchical
MIBs plus associated control software, while flexible, have many of
the same drawbacks as XML -- space, complexity, and poor support for
ad hoc data.

For Grid environments, a popular monitoring tool is
Ganglia~\cite{ganglia}, which has also been adapted for use with
PlanetLab. It presents much of the system monitoring information
provided by OS tools like vmstat, iostat, uptime, etc. For data
transmission, Ganglia uses an XDR wire format, with raw data for all
of its native fields.  It can be extended by adding XML-encapsulated
fields for any other node-level measurements. 

What distinguishes this proposal from systems like SNMP or Ganglia is
that we want to be able to automatically parse and monitor virtually
any kind of ad hoc data, from node-level information like that
collected by Ganglia or SNMP, all the way down to application-level
data as well as protocol-level data. These areas are the ones that are
not well-served by today's general-purpose monitoring
systems. Moreover, the ability to use the same data description to
automatically build parsers, in-situ tools, and monitoring systems
represents an ease of use that we believe is not available in other
systems.

Another monitoring system of interest is PsEPR~\cite{psepr} (formerly
known as Trumpet), which focuses on finding problems via several tests
to gauge node health. What makes PsEPR interesting to consider is that
its design is completely decentralized, and all information is pushed
to all participating nodes via a publish/subscribe mechanism in the
Jabber protocol~\cite{jabber}. While this approach can be more
scalable in theory, it currently appears to be hitting the limits of
Jabber messaging servers. In the event that we decide to support
fully distributed monitoring (as opposed to replicated monitoring at
several sites), we will examine the lessons of PsEPR when deciding how
to proceed.

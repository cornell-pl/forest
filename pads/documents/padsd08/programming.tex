\section{Tools}
\label{sec:tools}
\begin{itemize}
\item tar, ...
\item Graph representation giving status
\item ??
\end{itemize}


\subsection{The Single-Minded Implementer}

In addition to the built-in tools, \padsd{} includes an API for
manipulating feeds created from a description. The API provides the
user with a feed abstraction representing a potentially infinite
series of elements. This abstraction is related to that of a lazy
list, but extends it with support for data timing and provenance
information. Therefore, the API provided for feeds is modeled on the
list APIs of common functional languages, like \ocaml and \haskell,
but provides two levels of abstraction. One level allows the user to
manipulate feeds like any other lazy list of data elements (ignoring
where they come from), while the other exposes the metadata along with
the data. 

% Need a new name for Feedmain module. I vote Feed and then Feed_core for the lower-level module.


\begin{figure}[tb]
\begin{codebox}
\kw{let} (sample, \_) = \textit{Feed.split_every} 600. comon \kw{in}
\kw{let} select_load = \kw{function}
    Some \{Comon_format.Source.
          loads = (_, load::_)\} -> Some load
  | None -> None \kw{in}
\kw{let} loads    = \textit{Feed.map} select_load sample \kw{in}
\kw{let} load_tbl = \textit{Feed.fold} update (create ()) loads 
\kw{in}  print_top 10 load_tbl
\end{codebox}
  \caption{Code fragment for finding \planetlab nodes with low load
  over the last 10 minutes (600 seconds).  \texttt{Feed.split\_every n}
  extracts the first \texttt{n} seconds of any feed. 
  Function \texttt{create} constructs an empty hash table; function 
  \texttt{update} adds
  an entry to the hash table; function
  \texttt{print\_top k} prints the \texttt{k} lowest loads from the table.}
\label{fig:sample-loads}
%\vskip -2ex
\end{figure}


%% \begin{figure}[tb]
%% \begin{codebox}
%% \kw{let} deadline = Time.now() +. 600. \kw{in}
%% \kw{let} (sample, remainder) = \textit{Feed.split_when} 
%%    (fun () -> Time.now() > deadline) comon_feed \kw{in}
%% \kw{let} select_load = \kw{function}
%%     Some \{Comon_format.Source.
%%           loads = (_, load::_)\} -> Some load
%%   | None -> None \kw{in}
%% \kw{let} loads = \textit{Feed.map} select_load sample \kw{in}
%% \kw{let} load_tbl = \textit{Feed.fold} update (create ()) loads 
%% \kw{in}  print_top 10 load_tbl
%% \end{codebox}
%%   \caption{Code fragment for sampling \planetlab loads for 10 minutes. Function
%%   \texttt{print\_top k} selects the \texttt{k} lowest loads from the table.}
%% \label{fig:sample-loads}
%% %\vskip -2ex
%% \end{figure}

\begin{figure}[tb]

\begin{codebox}
\kw{let} update tbl idata =
  \kw{let} meta = IData.get_meta idata \kw{in}
  \kw{let} data = IData.get_contents idata \kw{in}
  \kw{match} meta, data \kw{with} 
    (h, Some basemeta), Some load ->
      \kw{let} location = Meta.get_link basemeta \kw{in}
      update tbl location data
  | _ -> tbl \textit{ (* no change to tbl *)}
\kw{in} ...
\kw{let} load_table = \textit{\textbf{Feed.fold_p}} update empty_tbl loads
\kw{in} print_top_10 load_table
\end{codebox}
  \caption{Revised code fragment which exploits provenance metadata. }
\label{fig:sample-loads-prov}
%\vskip -2ex
\end{figure}

For example, consider a \planetlab user looking for a desirable set of
nodes on which to run their experiments. Based on the \comon
description, they can use our API to monitor \planetlab for a few
minutes to find the least loaded nodes. In \figref{fig:sample-loads},
we show an \ocaml code fragment that collects a list of nodes with
lowest average loads over $10$ minutes, and then prints them. We leave
out the exact details on maintaining the table of top values, as it is
orthogonal to our discussion. First, we use \cd{Feed.split_when} to
split the feed when 600 seconds (10 minutes) have elapsed. Then, we
use \cd{Feed.map} to project the loads data from the\comon
elements. Finally, we use \cd{Feed.fold} to collect the load values
into a table after which we can average the data on a per node basis
and extract the $10$ lowest values.

However, this solution is not quite enough. The \comon data format
does not include the host name in the data itself, so the code in
\figref{fig:sample-loads} will only be able to return the lowest
average loads, but not the names of the machines associated with
them. In situations like this, the provenance data is essential.  In
\figref{fig:sample-loads-prov}, we replace the last three lines of
\figref{fig:sample-loads} with a call to the lower-level fold,
\cd{fold_p}, which provides both the data and metadata to the folding
function. We also sketch an \cd{update} function which makes use of 
the metadata.   Notice the use of the lower level \cd{IData} and
\cd{Meta} interfaces to facilitate mangement of both data and 
metadata from the feed. 

It should be clear from these examples that the single-minded implementer
has a number of new interfaces to master relative to the quick-and-dirty
hacker, but gains a correspondingly higher degree of flexibilty and can
still write relatively concise programs.

\subsection{The Generic Programmer}

% Motivate

Occasionally, a user might want to develop a function that can
manipulate {\it any} feed. This desire might arise because the user
has a number of different feeds to process in the same way, or simply
because the user wishes to provide a new tool for other \padsd{}
users. Some such functions can be written simply by designing them to
be parametric in the type of the feed element, much like the feed
library functions discussed above. However, the behavior of many feeds
functions will depends on the structure of the feed and its
elements. Such functions can be viewed as {\it interpretations} of
feed descriptions, so, to support their development, we provide a
framework for writing feed interpreters.

% Examples

Two core examples of feed interpretations are the feed creator and the
feed accumulator. The behavior of both of these tools depend
essentially on the structure of the feed. These functions, and others
like them, require as input a runtime representation of the feed,
complete with all of the details included in the feed description that
they represent. The obvious choice for representing feed descriptions
in \ocaml is a datatype. However, standard \ocaml datatypes are not
sufficiently typeful to express the types of many generic feed
functions. For example, the feed creation function has the type:
\begin{code}
feed_create : 'a prefeed -> 'a feed
\end{code} where \cd{'a prefeed} is an AST of a feed description and feed 
elements have type \cd{'a}.
%
% Tools as interpreters
%
This limitation of datatypes has been widely discussed in the
literature, and various solutions have been 
proposed~\cite{yang:icfp98,weirich:encodingtypecase,hinz:icfp04,padsml-padl}. We have 
chosen to represent our AST using a variant of the Mogensen-Scott
encoding in which higher-order abstract syntax is exploited 
to encode variable binding in feed descriptions.  This implementation strategy 
exploits \ocaml's module system to type the encodings in $F_\omega$. 
A similar strategy is exploited in our earlier work on \padsml~\cite{padsml-padl}, 
but there we only sought to encode the \ocaml type of data, not entire \padsml
description, which is where the higher-order abstract syntax enters the picture.
% To effectively encode the dependency present in feeds
% descriptions, we instead 
% encodings~\cite{mogensen}, employing HOAS to encode variable binding
% in feed descriptions.
% Given that \ocaml is our target language, we follow
% the method used in \padsml{} to encode type
% representations~\cite{padsml-padl}. Because of space limitations, we
% do not provide details here.  

The result of our work is that developers
can interpret feed-description representations by case analysis on
their structure, while still achieving the desired static
guarantees. Moreover, we have successfully used this framework to
develop {\it all} of the tools presented in this paper, including the
feed creator. The only role of the compiler is to infer appropriate
type declarations from feed descriptions, and compile the feed syntax
into our representations.  However, as one might expect, interfaces using
higher-order abstract syntax and Mogensen-Scott encodings are one step more
complex than those involving the more familiar maps and folds.  Consequently, the
learning curve for the generic programmer is one step steeper than
the curve for the single-minded implementor, and two (or perhaps ten) steps steeper
than the curve for the quick-and-dirty hacker.

% There, we represent our AST using a
% Scott-encoded datatype, and leverage \ocaml's module system to, in
% essence, type the Scott encodings in $F_\omega$. However, in \padsml
% we only sought to encode the \ocaml type of data, not entire \padsml
% descriptions. To effectively encode the dependency present in feeds
% descriptions, we instead use a variant of Mogensen-Scott
% encodings~\cite{mogensen}, employing HOAS to encode variable binding
% in feed descriptions.





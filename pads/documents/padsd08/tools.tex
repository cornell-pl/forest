\subsection{Using the Tools Library}
\label{sec:tools}
The \padsd{} system comes with a suite of useful tools that help users 
cope with standard data adminstration needs. A tool is typically an
O'caml module with a standard interface: a tool specification, a
feed representation or prefeed compiled from the feed description,
and a feed of interest. We now briefly introduce them and discuss their uses. 

\begin{figure*}[th]
\centering
\begin{codebox}
===========================================================================
Summary of network transmission errors
===========================================================================
ErrCode: 1      ErrMsg: Misc HTTP error Count: 12
ErrCode: 5      ErrMsg: Bad message     Count: 27
ErrCode: 6      ErrMsg: No reply        Count: 2

===========================================================================
Top 10 locations with most network errors
===========================================================================
Loc: http://planetlab01.cnds.unibe.ch:3121    Count: 2
Loc: http://pepper.planetlab.cs.umd.edu:3121  Count: 2
Loc: http://planetlab3.cs.uchicago.edu:3121   Count: 2
... omitted ...
===========================================================================
<top>.<listitem>.Cpu_hog.elt1.Some.elt2 : string
===========================================================================
good vals:      554     bad vals:       0       pcnt-bad:       0.00%
min val: 4      max val: 22     avg val: 11     std dev: 5.29   

=> Distribution of the top 10 values out of 47 distinct values:
    Val: "nyu_d"                    Count: 111     pcnt-of-good-vals: 20.04%
    Val: "ethzcs_q"                 Count: 86      pcnt-of-good-vals: 15.52%
    Val: "princeton_coblitz"        Count: 68      pcnt-of-good-vals: 12.27%
    ... omitted ...
    Val: "root"                     Count: 12      pcnt-of-good-vals: 2.17%
. . . . . . . . . . . . . . . . . . . . . . . . . . . . . . . . . . . . . .
    SUMMING                         Count: 434     pcnt-of-good-vals: 78.34%
\end{codebox}
\caption{Portion of an accumulator output} \label{fig:acc}
\end{figure*} 

\begin{itemize}
\item {\em Archiver.} The archiver takes a feed, and archives all the files
remotely fetched by this feed into the local file system. The archive
is organized by the structure of the feed, one directory per base feed,
where each directory stores all the files fetched by a base feed. 
It also keeps a catalog file in each directory that documents where 
the data come from, when the data was scheduled to arrive and when
the data actually arrived. The user has the option of compressing
the files in each directory when they reach certain number. 

\item {\em Printer.} The printer prints the content of a feed to a file or
multiple files. When it prints to a single file, all data from a feed is
concatenated with an optional separator. When it is configured to print
to multiple files, it outputs the content of each base feed into a separate
file.

\item {\em Profiler.} The profiler is a performance monitor of the feeds which
provides the user with the throughputs, the average network and system latencies
of the feed over a period of time. Users can specify when to profile the feed and
for how long using the tool's specification. This tool has been used to produce
some of the experimental results in section \ref{sec:experiments}.

\item {\em Accumulator.} The accumulator is one of the most useful tools to system
adminstrators. It maintains a statistical profile of the feed and
its error characteristics. For numerical data, aggregated information such as 
average values, max/min values and standard deviations are maintained. For other
types of data such as strings, URL's and ip addresses, the counts of the top N
most frequently seen values are presented. Error rates and most common errors
and where they come from are also maintained. The tool can be configured to
accumulate the information over the entire feed or its slices. The latter is
useful if the feed is infinite so the user can constantly monitor the 
current situation and compare with historical statistics. 
The output the tool can be either plain text and XML with an
accompanying DTD file. Figure \ref{fig:acc} shows portions of an accumulator
output for the Comon example in text format.

\item {\em Alerter.} The alerter allows the user to specify certain conditions
upon which an alert will be raised. The library has several predefined
conditions such as value exceeding the max/min or $n$ standard deviation away. 
Users can also define custom conditions by specifying an arbitrary predicate. 
This tool has been integrated with the accumulator above. Alerts are sent by
appending to a file which can then be piped to other tools such as email
client.

\item {\em Database loader.} The DB loader allows the user to load numerical
data from a feed into a round robin database (RRD tool). User needs to specify
a custom function that transform the feed items into numerical values
and also some other parameters required by the rrdtools, such as datasource
type, sampling rate. The data is indexed by arrival time in the RRD and
older data is discarded to make space for new data.

\item {\em Selector.} The selector is a simple query tool that selects
into the data source in a feed, and returns a feed of the subcomponents 
such as a field of a record. The query language is a path similar to XPath
and defined as follows.

{\small
\begin{verbatim}
path :: =  
  "top"
| path.ID  (field/variant name)
| path.INT (branch number (from 1) of a tuple)
| path.[?] (any one elements of array/table) 
| path.[*] (all elements of array/table) 
| path.[INT] (nth element of array (from 0))
| path.[Key] (a table entry indexed by the Key)
\end{verbatim}
}

\item {\em RSS feed generator.} The RSS feed generator converts a given
\padsd{} feed into properly formatted RSS feed in XML. User specifies
the title, link, description as well as update schedule of this RSS feed.
The content of the RSS feed is specified by a selection path into
the data source. The tool uses the selector tool to
select into the data source and returns the subcomponent which is then
wrapped in XML and presented as an RSS feed.
\end{itemize}

The most straight-forward use of the \padsd{} system is to create a feed 
based on a \padsd{} description, and then apply one or more of built-in tools 
to the feed. \padsd{} provides a very 
easy-to-use interface, a tool configuration, to accomplish this simple task. 
Figure~\ref{fig:toolconfigs} presents an example tool config file.

\begin{figure}[tb]
\centering
\begin{codebox}
\kw{feed} comon.fml/comon

\kw{tool} feedaccum
\{
  minalert  = true;
  maxalert  = true;
  lesssig   = Some 3;
  moresig   = Some 3;
  useralert = (fun x -> false);
  slicesize = Some 10;
  slicefile = Some "slice.acc";
  totalfile = Some "total.acc"
\}

\kw{tool} rss
\{
  title = "Comon RSS";
  link  = "http://www.cs.princeton.edu/~kzhu/rss.xml";
  desc  = "Memory Info of Monall";
  sched = None; 
  path  = ["[?]"; "Mem_info"];
  rssfile = None; 
\}
\end{codebox}
\caption{Example Tool Configuration File}
\label{fig:toolconfigs}
\end{figure}

A tool configuration includes a feed declaration header 
(the first line in the config),
and a sequence of tool specifications. The header specifies the path to the 
feed description file (comon.fml) and the name of the feed 
(described in the fml file) to be created (comon).
Each tool specification starts with the the keyword {\tt tool} followed by the
name of the tool (feedaccum and rss for example). Only system built-in tools may be 
included in the tool config file. The body of each tool specification is essentially
an O'Caml record. The exact syntax and meanings of these tool specs can be found in
the \padsd{} library. 

Once the tool config is edited, it can be compiled into an O'Caml user program
that calls the \padsd{} tools library properly. The program can then be built with the the
fml description and (optionally) \padsml{} descriptions into the final executable
code. At runtime, the specified feed will be created and archived, and then the
configured tools will be applied to the feed in parallel. This is possible because feeds in
\padsd{} are functional and immutable. Typically the tools process the data items
in the feed, produce an output, and return unit type. We apply the tools in parallel to
achieve maximum throughput and also because some of the tools, such as feedaccum,
operate on a per feed basis, rather than a per item basis. And thus running the tools
in pipeline is not optimal.

In summary, the tool configuration in \padsd{} is a very simple programming 
paradigm where minimum O'Caml language knowledge is required. Feed designers
and system administrators who hardly know any functional programming can still 
perform day-to-day data management tasks. For users who do know O'Caml programming, 
more control is available by programming directly against the \padsd{} 
tools library, which will be discussed in the next section.


%\documentclass[fleqn]{article}
\documentclass[nocopyrightspace]{sigplanconf}

\usepackage{xspace,amsmath,math-cmds,
            math-envs,inference-rules,times,
            verbatim,alltt,multicol,proof,url}
\usepackage{epsfig}
\usepackage{code} 
%\setlength{\oddsidemargin}{0in}
%\setlength{\evensidemargin}{0in}
%\setlength{\textwidth}{6.5in}
%\setlength{\textheight}{8.5in}

\begin{document}
\title{Automatic Tool Generation for Monitoring Distributed Systems}

\authorinfo{
Daniel S. Dantas$^1$ \quad
Kathleen Fisher$^2$ \quad
Limin Jia$^1$ \\
Yitzhak Mandelbaum$^2$ \quad
David Walker$^1$ \quad
Kenny Q. Zhu$^1$
}{
$^1$ Princeton University \\
$^2$ AT\&T Labs Research
}{}

%% \authorinfo{Kathleen Fisher}{
%% 	   AT\&T Labs Research}
%%        {\mono{kfisher@research.att.com}}
%% \authorinfo{Yitzhak Mandelbaum}{
%% 	   AT\&T Labs Research}
%%        {\mono{yitzhakm@research.att.com}}
%% \authorinfo{David Walker}{
%% 	   Princeton University}
%%        {\mono{dpw@CS.Princeton.EDU}}
%% \authorinfo{Kenny Q. Zhu}{
%%            Princeton University}
%%        {\mono{kzhu@CS.Princeton.EDU}}

\newcommand{\cut}[1]{}
\newcommand{\reminder}[1]{{\it #1 }}
\newcommand{\poplversion}[1]{#1}
\newcommand{\trversion}[1]{}

\newcommand{\appref}[1]{Appendix~\ref{#1}}
\newcommand{\secref}[1]{Section~\ref{#1}}
\newcommand{\tblref}[1]{Table~\ref{#1}}
\newcommand{\figref}[1]{Figure~\ref{#1}}
\newcommand{\listingref}[1]{Listing~\ref{#1}}
%\newcommand{\pref}[1]{{page~\pageref{#1}}}

\newcommand{\eg}{{\em e.g.}}
\newcommand{\cf}{{\em cf.}}
\newcommand{\ie}{{\em i.e.}}
\newcommand{\etc}{{\em etc.\/}}
\newcommand{\naive}{na\"{\i}ve}
\newcommand{\role}{r\^{o}le}
\newcommand{\forte}{{fort\'{e}\/}}
\newcommand{\appr}{\~{}}

%\newcommand{\bftt}[1]{{\ttfamily\bfseries{}#1}}
\newcommand{\kw}[1]{\bftt{#1}}
\newcommand{\pads}{\textsc{pads}}
\newcommand{\padsc}{\textsc{pads/c}}
\newcommand{\ipads}{\textsc{ipads}}
\newcommand{\padsl}{\textsc{padsl}}
\newcommand{\blt}{\textsc{blt}}
\newcommand{\ddc}{\textsc{ddc}$^{\alpha}$}
\newcommand{\ddcold}{\textsc{ddc}}
\newcommand{\padsml}{\textsc{pads/ml}}
\newcommand{\padsmlbig}{\textsc{PADS/ML}}
\newcommand{\ddl}{\textsc{ddl}}
\newcommand{\C}{\textsc{c}}
\newcommand{\perl}{\textsc{perl}}
\newcommand{\ml}{\textsc{ml}}
\newcommand{\smlnj}{\textsc{sml/nj}}
\newcommand{\ocaml}{\textsc{o'caml}}
\newcommand{\java}{\textsc{java}}
\newcommand{\xml}{\textsc{xml}}
\newcommand{\xquery}{\textsc{xquery}}
\newcommand{\datascript}{\textsc{datascript}}
\newcommand{\packettypes}{\textsc{packettypes}}
\newcommand{\erlang}{\textsc{Erlang}}

\newcommand{\dibbler}{Sirius}
\newcommand{\ningaui}{Altair}
\newcommand{\darkstar}{Regulus}

%% \newcommand{\IParray}[4]{{\tt Parray} \; #1 \; \[#2, #3, #4\]}

\newcommand{\figHeight}[4]{\begin{figure}[tb]
	\centerline{
	            \epsfig{file=#1,height=#4}}
	\caption{#2}
	\label{#3}
	\end{figure}}


\maketitle{}

\begin{abstract}  
Many applications use the file system as a simple persistent data
store.  This approach is expedient, but it is not robust.  In general,
the overall correctness of such an application dependso on the
collection of files, directories, and symbolic links having some
precise hierarchical organization. Furthermore file system properties
such as file ownership, permissions, and timestamps must have
acceptable values. Unfortunately, current programming languages do not
provide support for documenting assumptions about the file system. In
addition, actually loading data from disk requires writing tedious
boilerplate code.

This paper describes \forest{}, a new domain-specific language for
describing directory structures embedded in \haskell{}. \forest{}
descriptions use a type-based metaphor to specify portions of the file
system in a simple, declarative manner.  \forest{} makes it easy to
connect data on disk to an isomorphic representation in memory
that can be manipulated by programmers as if it were any other
data structure in their program.  \forest{} generates
metadata that describes to what degree the files on disk conform to
the specification, making error detection easy. The system greatly
lowers the divide between on-disk and in-memory representations of
data. \forest{} leverages \haskell{}'s powerful generic programming
infrastructure to make it easy for third-party developers to build
tools that work for any \forest{} description.  We illustrate the use
of this infrastructure to build a number of useful tools, including a
visualizer, permission checker, and description-specific tools for a
number of standard shell tools.

\cut{
We present the design for \forest{} and describe the implementation of
a full working prototype. From a single compact description, the
\forest{} compiler generates a collection of \haskell{} types and
functions for validating and analyzing file system data.  In addition,
\forest{} generates type class instance declarations that make it
possible to exploit powerful generic programming paradigms that allow
third-party developers to build tools for querying, visualizing, and
debugging on-disk data in a generic way. We present examples
illustrating the use of \forest{} on a number of real-world directory
structures and programming tasks, including description-specific
replacements for a number of standard shell tools. Finally, we
formalize the core elements of the language as a simple calculus based
on classical tree logics.}
\end{abstract}


\section {Introduction}
\label{sec:intro}

An {\em ad hoc data source} is any semistructured data source for
which useful data analysis and transformation tools are not readily
available.  The data that constitutes a single, abstract source 
often comes from many different concrete, physical destinations
distributed across the Internet.  It may also become available
over a range of times and in several, evolving formats.
Before the users can extract the information they need from the data,
it usually must be fetched, archived locally for historical analysis, 
compressed, perhaps encrypted or anonymized, and monitored for errors 
or deviations from the norm.

Managing ad hoc data 
is a particular bane of the implementers of distributed systems.
Depending on the size, these systems may have hundreds or thousands
of heterogenous, distributed components.  Keeping all these components
up and running is an enourmous continuous maintenance task.  Consequently,
each component in a well-designed system will produce endless log files
that measure its performance and heath.  As an example, consider the data
manipulated by CoMon~\cite{comon}, a system designed to monitor the
health, performance and security of PlanetLab~\cite{planetlab}.  Every
five minutes, CoMon attempts to contact each of 842 PlanetLab nodes
across 416 sites worldwide.\footnote{Data as of writing this manuscript;
PlanetLab membership varies over time.}  
When all is running smoothly, which it
never is, each node responds by sending back an ASCII data file
in mail-header format containing information ranging from
the kernel version to the uptime to the memory usage to the id of the
user with the greatest CPU utilization.  CoMon archives this data in
compressed form and its backend processes the information for display
to PlanetLab users.  It is an invaluable resource for Planet users
who need to monitor the health and performance of their applications
or experiments.  

Almost all distributed systems have (or should have) similar sorts of
monitoring infrastructure.  Unfortunately, system implementers are
often left to hack ``one-off'' monitoring tools of their own, which
are invariably less reliable, unoptimized, insecure, and difficult or
impossible to evolve when new requirements become known.  A
substantial part of the difficulty simply comes from the diversity,
quality, and volume of data these systems must often handle. Often,
new monitoring systems also face the problem of having to interact
with legacy devices, legacy software and legacy data, leaving
implementers in a situation where they cannot use robust off-the-shelf
data management tools built for standard formats like XML.  XML-based
tools also have the disadvantage of significant bloat (often 8-10 times
the size of a more natural, even uncompressed representation) caused by
using a generic reprensentation.

Somewhat similar problems also appear
across the natural and social sciences, including biology,
physics and economics.  For example, systems such as BioPixie~\cite{biopixie}, Grifn~\cite{grifn} and Golem~\cite{golem}, built by
computational biologists at Princeton,
routinely obtain data supplied from a number of sources scattered 
across the net.  The data is often archived and later analyzed or mined 
for valuable information about gene structure and regulation.  Likewise, 
cosmologists need access to data uploaded from major telescopes~\cite{sdss}
and economists can make use of vast data repositories at FedStats.org
amongst other sources.  These and other selected ad hoc data sources 
are presented in Figure~\ref{fig:exampledata}.

\begin{figure*}
\begin{center}
\begin{tabular}{|l|l|l|}
\hline\hline
Name & Use & Properties 
\\\hline\hline
CoMon~\cite{comon} & PlanetLab host monitoring & Multiple data sets in mail-header formats\\
                                       && Archiving every 5 minutes \\
                                       && From evolving set of ~800 nodes \\\hline
CoBlitz~\cite{coblitz} & File transfer system monitoring & Multiple data sets \\
                                       && Archiving every 5 minutes \\
                                       && From evolving set of ~800 nodes \\\hline
CoralCDN~\cite{coral} & Log files from CDN monitoring & Single Format \\
                                       && Periodic archiving \\
                                       && From evolving set of ~250 hosts \\\hline
\vizGems{}       & Website Host Monitoring & Many and varied machines \\
                 &                         & Execute programs remotely\\
                 &                         & to collect data\\\hline
\darkstar{}      & AT\&T network monitoring & Archiving for future analysis \\\hline
\ningaui{}       & AT\&T billing auditing   & Thousands of data sources\\
                 &                          & Archiving and error analysis\\\hline
GO DB (Gene Ontology)~\cite{geneontology} & Gene Function Information & Multiple Formats \\
                                             && Uploaded in daily, weekly, monthly intervals \\\hline
BioGrid~\cite{biogrid} & Curated Gene and Protein Data & XML and Tab-separated Formats \\
          & & multiple data sets $<=$ 50MB each \\
          & & monthly data releases \\\hline
NCBI~\cite{ncbi} & National Center for Biotechnology Information & Links to multiple bioinformatics datasets \\
                                                     && and online databases\\
\hline\hline
\end{tabular}
\end{center}
\caption{Example ad hoc data sources}
\label{fig:exampledata}
\end{figure*}

The purpose of our research is to develop a system that makes it easier to create,
maintain, and evolve tools for monitoring such distributed systems.  We propose to
do so by developing a domain-specific language, called \padsd, in which software developers specify
a number of key aspects of the data sources they wish to monitor including any of the
following.

\begin{itemize}
\item {\bf where} the data is located.  The data may be in some directory
on the current machine (perhaps placed there by another process) or at some remote location
or collection of locations.
\item {\bf when} to get the data.  The data may need to be fetched just once (right now!) or
according to some repeated schedule in time series indexed by minutes, days or months.
\item {\bf how} to obtain it.  The data may be accessible through standard protocols such as
http or ftp or it may be created through remote execution of a non-standard script. 
\item {\bf what preprocessing} the system should do when it arrives.  The data may be compressed
or encrypted and therefore need to be decompressed or decrypted before it can be processed.  Privacy 
considerations may require the data be anonymized in some way.
\item {\bf what format} the data source arrives in.  The data may be ASCII or binary; it may
be tab- or comma-separated.  The data may also be represented in some completely ad hoc, non standard 
format consisting of floating point numbers, integers, strings, vertical bars and curly-q's.  
\end{itemize}

These rich, high-level specifications are then compiled into a collection of programming libraries and
end-to-end tools for distributed systems monitoring.  Our current tool suite includes a number of useful
artifacts, inspired by the common needs we have observed in ad hoc monitoring systems:

\begin{itemize}
\item {\bf an archiver} that
collects distributed data on the specified schedule, archives it locally, and maintains a 
``table of contents.''
\item {\bf a database loader} that takes the data and extracts specified pieces to load into the
RRD database tool~\cite{rrdtool}.  The data is indexed by its arrival time and supports time-based
queries.  For performance, as more recent data arrives, older data is discarded.
\item {\bf an accumulator tool} that maintains a statistical profile of the data and its error characteristics.
For numeric data, information about average values and standard deviations are maintained.
For other kinds of data, such as strings, urls, ip addresses, times, dates, and ad hoc enumerations,
information counts of the top $N$ most commonly occurring items are maintained.  For all data, error rates
and information about common errors are maintained.
\item {\bf an alert system} that generates alerts based on programmable conditions.
\item {\bf a selector tool} that extracts and records specified subcomponents of a larger data source.
\item {\bf an RSS feed generator} that wraps data in the appropriate XML headers and creates a single RSS feed
from possibly multiple diverse ad hoc data sources.
\end{itemize}

In addition to these standard tools, the system provides support for
creating new tools by automatically generating a collection of
libraries.  The libraries include a run-time system for fetching data,
libraries for parsing data in a specified format and for printing data
in that format.  There is also infrastructure for type-safe data
traversal and stream processing using classic functional programming
paradigms such as map, fold and iterate.  The generated libraries make
it straightforward for programmers to create their own custom tool
specific to a single data source or collection of sources.  In
addition, there is advanced support for creating new, {\em generic}
programs, where a generic program is one that operates correctly over
{\em any} well-specified data source.  For example, the RRDtool loader
is generic, because it is possible to load data from any specified
source into the RRDtool without doing any substantial additional ``programming.''
Likewise, the alert system, selector, RSS feed generator, parsers,
printers and traversal libraries are all generic programs.

In the remainder of the paper, we will explain the design and
implementation of our system in further detail.  First, in Section~\ref{sec:related}
we explain the relationship with other research in this area.  Next, in Section~\ref{sec:examples},
we will outline two running examples we will use for expository purposes throughout the paper, one 
involving CoMon, a system built at Princeton to monitor PlanetLab, and a second involving
\ningaui{}, built at AT\&T for monitoring AT\&T's web hosting service.  After introducing the examples,
we will explain how to specify the attributes of the data sources they depend upon in 
Section~\ref{sec:informal}.  In the next section, we will explain the architecture of our compiler
and tool generation system.  Finally, in Section~\ref{sec:conclusions} we will touch upon future work
and conclude.

\section{Related Work}
\label{sec:related}
Researchers have been studying {\em grammar induction}, the process of
inferring descriptions of text-based data, for decades.  Nevertheless,
the work we present in this paper represents an important and novel 
contribution to the field for three key reasons:

\begin{enumerate}
\item Our system solves {\em a new end-to-end problem} not treated in
past work --- the problem of generating an extensible suite of fully
functional data processing tools directly from ad hoc data.  
%%We can
%%currently generate an XML translator, a normalizing reformatter, a
%%graphing tool, a full query engine allowing users to write arbitrary XQueries
%%against the ad hoc data, an accumulator tool, and
%%programming libraries for parsing, printing and data validation.
Generating this suite requires the combination of three elements:
grammar induction, automatic intermediate representation generation
and type-directed programming.  A key contribution of this work is the
conception, development and evaluation of this end-to-end system.

%%After surveying
%%experts at the CAGI 2007 workshop on grammar induction, where we
%%presented a two-page overview of our system~\cite{burke+:cagi07}, and
%%searching the literature, we could find no existing system that
%%provides this end-to-end functionality.

\item Past work on grammar induction has focused primarily on
either (1) theoretical problems, (2) natural language processing, 
(3) web page analysis, or
(4) XML typing.  Our work tackles an understudied domain, that of complex system
logs and other ad hoc data sources.  Since ad hoc data has
different characteristics from the previously studied domains, naive
adaptations of the existing algorithms are unlikely to be %the most
effective.  
%%As the evaluation in this paper shows, our system is tuned
%%to perform well on ad hoc data, particularly system logs and
%%networking data.  
%%One of the conclusions of the chair of the CAGI 2007
%%workshop, presented in the final discussion session of the workshop,
%%was that ``ad hoc data'' was indeed a new domain for the study of
%%grammar induction and that more research in this area was an important
%%future direction for the community.

\item  From a technical standpoint, we developed a new top-down 
structure-discovery algorithm and showed how to combine that 
productively with a classic bottom-up rewriting system based on 
the minimum description length principle. We demonstrate that our
new algorithm has good practical properties on ad hoc data sources:  
it usually infers correct descriptions on a small amount of training
data and its performance scales linearly relative to the amount of training
data used.
\end{enumerate}

\noindent
%We presented a two-page overview of our system~\cite{burke+:cagi07} at
%the CAGI 2007 workshop on grammar induction. 
In the rest of this section, we analyze
the most closely related work in more depth.

\paragraph*{Traditional Grammar Induction.}
Classic grammar induction algorithms \cite{vidal:gisurvey} 
can be divided into two classes: those that require both
positive and negative examples to discover a grammar and those that
only require positive examples. The problem our system solves is the latter;
negative examples of ad hoc data sources are not available in
practice.  Consequently, effective theoretical algorithms for learning
from both positive and negative
examples such as RPNI~\cite{rpni}
%~\cite{lemay+:tree-transducers,rpni,raeymaekers+:learning-tree-languages},
are not applicable in our context.

Unfortunately, an early result by \citet{gold:inference} showed
that perfect grammar induction is impossible for any superfinite class
of languages when the algorithm has no access to negative examples.  A
{\em superfinite} class of languages is any set of languages that
includes all finite languages and at least one infinite
language. Hence, all the most familiar classes of languages, including
regular expressions, context free grammars and PADS are superfinite.
There are two main tactics one can use to avoid this negative
result: 
(1) use domain knowledge to explicitly limit the class of languages to a
non-superfinite class, or
(2) give up on perfect language identification and instead settle for {\em approximate
identification}~\cite{wharton:approximate-language-identification}
through the use of probabilistic language models.

Examples of non-trivial, non-superfinite
language classes with known inference algorithms include
k-reversible languages~\cite{angluin:revesible-language-inference},
%k-testable regular languages~\cite{garcia+:k-testable-languages},
SOREs and CHAREs~\cite{bex+:dtd-inference}.
None of these languages and the associated algorithms 
are a good fit for inferring PADS descriptions (even the
regular subset of PADS without dependencies and constraints).  
For example, ad hoc data is unlikely to be reversible and hence
k-reversible languages are not relevant.  
%K-testable regular languages are
%more relevant, but algorithms for inferring them
%operate by finding a finite automaton and converting that 
%automaton into a regular expression.  Unfortunately, the conversion process
%often leads to overly verbose regular expressions, sometimes 
%exponential in the size of the automaton~\cite{bex+:dtd-inference}. 
SOREs are a subset of the k-testable
regular languages with a linear-size translation from automata to
regular expressions, but they carry the restriction that each symbol
in the regular expression appear at most once.  A cursory glance at
our hand-written PADS descriptions reveals that many such descriptions
include repeated use of the same symbol.  Finally, it appears that
CHAREs restrict the nesting of regular expression operators too severely to 
be of much use to us.  For example, when $a$, $b$, and $c$ are atomic symbols,
even the simple expression $(ab + c)*$ is not a CHARE.

Given the difficulty of finding useful non-superfinite language classes,
it is reasonable to turn to algorithms for approximate
inference that use probabilistic models.    
Classic examples of such procedures include work by~\citet{stolcke94inducing} 
%%{\em insert other references here -- see Hong thesis related work
%%for other work...} 
and 
\citet{hong:thesis}.  These and a number of other algorithms
operate by repeatedly rewriting a candidate grammar (or set of candidate
grammars) until an objective function is optimized.
If the training data for the learning system is the strings
$s_1$, $s_2$, $\ldots$, $s_n$, these algorithms normally start their
process using the grammar $s_1 + s_2 + \cdots + s_n$.  Consequently,  
an enormous number of different rewrites may apply to the
initial candidate grammar.  Our structure refinement
phase avoids these problems 
because it is preceded by a highly efficient
histogram-based structure-discovery algorithm 
that identifies a good candidate grammar from which to start the search.  
%%Another interesting, non-standard element of our algorithm is the way 
%%it is tuned to include specialized rules for finding constraints and 
%%rewrite tokens.
%%These rules are very useful in the domain of ad hoc data; different
%%considerations are appropriate in other domains such as XML or HTML.

%% is also tuned in a variety of ways to make it effective

%% The effectiveness of structure-discovery allows us to 
%% simplify our search algorithm and cut down the search space we 
%% look at substantially.  In addition to worrying about
%% performance considerations, we tuned our structure refinement phase
%% specifically for ad hoc data by including domain-specific rules
%% for finding constraints and 
%% dependencies as well as those for introducing constants, enumerations and
%% good basic types/tokens that cannot be found effectively at earlier stages.
%% Some of these rules are needed in our system, but not other systems
%% that work in different domains, because
%% tokenization is highly ambiguous in ad hoc data.
%% Our initial tokenization and structure-discovery algorithms often 
%% over-generalize and this over-generalization must be undone during
%% the rewriting phase.  {\em NOTE: end of that paragraph was highly  run-on}

Another category of algorithms are those that learn various kinds of
automata as opposed to regular expressions or 
grammars~\cite{denis:learning-regular-languages,rpni,raeymaekers+:learning-tree-languages}.  
One difficulty with adapting these algorithms to our task is that 
we would need to convert the inferred automata into a 
grammatical representation so that we can
present the result to users and funnel it
to our tool-generation infrastructure.
Unfortunately, in theory, conversion from automata into
regular expressions can result in an exponential blowup in the
size of the representation.
Moreover, a substantial blowup appears to be relatively common in
practice~\cite{bex+:dtd-inference}.  Consequently, these algorithms
are not appropriate for our domain.



%% developed another system for information extraction from web pages
%% based on learning ($k$,$l$)-Contextual Tree Languages.  They show that
%% these tree languages can be learned from positive examples 
%% (from which one may infer they are not superfinite) and apply
%% their techniques to the problem of information extraction.  One of the
%% difficulties they face involves estimating the parameters involved in

  
%% {\em NOTE: I'm leaving out reference to denis:learning-regular-languages (see
%% pads.bib file).
%% Vincent Danos mentioned it but it contains no references to any real data.
%% It's interesting theoretical result that infers a new kind of automaton.
%% This automaton will likely have the same potential difficulties 
%% (possibly exponential explosion -- I haven't prove that though) 
%% when conversion to regular expressions happens. Anyway, I just didn't
%% want to bother studying the paper because they're so much other stuff
%% that is more relevant.  basically, I just couldn't figure out how to cram
%% the reference in elegantly.  I actually couldn't even figure out when I skimmed
%% the paper whether or not it uses both positive and negative examples.
%% It gets compared to RPNI, so I think it must use negative examples.
%% }

%% One disadvantage of such
%% techniques is that the initial state is large (representing
%% the entire training data set explicitly) and the search space is 
%% enormous.  Nevertheless, bottom-up state-merging is often used because
%% it has been difficult to find an effective state-splitting algorithm.
%% Our histogram-based structure-discovery procedure is a new state-splitting
%% algorithm that appears to work well on ad hoc data when coupled with
%% bottom-up rewriting.


%% The classic grammar induction problem~\cite{vidal:gisurvey} requires we find an
%% algorithm that discovers a grammar $G$ given a set of
%% positive examples $R+$ (example strings in the language to be inferred)
%% and a set of negative examples $R-$ (example strings {\em not}
%% in the language to be inferred).  To be more specific, in the limit,
%% as the sets of positive and negative examples grow, the
%% algorithm is expected to converge on the language that defines them.  
%% Unfortunately, very early on,
%% Gold~\cite{gold:inference} proved a key negative result about this problem:  If
%% the algorithm is presented with no negative examples, grammar
%% induction for any super-finite class of languages is impossible.
%% A {\em super-finite} class of languages is any set of languages
%% that includes all finite languages and at least one infinite language.
%% All the most familiar classes of languages, including regular expressions, 
%% context free grammars and PADS, fall into this class.

%% Traditional
%% Some traditional grammar induction algorithms assume that
%% both positive and negative training data
%% One way to categorize research in traditional
%% grammar induction is to ask whether the research in question
%% assumes that both positive and negative training data is available
%% or whether only positive training data is available.

%% analyze the assumptions made
%% about the training data.  
%% Very early in the study of grammar induction, Gold proved a key
%% negative result:


%% Other researchers have defined grammar induction algorithms that use
%% bottom-up rewriting to search through description space for an optimal
%% description.  Many of these techniques, such as 
%% require the availability of both
%% positive and negative examples.  In our context, negative examples
%% never exist, making such techniques inapplicable.
%% % since Gold's early result proved the
%% %impossibility of {\em perfect} grammar induction for any useful family of
%% %languages when no negative examples are
%% %available~\cite{gold:inference}.  
%% However, others, such as Stolcke and
%% Omohundro~\cite{stolcke94inducing} and Hong~\cite{hong01using}, do not
%% assume the existence of negative examples.  These and a number of other systems
%% search through solution space using
%% state-merging rewriting rules.  One disadvantage of such
%% techniques is that the initial state is large (representing
%% the entire training data set explicitly) and the search space is 
%% enormous.  Nevertheless, bottom-up state-merging is often used because
%% it has been difficult to find an effective state-splitting algorithm.
%% Our histogram-based structure-discovery procedure is a new state-splitting
%% algorithm that appears to work well on ad hoc data when coupled with
%% bottom-up rewriting.

% State-merging rewriting rules seem to be
% more popular than 
% that use bottom-up rewriting to find good grammars may suffer from the
% problem of running into local maxima.  The rewriting component of our
% algorithm can also run into a local maximum, but because we start with
% a relatively good candidate generated from our recursive, top-down
% algorithm, this does not appear to be much of a problem for us.  We
% also believe that combining top-down structure-discovery with
% bottom-up rewriting has the potential to deal with larger data sources
% than a pure bottom-up approach.  Our empirical experiments demonstrate
% that the top-down structure-discovery phase is extremely efficient
% when compared with the cost of rewriting.  However, proposals for
% bottom-up-only inference techniques use the (possibly enormous) data
% source itself as the first description.  We are unaware of other
% systems that combine two techniques similar to ours.


%% ; De La Higuera
%% surveys some recent trends~\cite{higuera01current}.  However,
%% our system is unique in two important ways.  First, our inference
%% algorithm does not stand alone; it is part of the more general \pads{}
%% programming environment.  The fusion of the
%% \pads{} system, including its automatic data representation generation,
%% its error detection facilities, its generic programming environment, 
%% and its powerful tool suite, together with grammar induction
%% is one of our key contributions.  Second, many researchers have
%% focused either on grammar induction for natural language processing or
%% for information extraction from \xml{} or \html{} documents.  In
%% contrast, we focus on ad hoc data sources such as system logs and
%% scientific data sets. Ad hoc data is substantially less
%% structured syntactically than \xml{}, and yet, unlike natural language, it is
%% possible to assign our data sources accurate, compact descriptions. After
%% searching the literature and consulting
%% with experts in grammar induction at the CAGI 2007
%% workshop, where we presented a two page overview of our system~\cite{burke+:cagi07},
%% we could find no existing work comparable to ours.

% Third, from a
% technical standpoint, we developed a new top-down structure-discovery
% algorithm and showed how to combine that productively with a
% classic bottom-up rewriting systems based on the minimum description
% length principle.  In what follows, we compare our system more
% specifically to the most closely related work of other researchers.

\paragraph*{Information Extraction.}

The basic goal of an information extraction system is to find and
separate the interesting and relevant bits of information (the
needles) from a haystack of data.  Such systems are fundamentally
different from ours, in that they choose which bits of information to
extract, while we learn a description of the entirety of a data
source, leaving the choice about which pieces are interesting to
down-stream applications.  Of course, this option is only feasible
because we target ad hoc data, which is fairly structured and dense in
useful information, rather than web pages or free text, which are the
usual targets for information extraction systems. 

A common approach to information extraction involves an inductive
learning process in which a user manually tags the relevant data in sample documents.
An example might be highlighting product names and prices on a
collection of shopping web pages from a particular site.  The learning
system then uses these labelled documents in two ways: first, to
decide which bits of information should be extracted from the page
(\ie, product names and prices), and second, to construct a
\textit{wrapper} function to extract those bits of information from
similar pages.  Soderland's WHISK system (\citeyear{soderland:whisk}) is an
example of such an extraction system.  It is particularly general as
it makes few assumptions about the form of the source text,
operating over structured data, stylized text such as Craig's List
descriptions, or free-form text.  WHISK differs from our system in
that it requires user labeling and then only extracts a collection of
tuples from the data source rather than returning the complete
structure of the data source.


Kushmerick and
colleagues (\citeyear{kushmerick-phd1997,KushmerickWD97:Wrapper}) focus on
more structured data to reduce the amount of labeling required during
training.  In particular, this work assumes the labelled pages conform
to one of six different templates, the most well-developed of which
has the form of a header, followed by a sequence of K-tuples each of
which is flanked by a pair of begin and end tags, followed by a
trailer.  For such documents, the system generates a wrapper to
extract the K-tuples.  
% To limit the amount of labeling, the system
% has provisions to automatically tag the desired tuples using {\it
% recognizers}, which are imperfect, but reusable heuristics for finding
% atomic pieces of data such as country names or phone numbers.  Such
% recognizers mean the user has to select which set of recognizers to
% use for a particular extraction task instead of labeling pages by
% hand.  The system requires one recognizer to be {\it perfect}, meaning
% it generates neither false positives nor false negatives.  It then
% uses a process called {\it corroboration} to correct the mistakes of
% the other recognisers.  The system is not robust in the presence of
% missing data, and it is not clear how it would handle multiple
% instances of the same kind of data within a single tuple. This
% approach differs from ours in that it requires the data to comform to
% one of a fixed collection of templates.  In addition, the 
% templates that support corroboration will only return relational data,
% whereas our system will return semi-structured data.
The use of fixed templates and the primary focus on relational data makes this
work quite different from ours.

\citet{muslea+:active-learning} tackle a similar
problem, but strive to reduce the amount of labeling by having the
learning system chose which documents to have the user label,
selecting documents by their probative value.  \citet{borkar+:text-segmentation} uses hand-labelled training
examples and a user-specified set of desired features to train Hidden
Markov Models to select the desired features from similar documents.
This work is quite successful at learning to select the relevant
features of addresses and bibliographic citations from a variety of
input formats. 
% Various researchers have leveraged the syntactic
% regularity and verbosity of XML/HTML to reduce the amount of user
% annotations required to train information extraction systems targetted
% at web pages~\cite{Ambite+:ariadne,doorenbos+:shopbot}.
% Work by Ireson {\em et al.}~\cite{ireson+:ml-evaluation} investigates
% how information extraction systems should be evaluated.  Soderland's
% WHISK paper~\cite{soderland:whisk} and Kushmerick's
% theis~\cite{kushmerick-phd1997} both contain detailed descriptions of
% other information extraction systems.  
In general, systems that depend
upon labeling are unlikely to be helpful in our context; rather than
spending time explicitly labeling documents, the user might as well
write a PADS description by hand.

% Another type of information extraction system strives to provide a
% high-level semantic characterization of the content of natural
% language documents to guide information retrieval
% queries~\cite{gubanov+:structural-text-search,rus+:information-capture}.
% This work differs from ours in that it is building a semantic rather
% than a syntatic description of the source data.

More closely related are various efforts to identify tabular data 
either from free-form text~\cite{Ng+:texttables,Pinto+:texttables} or
from web pages~\cite{Lerman+:webtables}.  These approaches typically
use hand-labelled examples to train machine learning systems to
identify the tables.  They then use heuristics specific to tabular
data to extract the tuples contained within those tables.  The portion
of this work related to identifying structured data from within more
free-form documents is complementary to ours.  The portion responsible
for deconstructing the identified tables uses more specific
domain-knowledge related to the form of tables than we do.

Web pages generated in response to queries tend to be formed by
sloting the resulting tuples into a standard template.  Another line
of work aims to separate such templates from the payload
data~\cite{arasu+:sigmod03,Cresenzi+:roadrunner}.  
Arasu and Garcia-Molina %~\cite{arasu+:sigmod03}
use a top-down grammar induction
algorithm somewhat similar to our rough structure-inference phase
(though it does not use histograms),
but has no description-rewriting engine.  
%However, in certain ways, Arasu has a much easier task than we do as html
%documents have far more regular structure than ad hoc data sources do.
This algorithm exploits the hierarchical nesting
structure of \xml{} documents in essential ways
and so cannot be applied directly to ad hoc data.  
%For example,
%we use histograms to summarize the contents of data chunks whereas
%Arasu does not.  In addition, a substantial portion of our system
%is a description rewriting engine, which Arasu seems not to need.  






% For further reading on
% information extraction from web pages, Hong's
% thesis~\cite{hong:thesis} includes an informative survey.  Though,
% Arasu's work and TSIMMIS appear more closely related to our work than
% the others Hong mentions.


\paragraph*{XML Type Inference.}
Many researchers have studied the problem of learning
a schema such as a DTD or XSchema from a collection 
of XML
documents~\cite{bex+:dtd-inference,bex+:inferring-xml-schema,fernau:learning-xml,garofalakis+:xtract}.  
At a high level, this task is similar to the format inference component of our system.  
However, the details differ because XML has different characteristics
from ad hoc data.  One difference is that XML documents come in a well-nested tree 
shape, with obvious delimiters defining the structure.  
A second important difference is that the appropriate tokenization for
a given ad hoc data source is often not known in advance.  
%%One of our strategies for dealing with
%%these ambiguities is to define simple approximate tokens for use
%%in the tokenization phase, but then to employ a collection of 
%%rules to improve token ({\em i.e.}, base type) choices in the rewriting phase
%%when more contextual information is available.  
In contrast,
tokens in XML documents are clearly demarcated using angle bracket syntax.
%%A third difference is that XML documents are often organized such that
%%the structure of a child node is dependent on its parent or grandparent.
%%In contrast, in the flatter ad hoc data we have considered, dependencies
%%generally arise between siblings -- some data item to the left influences
%%the structure of data to the right.  
As a result of these differences,
XML inference algorithms cannot be used ``off-the-shelf'' for understanding
the structure of ad hoc data.  They must be modified, tuned and
empirically evaluated on this new task.

One line of research on schema inference for XML makes use of the 
observation that 99\% of the content models for XML nodes are defined as
SOREs or CHAREs~\cite{martens+:expressiveness-xml-schema}. 
%(recall, these
%are heavily restricted forms of regular expressions).  
This observation allows \citet{bex+:dtd-inference} to define
an efficient algorithm for inferring concise DTDs.  Later 
\citet{bex+:inferring-xml-schema} build on this work 
by showing how to infer $k$-local XML Schema definitions also based on
SORES.  A $k$-local definition allows node content to depend on the parent
tag, grandparent tag, etc. (up to $k$ levels for some fixed $k$).
As mentioned earlier, hand-written PADS descriptions do not generally obey
the SOREs or CHAREs restriction, nor are they generally arranged with a nesting
structure that suggests $k$-local inference will be particularly useful.
The successful application of these techniques to XML data reinforces 
the idea that the ad hoc data we analyze has quite different characteristics
from XML, and therefore the ad hoc data inference problem merits study
independent of the XML inference problem.

XTRACT~\cite{garofalakis+:xtract} is another system for inferring DTDs
for XML documents.  It operates in three phases: generalization,
factoring and MDL optimization.  The first phase plays a role similar to
our structure discovery phase in that it generates a
collection of candidate structures from a series of XML examples.
This generalization phase searches for patterns in XML
data; it is tuned using the authors' knowledge of common DTD
structures.  Factoring decreases the size of generated candidate DTDs;
some of the factoring rules resemble our rewriting rules.
Finally, they tackle the MDL optimization problem by mapping the
problem into an instance of the NP-complete Facility Location Problem,
which they solve using a quadratic approximation algorithm.
Our MDL-guided rewriting problem considers a more general set of
rewriting rules and hence we cannot reuse their technique.

%% Another related problem of great interest in the XML and database world
%% involves finding a mapping between two data sources with different schema.
%% \citet{doan+:disparate-data-sources} is one example amongst
%% many which attempts to solve this problem using a machine learning approach.
%% While some of our PADS tools do involve translations between
%% different formats, our learning system does not attempt to discover
%% translation tools for which the output is guaranteed to match 
%% the characteristics of a second data set.


\paragraph*{Other work.}
Potter's Wheel~\cite{raman+:potterwheel} is a system that attempts to
help users find and purge errors from
relational data sources.  It does so through the use of a spread-sheet
style interface, but in the background, a grammar inference algorithm
infers the structure of the input data, which may be ``ad hoc,'' 
somewhat like ours.  This inference algorithm operates by
enumerating all possible sequences of base types that appear
in the training data.  
%As in our work,
%users can specify custom base types, and search for a description
%is based on the minimum description length principle.  
Since Potter's Wheel is aimed at processing
relational data, they only infer \cd{struct} types
as opposed to enumerations, arrays, switches or unions.  

The TSIMMIS project~\cite{chawathe+:tsimmis} aims to
allow users to manage and query collections of heterogeneous, ad hoc
data sources.  TSIMMIS sits on top of the Rufus
system~\cite{shoens+:rufus}, which supports automatic classification
of data sources based on features such as the presence of certain
keywords, magic numbers appearing at the beginning of files and file
type.  
%The sources are classified using categories such as ``email''
%and ``C program.''  
This sort of classification is materially
different from the syntactic analysis we have developed.


\section{Running Examples}
\label{sec:examples}
In this section, we describe two examples that we will use throughout
the paper to motivate and explain our system.

AT\&T provides a web hosting service.  AT\&T's infrastructure for this
service includes a variety of components including routers, firewalls,
load balanacing machines, actual web servers, and databases,
replicated and geographically distributed.  Hence, a given web site
may be distributed across a variety of machines running a variety of
operating systems in a variety of locations.  When a customer signs up
for AT\&T's hosting service, part of the contract specifies what kinds
of monitoring AT\&T will provide for site.  The monitoring includes a
variety of resources, including network bandwidth, packet loss, cpu
utilization, disk utlization, memory usage, load averages, \etc{} For
each machine in the hosting service and each such resource, the
monitoring system archives the values at regular intervals and issues
alerts when the values exceed resource- and contract-specific levels.
The archive is used to track long-term behavior of the service,
allowing engineers to determine when more resources need to be
provisioned, for example, adding additional cpus, memory, or disk
space.  It also allows engineers to understand the ``normal'' behavior
for a particular site such as daily or seasonal cycles for a
particular site. 




%% Notes on the visgems example.
%smaug:/fs/swift/proj/vg/4yitzhak
% inventory file:
% labems-test-inv.txt
%    for each asset, defines its type: linux, ip address, password
%    url1, url2, ip,
%    systype(url,url-win,win32.i386,vmware, solaris.sun4,linux.i386,cisco,cisco3750, alteonsw, alteon,...), 
%    user, password, 
%    snmpcommunity(public,CompuLert,monitor, MT1HostingMgmt!,R1cd4Win+g1A, private), 
%    sysfunc(client,ems), 
%    servicelevel(os,man,mon,soss), 
%    need_tags(eastcoast), nets(ip/port), weight (1000,300),
%    ticketmodel(keep)
%    realid(esxhost-122)
%    scopeinv_port22(22), scopeinv_port443(443), scopeinv_port80(80)
%    implappend_protSNMP(version=1)
% class file:
% parameter.txt
%    for each asset, what type of info to collect and how
%    including what kind of scope machine (windows, linux) to use
%    bindings from inventory file are in scope in single brackets
%    what are double brackets: [[scopeinv_cpu]]?
%       the single brackets mean if there's an inventory entry with
%       that key, find it and replace the thing in brackets with the
%       value. if there's no entry, abort processing that metric
%       rule. the double brackets are similar except that the thing in
%       brackets is assumed to be a prefix. so in the above, the tool
%       searches the inventory for entries with key == scopeinv_cpu*
%       and for each one found, it generates a metric collection entry
%       in the schedule. this is how monitoring of multiple
%       filesystems, or multiple cpus is implemented. the scopes query
%       the assets and collect info about filesystems, and cpus which
%       are sent back to the main server that adds them to the
%       inventory. 
%          scopeinv_cpu
%          scopeinv_fs
%          scopeinv_iface
%          scopeinv_port

%    what are counts: count=10, count=5?
%       these are collection type specific. for example, in PING
%       rules, it means send 5 packets.  in calls to vmstat / mpstat /
%       etc, means collect 5 samples. 


%    what are inst parameters (inst=_total), etc
%       inst goes with the 'var' attribute: var=cpu_used and inst=0
%       would collect data for cpu usage on cpu #0 and return it as
%       metric: cpu_used.0 

%    what are labels used for?
%      CPU Used ([[scopeinv_cpu!All]])
%      CPU System
%      CPU User
%      Number of Threads
%      Pages In
%      Pages Out
%      Run Queue
%      Swap In
%      Swap Out
%      Used Memory
%      they are used for tools like WMI where it's simpler to override
%      the label of the returned stats instead of generating them on
%      the scope. 


%    what are val fields
%       val=* */%v *
%       collection specific, in this case it's a regular expression
%       that means the value is the text after a '/' and before a
%       space. 

%    What are file fields
%       file=loadavg
%       file=vmstat
%       file=stat
%       collection specific, in this case it tells the tool to look for /proc/loadavg etc

%    what are exclude fields?
%       exclude=*:top
%       collection specific, in this case it tells the top tool to not
%       include itself in the top process discovery. 


%    pipe separated
%    servicelevel: man, os, soss, mon, colo
%       monitor fewer things for less expensive levels of support
%    asset machine type: linux.i386
%    scope machine type: linux.i386
%    collected info
%       ping_loss (_main)
%       ping_time (_main)
%       cpu_free
%       cpu_sys
%       cpu_used
%       cpu_usr
%       cpu_wait
%       fs_used
%       memory_free (_total)
%       memory_total (_total)
%       memory_used (_total)
%       os_loadavg (_main)
%       os_nproc (_total)
%       os_nthread (_total)
%       os_nuser (_total)
%       os_pagein (_total)
%       os_pageout (_total)
%       os_runqueue (_total)
%       os_swapin (_total)
%       os_swapout (_total)
%       proc_topcpu (1)
%       swap_free (_total)
%       swap_total (_total)
%       swap_used (_total)
%       tcpip_inpkt
%       tcpip_outpkt
%       tcpip_inerrpkt
%       tcpip_outerrpkt
%       url_avail (_main)
%       url_time (_main)
%       port_avail
%       port_time
%       log.hardware
%       log.console
%       log.application
%       log.system
%       host_cpuused,....
%       pool_cpumax,...
%       guest_numvcpu,...
%       collection

%       any instance starting with '_' is meant to be special, as in
%       'overall' or 'main' metric instance. so you may have
%       cpu_used.0, cpu_used.1, ..., for each cpu and also
%       cpu_used._total that is the average of the individual ones.


%    y/n
%       the y/n is a boolean that says to report or not report the
%       stat value back to the server. you'd set it to 'n' when the
%       metric is't important, but you either need it to generate
%       another metric (using the CALC methods), or to generate an
%       alarm. for example, for network interfaces, we don't really
%       care to chart the in/out errors and discards since they are
%       usually 0. but we still monitor them and when errors do occur
%       we create an alarm. 

%    command: 
%      what is distinction between raw, cooked, and embedded?
%         - raw means run a simple command and return the output,
%         e.g. collect SNMP oid .a.b.c.d and return its value.
%         - cooked means runs a more elaborate tool that interacts with
%         the remote side. for example, most SSH collections are like
%         that because they run either multiple commands or need to
%         parse the results and perform calculations. 
%         - embedded is similar to cooked except that the remote end
%         is assumed to not be a full POSIX shell environment, so the
%         mechanism for collection needs to be a little
%         different. this happens for network switches that support a
%         limited shell type environment. 

%      PING:..., 
%        loss, time
%      SSH:...
%        mpstat, df, free, uptime, top, proc, uptime, netstat, sar,
%        swap, ibmhmc, vmwarei, vmwarevires
%      CALC:...
%        [[!scopeinv_cpu]]
%      PORT:...
%      URL:
%         url=
%      WMI:
%      NOOP
%      SNMP:
%         community
%         version
%         var, label, unit, helper, unique
%    units:
%        %,ms,GB,<empty>, pkts, mbps
%    number, counter
%    alarm spec:  >=100:1:2/2:1/3600:CLEAR:5:2/2
%       <vrange>:<severity>:<m hits/n collections>:<alarm refresh count/time>
%     or
%       CLEAR:<severity>:<m clears/n collections>
%     vrange can be >= v, <= v, [v1,v2], (v1,v2) (inclusive / exclusive intervals)
%     1/3600 means resend this alarm once every 3600 secs, e.g. 1hr.
%     so the above means: alarm if the value is >= 100 for 2 consecutive intervals,
%     refresh the alarm every hour while the condition persists, and clear the alarm
%     if you get 2 intervals < 100.



% scopemgr script assigns a scope based on inventory and class files
% and generates a schedule for the asset.  
% schedules are grouped by customer and scope
%  asset schedule file:
%  labems-test-sched.scope3.txt
% all schedules for a scope are concatentated it single schedule for
% scope

% vg_collector invoked with segment of scheduler for given asset
% examples:
%  ssh-schedule.txt
%  snmp-schedule.txt
% "XML code between <cfg> tags
% "vars" table with variables to collect and how to do it
% "alarms" table with threshold limits for these variables.

% scopes come in linux and windows flavors because hard to monitor
% windows machines from non-windows machines.

% scopes are assigned based on network reachablity and grouping by
% tags:
%  a scope will be assigned to an asset if their tag sets intersect
%  assignment also considers "cost" which accounts for bandwidth and
%  load issues.

% if a scope fails, tasks are reassigned to other scopes until it
% comes back on-line

\section{\padsd{}: An Informal Introduction}
\label{sec:informal}
\begin{verbatim}
Introduce running examples
  Ace satellite data?
  comon application?
  visigems application?

Base feeds
  description, locations, frequency
  Description
    PADS/ML
  Locations
    local, http, ftp, shell commands
  Frequency
    time data type

Simple Compound Feed

Dependent Compound Feed
  list comprehension syntax
  comon example

\end{verbatim}


\section{\padsd{} Semantics}
\label{sec:semantics}
\section{\forest{} Design}
\label{sec:language}

\forest{} is a domain-specific language embedded within Haskell using
the Quasiquote mechanism~\cite{Mainland:quasi}.  In a typical
\forest{} description, \forest{} declarations are interleaved with ordinary
Haskell declarations.  To introduce new \forest{} declarations,
the programmer simply opens the \forest{} sublanguage scope:
\begin{code}
[forest| 
  ... forest declarations ...
|]
\end{code}

\forest{} uses a type-based metaphor to describe directory
structures, so once  within the \forest{} sublanguage, the programmer
writes declarations that look and feel very much like extended Haskell
type declarations. 
Each such type declaration serves three purposes: 
(1) it describes a fragment of the file system,
(2) it specifies the structure of the in-memory {\em representation}
    that will be constructed when the fragment is (lazily) loaded into a Haskell program, and 
(3) it specifies the structure of the in-memory {\em metadata}
    that will be generated when the fragment loaded.
Such metadata includes error information (missing file, insufficient
permissions, \etc{})  as well as file system attributes (owner, size, \etc{}).
As we explain the design of \forest{}, 
readers should keep these three different aspects in mind.  
The effectiveness of the \forest{} language comes in part from the fact that these three 
elements can all be specified in a single compact description.  

Every \forest{} description is defined relative to a
{\em current path} within the file system.  As 
\forest{} matches a description against the file system, it 
adjusts the current path to reflect its navigation.
\cut{
In general, descriptions
do not assume that the current path is a valid path in the file
system.  
When the current
path is invalid, a description will generally\footnote{The \cd{Maybe} constructor,
discussed later, is an example of an exception to the general rule.} 
register an attempt to load from that
path as an error in the associated metadata structure.}

At its core, \forest{} is a simple dependent type system in which base
types denote files of various flavors and record types describe
directories.  \forest{} also includes a list type to describe
collections of files with finer granularity than directories.  We use
other type constructors to build more refined structures from these
basic building blocks.  We discuss each of these constructs in turn in
the remainder of this section.

\subsection{Files}
\label{sec:basics}
\forest{} provides a small collection of base types for describing
files: \cd{Text} for ASCII files, \cd{Binary} for binary files, and
\cd{Any} for arbitrary files.  As with all \forest{} types, each of
these types specifies a representation type, a metadata type, and a
loading function.  The in-memory representation for an ASCII file is a
Haskell \cd{String}; for binary and arbitrary files, it is a
\cd{ByteString}.  For all three file types, the metadata type pairs
file-system metadata with metadata describing properties of the file
contents.  The file-system metadata has type \cd{Forest\_md}, shown in
\figref{fig:forest-md}.  This structure stores two kinds of
information: (1) the number and kind of any errors that occurred while
loading the file and (2) the attributes associated with the file.
File-content metadata typically describes errors within the file, but
can be used for other purposes.  For these three file types, there is
no meaningful content metadata and so this type is the unit type.
Leveraging Haskell's laziness, the loading functions create the
in-memory representations and set the metadata on demand.

\begin{figure}
\begin{code}
\kw{data} Forest_md = Forest_md 
   \{ numErrors :: Int
   , errorMsg  :: Maybe ErrMsg
   , fileInfo  :: FileInfo      
   \}
\mbox{}
\kw{data} FileInfo = FileInfo 
   \{ fullpath   :: FilePath
   , owner       :: String
   , group       :: String
   , size        :: COff
   , access_time :: EpochTime
   , mod_time    :: EpochTime
   , read_time   :: EpochTime
   , mode        :: FileMode
   , isSymLink   :: Bool
   , kind        :: FileType     
   \}
\end{code}
\caption{\forest{} metadata types.}
\label{fig:forest-md}
\end{figure}

Of course, there are many kinds of files, and the appropriate
representation and content metadata type for each such file varies.
Possible examples include XML documents, Makefiles, source files in
various languages, shell scripts, \etc{}  To support such files,
\forest{} provides a plug-in architecture, allowing third party users
to define new file base types by specifying a representation type, a
metadata type, and a corresponding loading function.

A common class of files are \textit{ad hoc data files} containing
semi-structured information, an example of which is the Princeton
student record file format (\figref{fig:student-file-example}). 
In such cases, \forest{} can leverage the \padshaskell{} data
description language to define format-specific in-memory
representations, content metadata, and loading functions. 
\padshaskell{} is a recently developed version of 
\pads{}~\cite{fisher+:pads,fisher+:popl06,mandelbaum+:pads-ml}.
Like \forest{}, \padshaskell{} is embedded in Haskell using
quasiquotation.  For example, the following code snippit 
begins the \pads{} specification of the Princeton student record
format: 
\noindent
\begin{code}
[pads| 
  \kw{data} Student (name :: String) = < pads decl >
|]
\end{code}
This description is parameterized by the name of the student whose
data is in the file; the complete description appears in the appendix.
From this specification, the \pads{} compiler
generates an in-memory representation type \cd{Student}, a content metadata
type \cd{Student_md}, and a parsing function.  

\forest{} provides the \cd{File} type constructor to lift \pads{}
types to \forest{} file types.  For example, the declaration
\begin{code}
[forest| 
  \kw{type} SFile (n::String) = File (Student n) 
|]
\end{code}
introduces a new file type named \cd{SFile} whose format is given by
the \pads{} type \cd{Student}.  As with the \pads{} type, \cd{SFile}
is parameterized by the name of the student.  

Using \padshaskell{} descriptions in \forest{} not only helps specify
the structure of ad hoc data files, but it also generates a structured
in-memory representation of the data, allowing Haskell programmers to
traverse, query and otherwise manipulate such data.  Indeed,
\padshaskell{} and \forest{} were designed to work seamlessly
together.  From the perspective of the Haskell programmer traversing
an in-memory data structure, there is effectively no difference
between iterating over files in a directory or structured sequences of
lines or tokens within a file.

While \padshaskell{} is independently interesting,
the rest of this paper focuses on \forest{}.  Henceforth, any
unadorned declarations occur within the \forest{} scope
\cd{[forest|...|]} unless otherwise noted.  Any declarations prefixed
by \cd{>} 
are ordinary Haskell declarations.

\subsection{The Maybe Type}
\label{sec:maybe}
Sometimes, a given file (or directory or symbolic link) may or may not
be present in the file system, and either situation is valid.
\forest{} provides the \cd{Maybe} type constructor for this
situation.  If \cd{T} is a \forest{} type, then \cd{Maybe T} is the
\forest{} type denoting an optional \cd{T}.  In particular, 
\cd{Maybe T} succeeds and returns representation \cd{None} when the
current path does not exist in the file system.  \cd{Maybe T} also
succeeds, returning \cd{Just v} for some \cd{v} of type \cd{T}, 
when applied to an file system object that matches \cd{T} at the
current path.  \cd{Maybe T} registers an error in the metadata when
the current path exists but the object at the current path does not
match \cd{T}. 


\subsection{Directories}
\label{sec:simple-directories}

The simplest way to specify the contents of a directory is to use
a record-like declaration.  For example, to specify the root directory
of the student repository in Figure~\ref{fig:student-pic}, we might use
the following declaration.  This declaration assumes that we have already
defined \cd{Class n}, a parameterized description that specifies
the structure of a directory holding data on the class of year \cd{n},
and \cd{Grads}, a description that specifies the structure of the directory holding
all graduated classes.   
\begin{code}
type PrincetonCS = Directory
  \{ seniors is "classof11" :: Class 11
  , juniors is "classof12" :: Class 12
  , grads is "graduates" :: Grads
  , notes is "README" :: File Ptext
  \}
\end{code}
Above, each field of the record has three components:  (1) a label
name ({\it e.g.,} seniors or juniors), (2) a file or directory name
({\it e.g.,} "classof11" or "classof12"), and (3) a \forest{} subdescription
for the contents of the named object ({\it e.g.,} \cd{Class 11} or \cd{Grads}
or \cd{File Ptext}).

With simple descriptions like this one, it is common for users to want
the label to be the same as the name of the file.  In such a case, users
may write the following abbreviated description:
\begin{code}
type All = Directory
  \{ classof11 :: Class 11
  , classof12 :: Class 12
  , graduates :: Grads
  , notes is "README" :: File Ptext
  \}
\end{code}
Above, the name of the label is used as the expected file name.  We did not
shorten \cd{notes is "README"} to \cd{README} because labels must start
with a lowercase letter in Haskell.

\paragraph*{Matching.}
In order for a file system object to match a description like the one above, it must be a
directory and each field of the record must match.  A field matches when the given
file name ({\em e.g.,} \cd{"README"}) is concatenated to the current path and the 
object at that new path matches the field's subdescription.

It is possible for the same file system object to to match multiple different fields of a description at
the same time.  For example, if "README" were a symbolic link that pointed to a text file, one 
might want to specify that using a pair of declarations:
\begin{code}
type All = Directory
  \{ ...
  , link is "README" :: SymLink
  , notes is "README" :: File Ptext
  \}
\end{code}

It is also possible for a directory to contain a number of objects that go unmatched by
a description.  We made the design choice to allow extra items in
a directory because we found it common for directories to contain files that users
simply do not care much about.  For example,
if programmers do not need to extract information out of the README file and
users do not really care whether it exists or not, one might well choose
to omit it from the description.  One might be concerned that this design choice makes it difficult to specify
the absence of files (as opposed to their presence), which can be important
for security purposes in some cases.  However, we will see shortly that in the
uncommon case that programmers want to specify absence information, they can
do so using constraints.

When a directory does match a record declaration like this one, 
\forest{} provides utilities to read the directory in
to memory and generate a convenient programmatic representation for it.  
In this case, the type of the in-memory representation is a record type
with labels \cd{seniors}, \cd{juniors}, {\em etc.} and with field types
generated from the associated \forest{} subdescriptions \cd{Class}, \cd{Grads},
{\em etc.}  When the same object is described in two different ways, one may
get two different in-memory representations of it.  For instance, in the example above,
the \cd{link} field will hold a representation of the filepath that describes the target of the
symbol link whereas the \cd{notes} field will hold a representation of the \cd{README}
file itself ({\em i.e.,} a string of characters).

When a directory does not match a record declaration, \forest{} will
construct representations for the subparts that do match and will insert dummy values
into the parts that do not match.  \forest{} will also record all errors encountered in the
matching process in the object metadata.

\paragraph*{Computed Paths}
\label{sec:computed-paths}

Both of the above descriptions are a good start for our application, but neither
are ideal.  Every year, the directory for the graduating seniors 
({\em i.e.,} \cd{classof11}) gets moved into the graduates directory,
the juniors get promoted to seniors and a new junior class gets created.
As it stands, this means we would also have to edit our description every year.
An alternative is to parameterize our description with the current year and
to construct the appropriate file names.  If we follow this strategy,
we might arrive at the following specification for the top-level student directory:
\begin{code}
> mkclass y = "classof" ++ (toString y)
\mbox{}
type All (y::Integer) = Directory
  \{ seniors is <|mkclass y|> :: Class y
  , juniors is <|mkclass (y+1)|> :: Class <|y+1|>
  , graduates :: Grads
  , notes is "README" :: File Ptext
  \}
\end{code}
%\begin{code}
%type All (sy::String, jy::String) = Directory
%  \{ seniors is <|"classof" ++ sy|> :: Class
%  , juniors is <|"classof" ++ jy|> :: Class
%  , graduates :: Grads
%  , notes is "README" :: File Ptext
%  \}
%\end{code}
In general, \cd{<|...|>} can be used to escape back in to
Haskell to perform arbitrary computations.  Note also that any description
may be parameterized by specifying a legal Haskell identifier and its type.
Parameterized specifications may be used by supplying their arguments
in the usual way.  When arguments are simple constants or variables,
they may be supplied directly.  When arguments are more complex
computed expressions, it is necessary to use explicit escapes back 
in to Haskell.

\paragraph*{Approximate Paths}
As repositories evolve over time, naming conventions may change.
Alternatively, programmers simply may not care to specify certain names
exactly.  To accomodate these possibilities, \forest{} includes mechanisms
for approximate naming of files.  For example, in each class, there may
(or may not) be some number of students that have withdrawn from the
program, transferred to a different Princeton program on gone on
temporary leave.  Over the years, slightly different directory names
have been used to represent these situations.  Given this circumstance,
we can use the following declarations to describe the class directory.
\reminder{dpw: I edited out some of the withdrawn RE options to make it
fit on a line.}\reminder{jnf:how about making all of them two characters long (``tr'', ``wd'', ``lv'')?}
\begin{code}
> transRE = RE "TRANSFER|Transfer"
> leaveRE = RE "LEAVE|Leave"
> wdRE    = RE "WITHDRAWN|WITHDRAWAL|Withdrawn"
\mbox{}
type Class (y::Integer) = Directory
  \{ bse is <|"BSE" ++ (toString y)|> :: Major
  , ab  is <|"AB"  ++ (toString y)|> :: Major   
  , trans matches transRE :: Maybe Major      
  , wd matches wdRE :: Maybe Major
  , leave matches leaveRE :: Maybe Major 
  \}
\end{code}
A field with the form \cd{<label> matches <regexp> :: T}
finds the set of paths in the files system that match \cd{currentPath/} \cd{<regexp>}.
If there are zero or one matches, the \cd{matches} form acts like the \cd{is} 
form.\footnote{Recall that when 
there are zero matches, the \cd{is} form attempts to match the type \cd{T} 
against a non-existant file system object at a non-existant path.  This may or may not be
an error, depending on \cd{T}.  The matches form constructs a non-existant path
and behaves the same way.}  If there is more than one match, one of the many matches
is selected non-deterministically, a multiple match error is registered in the metadata,
and matching once again continues as in the \cd{is} form.


\subsection{Comprehensions}
\label{sec:comprehensions}

Record directories allow programmers to specify a fixed number of file system objects.
Comprehensions, on the other hand, allow programmers to specify arbitrary numbers
of file system objects.  As an example, we might specify the contents of the
\cd{Grads} directory from Figure~\ref{fig:student-file-example} as follows.
\reminder{dpw: changed Prelude.take into take below for space reasons.}
\reminder{dpw: perhaps the definition of getYear should use Haskell composition "." instead
of all the parens? Also, I changed the type of the Class parameter to an integer. Hence, I needed the toInteger conversion -- it wasn't obvious to me from the Haskell docs if in fact toInteger will
convert from a string to an integer. In the current Students2.hs (Oct. 23), all comprehensions
are nested within records.  I unnested here. I can imagine syntax is a possible issue with doing
that but not semantics.  The record wrapper does seem unnecessary overhead for the programmer.
(But really, it just helps me with the explanation here, because it makes comprehensions orthogonal.}
\begin{code}
> cYear s = 
>   toInteger (reverse (take 2 (reverse s)))
> cRE = RE "classof[0-9][0-9]"
\mbox{}
type Grads = 
  [c :: Class <|cYear c|> | c <- matches cRE]
\end{code}
In the specification above, \cd{Grads} is a directory that contains a number of
\cd{Class} subdirectories with names \cd{c} that match the regular expression
\cd{cRE}.  The Haskell function \cd{cYear} extracts the last two digits from the
name of the directory, converts the string digits to an integer year, and passes
the year to the underlying \cd{Class} specification.
More generally, comprehensions have the following form.
\begin{code}
[path :: T | id <- gen, pred]
\end{code}
Here, \cd{id} is bound to each of the strings generated by the \cd{gen},
which may be a \cd{matches} function (used to match against the files
at the current path) or may be some other list computed in Haskell. The allowed
\cd{id}s may be filtered by \cd{pred} and they may be used in the computation
\cd{path}.  Each such computed path is concatentated to the current path and
used to match a file system object with type \cd{T}.  The predicate and path
computation may also use the path metadata by referring to the variable
\cd{id_md}, which will be in scope.  The in-memory representation of a comprehension
is a list.

Another example of a comprehension occurs in specification the \cd{Major} 
directory.  These directories contain a list of student files, and one additional
template file named \cd{sss.txt} or \cd{sxx.txt}.  The declaration below
specifies all the student files except the template files.  Note that it uses
a glob pattern as opposed to a regular expression
to describe a set of text files.\footnote{A glob pattern
is a pattern used to describe file paths in UNIX-based systems.  Such patterns include
constant strings and wildcards \cd{*} and \cd{?}.} 
\begin{code}
> tpl s = (s == "sss.txt") || (s == "sxx.txt")
> txt = GL "*.txt"
\mbox{}
type Major =   
  [ s :: File (Student <|getName s|>) 
  | s <- matches txt, <|not (tpl s)|>]
\end{code}

\paragraph*{Representation Coercions}
One of the benefits of using \forest{} is that every description serves multiple purposes:
(1) it specifies the structure of the file system on disk, (2) it specifies structure of in-memory
representations of the file system, (3) it generates type declarations describing those in-memory
structures, and (4) it generates data structures representing error conditions and other meta data.
\reminder{jnf: above, we identified three things generated from descriptions...}

Sometimes, however, it is useful to be able to override the default structure of the generated items.
In particular, we have found that programmers occasionally want the representation of their in-memory data
structured differently than the default.  In order to give programmers choice on this matter when they want it,
\forest{} permits them to use {\em representation coercions} within a description.  As an example, consider
again the definition of \cd{Major} from subsection~\ref{sec:comprehensions}.  In that definition, the
in-memory representation of the set of majors is a list of \cd{(name.txt,filecontents)} pairs.  A more convenient
representation for programming, however, is likely a map from \cd{names} to \cd{filecontents}.  Assuming
\cd{MapName} is the coercion from lists to maps suggested, 
we insert it into the description as follows.\reminder{dpw: can we put the definition of mapname in?  I don't know
how to write it.  I can't find Kathleen's email with some mention that any Builder something something toList
function something something is how you write these functions.}
\begin{code}
type Major = 
  MapName [ s :: File (Student <|getName s|>) 
          | s <- matches txt, <|not (tpl s)|>]
\end{code}
While programmers may build their own custom coercions as above, \forest{} does supply a set of 
generic, built-in coercions as well such as \cd{Map} (convert to a map from file name to contents)
and \cd{Set} (convert to a set of contents).


\subsection{Attributes and Constraints}
\label{sec:constraints}

Every file system object has a number of attributes associated with it.  In general,
if a \forest{} identifier \cd{id} refers to a path, the attributes for the object at that
path are available through the identifier named \cd{id_att}. Figure~\ref{fig:metadata-components}
lists a set of functions that extract useful information from metadata structures.

Attributes are also commonly used in {\em constraints}.  For example, if a user
wants to ensure that all her text files are private,
she might replace the use of \cd{File Ptext} in a description with the following
\cd{PrivateFile} definition.  
\reminder{dpw: The following two examples need to be tested to ensure correctness.}
\begin{code}
type PrivateFile = 
  File Ptext 
    where <|get_modes this_att == "-rw-------"|>
\end{code}
Here, the keyword \cd{where} introduces a constraint
on the underlying type.  If a constraint evaluates to false, an error
is registered in the metadata.  Within the constraint, \cd{this} refers to the representation 
of the underlying object, \cd{this_att} refers to its attributes and \cd{this_md} refers
to its metadata.


Below, metadata functions used to write a universal description.
Notice that the description also happens to be recursive, another useful feature of
\forest{}.  In the case that a symbolic link creates a cycle in the file system, the Haskell
in-memory representation will be a (lazy) infinite data structure. 
\reminder{jnf: the preceding remark is only true if the symbolic link points to a directory, right?}
\begin{code}
type Universal = Directory 
  \{ asc is [ f :: File Ptext   
           | f <- matches (GL "*"), 
            <|get_kind f_att == AsciiK|>]
  , bin is [ b :: File Pbinary
           | b <- matches (GL "*") 
           where <|get_kind b_att == BinaryK|>]
  , dir is [ d :: Universal  
           | d <- matches (GL "*") 
           where <|get_kind d_att == DirectoryK|>]
  , sym is [s :: SymLink      
           | s <- matches (GL "*") 
           where <|get_sym s_att == True|>]
  \}
\end{code}


Another use for constraints involves specifying emptiness conditions or conditions that
certain files do not appear in certain places.  As an example, one might want to specify
that no binaries appear in a directory given to an untrusted user as scratch space.
The description below registers an error whenever a binary file appears in the described
directory. 
\begin{code}
type NoBin =
  [ b :: File Pbinary 
  | b <- matches (GL "*"), 
  <|get_kind b_att == BinaryK|> ]
  where <|List.length this == 0|>
\end{code}

\begin{figure}
\begin{center}
\begin{tabular}{l|l}
function name &  information \\
\hline
\cd{get_group} & object group\\
\cd{get_kind} & the sort of file or directory \\
\cd{get_modes} & permission string\\
\cd{get_owner} & object owner\\
\cd{get_size} & object size \\
\end{tabular}
\end{center}
\caption{Selected file attribute functions}
\label{fig:metadata-components}
\end{figure}


\subsection{Gzip and Tar Type Constructors}
\label{sec:file-modifiers}

Some files need processing by a utility before they can be used.  A typical
example is a compressed file such as the gzipped log files in the Coral
example shown in Figure~\ref{fig:coral-pic}.  \forest{} allows users to describe such 
files using various modifiers.  For example, if \cd{CoralInfo} is a \padshaskell{}
description of a Coral log file then the following describes a gzipped log file.
\begin{code}
type CoralLog = Gzip (File CoralInfo)
\end{code}
Likewise, suppose \cd{logs.tar.gz} is a gzipped tar file and suppose also that 
\cd{AllLogs} describes the directory of log files that \cd{logs.tar} expands
to.  In this case, the following description will describe the contents of
\cd{logs.tar.gz} properly.
\begin{code}
type CoralLogs = Gzip (Tar AllLogs)
\end{code}

\subsection{Symbolic Links}
\label{sec:symlinks}

When symbolic links are present in the file system described,
the default behavior is to read through the symbolic links to their
targets.  Hence, the behavior of \forest{} will be intuitive to programmers who
are used to working in a standard UNIX-like environment.  In addition, however, it is possible
for a programmer to explicitly specify that she expects a symbolic link in a particular
position.  To do so, the programmer will use the \cd{SymLink} base type.  For
example:
\begin{code}
type MyFile = SymLink
\end{code}
In \forest{}, any file system object may be described in multiple different ways.
Hence, in the case of symbolic links, it is possible to use one declaration to
specify the fact that an object is a symbolic link, and a second declaration to specify
the nature of the target that link ({\em e.g.,} the target is a text file).  We will see such 
a specification in later subsections.

\subsection{Putting it all together}

The previous subsections explain most of the interesting features of \forest{}.
These features are put to work in Figures~\ref{fig:student-description}
and~\ref{fig:coral-description}.  A set of other descriptions are available on 
the \forest{} web site~\cite{forest-web-site}.

\begin{figure}
{
\begin{code}
tpl s = (s == "sss.txt") || (s == "sxx.txt")
cYear s = 
  toInteger (reverse (take 2 (reverse s)))
mkclass y = "classof" ++ (toString y)
\mbox{}
transferRE  = RE "TRANSFER|Transfer"
leaveRE     = RE "LEAVE|Leave"
withdrawnRE = RE "WITHDRAWN|WITHDRAWAL|Withdrawn"
cRE         = RE "classof[0-9][0-9]"
txt         = GL "*.txt"
\mbox{}
[pads|
  data Student (name::String) = ...
|]
\mbox{}
[forest|
  type Major = MapName  
    [ s :: File (Student <|getName s|>) 
    | s <- matches txt, <|not (tpl s)|>]
\mbox{}
  type Grads = 
    [c :: Class <|cYear c|> | c <- matches cRE]
\mbox{}
  type Class (y::Integer) = Directory
    \{ bse is <|"BSE" ++ (toString y)|> :: Major
    , ab  is <|"AB"  ++ (toString y)|> :: Major   
    , trans matches transRE :: Maybe Major      
    , wd matches wdRE :: Maybe Major
    , leave matches leaveRE :: Maybe Major 
    \}
\mbox{}
  type All (y::Integer) = Directory
    \{ seniors is <|mkclass y|> :: Class y
    , juniors is <|mkclass (y+1)|> :: Class <|y+1|>
    , graduates :: Grads
    , notes is "README" :: File Ptext
    \}
|]
\end{code}
}
\caption{\forest{} student description -- out of date (nov 1)}
\label{fig:student-description}
\end{figure}

\begin{figure}
\begin{code}
[pads|
  type WebSrv = ...
|]
\mbox{}
coralwebsrv = "coralwebsrv.log.gz"
time = RE "[0-9]\{4\}(_[0-9]\{2\})\{2\}-[0-9]\{2\}_[0-2]\{2\}"
site = RE "[^.].*"
\mbox{}
[forest|
  type Log = Directory 
    \{ log is coralwebsrv :: Gzip (File CoralLog) \}

  type Site = [ d :: Log | d <- matches time ]

  type Top = [ s :: Site | s <- matches site ]
|]
\end{code}
\caption{\forest{} coral description -- out of date (nov 1)}
\label{fig:coral-description}
\end{figure}


\section{\padsd{}:  Working with Feeds}
\label{sec:programming}
\section{Tools}
\label{sec:tools}
\begin{itemize}
\item tar, ...
\item Graph representation giving status
\item ??
\end{itemize}


\subsection{The Single-Minded Implementer}

In addition to the built-in tools, \padsd{} includes an API for
manipulating feeds created from a description. The API provides the
user with a feed abstraction representing a potentially infinite
series of elements. This abstraction is related to that of a lazy
list, but extends it with support for data timing and provenance
information. Therefore, the API provided for feeds is modeled on the
list APIs of common functional languages, like \ocaml and \haskell,
but provides two levels of abstraction. One level allows the user to
manipulate feeds like any other lazy list of data elements (ignoring
where they come from), while the other exposes the metadata along with
the data. 

% Need a new name for Feedmain module. I vote Feed and then Feed_core for the lower-level module.


\begin{figure}[tb]
\begin{codebox}
\kw{let} (sample, \_) = \textit{Feed.split_every} 600. comon \kw{in}
\kw{let} select_load = \kw{function}
    Some \{Comon_format.Source.
          loads = (_, load::_)\} -> Some load
  | None -> None \kw{in}
\kw{let} loads    = \textit{Feed.map} select_load sample \kw{in}
\kw{let} load_tbl = \textit{Feed.fold} update (create ()) loads 
\kw{in}  print_top 10 load_tbl
\end{codebox}
  \caption{Code fragment for finding \planetlab nodes with low load
  over the last 10 minutes (600 seconds).  \texttt{Feed.split\_every n}
  extracts the first \texttt{n} seconds of any feed. 
  Function \texttt{create} constructs an empty hash table; function 
  \texttt{update} adds
  an entry to the hash table; function
  \texttt{print\_top k} prints the \texttt{k} lowest loads from the table.}
\label{fig:sample-loads}
%\vskip -2ex
\end{figure}


%% \begin{figure}[tb]
%% \begin{codebox}
%% \kw{let} deadline = Time.now() +. 600. \kw{in}
%% \kw{let} (sample, remainder) = \textit{Feed.split_when} 
%%    (fun () -> Time.now() > deadline) comon_feed \kw{in}
%% \kw{let} select_load = \kw{function}
%%     Some \{Comon_format.Source.
%%           loads = (_, load::_)\} -> Some load
%%   | None -> None \kw{in}
%% \kw{let} loads = \textit{Feed.map} select_load sample \kw{in}
%% \kw{let} load_tbl = \textit{Feed.fold} update (create ()) loads 
%% \kw{in}  print_top 10 load_tbl
%% \end{codebox}
%%   \caption{Code fragment for sampling \planetlab loads for 10 minutes. Function
%%   \texttt{print\_top k} selects the \texttt{k} lowest loads from the table.}
%% \label{fig:sample-loads}
%% %\vskip -2ex
%% \end{figure}

\begin{figure}[tb]

\begin{codebox}
\kw{let} update tbl idata =
  \kw{let} meta = IData.get_meta idata \kw{in}
  \kw{let} data = IData.get_contents idata \kw{in}
  \kw{match} meta, data \kw{with} 
    (h, Some basemeta), Some load ->
      \kw{let} location = Meta.get_link basemeta \kw{in}
      update tbl location data
  | _ -> tbl \textit{ (* no change to tbl *)}
\kw{in} ...
\kw{let} load_table = \textit{\textbf{Feed.fold_p}} update empty_tbl loads
\kw{in} print_top_10 load_table
\end{codebox}
  \caption{Revised code fragment which exploits provenance metadata. }
\label{fig:sample-loads-prov}
%\vskip -2ex
\end{figure}

For example, consider a \planetlab user looking for a desirable set of
nodes on which to run their experiments. Based on the \comon
description, they can use our API to monitor \planetlab for a few
minutes to find the least loaded nodes. In \figref{fig:sample-loads},
we show an \ocaml code fragment that collects a list of nodes with
lowest average loads over $10$ minutes, and then prints them. We leave
out the exact details on maintaining the table of top values, as it is
orthogonal to our discussion. First, we use \cd{Feed.split_when} to
split the feed when 600 seconds (10 minutes) have elapsed. Then, we
use \cd{Feed.map} to project the loads data from the\comon
elements. Finally, we use \cd{Feed.fold} to collect the load values
into a table after which we can average the data on a per node basis
and extract the $10$ lowest values.

However, this solution is not quite enough. The \comon data format
does not include the host name in the data itself, so the code in
\figref{fig:sample-loads} will only be able to return the lowest
average loads, but not the names of the machines associated with
them. In situations like this, the provenance data is essential.  In
\figref{fig:sample-loads-prov}, we replace the last three lines of
\figref{fig:sample-loads} with a call to the lower-level fold,
\cd{fold_p}, which provides both the data and metadata to the folding
function. We also sketch an \cd{update} function which makes use of 
the metadata.   Notice the use of the lower level \cd{IData} and
\cd{Meta} interfaces to facilitate mangement of both data and 
metadata from the feed. 

It should be clear from these examples that the single-minded implementer
has a number of new interfaces to master relative to the quick-and-dirty
hacker, but gains a correspondingly higher degree of flexibilty and can
still write relatively concise programs.

\subsection{The Generic Programmer}

% Motivate

Occasionally, a user might want to develop a function that can
manipulate {\it any} feed. This desire might arise because the user
has a number of different feeds to process in the same way, or simply
because the user wishes to provide a new tool for other \padsd{}
users. Some such functions can be written simply by designing them to
be parametric in the type of the feed element, much like the feed
library functions discussed above. However, the behavior of many feeds
functions will depends on the structure of the feed and its
elements. Such functions can be viewed as {\it interpretations} of
feed descriptions, so, to support their development, we provide a
framework for writing feed interpreters.

% Examples

Two core examples of feed interpretations are the feed creator and the
feed accumulator. The behavior of both of these tools depend
essentially on the structure of the feed. These functions, and others
like them, require as input a runtime representation of the feed,
complete with all of the details included in the feed description that
they represent. The obvious choice for representing feed descriptions
in \ocaml is a datatype. However, standard \ocaml datatypes are not
sufficiently typeful to express the types of many generic feed
functions. For example, the feed creation function has the type:
\begin{code}
feed_create : 'a prefeed -> 'a feed
\end{code} where \cd{'a prefeed} is an AST of a feed description and feed 
elements have type \cd{'a}.
%
% Tools as interpreters
%
This limitation of datatypes has been widely discussed in the
literature, and various solutions have been 
proposed~\cite{yang:icfp98,weirich:encodingtypecase,hinz:icfp04,padsml-padl}. We have 
chosen to represent our AST using a variant of the Mogensen-Scott
encoding in which higher-order abstract syntax is exploited 
to encode variable binding in feed descriptions.  This implementation strategy 
exploits \ocaml's module system to type the encodings in $F_\omega$. 
A similar strategy is exploited in our earlier work on \padsml~\cite{padsml-padl}, 
but there we only sought to encode the \ocaml type of data, not entire \padsml
description, which is where the higher-order abstract syntax enters the picture.
% To effectively encode the dependency present in feeds
% descriptions, we instead 
% encodings~\cite{mogensen}, employing HOAS to encode variable binding
% in feed descriptions.
% Given that \ocaml is our target language, we follow
% the method used in \padsml{} to encode type
% representations~\cite{padsml-padl}. Because of space limitations, we
% do not provide details here.  

The result of our work is that developers
can interpret feed-description representations by case analysis on
their structure, while still achieving the desired static
guarantees. Moreover, we have successfully used this framework to
develop {\it all} of the tools presented in this paper, including the
feed creator. The only role of the compiler is to infer appropriate
type declarations from feed descriptions, and compile the feed syntax
into our representations.  However, as one might expect, interfaces using
higher-order abstract syntax and Mogensen-Scott encodings are one step more
complex than those involving the more familiar maps and folds.  Consequently, the
learning curve for the generic programmer is one step steeper than
the curve for the single-minded implementor, and two (or perhaps ten) steps steeper
than the curve for the quick-and-dirty hacker.

% There, we represent our AST using a
% Scott-encoded datatype, and leverage \ocaml's module system to, in
% essence, type the Scott encodings in $F_\omega$. However, in \padsml
% we only sought to encode the \ocaml type of data, not entire \padsml
% descriptions. To effectively encode the dependency present in feeds
% descriptions, we instead use a variant of Mogensen-Scott
% encodings~\cite{mogensen}, employing HOAS to encode variable binding
% in feed descriptions.






\section{Conclusions}
\label{sec:conclusions}

\section*{Acknowledgments}

This material is based upon work 
supported by the NSF
   under grants 0612147 and 0615062.
Any opinions, findings, and conclusions or recommendations
   expressed in this material are those of the authors and do not
   necessarily reflect the views of the NSF.

\bibliographystyle{abbrv}
\bibliography{pads,vivek}

\end{document}

%%% Local Variables:
%%% mode: outline-minor
%%% End:


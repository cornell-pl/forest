%\documentclass{acm_proc_article-sp-sigmod06}
\documentclass{sig-alternate-sigmod06}
\usepackage{url}
\usepackage{stmaryrd}
\usepackage{epsfig}
\usepackage{alltt}
\usepackage{times}
\usepackage{code}
\usepackage{xspace}

\renewcommand{\floatpagefraction}{0.9}
\renewcommand{\dbltopfraction}{0.9}
\renewcommand{\dblfloatpagefraction}{0.9}

\newcommand{\cut}[1]{}

\newcommand{\appref}[1]{Appendix~\ref{#1}}
\newcommand{\secref}[1]{Section~\ref{#1}}
\newcommand{\tblref}[1]{Table~\ref{#1}}
\newcommand{\figref}[1]{Figure~\ref{#1}}
\newcommand{\listingref}[1]{Listing~\ref{#1}}
%\newcommand{\pref}[1]{{page~\pageref{#1}}}

\newcommand{\eg}{{\em e.g.}}
\newcommand{\cf}{{\em cf.}}
\newcommand{\ie}{{\em i.e.}}
\newcommand{\etc}{{\em etc.\/}}
\newcommand{\naive}{na\"{\i}ve}
\newcommand{\role}{r\^{o}le}
\newcommand{\forte}{{fort\'{e}\/}}

\newcommand{\bftt}[1]{{\ttfamily\bfseries{}#1}}
\newcommand{\kw}[1]{{\tt #1 }}
\newcommand{\pads}{\textsc{pads}}
\newcommand{\padsc}{\textsc{pads/c}}
\newcommand{\padsl}{\textsc{padsl}}
\newcommand{\C}{\textsc{C}}
\newcommand{\ocaml}{\textsc{O'Caml}}
\newcommand{\java}{\textsc{Java}}

\newcommand{\mono}[1]{\texttt{#1}}
\newcommand{\dibbler}{Sirius}
\newcommand{\ningaui}{Altair}
\newcommand{\darkstar}{Regulus}

\newcommand{\abstractdm}{abstract data model}
\newcommand{\concretedm}{concrete data model}
\newcommand{\typeddm}{type-specialized concrete data model}

\title{PADS: An End-to-end System for Processing Ad Hoc Data}
\author{Mark Daly\\Yitzhak Mandelbaum\\David Walker\\
       {Princeton University}\\
       {\mono{mdaly@princeton.edu}}\\{\mono{\{yitzhakm,dpw\}@cs.princeton.edu}}
\and Mary Fern\'andez\\Kathleen Fisher\\
       {AT\& Labs Research}\\
       {\mono{\{mff,kfisher\}@research.att.com}}
\and   {Robert Gruber\titlenote{Work conducted while at AT\&T
                                     Labs Research.}}\\
       {Google}\\
       {\mono{gruber@google.com}}}

\date{\today}


\begin{document}

\maketitle
\begin{abstract}
Enormous amounts of data exist in ``well-behaved'' formats such as
relational tables and XML, which come equipped with extensive tool
support.  However, vast amounts of data also exist in non-standard or
\textit{ad hoc} data formats, which often lack standard or extensible
tools. This deficiency forces data analysts to implement
their own tools for parsing, querying, and analyzing their ad hoc
data.  The resulting tools typically interleave parsing, querying, and
analysis, obscuring the semantics of the data format and making it
nearly impossible for others to resuse the tools.

This proposal describes \pads{}, an end-to-end system for processing
ad hoc data sources.  The core of \pads{} is a declarative
language for describing ad hoc data sources and a data-description
compiler that produces customizable libraries for parsing the ad hoc
data.  A suite of tools built around this core include statistical
data-profiling tools, a query engine that permits viewing ad hoc
sources as XML and for querying them with XQuery, and an interactive
front-end that helps users produce \pads{} descriptions quickly.

Details about the \pads{} system are reported in technical
papers~\cite{fernandez+:padx,fisher+:pldi05,fisher+:popl06}.  A
shorter version of this proposal was accepted to the PLAN-X 2006
workshop~\cite{daly+:launchpads}.  An open-source implementation of
\pads{} is available for download~\cite{padsmanual}.
\end{abstract}

\section{Introduction}
\label{sec:intro}

\datascript{}~\cite{gpce02}. \packettypes{}~\cite{sigcomm00}. \padsc{}~\cite{fisher+:pads}
and \padsml{}~\cite{mandelbaum+:padsml}. Bro\cite{paxson:bro}. These
are but a few of the many languages designed for describing data
formats. In his classic paper {\em The Next 700 Programming
  Languages}, 1966~\cite{landin:700}, Landin asserts that principled
programming language design involves thinking in terms of ``families
of languages'' and choosing from a ``well-mapped space.''  However,
when it comes to the domain of processing ad hoc data, there is no
well-mapped space and no systematic understanding of the family of
languages one might be dealing with.

In our previous work, we developed the data description calculus
\ddcold{} to capture the core features of many existing data
description languages~\cite{fisher+:next700ddl}, like \padsc{},
\packettypes{} and \datascript{}. Given the broad applicability of
\ddcold{}, we wanted to use it to define the semantics of
\padsml{}. However, the polymorphic types that we wished to include in
\padsml{} can not be formalized with \ddcold{}.  In addition, both
\padsc{} and \padsml{} generate tools from data format descriptions to
{\em print} data in the specified format. For
many applications, printing data correctly can be as important as
parsing it correctly. Yet, our previous work
specified only the type and parsing semantics of \ddcold{}. 

In this work, we address both of these limitations of
\ddcold{}. First, we extend \ddcold{} with abstractions over types to
create \ddc. In the process, we also improve the \ddc\ theory, as
noted in \secref{sec:ddc-sem}. The new \ddc provides basis for
specifying the semantics of \padsml{}. Second, we specify the a
printing semantics for the new \ddc{}.  We used this new
semantics to guide the \padsml{} implementation of printing.
\secref{sec:ddc} presents the extended \ddc{} calculus, focusing on
the semantics of polymorphic types for parsing and the key elements of
the printing semantics.  We show that both parsers and printers in the
\ddc{} are type correct and furthermore that parsers produce pairs of
parsed data and parse descriptors in {\em canonical form}, and that
printers, given data in canonical form, print successfully.

In summary, this work makes the following key contributions:
\begin{itemize}
\item We have defined the formal semantics of both \padsml{} parsers 
and printers. 
\item We have proven our generated code is type safe and
well-behaved as defined by a canonical forms theorem.
\end{itemize}

%%% Local Variables: 
%%% mode: latex
%%% TeX-master: "paper"
%%% End: 

% pads.sty
%% Command file for pads core calculus papers
\NeedsTeXFormat{LaTeX2e}
\ProvidesPackage{pads}

\RequirePackage{xspace}
\RequirePackage{math-cmds}
\RequirePackage{proof}
\RequirePackage{amsmath}
\RequirePackage{amssymb}
\RequirePackage{multirow}

%\RequirePackage{inference-rules}

\newcommand{\gD}{\Delta\xspace}
\newcommand{\gG}{\Gamma\xspace}
\newcommand{\gM}{\textrm{M}\xspace}
\newcommand{\gP}{\Pi\xspace}
\newcommand{\gS}{\Sigma\xspace}

\newcommand{\ga}{\alpha\xspace}
\newcommand{\gb}{\beta\xspace}
\newcommand{\gk}{\kappa\xspace}
\newcommand{\gl}{\lambda\xspace}
\newcommand{\gm}{\mu\xspace}
\newcommand{\gn}{\nu\xspace}
\newcommand{\gp}{\pi\xspace}
\newcommand{\go}{\omega\xspace}
\newcommand{\gs}{\sigma\xspace}
\newcommand{\gt}{\tau\xspace}

\newcommand{\mth}[1]{$#1$}
\newcommand{\turn}{\mathrel{\vdash}}
\newcommand{\fv}[1]{\mathsf{FV}(#1)}
\newcommand{\ftv}[1]{\mathsf{FTV}(#1)}
\newcommand{\dom}[1]{\mathsf{dom}(#1)}
\newcommand{\ltran}{{\{\!|}}
\newcommand{\rtran}{{|\!\}}}
\newcommand{\lsem}{{[\![}}
\newcommand{\rsem}{{]\!]}}

% code-font vertical bar
\newcommand{\cvb}{\mathrel{\mbox{\tt |}}}
% code-font equals
\newcommand{\ceq}{\mathrel{\mbox{\tt =}}}

  %% parser type of a context
\newcommand{\ptyc}[1]{\mathsf{PT}(#1)}
\def\itsem[#1]{\lsem#1\rsem_{\text{rep}}}
\def\itpdsem[#1]{\lsem#1\rsem_{\text{pd}}}
\def\itbdsem[#1]{\lsem#1\rsem_{\text{body}}}
\def\kTrans[#1,#2]{\mathsf{PT}({#2}{:}{#1})}
\def\tTrans[#1]{\lsem#1\rsem}
\def\pdtTrans[#1]{\lsem#1\rsem_{pd}}
\def\stsem[#1,#2,#3]{#2 \turn #1 : #3}
\def\ddck[#1,#2,#3,#4]{#2 \turn_{#4} #1 : #3}
%% \def\trans[#1,#2,#3]{\lsem#1\rsem^{#2}_{#3}}
\def\trans[#1,#2,#3]{\lsem#1\rsem}
\def\arrTrans[#1,#2]{\lsem#1\rsem^{#2}_{array}}
\def\sTrans[#1,#2]{\lsem#1\rsem^{#2}_{scan}}
\newcommand{\ityeq}[2]{#1 \equiv #2}
\newcommand{\asub}[2]{#1(#2)} %% apply a substitution
\newcommand{\wfd}[2]{#1 \turn #2 \; \mathsf{ok}}
\newcommand{\fnm}[1]{$\mathsf{#1}$} %% function name
\newcommand{\kwd}[1]{\mathbf{#1}} %% keyword
\newcommand{\codefont}[1]{\mathtt{#1}} %% a piece of code
\newcommand{\nrm}[1]{\mathnormal {#1}} %% normal math

%% Meta-variables
  %% contractiveness
\newcommand{\mcon}{c}
\newcommand{\const}{c}
\newcommand{\var}{x}
\newcommand{\kind}{\gk}    %% kind
\newcommand{\ty}{\gt}      %% type
\newcommand{\tyval}{\gn}   %% a fully reduced type
\newcommand{\ity}{\gs}     %% internal language type
\newcommand{\ctxt}{\gG}    %% context
\newcommand{\pctxt}{\gD}   %% context for poly. vars
\newcommand{\rctxt}{\gM}   %% context for rec. vars
\newcommand{\dmn}{D}       %% semantic domain
\newcommand{\obj}{o}       %% semantic object
\newcommand{\data}{B}      %% external data
\newcommand{\loc}{\ell}    %% location
\newcommand{\off}{\go}     %% offset
\newcommand{\term}{{term}} %% term
\newcommand{\bterm}{{bt}}  %% back term
\newcommand{\expr}{e}      %% expression

%% PADS core language syntax
% \newcommand{\pterm}{\kwd{term}} 
  %% The kind of types, *.
\newcommand{\kty}{\mathsf{T}}
  %% contractive
\newcommand{\con}{y}
  %% non-contractive
\newcommand{\ncon}{n}
\newcommand{\pvar}{x} 
\newcommand{\ptyvar}{\ga}
% \newcommand{\plam}[3]{\gl #1{:}#2.#3}
\newcommand{\plam}[3]{\gl #1.#3}
\newcommand{\plamA}[2]{\gl^* #1.#2}
\newcommand{\papp}[2]{#1 \, #2}
\newcommand{\pbase}[1]{C(#1)} 
% \newcommand{\pnone}{\kwd{none}}
\newcommand{\pnone}{\kwd{true}}
\newcommand{\ptrue}{\kwd{true}}
\newcommand{\pfalse}{\kwd{false}}
\newcommand{\psig}[3]{\gS \, #1{:}#2.#3} 
%\newcommand{\psum}[3]{#1 +_{#2} #3} 
\newcommand{\psum}[3]{#1 + #3} 
\newcommand{\pand}[2]{#1 \, \& \, #2}
\newcommand{\pseq}[3]{#1 \, \kwd{seq}(#2,#3)}
\newcommand{\pterm}[2]{#1,#2}
\newcommand{\pset}[3]{\{#1{:}#2 \,|\, #3\}} 
\newcommand{\pcase}[4]
   {\kwd{case} \; #1 \; \kwd{of} \, (#2 \Rightarrow #3 \, | \, \_ \Rightarrow #4) } 
\newcommand{\pcaseA}[3]
   {\kwd{case}^* \; #1 \; \kwd{of} \, (#2 \Rightarrow #3)}
\newcommand{\pcomputen}{\kwd{compute}}
\newcommand{\pcompute}[2]{\pcomputen{}(#1{:}#2)}
\newcommand{\pabsorbn}{\kwd{absorb})}
\newcommand{\pabsorb}[1]{\pabsorbn{}(#1)}
  %% transfrom pd
\newcommand{\pxpd}[2]{\kwd{pdtrans}(#1,#2)}
\newcommand{\pscann}{\kwd{scan}}
\newcommand{\pscan}[1]{\pscann{}(#1)}
\newcommand{\ptry}[2]{\kwd{try}(#1,#2)}
\newcommand{\ptransform}[3]{#1 \; #2 \rightsquigarrow #3}
\newcommand{\pmu}[2]{\gm #1.#2}

%% \newcommand{\p}[]{}

  %% generic interface macro
\newcommand{\Igen}[1]{\mathcal{I}_{\text{#1}}}
  %% base type kind interface
\newcommand{\Ikind}{\Igen {kind}}
  %% base type rep type interface
\newcommand{\Irty}{\Igen {type}}
  %% base type pd type interface
\newcommand{\Ipdty}{\Igen {pdType}}
  %% Base type implementation interface
\newcommand{\Iimp}{\Igen {imp}}
  %% constant type interface
\newcommand{\Icty}{\Igen {cty}}
  %% operator type interface
\newcommand{\Iopty}{\Igen {opty}}
\newcommand{\defty}[2]{{\rm default\_ty}(#1) = #2}

% Terminator syntax
  % bw plus
\newcommand{\BWp}[1]{\text{BW+}(#1)}
  % bw minus
\newcommand{\BWm}[1]{\text{BW-}(#1)}
  % fw
\newcommand{\FW}[1]{\text{FW}(#1)}
  % terminator combination
\newcommand{\tand}{\mathrel{\&\&}}


  % Internal language name
\newcommand{\implang}{host\xspace}
 %% capitalized version
\newcommand{\Implang}{Host\xspace}

% Internal language types syntax
\newcommand{\tyface}[1]{\texttt{#1}}

\newcommand{\ibasety}{a}
\newcommand{\ibitsty}{\tyface{bits}}
\newcommand{\iunitty}{\tyface{unit}}
\newcommand{\ioffty}{\tyface{offset}}
\newcommand{\ispty}{\tyface{span}}
\newcommand{\ilocty}{\tyface{loc}}
\newcommand{\iecty}{\tyface{errcode}}
  % int type. ``ty'' appended to avoid conflict with amsmath.iint.
\newcommand{\iintty}{\tyface{int}}
\newcommand{\iboolty}{\tyface{bool}}
\newcommand{\iprodty}[2]{#1 * #2}
  % deprecated version of \iprodty
\newcommand{\iprod}[2]{\iprodty {#1}{#2}}
  %% infix version - just the operator
\newcommand{\iprodi}{*}
\newcommand{\isum}[2]{#1 + #2}
  % type for arrays
\newcommand{\iseqty}[1]{#1 \, \tyface{seq}}
  % deprecated version of \iarrty
\newcommand{\iseq}[1]{\iseqty {#1}}
\newcommand{\iarrow}[2]{#1 \rightarrow #2}
\newcommand{\iarrowi}{\rightarrow}
\newcommand{\ioparrow}[2]{#1 \rightharpoonup #2}
\newcommand{\imu}[2]{\gm #1.#2}
  % noval type
\newcommand{\invty}{\tyface{noval}}
  % error type
\newcommand{\ierrty}[1]{#1 \; \tyface{error}}
  % standard pd type
\newcommand{\ipty}[1]{\iprod {\tyface{pd\_hdr}}{#1}}
\newcommand{\pdtyvar}[1]{{#1}\, \mbox{s.t.}\, \ispdty {#1}}
\newcommand{\pdtyvars}[2]{#1,#2\; \mbox{s.t.}\; \ispdty {#1},\ispdty {#2}}
% has header ddc ty vars
%\newcommand{\hdtvs}[2]{#1,#2\; \mbox{s.t.}\; \ispdty {\itpdsem[{#1}]},\ispdty {\itpdsem[{#2}]}}
\newcommand{\hdtvs}[2]{#1,#2}
\newcommand{\iaptyname}{\tyface{arr\_pd}}
\newcommand{\iapty}[1]{\ipty {(\iaptyname \; #1)}}
\newcommand{\ityvar}{\ga}

%% Internal language term syntax
\newcommand{\iconst}{\codefont{c}}
\newcommand{\ivar}{\codefont{x}}
\newcommand{\iexp}{\codefont{e}}
\newcommand{\idata}{\codefont B}
\newcommand{\iuval}{\codefont{()}}
\newcommand{\iok}{\codefont{ok}}
\newcommand{\iecerr}{\codefont{err}}
\newcommand{\iecpc}{\codefont{pc}}
\newcommand{\iop}[1]{{op}(#1)}
\newcommand{\inotop}{\codefont{not}}
\newcommand{\isizeofop}{\codefont{sizeof}}
\newcommand{\isizeof}[1]{\isizeofop(#1)}
% \newcommand{\iinl}[2]{\kwd{inl}_{#1} \; #2}
% \newcommand{\iinr}[2]{\kwd{inr}_{#1} \; #2}
\newcommand{\iinl}[1]{\codefont{inl} \; #1}
  % deprecated version:
\newcommand{\iinld}[2]{\codefont{inl} \; #2}
\newcommand{\iinr}[1]{\codefont{inr} \; #1}
  % deprecated version:
\newcommand{\iinrd}[2]{\codefont{inr} \; #2}
\newcommand{\icase}[4]
  {\codefont{case} \; #1 \; \codefont{of} \, (\codefont{inl \, x} \Rightarrow #3
    \, | \, \codefont{inr \, x}  \Rightarrow #4) }
  % more general version
\newcommand{\icaseg}[5]
  {\codefont{case} \; #1 \; \codefont{of} \, (\codefont{inl \, #2} \Rightarrow #3
    \cvb \codefont{inr \, #4}  \Rightarrow #5) }
% \newcommand{\icase}[4]
%   {\kwd{case} \; #1 : #2 \; \kwd{of} \, (\kwd{inl} \, x \Rightarrow #3
%     \, | \, \kwd{inr} \, x  \Rightarrow #4) }
\newcommand{\ilam}[3]{\gl \codefont{#1}.#3}
%\newcommand{\ilam}[3]{\gl #1{:}#2.#3}
\newcommand{\ipair}[2]{(#1,#2)}
\newcommand{\itup}[1]{\codefont({#1}\codefont)}
\newcommand{\ictup}[1]{\codefont{(#1)}}
\newcommand{\ipi}[2]{\codefont{\gp_{#1}} \, #2}
\newcommand{\iarr}[1]{\codefont{[}#1\codefont{]}}
\newcommand{\ieseq}{\codefont{[]}}
\newcommand{\iappend}[2]{#1 \; @ \; #2}
%%\newcommand{\isub}[2]{\codefont{sub_{#1}} \; #2}
\newcommand{\isub}[2]{#1\,\codefont{[{#2}]}}
\newcommand{\ifoldr}[3]{\codefont{foldr} \; #1 \, #2 \, #3}
\newcommand{\iroll}[2]{\codefont{roll}(#1,#2)}
\newcommand{\iunroll}[1]{\codefont{unroll}(#1)}
\newcommand{\iapp}[2]{#1 \; #2}
\newcommand{\iappi}{\;}
  %% Exception
\newcommand{\ifail}{\codefont{fail}}
  %% Plain error, no value.
\newcommand{\ierr}{\codefont{noval}}
  %% Error with value.
\newcommand{\ierror}[1]{\codefont{error}(#1)}
\newcommand{\iexamine}[1]{\codefont{examine}(#1)}
\newcommand{\itrue}{\codefont{true}}
\newcommand{\ifalse}{\codefont{false}}
\newcommand{\ilet}[2]{\codefont{let \; #1} \ceq #2 \; \codefont{in}}
%\newcommand{\icasess}[1]{\codefont{case} \; #1 \; \codefont{of}}
%\newcommand{\ipattss}[2]{| \; \codefont{#1} \Rightarrow \codefont{#2}}
\newcommand{\iif}[1]{\codefont{if} \; {#1}}
\newcommand{\ithen}[1]{\codefont{then} \; #1}
\newcommand{\ielse}[1]{\codefont{else} \; #1}
\newcommand{\iin}{\codefont{in}}
% \newcommand{\iletinend}[2]{
%   \codefont{let} \\ \quad
%   \begin{array}{l}
%     #1
%   \end{array} \\
%   \codefont{in} \\ \quad
%   \begin{array}{l}
%     #2
%   \end{array} \\
%   \codefont{end}
% }
% \newcommand{\ival}[2]{\codefont{val} \; #1 = #2}
\newcommand{\ifun}[3]{\codefont{fun} \; \codefont{#1 \; #2} \ceq {#3}}
\newcommand{\iletfun}[3]{\codefont{letfun} \; \codefont{#1 \; #2} \ceq #3}
\newcommand{\ifunand}[3]{\codefont{and} \; \codefont{#1 \, #2} \ceq #3}
\newcommand{\ior}[2]{#1 \; \codefont{or} \; #2}
\newcommand{\iori}{\; \codefont{or} \;}
\newcommand{\iandi}{\; \codefont{and} \;}

%% Semantic Domain syntax
  %% semantic definition
%\newcommand{sbox
\newcommand{\sdefm}[1]{\gather* #1\endgather}
\newcommand{\sdef}[1]{\begin{array}{l}#1\end{array}}
\newcommand{\sfn}[3]{\ilam {\codefont{#1}}{#2}{#3}}
  %% recursive function
\newcommand{\sfun}[4]{\text{fun} \, #1(#2{:}#3).#4}
  %% prefix application macro
\newcommand{\sappp}[2]{\iapp {#1}{#2}}
  %% infix application (just a space).
\newcommand{\sapp}{\iappi}
\def\spair<#1>{\itup {#1}} 
\newcommand{\spi}[2]{\ipi {#1} {#2}}
\newcommand{\strue}{\codefont{true}}
  %% Equivalence in sem. dom.
\newcommand{\semeq}{\mathrel{==}}
  %% Semantics Domains
\newcommand{\extdom}{\tyface{bits}}
\newcommand{\intdom}{\tyface{IV}}
\newcommand{\locdom}{\tyface{Loc}}
\newcommand{\offdom}{\ioffty}
\newcommand{\consdom}{\tyface{Consume}}
\newcommand{\modedom}{\tyface{Mode}}

% Built-ins
  %% rep and pd constructors themselves (i.e. not applied).
\newcommand{\newrepf}[1]{\codefont{R_{#1}}}
\newcommand{\newpdf}[1]{\codefont{P_{#1}}}
  %% generic rep constructor
\newcommand{\newrep}[2]{\newrepf{#1}(\codefont{#2})}
  %% generic pd constructor
\newcommand{\newpd}[2]{\newpdf{#1}(\codefont{#2})}
  %% EoF predicate
\newcommand{\eofpred}[1]{\codefont{EoF(#1)}}
  %% isOk function
\newcommand{\pdok}[1]{\codefont{isOk(#1)}}
  %% isErr function
\newcommand{\pderr}[1]{\codefont{isErr(#1)}}
\newcommand{\isdone}[1]{\codefont{isDone} \sapp \ictup{#1}}
\newcommand{\scanmax}{\codefont{SCAN\_MAX}}
\newcommand{\seterr}[1]{\kwd{SetErr} \, #1}
\newcommand{\setmode}[2]{\kwd{SetMode} \, #1 \, #2}
\newcommand{\incnr}[1]{\kwd{IncNR} \, #1}

%% Abbreviations
\newcommand{\lampair}[1]{\gl \spair<\codefont{\data},\off>.#1}
%\newcommand{\funpair}[2]{\text{fun} \, #1\spair<\data,\off>.#2}

%% Misc. operations
  %% well typed judgment
\def\wellty(#1,#2,#3){#1 \turn #2 : #3}
  %% extended context
\newcommand{\ectxt}[1]{\ctxt,{#1}}
  %% extended rec. context
\newcommand{\erctxt}[2]{\rctxt,{#1}{=}\pmu {#1} {#2}}
  %% truth judgment
\newcommand{\tjudge}[2]{#1 \vDash #2}
  %% negated truth judgment
\newcommand{\tnjudge}[2]{#1 \nvDash #2}
  %% ok parse descr. judgment
\newcommand{\okjudge}[1]{\vDash #1 \; \kwd{ok}}
  %% err parse descr. judgment
\newcommand{\errjudge}[1]{\vDash #1 \; \kwd{err}}
  %% instruction-level dynamic semantics ``steps to''.
\newcommand{\stepstoi}{\hookrightarrow_i}
  %% top-level dynamic semantics ``steps to''.
\newcommand{\stepsto}{\hookrightarrow}
  %% Klean-closure of top-level ``steps to''.
\newcommand{\kstepsto}{\stepsto^*}
  %% indicates a problem to be fixed.
\newcommand{\fixme}{\mathbf{???}}
  %% error correllation ind. hypothesis
\newcommand{\ecpred}[2]{\mathrm{EC}(#1:#2)}
  %% pd error count
\newcommand{\pecnt}[1]{\mathrm{PEC}( \, #1)}
  %% Type preservation predicate.
\newcommand{\tppred}[2]{\mathrm{TP}_{#1}(#2)}
  %% Deep error correlation relation
\newcommand{\dcorr}[2]{\mathrm{DeepCorr}(#1,#2)}
  %% Another error correlation relation (generic)
\newcommand{\corrg}[2]{\mathrm{Corr}(#1,#2)}
  %% Another error correlation relation
\newcommand{\corr}[3]{\mathrm{Corr_{#1}}(#2,#3)}
  %% Clean closure version of correlation relation
\newcommand{\corrkl}[3]{\mathrm{Corr^*_{#1}}(#2,#3)}
  %% Error Correlation predicate.
\newcommand{\cepred}[2]{\mathrm{CE}_{#1}(#2)}
  %% Function Error Correlation predicate.
\newcommand{\cefpred}[2]{\mathrm{CE}_{#1}(#2)}
  %% Cannonical Formas pred.
\newcommand{\canfm}[3]{\mathrm{CF_{#1}}(#2,#3)}
  %% Is-pd predicate:
\newcommand{\ispdty}[1]{\mathrm{hh}(#1)}
  %% No Errors predicate
\newcommand{\noerr}[1]{\mathsf{Clean}(#1)}

  %% BW termination judgment
\newcommand{\btjudge}[3]{#1 \vDash #2 \Rightarrow #3}
  %% termination judgment
\newcommand{\ttjudge}[4]{#1 \vDash #2 (#3) \Rightarrow #4}
  %% negated termination judgment
\newcommand{\ttnjudge}[3]{#1 \nvDash #2 (#3)}

  %% inverse infer
\newcommand{\iinfer}[2]{\infer{#2}{#1}}

%% The following macros are taken from the pads manual, 
%% file defs.tex.

%% dave added a couple here.

  %% keywords and PADSL types: added micro space on either side
\newcommand{\bftt}[1]{{\ttfamily\bfseries{}#1}}
\newcommand{\padskw}[1]{\text{\/\/\bftt{#1}\/\/}}
\newcommand{\cd}[1]{\texttt{#1}}

\newcommand{\Pbase}[1]{\padskw{C}(#1)} 
\newcommand{\Pomit}{\padskw{Pomit}}
\newcommand{\Pcompute}{\padskw{Pcompute}}
\newcommand{\Pendian}{\padskw{Pendian}}
\newcommand{\Pstruct}{\padskw{Pstruct}}
\newcommand{\Punion}{\padskw{Punion}}
\newcommand{\Popt}{\padskw{Popt}}
\newcommand{\Pchar}{\padskw{Pchar}}
\newcommand{\Pdate}{\padskw{Pdate}}
\newcommand{\Puint}{\padskw{Puint32}}
\newcommand{\Pip}{\padskw{Pip}}
\newcommand{\Pstring}{\padskw{Pstring}}
\newcommand{\Prec}{\padskw{Prec}}
\newcommand{\Pfun}{\padskw{Pfun}}
\newcommand{\Parray}{\padskw{Parray}}
\newcommand{\Palternate}{\padskw{Palternate}}
\newcommand{\Ptypedef}{\padskw{Ptypedef}}
\newcommand{\Penum}{\padskw{Penum}}
\newcommand{\Pwhere}{\padskw{Pwhere}}
\newcommand{\Palt}{\padskw{Palt}}
\newcommand{\Pparsecheck}{\padskw{Pparsecheck}}
\newcommand{\Pterm}{\padskw{Pterm}}
\newcommand{\Psep}{\padskw{Psep}}
\newcommand{\Pre}{\padskw{Pre}}
\newcommand{\Pnosep}{\padskw{Pnosep}}
\newcommand{\Plongest}{\padskw{Plongest}}
\newcommand{\Plast}{\padskw{Plast}}
\newcommand{\Pended}{\padskw{Pended}}
\newcommand{\Peor}{\padskw{Peor}}
\newcommand{\Peof}{\padskw{Peof}}
\newcommand{\Pforall}{\padskw{Pforall}}
\newcommand{\Pfrom}{\padskw{Pfrom}}
\newcommand{\Pin}{\padskw{Pin}}
\newcommand{\Precord}{\padskw{Precord}}
\newcommand{\Psource}{\padskw{Psource}}
\newcommand{\Pcase}{\padskw{Pcase}}
\newcommand{\Pswitch}{\padskw{Pswitch}}
\newcommand{\Pdefault}{\padskw{Pdefault}}
\newcommand{\Psome}{\padskw{Psome}}
\newcommand{\Pnone}{\padskw{Pnone}}
\newcommand{\Pcharclass}{\padskw{Pcharclass}}
\newcommand{\Pprefix}{\padskw{Pprefix}}
\newcommand{\Plit}[1]{\padskw{Plit} \; #1}

  %% IPADS Parray 
\newcommand{\iParray}[4]{#1 \; \Parray{}(#2,#3)}

  %% Conversion from surface language to core calculus
\newcommand{\conv}[2]{#1 \Longrightarrow #2}


%%%%%%%%%%%%%%%%%%%%%%%%%%%%%%%%%%%%%%%%%%%%%%%%%%%%%%
%%                 Environments                     %%
%%%%%%%%%%%%%%%%%%%%%%%%%%%%%%%%%%%%%%%%%%%%%%%%%%%%%%

% empty environment used for scoping of declarations.
\newenvironment{scope}{}{}

%% Environment for typesetting BNF grammars. Uses display math mode.
\newenvironment{bnf}
     {%% local command definitions:
        %% BNF definition symbol
      \def\->{\rightarrow}
%%      \def\::={{::=} &}
%      \def\::={\bnfdef &}
      \def\::={\mathrel{::=} &}
      \def\|{\bnfalt}
      \newcommand{\name}[1]{\text{##1}}
      % name spanning multiple rows
%      \newcommand{\mname}[2]{\multirow{##2}{*}{\begin{tabular}{l}##1\end{tabular}}}
      \newcommand{\mname}[2]{\multirow{##2}{.6in}{##1}}
        %% non-terminal
      \newcommand{\nont}[1]{\mathit{##1}}
      \newcommand{\meta}[1]{& \mathit{##1} &}
      \newcommand{\descr}[1]{& \text{// ##1}}
      \newcommand{\opt}[1]{ [##1] }
      \newcommand{\opnon}[1]{\opt{\nont{##1}}}
      \newcommand{\none}{\epsilon}
      \newcommand{\nwln}{\\ &&&}
      \newcommand{\nlalt}{\\ && \| &}
      \[\begin{array}{lrcll}
     }
     {\end{array}\]}

%% Environment for typesetting BNF grammars. Uses standard text mode.
\newenvironment{bnft}
     {%% local command definitions:
        %% BNF definition symbol
      \def\->{$\rightarrow$}
      \def\::={{::=} &}
      \def\|{\ $\mid$ \ }
      \newcommand{\name}[1]{\textit{##1} &}
        %% non-terminal
      \newcommand{\nont}[1]{\textit{##1}}
      \newcommand{\meta}[1]{}
      \newcommand{\descr}[1]{& // ##1}
      \newcommand{\opt}[1]{ [##1] }
      \newcommand{\opnon}[1]{\opt{\nont{##1}}}
      \newcommand{\none}{$\epsilon$}
      \newcommand{\nwln}{\\ &&}
      \newcommand{\nlalt}{\\ & \| &}
%      \newcommand{\meta}[1]{##1 &}
%      \newcommand{\nwln}{\\ &&&}
%      \newcommand{\nlalt}{\\ && \| &}
%      \begin{tabular}{lrcll}
      \begin{tabular}{lcll}
     }
     {\end{tabular}}

\newenvironment{semdef}
    {%% local command definitions:
       %% arrow kind
     \def\->{\rightarrow}
     \allowdisplaybreaks
     \gather}
    {\endgather}

%% \newenvironment{semdef}
%%     {%% local command definitions:
%%        %% arrow kind
%%      \def\->{\rightarrow}
%%      \[\begin{array}{l}\allowdisplaybreaks
%%     }
%%     {\end{array}\]}

% future tag macro. does nothing now.
\newcommand{\stag}[1]{}
\newenvironment{semcasedef}[1]{
        %% label for an alternative.
      \newcommand{\altlbl}[1]{\qquad \text{(Case ##1:)}}
        %% value of an alternative.
      \newcommand{\altval}[1]{\quad ##1}
        %% semantic let: let a variable/pattern equal a value
      \newcommand{\slet}[2]{\texttt{let} \; ##1 = ##2 \; \texttt{in}}
     \tag{#1}
     \begin{array}{l}}
    {\end{array}}

%% \newenvironment{semrule}
%%     {\begin{array}{l}}
%%     {\end{array}}

\newenvironment{semcond}
    {% Local command definitions:
       %% ``if'' condition
     \newcommand{\cif}{\text{if} &}
       %% ``and'' condition
     \newcommand{\cand}{\text{and} &}
       %% let a variable/pattern equal a value
     \newcommand{\vlet}[2]{##2 = ##1}
     %\qquad
     \begin{array}{ll}}
    {\end{array}}

\newcommand{\phide}[1]{}
\newcommand{\preplace}[2]{#2}
\newcommand{\pext}[1]{}

% \subsection{Related Work}

\subsection {Current and Future Work}

thoughts on problems with tokenization; information theory; partial
descriptions; user interface; recursion; more experimentation with a
broader range of formats


\bibliographystyle{abbrv}

{\small
\bibliography{../pads,../galax}}
\end{document}


\section{Using XQuery and \Galax{}[Mary]}
\label{section:galax}

XML is blah. One paragraph on XQuery and where to learn about it.

XQuery is a typed, functional language that supports user-defined
functions and modules for structuring large queries.  It contains
XPath 2.0~\cite{xpath} as a proper sublanguage, which supports
navigation, selection, and extraction of fragments of XML documents.
XQuery also includes expressions to construct new XML values, and to
integrate or join values from multiple documents.  XQuery's type
system is based on XML Schema. 

Establish why XQuery was a reasonable query language for PADS.
Standard.  Statically typed and PADS descriptions are types.  

XML is obvious target for all data sources.  Cite numerous
tools for converting to XML and commercial DB support for non-XML
sources.  Do not re-invent the wheel.

\subsection{Galax's Abstract Data Model}

Abstract object-oriented data model that permits querying of virtual
XML sources.  

Tree data model.  Data model accessors (axis::node-tests) that can/should be implemented
efficiently by the underlying source are pushed into the OO tree data
model.  Default implementations for sources that don't do anything
clever.   
\begin{figure}
\begin{small}
\begin{code}
  method virtual node_name  : unit -> atomicQName option
  method virtual parent     : node_test option -> node option
  method virtual children   : node_test option -> node cursor
  method descendant_or_self : node_test option -> node cursor
\end{code}
\end{small}
\caption{Signatures for methods in Galax's abstract data model}
\end{figure}

Motivation for abstract data model: Supporting PADS and secondary
storage system occurred at same time.  Data and information
integration.  

Implementations provided: Main-memory DOM-like rep; Main-memory
shredded (HashTable) rep; Secondary-storage shredded rep; PADS.

Part of PADX implementation are generic functions that implement data
model accessors.  PADS compiler generates type-specific functions for
walking virtual XML tree.  Relate back to type-specific library
functions mentioned in last section.


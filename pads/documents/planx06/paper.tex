\documentclass[nocopyrightspace]{sigplanconf}
\usepackage{url}
\usepackage{stmaryrd}
\usepackage{epsfig}
\usepackage{alltt}
\usepackage{times}
\usepackage{code}
\usepackage{xspace}

\renewcommand{\floatpagefraction}{0.9}

\newcommand{\cut}[1]{}

\newcommand{\appref}[1]{Appendix~\ref{#1}}
\newcommand{\secref}[1]{Section~\ref{#1}}
\newcommand{\tblref}[1]{Table~\ref{#1}}
\newcommand{\figref}[1]{Figure~\ref{#1}}
\newcommand{\listingref}[1]{Listing~\ref{#1}}
%\newcommand{\pref}[1]{{page~\pageref{#1}}}

\newcommand{\eg}{{\em e.g.}}
\newcommand{\cf}{{\em cf.}}
\newcommand{\ie}{{\em i.e.}}
\newcommand{\etc}{{\em etc.\/}}
\newcommand{\naive}{na\"{\i}ve}
\newcommand{\role}{r\^{o}le}
\newcommand{\forte}{{fort\'{e}\/}}

\newcommand{\bftt}[1]{{\ttfamily\bfseries{}#1}}
\newcommand{\kw}[1]{{\tt #1 }}
\newcommand{\pads}{\textsc{pads}}
\newcommand{\padsc}{\textsc{pads/c}}
\newcommand{\padsl}{\textsc{padsl}}
\newcommand{\C}{\textsc{C}}
\newcommand{\ocaml}{\textsc{O'Caml}}
\newcommand{\java}{\textsc{Java}}

\newcommand{\dibbler}{Sirius}
\newcommand{\ningaui}{Altair}
\newcommand{\darkstar}{Regulus}

\newcommand{\abstractdm}{abstract data model}
\newcommand{\concretedm}{concrete data model}
\newcommand{\typeddm}{type-specialized concrete data model}
% (or just type-specialized data model or type-specific data model)


\title{PADX : Querying Large-scale Ad Hoc Data with XQuery}
%\title{PADX : An XQuery Interface to Ad Hoc Data Sources}
%\title{PADX : A System for Querying Ad Hoc Data Sources with XQuery}

\authorinfo{Mary Fern\'andez\\Kathleen Fisher}
       {AT\& Labs Research}
       {\mono{\{mff,kfisher\}@research.att.com}}

\authorinfo{ Robert Gruber\titlenote{Work carried out while at AT\&T
                                     Labs Research.}}
       {Google}
       {\mono{gruber@google.com}}

\authorinfo{Yitzhak Mandelbaum}
       {Princeton University}
       {\mono{yitzhakm@cs.princeton.edu}}

\date{\today}


\begin{document}

\maketitle
\begin{abstract}
Enormous amounts of data exist in ``well-behaved'' formats
such as XML and relational databases which come equipped with
extensive tool support.  However,  vast amounts of data also exist
in non-standard or \textit{ad hoc} data formats, which lack standard
tools. This lack forces data analysts to implement their own tools for
parsing, querying, and analyzing their ad hoc data.  The 
resulting tools typically interleave parsing, querying, and analysis,
obscuring the data format and making it nearly impossible for others
to resuse the code artifacts.

This paper describes our experience designing and implementing
\padx{}, a system for querying large-scale ad hoc data sources with
XQuery.  \padx{} is the synthesis and extension of two existing
systems: \pads{} and \Galax{}. With \padx{}, an analyst writes a
declarative data description of the physical layout of her ad hoc
data, and the \pads{} compiler produces customizable libraries for
parsing the data and for viewing it as XML.  The resulting library is
linked with an XQuery engine, permitting the analyst to view and query
her ad hoc data sources using XQuery.
\end{abstract}

\section{Introduction}
\label{sec:intro}

\datascript{}~\cite{gpce02}. \packettypes{}~\cite{sigcomm00}. \padsc{}~\cite{fisher+:pads}
and \padsml{}~\cite{mandelbaum+:padsml}. Bro\cite{paxson:bro}. These
are but a few of the many languages designed for describing data
formats. In his classic paper {\em The Next 700 Programming
  Languages}, 1966~\cite{landin:700}, Landin asserts that principled
programming language design involves thinking in terms of ``families
of languages'' and choosing from a ``well-mapped space.''  However,
when it comes to the domain of processing ad hoc data, there is no
well-mapped space and no systematic understanding of the family of
languages one might be dealing with.

In our previous work, we developed the data description calculus
\ddcold{} to capture the core features of many existing data
description languages~\cite{fisher+:next700ddl}, like \padsc{},
\packettypes{} and \datascript{}. Given the broad applicability of
\ddcold{}, we wanted to use it to define the semantics of
\padsml{}. However, the polymorphic types that we wished to include in
\padsml{} can not be formalized with \ddcold{}.  In addition, both
\padsc{} and \padsml{} generate tools from data format descriptions to
{\em print} data in the specified format. For
many applications, printing data correctly can be as important as
parsing it correctly. Yet, our previous work
specified only the type and parsing semantics of \ddcold{}. 

In this work, we address both of these limitations of
\ddcold{}. First, we extend \ddcold{} with abstractions over types to
create \ddc. In the process, we also improve the \ddc\ theory, as
noted in \secref{sec:ddc-sem}. The new \ddc provides basis for
specifying the semantics of \padsml{}. Second, we specify the a
printing semantics for the new \ddc{}.  We used this new
semantics to guide the \padsml{} implementation of printing.
\secref{sec:ddc} presents the extended \ddc{} calculus, focusing on
the semantics of polymorphic types for parsing and the key elements of
the printing semantics.  We show that both parsers and printers in the
\ddc{} are type correct and furthermore that parsers produce pairs of
parsed data and parse descriptors in {\em canonical form}, and that
printers, given data in canonical form, print successfully.

In summary, this work makes the following key contributions:
\begin{itemize}
\item We have defined the formal semantics of both \padsml{} parsers 
and printers. 
\item We have proven our generated code is type safe and
well-behaved as defined by a canonical forms theorem.
\end{itemize}

%%% Local Variables: 
%%% mode: latex
%%% TeX-master: "paper"
%%% End: 

% pads.sty
%% Command file for pads core calculus papers
\NeedsTeXFormat{LaTeX2e}
\ProvidesPackage{pads}

\RequirePackage{xspace}
\RequirePackage{math-cmds}
\RequirePackage{proof}
\RequirePackage{amsmath}
\RequirePackage{amssymb}
\RequirePackage{multirow}

%\RequirePackage{inference-rules}

\newcommand{\gD}{\Delta\xspace}
\newcommand{\gG}{\Gamma\xspace}
\newcommand{\gM}{\textrm{M}\xspace}
\newcommand{\gP}{\Pi\xspace}
\newcommand{\gS}{\Sigma\xspace}

\newcommand{\ga}{\alpha\xspace}
\newcommand{\gb}{\beta\xspace}
\newcommand{\gk}{\kappa\xspace}
\newcommand{\gl}{\lambda\xspace}
\newcommand{\gm}{\mu\xspace}
\newcommand{\gn}{\nu\xspace}
\newcommand{\gp}{\pi\xspace}
\newcommand{\go}{\omega\xspace}
\newcommand{\gs}{\sigma\xspace}
\newcommand{\gt}{\tau\xspace}

\newcommand{\mth}[1]{$#1$}
\newcommand{\turn}{\mathrel{\vdash}}
\newcommand{\fv}[1]{\mathsf{FV}(#1)}
\newcommand{\ftv}[1]{\mathsf{FTV}(#1)}
\newcommand{\dom}[1]{\mathsf{dom}(#1)}
\newcommand{\ltran}{{\{\!|}}
\newcommand{\rtran}{{|\!\}}}
\newcommand{\lsem}{{[\![}}
\newcommand{\rsem}{{]\!]}}

% code-font vertical bar
\newcommand{\cvb}{\mathrel{\mbox{\tt |}}}
% code-font equals
\newcommand{\ceq}{\mathrel{\mbox{\tt =}}}

  %% parser type of a context
\newcommand{\ptyc}[1]{\mathsf{PT}(#1)}
\def\itsem[#1]{\lsem#1\rsem_{\text{rep}}}
\def\itpdsem[#1]{\lsem#1\rsem_{\text{pd}}}
\def\itbdsem[#1]{\lsem#1\rsem_{\text{body}}}
\def\kTrans[#1,#2]{\mathsf{PT}({#2}{:}{#1})}
\def\tTrans[#1]{\lsem#1\rsem}
\def\pdtTrans[#1]{\lsem#1\rsem_{pd}}
\def\stsem[#1,#2,#3]{#2 \turn #1 : #3}
\def\ddck[#1,#2,#3,#4]{#2 \turn_{#4} #1 : #3}
%% \def\trans[#1,#2,#3]{\lsem#1\rsem^{#2}_{#3}}
\def\trans[#1,#2,#3]{\lsem#1\rsem}
\def\arrTrans[#1,#2]{\lsem#1\rsem^{#2}_{array}}
\def\sTrans[#1,#2]{\lsem#1\rsem^{#2}_{scan}}
\newcommand{\ityeq}[2]{#1 \equiv #2}
\newcommand{\asub}[2]{#1(#2)} %% apply a substitution
\newcommand{\wfd}[2]{#1 \turn #2 \; \mathsf{ok}}
\newcommand{\fnm}[1]{$\mathsf{#1}$} %% function name
\newcommand{\kwd}[1]{\mathbf{#1}} %% keyword
\newcommand{\codefont}[1]{\mathtt{#1}} %% a piece of code
\newcommand{\nrm}[1]{\mathnormal {#1}} %% normal math

%% Meta-variables
  %% contractiveness
\newcommand{\mcon}{c}
\newcommand{\const}{c}
\newcommand{\var}{x}
\newcommand{\kind}{\gk}    %% kind
\newcommand{\ty}{\gt}      %% type
\newcommand{\tyval}{\gn}   %% a fully reduced type
\newcommand{\ity}{\gs}     %% internal language type
\newcommand{\ctxt}{\gG}    %% context
\newcommand{\pctxt}{\gD}   %% context for poly. vars
\newcommand{\rctxt}{\gM}   %% context for rec. vars
\newcommand{\dmn}{D}       %% semantic domain
\newcommand{\obj}{o}       %% semantic object
\newcommand{\data}{B}      %% external data
\newcommand{\loc}{\ell}    %% location
\newcommand{\off}{\go}     %% offset
\newcommand{\term}{{term}} %% term
\newcommand{\bterm}{{bt}}  %% back term
\newcommand{\expr}{e}      %% expression

%% PADS core language syntax
% \newcommand{\pterm}{\kwd{term}} 
  %% The kind of types, *.
\newcommand{\kty}{\mathsf{T}}
  %% contractive
\newcommand{\con}{y}
  %% non-contractive
\newcommand{\ncon}{n}
\newcommand{\pvar}{x} 
\newcommand{\ptyvar}{\ga}
% \newcommand{\plam}[3]{\gl #1{:}#2.#3}
\newcommand{\plam}[3]{\gl #1.#3}
\newcommand{\plamA}[2]{\gl^* #1.#2}
\newcommand{\papp}[2]{#1 \, #2}
\newcommand{\pbase}[1]{C(#1)} 
% \newcommand{\pnone}{\kwd{none}}
\newcommand{\pnone}{\kwd{true}}
\newcommand{\ptrue}{\kwd{true}}
\newcommand{\pfalse}{\kwd{false}}
\newcommand{\psig}[3]{\gS \, #1{:}#2.#3} 
%\newcommand{\psum}[3]{#1 +_{#2} #3} 
\newcommand{\psum}[3]{#1 + #3} 
\newcommand{\pand}[2]{#1 \, \& \, #2}
\newcommand{\pseq}[3]{#1 \, \kwd{seq}(#2,#3)}
\newcommand{\pterm}[2]{#1,#2}
\newcommand{\pset}[3]{\{#1{:}#2 \,|\, #3\}} 
\newcommand{\pcase}[4]
   {\kwd{case} \; #1 \; \kwd{of} \, (#2 \Rightarrow #3 \, | \, \_ \Rightarrow #4) } 
\newcommand{\pcaseA}[3]
   {\kwd{case}^* \; #1 \; \kwd{of} \, (#2 \Rightarrow #3)}
\newcommand{\pcomputen}{\kwd{compute}}
\newcommand{\pcompute}[2]{\pcomputen{}(#1{:}#2)}
\newcommand{\pabsorbn}{\kwd{absorb})}
\newcommand{\pabsorb}[1]{\pabsorbn{}(#1)}
  %% transfrom pd
\newcommand{\pxpd}[2]{\kwd{pdtrans}(#1,#2)}
\newcommand{\pscann}{\kwd{scan}}
\newcommand{\pscan}[1]{\pscann{}(#1)}
\newcommand{\ptry}[2]{\kwd{try}(#1,#2)}
\newcommand{\ptransform}[3]{#1 \; #2 \rightsquigarrow #3}
\newcommand{\pmu}[2]{\gm #1.#2}

%% \newcommand{\p}[]{}

  %% generic interface macro
\newcommand{\Igen}[1]{\mathcal{I}_{\text{#1}}}
  %% base type kind interface
\newcommand{\Ikind}{\Igen {kind}}
  %% base type rep type interface
\newcommand{\Irty}{\Igen {type}}
  %% base type pd type interface
\newcommand{\Ipdty}{\Igen {pdType}}
  %% Base type implementation interface
\newcommand{\Iimp}{\Igen {imp}}
  %% constant type interface
\newcommand{\Icty}{\Igen {cty}}
  %% operator type interface
\newcommand{\Iopty}{\Igen {opty}}
\newcommand{\defty}[2]{{\rm default\_ty}(#1) = #2}

% Terminator syntax
  % bw plus
\newcommand{\BWp}[1]{\text{BW+}(#1)}
  % bw minus
\newcommand{\BWm}[1]{\text{BW-}(#1)}
  % fw
\newcommand{\FW}[1]{\text{FW}(#1)}
  % terminator combination
\newcommand{\tand}{\mathrel{\&\&}}


  % Internal language name
\newcommand{\implang}{host\xspace}
 %% capitalized version
\newcommand{\Implang}{Host\xspace}

% Internal language types syntax
\newcommand{\tyface}[1]{\texttt{#1}}

\newcommand{\ibasety}{a}
\newcommand{\ibitsty}{\tyface{bits}}
\newcommand{\iunitty}{\tyface{unit}}
\newcommand{\ioffty}{\tyface{offset}}
\newcommand{\ispty}{\tyface{span}}
\newcommand{\ilocty}{\tyface{loc}}
\newcommand{\iecty}{\tyface{errcode}}
  % int type. ``ty'' appended to avoid conflict with amsmath.iint.
\newcommand{\iintty}{\tyface{int}}
\newcommand{\iboolty}{\tyface{bool}}
\newcommand{\iprodty}[2]{#1 * #2}
  % deprecated version of \iprodty
\newcommand{\iprod}[2]{\iprodty {#1}{#2}}
  %% infix version - just the operator
\newcommand{\iprodi}{*}
\newcommand{\isum}[2]{#1 + #2}
  % type for arrays
\newcommand{\iseqty}[1]{#1 \, \tyface{seq}}
  % deprecated version of \iarrty
\newcommand{\iseq}[1]{\iseqty {#1}}
\newcommand{\iarrow}[2]{#1 \rightarrow #2}
\newcommand{\iarrowi}{\rightarrow}
\newcommand{\ioparrow}[2]{#1 \rightharpoonup #2}
\newcommand{\imu}[2]{\gm #1.#2}
  % noval type
\newcommand{\invty}{\tyface{noval}}
  % error type
\newcommand{\ierrty}[1]{#1 \; \tyface{error}}
  % standard pd type
\newcommand{\ipty}[1]{\iprod {\tyface{pd\_hdr}}{#1}}
\newcommand{\pdtyvar}[1]{{#1}\, \mbox{s.t.}\, \ispdty {#1}}
\newcommand{\pdtyvars}[2]{#1,#2\; \mbox{s.t.}\; \ispdty {#1},\ispdty {#2}}
% has header ddc ty vars
%\newcommand{\hdtvs}[2]{#1,#2\; \mbox{s.t.}\; \ispdty {\itpdsem[{#1}]},\ispdty {\itpdsem[{#2}]}}
\newcommand{\hdtvs}[2]{#1,#2}
\newcommand{\iaptyname}{\tyface{arr\_pd}}
\newcommand{\iapty}[1]{\ipty {(\iaptyname \; #1)}}
\newcommand{\ityvar}{\ga}

%% Internal language term syntax
\newcommand{\iconst}{\codefont{c}}
\newcommand{\ivar}{\codefont{x}}
\newcommand{\iexp}{\codefont{e}}
\newcommand{\idata}{\codefont B}
\newcommand{\iuval}{\codefont{()}}
\newcommand{\iok}{\codefont{ok}}
\newcommand{\iecerr}{\codefont{err}}
\newcommand{\iecpc}{\codefont{pc}}
\newcommand{\iop}[1]{{op}(#1)}
\newcommand{\inotop}{\codefont{not}}
\newcommand{\isizeofop}{\codefont{sizeof}}
\newcommand{\isizeof}[1]{\isizeofop(#1)}
% \newcommand{\iinl}[2]{\kwd{inl}_{#1} \; #2}
% \newcommand{\iinr}[2]{\kwd{inr}_{#1} \; #2}
\newcommand{\iinl}[1]{\codefont{inl} \; #1}
  % deprecated version:
\newcommand{\iinld}[2]{\codefont{inl} \; #2}
\newcommand{\iinr}[1]{\codefont{inr} \; #1}
  % deprecated version:
\newcommand{\iinrd}[2]{\codefont{inr} \; #2}
\newcommand{\icase}[4]
  {\codefont{case} \; #1 \; \codefont{of} \, (\codefont{inl \, x} \Rightarrow #3
    \, | \, \codefont{inr \, x}  \Rightarrow #4) }
  % more general version
\newcommand{\icaseg}[5]
  {\codefont{case} \; #1 \; \codefont{of} \, (\codefont{inl \, #2} \Rightarrow #3
    \cvb \codefont{inr \, #4}  \Rightarrow #5) }
% \newcommand{\icase}[4]
%   {\kwd{case} \; #1 : #2 \; \kwd{of} \, (\kwd{inl} \, x \Rightarrow #3
%     \, | \, \kwd{inr} \, x  \Rightarrow #4) }
\newcommand{\ilam}[3]{\gl \codefont{#1}.#3}
%\newcommand{\ilam}[3]{\gl #1{:}#2.#3}
\newcommand{\ipair}[2]{(#1,#2)}
\newcommand{\itup}[1]{\codefont({#1}\codefont)}
\newcommand{\ictup}[1]{\codefont{(#1)}}
\newcommand{\ipi}[2]{\codefont{\gp_{#1}} \, #2}
\newcommand{\iarr}[1]{\codefont{[}#1\codefont{]}}
\newcommand{\ieseq}{\codefont{[]}}
\newcommand{\iappend}[2]{#1 \; @ \; #2}
%%\newcommand{\isub}[2]{\codefont{sub_{#1}} \; #2}
\newcommand{\isub}[2]{#1\,\codefont{[{#2}]}}
\newcommand{\ifoldr}[3]{\codefont{foldr} \; #1 \, #2 \, #3}
\newcommand{\iroll}[2]{\codefont{roll}(#1,#2)}
\newcommand{\iunroll}[1]{\codefont{unroll}(#1)}
\newcommand{\iapp}[2]{#1 \; #2}
\newcommand{\iappi}{\;}
  %% Exception
\newcommand{\ifail}{\codefont{fail}}
  %% Plain error, no value.
\newcommand{\ierr}{\codefont{noval}}
  %% Error with value.
\newcommand{\ierror}[1]{\codefont{error}(#1)}
\newcommand{\iexamine}[1]{\codefont{examine}(#1)}
\newcommand{\itrue}{\codefont{true}}
\newcommand{\ifalse}{\codefont{false}}
\newcommand{\ilet}[2]{\codefont{let \; #1} \ceq #2 \; \codefont{in}}
%\newcommand{\icasess}[1]{\codefont{case} \; #1 \; \codefont{of}}
%\newcommand{\ipattss}[2]{| \; \codefont{#1} \Rightarrow \codefont{#2}}
\newcommand{\iif}[1]{\codefont{if} \; {#1}}
\newcommand{\ithen}[1]{\codefont{then} \; #1}
\newcommand{\ielse}[1]{\codefont{else} \; #1}
\newcommand{\iin}{\codefont{in}}
% \newcommand{\iletinend}[2]{
%   \codefont{let} \\ \quad
%   \begin{array}{l}
%     #1
%   \end{array} \\
%   \codefont{in} \\ \quad
%   \begin{array}{l}
%     #2
%   \end{array} \\
%   \codefont{end}
% }
% \newcommand{\ival}[2]{\codefont{val} \; #1 = #2}
\newcommand{\ifun}[3]{\codefont{fun} \; \codefont{#1 \; #2} \ceq {#3}}
\newcommand{\iletfun}[3]{\codefont{letfun} \; \codefont{#1 \; #2} \ceq #3}
\newcommand{\ifunand}[3]{\codefont{and} \; \codefont{#1 \, #2} \ceq #3}
\newcommand{\ior}[2]{#1 \; \codefont{or} \; #2}
\newcommand{\iori}{\; \codefont{or} \;}
\newcommand{\iandi}{\; \codefont{and} \;}

%% Semantic Domain syntax
  %% semantic definition
%\newcommand{sbox
\newcommand{\sdefm}[1]{\gather* #1\endgather}
\newcommand{\sdef}[1]{\begin{array}{l}#1\end{array}}
\newcommand{\sfn}[3]{\ilam {\codefont{#1}}{#2}{#3}}
  %% recursive function
\newcommand{\sfun}[4]{\text{fun} \, #1(#2{:}#3).#4}
  %% prefix application macro
\newcommand{\sappp}[2]{\iapp {#1}{#2}}
  %% infix application (just a space).
\newcommand{\sapp}{\iappi}
\def\spair<#1>{\itup {#1}} 
\newcommand{\spi}[2]{\ipi {#1} {#2}}
\newcommand{\strue}{\codefont{true}}
  %% Equivalence in sem. dom.
\newcommand{\semeq}{\mathrel{==}}
  %% Semantics Domains
\newcommand{\extdom}{\tyface{bits}}
\newcommand{\intdom}{\tyface{IV}}
\newcommand{\locdom}{\tyface{Loc}}
\newcommand{\offdom}{\ioffty}
\newcommand{\consdom}{\tyface{Consume}}
\newcommand{\modedom}{\tyface{Mode}}

% Built-ins
  %% rep and pd constructors themselves (i.e. not applied).
\newcommand{\newrepf}[1]{\codefont{R_{#1}}}
\newcommand{\newpdf}[1]{\codefont{P_{#1}}}
  %% generic rep constructor
\newcommand{\newrep}[2]{\newrepf{#1}(\codefont{#2})}
  %% generic pd constructor
\newcommand{\newpd}[2]{\newpdf{#1}(\codefont{#2})}
  %% EoF predicate
\newcommand{\eofpred}[1]{\codefont{EoF(#1)}}
  %% isOk function
\newcommand{\pdok}[1]{\codefont{isOk(#1)}}
  %% isErr function
\newcommand{\pderr}[1]{\codefont{isErr(#1)}}
\newcommand{\isdone}[1]{\codefont{isDone} \sapp \ictup{#1}}
\newcommand{\scanmax}{\codefont{SCAN\_MAX}}
\newcommand{\seterr}[1]{\kwd{SetErr} \, #1}
\newcommand{\setmode}[2]{\kwd{SetMode} \, #1 \, #2}
\newcommand{\incnr}[1]{\kwd{IncNR} \, #1}

%% Abbreviations
\newcommand{\lampair}[1]{\gl \spair<\codefont{\data},\off>.#1}
%\newcommand{\funpair}[2]{\text{fun} \, #1\spair<\data,\off>.#2}

%% Misc. operations
  %% well typed judgment
\def\wellty(#1,#2,#3){#1 \turn #2 : #3}
  %% extended context
\newcommand{\ectxt}[1]{\ctxt,{#1}}
  %% extended rec. context
\newcommand{\erctxt}[2]{\rctxt,{#1}{=}\pmu {#1} {#2}}
  %% truth judgment
\newcommand{\tjudge}[2]{#1 \vDash #2}
  %% negated truth judgment
\newcommand{\tnjudge}[2]{#1 \nvDash #2}
  %% ok parse descr. judgment
\newcommand{\okjudge}[1]{\vDash #1 \; \kwd{ok}}
  %% err parse descr. judgment
\newcommand{\errjudge}[1]{\vDash #1 \; \kwd{err}}
  %% instruction-level dynamic semantics ``steps to''.
\newcommand{\stepstoi}{\hookrightarrow_i}
  %% top-level dynamic semantics ``steps to''.
\newcommand{\stepsto}{\hookrightarrow}
  %% Klean-closure of top-level ``steps to''.
\newcommand{\kstepsto}{\stepsto^*}
  %% indicates a problem to be fixed.
\newcommand{\fixme}{\mathbf{???}}
  %% error correllation ind. hypothesis
\newcommand{\ecpred}[2]{\mathrm{EC}(#1:#2)}
  %% pd error count
\newcommand{\pecnt}[1]{\mathrm{PEC}( \, #1)}
  %% Type preservation predicate.
\newcommand{\tppred}[2]{\mathrm{TP}_{#1}(#2)}
  %% Deep error correlation relation
\newcommand{\dcorr}[2]{\mathrm{DeepCorr}(#1,#2)}
  %% Another error correlation relation (generic)
\newcommand{\corrg}[2]{\mathrm{Corr}(#1,#2)}
  %% Another error correlation relation
\newcommand{\corr}[3]{\mathrm{Corr_{#1}}(#2,#3)}
  %% Clean closure version of correlation relation
\newcommand{\corrkl}[3]{\mathrm{Corr^*_{#1}}(#2,#3)}
  %% Error Correlation predicate.
\newcommand{\cepred}[2]{\mathrm{CE}_{#1}(#2)}
  %% Function Error Correlation predicate.
\newcommand{\cefpred}[2]{\mathrm{CE}_{#1}(#2)}
  %% Cannonical Formas pred.
\newcommand{\canfm}[3]{\mathrm{CF_{#1}}(#2,#3)}
  %% Is-pd predicate:
\newcommand{\ispdty}[1]{\mathrm{hh}(#1)}
  %% No Errors predicate
\newcommand{\noerr}[1]{\mathsf{Clean}(#1)}

  %% BW termination judgment
\newcommand{\btjudge}[3]{#1 \vDash #2 \Rightarrow #3}
  %% termination judgment
\newcommand{\ttjudge}[4]{#1 \vDash #2 (#3) \Rightarrow #4}
  %% negated termination judgment
\newcommand{\ttnjudge}[3]{#1 \nvDash #2 (#3)}

  %% inverse infer
\newcommand{\iinfer}[2]{\infer{#2}{#1}}

%% The following macros are taken from the pads manual, 
%% file defs.tex.

%% dave added a couple here.

  %% keywords and PADSL types: added micro space on either side
\newcommand{\bftt}[1]{{\ttfamily\bfseries{}#1}}
\newcommand{\padskw}[1]{\text{\/\/\bftt{#1}\/\/}}
\newcommand{\cd}[1]{\texttt{#1}}

\newcommand{\Pbase}[1]{\padskw{C}(#1)} 
\newcommand{\Pomit}{\padskw{Pomit}}
\newcommand{\Pcompute}{\padskw{Pcompute}}
\newcommand{\Pendian}{\padskw{Pendian}}
\newcommand{\Pstruct}{\padskw{Pstruct}}
\newcommand{\Punion}{\padskw{Punion}}
\newcommand{\Popt}{\padskw{Popt}}
\newcommand{\Pchar}{\padskw{Pchar}}
\newcommand{\Pdate}{\padskw{Pdate}}
\newcommand{\Puint}{\padskw{Puint32}}
\newcommand{\Pip}{\padskw{Pip}}
\newcommand{\Pstring}{\padskw{Pstring}}
\newcommand{\Prec}{\padskw{Prec}}
\newcommand{\Pfun}{\padskw{Pfun}}
\newcommand{\Parray}{\padskw{Parray}}
\newcommand{\Palternate}{\padskw{Palternate}}
\newcommand{\Ptypedef}{\padskw{Ptypedef}}
\newcommand{\Penum}{\padskw{Penum}}
\newcommand{\Pwhere}{\padskw{Pwhere}}
\newcommand{\Palt}{\padskw{Palt}}
\newcommand{\Pparsecheck}{\padskw{Pparsecheck}}
\newcommand{\Pterm}{\padskw{Pterm}}
\newcommand{\Psep}{\padskw{Psep}}
\newcommand{\Pre}{\padskw{Pre}}
\newcommand{\Pnosep}{\padskw{Pnosep}}
\newcommand{\Plongest}{\padskw{Plongest}}
\newcommand{\Plast}{\padskw{Plast}}
\newcommand{\Pended}{\padskw{Pended}}
\newcommand{\Peor}{\padskw{Peor}}
\newcommand{\Peof}{\padskw{Peof}}
\newcommand{\Pforall}{\padskw{Pforall}}
\newcommand{\Pfrom}{\padskw{Pfrom}}
\newcommand{\Pin}{\padskw{Pin}}
\newcommand{\Precord}{\padskw{Precord}}
\newcommand{\Psource}{\padskw{Psource}}
\newcommand{\Pcase}{\padskw{Pcase}}
\newcommand{\Pswitch}{\padskw{Pswitch}}
\newcommand{\Pdefault}{\padskw{Pdefault}}
\newcommand{\Psome}{\padskw{Psome}}
\newcommand{\Pnone}{\padskw{Pnone}}
\newcommand{\Pcharclass}{\padskw{Pcharclass}}
\newcommand{\Pprefix}{\padskw{Pprefix}}
\newcommand{\Plit}[1]{\padskw{Plit} \; #1}

  %% IPADS Parray 
\newcommand{\iParray}[4]{#1 \; \Parray{}(#2,#3)}

  %% Conversion from surface language to core calculus
\newcommand{\conv}[2]{#1 \Longrightarrow #2}


%%%%%%%%%%%%%%%%%%%%%%%%%%%%%%%%%%%%%%%%%%%%%%%%%%%%%%
%%                 Environments                     %%
%%%%%%%%%%%%%%%%%%%%%%%%%%%%%%%%%%%%%%%%%%%%%%%%%%%%%%

% empty environment used for scoping of declarations.
\newenvironment{scope}{}{}

%% Environment for typesetting BNF grammars. Uses display math mode.
\newenvironment{bnf}
     {%% local command definitions:
        %% BNF definition symbol
      \def\->{\rightarrow}
%%      \def\::={{::=} &}
%      \def\::={\bnfdef &}
      \def\::={\mathrel{::=} &}
      \def\|{\bnfalt}
      \newcommand{\name}[1]{\text{##1}}
      % name spanning multiple rows
%      \newcommand{\mname}[2]{\multirow{##2}{*}{\begin{tabular}{l}##1\end{tabular}}}
      \newcommand{\mname}[2]{\multirow{##2}{.6in}{##1}}
        %% non-terminal
      \newcommand{\nont}[1]{\mathit{##1}}
      \newcommand{\meta}[1]{& \mathit{##1} &}
      \newcommand{\descr}[1]{& \text{// ##1}}
      \newcommand{\opt}[1]{ [##1] }
      \newcommand{\opnon}[1]{\opt{\nont{##1}}}
      \newcommand{\none}{\epsilon}
      \newcommand{\nwln}{\\ &&&}
      \newcommand{\nlalt}{\\ && \| &}
      \[\begin{array}{lrcll}
     }
     {\end{array}\]}

%% Environment for typesetting BNF grammars. Uses standard text mode.
\newenvironment{bnft}
     {%% local command definitions:
        %% BNF definition symbol
      \def\->{$\rightarrow$}
      \def\::={{::=} &}
      \def\|{\ $\mid$ \ }
      \newcommand{\name}[1]{\textit{##1} &}
        %% non-terminal
      \newcommand{\nont}[1]{\textit{##1}}
      \newcommand{\meta}[1]{}
      \newcommand{\descr}[1]{& // ##1}
      \newcommand{\opt}[1]{ [##1] }
      \newcommand{\opnon}[1]{\opt{\nont{##1}}}
      \newcommand{\none}{$\epsilon$}
      \newcommand{\nwln}{\\ &&}
      \newcommand{\nlalt}{\\ & \| &}
%      \newcommand{\meta}[1]{##1 &}
%      \newcommand{\nwln}{\\ &&&}
%      \newcommand{\nlalt}{\\ && \| &}
%      \begin{tabular}{lrcll}
      \begin{tabular}{lcll}
     }
     {\end{tabular}}

\newenvironment{semdef}
    {%% local command definitions:
       %% arrow kind
     \def\->{\rightarrow}
     \allowdisplaybreaks
     \gather}
    {\endgather}

%% \newenvironment{semdef}
%%     {%% local command definitions:
%%        %% arrow kind
%%      \def\->{\rightarrow}
%%      \[\begin{array}{l}\allowdisplaybreaks
%%     }
%%     {\end{array}\]}

% future tag macro. does nothing now.
\newcommand{\stag}[1]{}
\newenvironment{semcasedef}[1]{
        %% label for an alternative.
      \newcommand{\altlbl}[1]{\qquad \text{(Case ##1:)}}
        %% value of an alternative.
      \newcommand{\altval}[1]{\quad ##1}
        %% semantic let: let a variable/pattern equal a value
      \newcommand{\slet}[2]{\texttt{let} \; ##1 = ##2 \; \texttt{in}}
     \tag{#1}
     \begin{array}{l}}
    {\end{array}}

%% \newenvironment{semrule}
%%     {\begin{array}{l}}
%%     {\end{array}}

\newenvironment{semcond}
    {% Local command definitions:
       %% ``if'' condition
     \newcommand{\cif}{\text{if} &}
       %% ``and'' condition
     \newcommand{\cand}{\text{and} &}
       %% let a variable/pattern equal a value
     \newcommand{\vlet}[2]{##2 = ##1}
     %\qquad
     \begin{array}{ll}}
    {\end{array}}

\newcommand{\phide}[1]{}
\newcommand{\preplace}[2]{#2}
\newcommand{\pext}[1]{}

\section{Using XQuery and \Galax{}}
\label{section:galax}

XML~\cite{xml10} is a flexible format that can represent many classes of
data: structured documents with large fragments of marked-up text;
homogeneous records such as those in relational databases; and
heterogeneous records with varied structure and content such as those
in ad hoc data sources.  XML makes it possible for applications to
handle all these classes of data simultaneously and to exchange such
data in a simple, extensible, and standard format.  This flexibility
has made XML the ``lingua franca'' of data
interoperability. \cut{making it possible for data to be exchanged
regardless of where it is stored or how it is processed.}

XQuery~\cite{xquery10} is a typed, functional query language for XML
that supports user-defined functions and modules for structuring large
queries.  Its type system is based on XML Schema.  XQuery contains
XPath 2.0~\cite{xpath} as a proper sub-language, which supports
navigation, selection, and extraction of fragments of XML documents.
XQuery also includes expressions to construct new XML values and to
integrate or join values from multiple documents.  Unusual among
industry standards, XQuery also has a formal semantics, which makes it
particularly interesting to database researchers. 

As XQuery was designed for querying semi-structured XML data, it is a
natural choice for querying semi-structured ad hoc data.  It naturally
handles irregularly structured data.  For example, the expressions in
Figure~\ref{figure:dibbler-query} are well-defined for order records
containing any number of events and for event records containing any
number of timestamps.  As noted in Section~\ref{subsec:example},
XQuery's static type system can detect common errors at compile time.
Such type safety is particularly valuable for long-running queries on
large ad hoc sources and for data sources whose schemata evolve.
XQuery is also ideal for specifying integrated views of multiple
sources.  Although here we focus on querying one ad hoc source at a
time, XQuery supports our goal of simultaneously querying multiple
\pads{} and other XML sources. Lastly, XQuery is practical: It will
become a standard; numerous language manuals already exist; and it is
widely implemented in commercial databases.

\Galax{}~\footnote{\texttt{http://www.galaxquery.org}} is a complete,
extensible, and efficient implementation of XQuery~1.0 and was
designed with database systems research in mind.  It supports all of
XML 1.0, most of XML Schema 1.0, which is the foundation of the XQuery
type system, and all of XQuery 1.0.  Its architecture is modular and
documented~\cite{galax:edbt2004}, which makes it possible for other
researchers to experiment with a complete XQuery implementation.  Its
abstract data model permits experimenting with various physical
representations of XML and non-XML data sources.  Its query compiler
produces evaluation plans in the first complete algebra for
XQuery~\cite{galax:icde2006}, which permits experimental comparison of
query-compilation techniques. 
Lastly, \Galax{}'s optimizer detects
joins and grouping constructs in algebraic plans and produces
efficient physical plans that employ traditional and novel join
algorithms~\cite{galax:icde2006}.

\subsection{Galax's Abstract Data Model}

The XQuery Data Model~\cite{XPath:DataModel} contains tree nodes,
atomic values and sequences of nodes and atomic values.  A tree node
corresponds to an entire XML document or to an individual element,
attribute, comment, or processing-instruction.  \Galax{}'s abstract
data model is an object-oriented realization of the XQuery Data Model.
Algebraic operators in a query-evaluation plan access documents via
this object-oriented interface.

Figure~\ref{figure:galax-dm} contains part of Galax's data model
interface~\footnote{Galax is implemented in O'Caml, so these
signatures are in O'Caml syntax.} for a node in the XQuery Data Model.
Node accessors return information such as a node's name
(\kw{node\_name}), the XML Schema type against which the node was
validated (\kw{type}), and the node's atomic-valued data if it was
validated against a XML Schema simple type (\kw{typed\_value}).  The
\kw{parent}, \kw{child}, and \kw{attribute} methods navigate the
document and return a node sequence containing the respective parent,
child, or attribute nodes of the given node.

\begin{figure*}
\begin{small}
\begin{code}
type sequence = cursor
class virtual \kw{node} : 
object
  (* Selected XQuery Data Model accessors *)
  method virtual \kw{node_name}   : unit -> atomicQName option
  method virtual \kw{type}        : unit -> (schema * atomicQName)
  method virtual \kw{typed_value} : unit -> atomicValue sequence

  (* Required axes *)
  method virtual \kw{parent}      : node_test option -> node option
  method virtual \kw{child}       : node_test option -> node sequence
  method virtual \kw{attribute}   : node_test option -> node sequence

  (* Galax's extensions *)
  method \kw{descendant_or_self}  : node_test option -> node sequence
  method \kw{descendant}          : node_test option -> node sequence
  method \kw{ancestor_or_self}    : node_test option -> node sequence
  method \kw{ancestor}            : node_test option -> node sequence

  ... Other accessors in XQuery Data Model ...
\end{code}
\end{small}
\caption{Signatures for methods in Galax's abstract node interface}
\label{figure :galax-dm}
\end{figure*}

The first six methods in Figure~\ref{figure:galax-dm} are virtual,
because they access the physical representation of a document, and
therefore a particular data model must provide their concrete
implementations.  \Galax{} provides default implementations for the
four {descendant} and ancestor axes, which are defined recursively in
terms of the {child} and {parent} methods.  These defaults are
overridden by implementations that can provide more efficient
implementations.  For example, some physical document representations
permit axes to be implemented by range queries over relational
tables~\cite{grust03staircase}.

All the axis methods take an optional node-test argument, which is a
boolean predicate on the names or types of nodes in the given axis.
For example, the \textit{Axis}:\textit{NodeTest} expression
\kw{descendant::order} returns nodes in the descendant axis with name
\kw{order}.  \Galax{} implements \textit{Axis}:\textit{NodeTest}
expressions by invoking the corresponding methods in the data model
and delegating evaluation of node tests to the underlying
implementation.  Some data-model implementations, like \padx{}, can
provide fast access to nodes by their name.

In addition to the \padx{} data-model implementation, \Galax{} has
three other data-model implementations: a DOM-like representation in
main memory and two ``shredded'' representations, one in main memory
and one in secondary storage for very large documents (e.g., $>$
100MB).  The shredded data model partitions a document into tables of
elements, attributes, and values that can be indexed on node names and
values~\cite{galax:ximep2004}.

\cut{Tree data model.  Data model accessors (axis::node-tests) that can/should be implemented
efficiently by the underlying source are pushed into the OO tree data
model.  Default implementations for sources that don't do anything
clever. }

\cut{The second sub-group contains operators for tree navigation and
projection (in the style of~\cite{projection}). The TreeJoinOp is a
set-at-a-time operator for navigation, which takes a set of nodes in
document order and returns a set of nodes in document order after
applying the given step. It can be implemented either as a combination
of iteration with navigational access on the data model or as an
advanced kind of XPath join, such as that proposed}

\cut{
.  which makes it possible for
\Galax{} to evaluate queries simultaneously over native and virtual
XML sources that implement the data model.   

Motivation for abstract data model: Supporting PADS and secondary
storage system occurred at same time.  Data and information
integration.  }



\section{Using \padx{} to Query Ad Hoc Data}
\label{section:padx}

\begin{figure}
\begin{center}
\epsfig{file=padx-arch.ps,width=0.47\textwidth}
\end{center}
\caption{\padx{} Architecture}
\label{figure:padx-arch}
\end{figure}

Put the pieces all together.  Sythesis of the two systems here. 
(Symbiotic)

\subsection{Virtual XML view of PADS data}

Embedding of PADS types in XML Schema.  One-to-one mapping from
PADS compound types to XML Schema complex types.  One-to-one mapping
from PADS base types to XML Schema simple types.  Field in compound
types are realized as local elements in XML Schema. 

All the compound types are annotated with an optional parse-descriptor
(absent if no errors occured).  Allows users to query error
structures, which may be most important data.  Other types annotated
with corresponding fields from PADS rep, e.g., arrays have a length. 

Extra level of indirection in representation of arrays---wrap each
item in an element. 

Extra level of indirection for base types: must contain the value of
the base type and an optional parse-descriptor, if an error has
occurred. 

We don't take complete advantage of XML Schema, e.g., Penum types
could be modeled by XML Schema enumeration simple types, but currently
unsupported.

Generated XML Schema.

\begin{figure*}
\begin{small}
\begin{code}
<xs:schema targetNamespace="\kw{file:/example/sirius.p}"
           xmlns="file:/example/sirius.p"
           xmlns:xs="http://www.w3.org/2001/XMLSchema"
           xmlns:p="http://www.padsproj.org/pads.xsd">
<xs:import namespace = "http://www.padsproj.org/pads.xsd".../>
...
<xs:complexType name="\kw{order_header_t}">
 <xs:sequence>
  <xs:element name="\kw{order_num}"     type="\kw{p:val_Puint32}"/>
  <xs:element name="\kw{att_order_num}" type="\kw{p:val_Puint32}"/>
  <xs:element name="\kw{ord_version}"   type="\kw{p:val_Puint32}"/>
  <!-- More local element declarations -->
  <xs:element name="\kw{pd}"            type="\kw{p:PStruct_pd}" minOccurs="0"/>
 </xs:sequence>
</xs:complexType>
<!-- More complex type declarations -->
<xs:complexType name="\kw{orders_t}">
 <xs:sequence>
  <xs:element name="\kw{elt}"    type="\kw{order_t}" maxOccurs="unbounded"/>
  <xs:element name="\kw{length}" type="\kw{p:Puint32}"/>
  <xs:element name="\kw{pd}"     type="\kw{p:Parray_pd}" minOccurs="0"/>
 </xs:sequence>
</xs:complexType>
...
<xs:element name="PSource" type="summary"/>
</xs:schema>
\end{code}
\end{small}
\caption{Fragment of XML Schema for \dibbler{} \pads{} description.}
\label{figure:dibbler-schema}
\end{figure*}

``Error-aware'' mapping from PADS type system to isomorphic XML
Schema. 
\begin{small}
\begin{code}
<xs:complexType name="\kw{val_Puint32}">
  <xs:\kw{choice}>
   <xs:element name="\kw{val}" type="\kw{p:Puint32}"/>
   <xs:element name="\kw{pd}"  type="\kw{p:Pbase_pd}"/>
  </xs:\kw{choice}>
</xs:complexType>
<xs:complexType name="\kw{Pbase_pd}">
 <xs:sequence>
   <xs:element name="\kw{pstate}"  type="\kw{p:Pflags_t}"/>
   <xs:element name="\kw{errCode}" type="\kw{p:PerrCode_t}"/>
   <xs:element name="\kw{loc}"     type="\kw{p:Ploc_t}"/>
 </xs:sequence>
</xs:complexType>
\end{code}
\end{small}

Example of query that uses generated schema.
\begin{figure*}
\begin{small}
\begin{code}
import schema default element namespace "\kw{file:/example/sirius.p}";
declare variable \kw{$pads} as \kw{document-node(PSource)} external; 

$pads/Psource/orders/elt[events/elt[1]
  [tstamp >= xs:dateTime("2004-10-01:00:00:00") and tstamp < xs:date("2004-11-01:00:00:00") ]]
\end{code}
\end{small}
\caption{\padx{} query with schema import}
\label{figure:padx-query}
\end{figure*}

How to access data files:  fn:doc("pads:/example/sirius.data");

\subsection{Physical Data Model}

Part of PADX implementation are generic functions that implement data
model accessors.  PADS compiler generates type-specific functions for
walking virtual XML tree.  Relate back to type-specific library
functions mentioned in last section.

Implementation of Galax's Abstract Tree Model.

Minimum necessary to implement Galax DM:

1. Generic implementations of the DM accessors: axis::node-test(), children(),
   attributes(), name(), etc. 

2. On PADX-side, we have a virtual handle for each node in the XML
   tree--we call that a node rep.  Node rep contains pads handle
   (maintains state for PADS parser); type-specific vtable of DM
   accessors; other stuff...

   Give example of vtable for event\_t and possibly code for
   kthChildByName. 

When to actually read from PADS data?

Options: 

1. Bulk read: Materialize entire PADS representation, populate all of the PADS
reps.  Then PADX DM lazily invoked the DM accessors over this data.
Problem: if data is big, it's all sitting in memory, even if the query
only touches a fragment of the virtual XML tree.

2. Smart read: 

Many common queries permit sequential, streamed access to underlying
XML source.  Give an example.  

Smart node rep, preserves meta-data about previously read records, but
re-uses memory for reading next item.  This rep permits multiple scans
of input (semantic problem is that DM must preserve node identity),
but slowly. 

Heuristic: records are a good level of granularity to read.   Each
smart node corresponds to one record.  When next smart node is
accessed, a little meta-data is preserved: the node rep and the
records location in the file (so we can re-read it if necessary).

3. Linear read: same as smart but does not preserve meta-data.
   Does not permit multiple scans of data source. 

Put in PADX signatures for constructing a new node and accessing
kthChild. 

Something about query evaluation:
Although we have not explored custom evaluation plans 
Galax's algebra or optimizer are particularly interesting in the 
\padx{}, we expect to do so 

\section{Performance}
\label{section:performance}

\subsection{Materialization and Loading}

Connect scalability to the "multiple entry point parsing routines" in
Sec 2.
\begin{itemize}
\item galax has a "virtual random-access view" of the XML
  source. Interface to PADX does not reveal details of how and when
  data is loaded. The interface puts no restrictions on access.
\item Indeed, \padx{} can supply a real random-access view of the data but at a
  high cost. Can prefetch entire data source into memory, but then
  depend on VM system to manage any overflows. Call this prefetch {\em
    bulk load}.
\item Problems crop up early due to quantity of data generated (on
  average) per byte.
\item Therefore, bulk reading will only work for the smallest of
  data sets.
\item This drawback is not a flaw in the system, but an intentionally
  chosen point in the design space. The parsing libraries were never
  intended to be used that way. Instead, \pads model is for programmer
  to use multiple entry points to control the parsing to produce
  manageable quantities of data. \pads generates a lot of meta data
  and lets programmer filter as desired.
\item Unfortunately, requires much programmer involvement. We want to
  capitilize on this behaviour, but automatically, without requiring
  any more involvement by Galax than what is already provided in
  interface. so we use the libraries as intended by reading
  sequentially and maintaining just enough state to satisfy
  Galax. That is, we take advantage of multiple entry points provided,
  but in an automatic way.
% \item In fact, the automatic nature of the \padx memory management,
%   offers a value-added tool to even the ``pure'' \pads
%   programmer. However, the memory management layer is not free, as we
%   now examine in more detail.
\end{itemize}

Description of three loading schemes:

% The flexibility of the node interface goes beyond support for
% arbitrary \pads{} types. It also allows us, for each type, to support
% alternative implementations of the interface. We take advantage of
% this flexibility to support multiple data input strategies: bulk-read
% and two forms of on-demand read.

\begin{enumerate}
\item Bulk read: Materialize entire PADS representation, populate all
  of the PADS reps.  Then PADX DM lazily invokes the DM accessors over
  this data.

\item On-demand sequential read: Keep storage for one record. Read
  records on demand into single memory slot. Does not permit multiple
  scans of data source.  Many common queries permit sequential,
  streamed access to underlying XML source.  Give an example.

\item On-demand, random access read: Store fixed number of records in
  memory at any one time. System preserves
  meta-data about previously read records, but re-uses memory for
  reading next item.  This rep permits multiple scans of input
  (semantic problem is that DM must preserve node identity), but
  slowly. Does not work for read once data that can't be ``rewound''
  such as live streams.
\end{enumerate}

Lasagna performance issues (i.e. big picture, not just the one layer):

\begin{itemize}
\item Each level
in the data model adds a small constant cost to touching
the data.
\item The linear reading is indeed linear.
\end{itemize}

\subsection{Querying}

Something about query evaluation:
Although we have not explored custom evaluation plans 
Galax's algebra or optimizer are particularly interesting in the 
\padx{}, we expect to do so 

Give examples of queries that analyst cares about. 

Example of query that can be evaluated in single scan over data
source, but is currently not 

Database person would balk at this point!  Why aren't you just loading
this data into a real database, building indices and getting good
query performance?  B/c data is ephmeral, queries are ephmeral, but
analyst/programmer should profit from disciplined access/querying of
their data.  Don't abandon them to Perl. 



\section{Related Work and Discussion}
\label{section:relatedwork}
\label{section:future}

DFDL~\cite{dfdl}. Contivo.  

Compiled-query architectures~\cite{daytona}.  \padx{} has customized
data model: between points on continuum from fully interpreted query
plan applied to generic data model to fully compiled query plan
applied to customized data model.

Streamable query plans. 
BEA~\cite{DBLP:journals/vldb/FlorescuHKLRWCS04}
Phantom XML~\cite{rose:villard:2005}
Big open problem: Given arbitrary XQuery expression, determine whether
it can be evaluated in single scan over data.  Point out that this
problem is of interest for a different reason: fully pipeline
evaluation of queries over XML token streams.

TODO LIST: Better use of XML Schema simple types.
We don't take complete advantage of XML Schema, e.g., Penum types
could be modeled by XML Schema enumeration simple types, but currently
unsupported.

Open problem: give result of XQuery and its corresponding type,
serialize result back into PADS rep.  How are the syntactic
constraints for the new values expressed?  Tool could pick default
delimiters automatically. 

Open problem: given an arbitrary PADS type, permit skipping and
reading at arbitrary positions within the data source. 

\bibliographystyle{abbrv}
\small
\bibliography{../pads,../galax} 
\end{document}


\chapter{Tutorial}
\label{chap:example}
In this chapter, we use examples to give an overview of \pads{}.

\section{CLF: Common log format}
\label{sec:example:common-log-format}
Web servers use the Common Log Format (CLF) to log client
requests~\cite{wpp}.  Researchers use such logs to measure
properties of web workloads and to evaluate protocol changes
by "replaying" the user activity recorded in the log.
This ASCII format consists of a sequence of
records, each of which has seven fields: the host name or IP address
of the client making the request, the account associated with the
request on the client side, the name the user provided for
authentication, the time of the request, the actual request, the
\textsc{http} response code, and the number of bytes returned as a
result of the request.  The actual request has three parts: the
request method (\eg, \texttt{GET}, \texttt{PUT}), the requested
\textsc{uri}, and the protocol version.  In addition, the second and
third fields are often recorded only as a '-' character to indicate
the server did not record the actual data.  \figref{figure:clf-records}
shows a couple of typical records.

\section{Provisioning data}

In the telecommunications industry, the term \textit{provisioning}
refers to the steps necessary to convert an order for phone service
into the actual service.  To track AT\&T's provisioning process, the
\dibbler{} project compiles weekly summaries of the state of certain
types of phone service orders.  These ASCII summaries store the
summary date and one record per order.  Each order record contains a
header followed by a sequence of events.  The header has 13 pipe
separated fields: the order number, AT\&T's internal order number, the
order version, four different telephone numbers associated with the
order, the zip code of the order, a billing identifier, the order
type, a measure of the complexity of the order, an unused field, and
the source of the order data.  Many of these fields are optional, in
which case nothing appears between the pipe characters.  The billing
identifier may not be available at the time of processing, in which
case the system generates a unique identifier, and prefixes this value
with the string ``no\_ii'' to indicate the number was generated. The
event sequence represents the various states a service order goes
through; it is represented as a new-line terminated, pipe separated
list of state, timestamp pairs.  There are over 400 distinct states
that an order may go through during provisioning.  The sequence is
sorted in order of increasing
timestamps. \figref{figure:dibbler-records} shows a small example of
this format.


\begin{figure*}[t!]
\begin{tiny}
\begin{code}
\chapter{Common log format example}
\section{pads description}
\section{c header file}
\end{code}
\vskip -2ex
\caption{Tiny example of web server log data.}
\label{figure:clf-records}
\end{tiny}
\end{figure*}

\begin{figure*}
\begin{small}
\begin{center}
\begin{verbatim}
0|1005022800
9152|9152|1|9735551212|0||9085551212|07988|no_ii152272|EDTF_6|0|APRL1|DUO|10|1000295291
9153|9153|1|0|0|0|0||152268|LOC_6|0|FRDW1|DUO|LOC_CRTE|1001476800|LOC_OS_10|1001649601
\end{verbatim}
\caption{Tiny example of \dibbler{} provisioning data.}
\label{figure:dibbler-records}
\end{center}
\end{small}
\end{figure*}

\begin{figure}
\begin{tiny}
\begin{code}
\kw{Punion} client\_t \{
  Pip       ip;      /- 135.207.23.32
  Phostname host;    /- www.research.att.com
\};
\mbox{}
\kw{Punion} auth\_id\_t \{
  Pchar unauthorized : unauthorized == '-';
  Pstring(:' ':) id;
\};
\mbox{}
\kw{Pstruct} version\_t \{
  "HTTP/";
  Puint8 major; '.';
  Puint8 minor;
\};
\mbox{}
\kw{Penum} method\_t \{
    GET,    PUT,  POST,  HEAD,
    DELETE, LINK, UNLINK
\};
\mbox{}
bool chkVersion(version\_t v, method\_t m) \{
  \kw{if} ((v.major == 1) && (v.minor == 1)) \kw{return} true;
  \kw{if} ((m == LINK) || (m == UNLINK)) \kw{return} false;
  \kw{return} true;
\};
\mbox{}
\kw{Pstruct} request\_t \{
  '\\"';   method\_t       meth;
  ' ';    Pstring(:' ':) req\_uri;
  ' ';    version\_t      version :
                  chkVersion(version, meth);
  '\\"';
\};
\mbox{}
\kw{Ptypedef} Puint16\_FW(:3:) response\_t :
         response\_t x => \{ 100 <= x && x < 600\};
\mbox{}
\kw{Precord} \kw{Pstruct} entry\_t \{
         client\_t       client;
   ' ';  auth\_id\_t      remoteID;
   ' ';  auth\_id\_t      auth;
   " ["; Pdate(:']':)   date;
   "] "; request\_t      request;
   ' ';  response\_t     response;
   ' ';  Puint32        length;
\};
\mbox{}
\kw{Psource} \kw{Parray} clt\_t \{
  entry\_t [];
\}
\end{code}

\caption{\pads{} description for web server log data.}
\label{figure:wsl}
\end{tiny}
\end{figure}

\begin{figure}
\begin{footnotesize}
\input{../pldi/dibbler_new}
\caption{\pads{} description for \dibbler{} provisioning data.}
\label{figure:dibbler}
\end{footnotesize}
\end{figure}



\section{\padsl{} description}
\label{sec:example:padsl-description}
\figref{figure:wsl} gives the \pads{} description for CLF web server
logs,  while \figref{figure:dibbler} gives the description for the
\dibbler{} provisioning data.  We will use these example to illustrate
various features of the \pads{} language.  In \pads{} descriptions,
types are declared before they are used, so the type that describes
the totality of the data source appears at the bottom of the
description.

\kw{Pstruct}s describe fixed sequences of data with unrelated types.
In the CLF description, the type declaration for \cd{version_t}
illustrates a simple \kw{Pstruct}. It starts with a string literal
that matches the constant \cd{HTTP/} in the data source.  It then has
two unsigned integers recording the major and minor version numbers
separated by the literal character \kw{'.'}.  \pads{} supports
character, string, and regular expression literals, which are
interpreted with the ambient character encoding. The type
\cd{request_t} similarly describes the request portion of a CLF
record.  In addition to physical format information, this \kw{Pstruct}
includes a semantic constraint on the \cd{version} field.
Specifically, it requires that obsolete methods \cd{LINK} and
\cd{UNLINK} occur only under HTTP/1.1.  This constraint illustrates
the use of predicate functions and the fact that earlier fields are in
scope during the processing of later fields, as the constraint refers
to both the \cd{meth} and \cd{version} fields in the \kw{Pstruct}.
\chapref{chap:structs} describes \kw{Pstruct}s in detail.

\kw{Punion}s describe variation in the data source.  For example, the
\cd{client_t} type in the CLF description indicates that the first
field in a CLF record can be either an IP address or a hostname.
During parsing, the branches of a \kw{Punion} are tried in order; the
first branch that parses without error is taken.  The \cd{auth_id_t}
type illustrates the use of a constraint: the branch \cd{unauthorized}
is chosen only if the parsed character is a dash.  \pads{} also
supports a \textit{switched} union that uses a selection expression to
determine the branch to parse.  Typically, this expression depends
upon already-parsed portions of the data source.

\pads{} provides \kw{Parray}s to describe varying-length sequences of
data all with the same type.  The \cd{eventSeq_t} declaration in the
\dibbler{} data description uses a \kw{Parray} to characterize the
sequence of events an order goes through during processing.  This
declaration indicates that the elements in the sequence have type
\cd{event_t}.  It also specifies that the elements will be separated
by vertical bars, and that the sequence will be terminated by an
end-of-record marker (\kw{Peor}).  In general, \pads{} provides a rich
collection of array-termination conditions: reaching a maximum size,
finding a terminating literal (including end-of-record and
end-of-source), or satisfying a user-supplied predicate over the
already-parsed portion of the \kw{Parray}.  Finally, this type
declaration includes a \kw{Pwhere} clause to specify that the sequence
of timestamps must be in sorted order.  It uses the \kw{Pforall}
construct to express this constraint.  In general, the body of a
\kw{Pwhere} clause can be any boolean expression.  In such a context
for arrays, the pseudo-variable \cd{elts} is bound to the in-memory
representation of the sequence and \cd{length} to its length.

Returning to the CLF description in \figref{figure:wsl}, the
\kw{Penum} type \cd{method_t} describes a collection of data literals.
During parsing, \pads{} interprets these constants using the ambient
character encoding.  The \kw{Ptypedef} \cd{response_t} describes
possible server response codes in CLF data by adding the constraint
that the three-digit integer must be between 100 and 600.

The \cd{order_header_t} type in the \dibbler{} data description
contains several anonymous uses of the \kw{Popt} type.  This type is
syntactic sugar for a stylized use of a \kw{Punion} with two branches:
the first with the indicated type, and the second with the ``void''
type, which always matches but never consumes any input.



\subsection{Generated library}
\label{sec:example:generated-library}
From a description, the \pads{} compiler generates a \C{} library
for parsing and manipulating the associated data source.  We chose \C{}
as the target language for pragmatic reasons: there were 
libraries that made building the compiler and run-time libraries easier,
our target users are comfortable with \C{}, and it can serve 
as a lingua franca in that essentially all languages have provisions for 
calling \C{} libraries.  Nothing about the \pads{} language mandates compiling
to \C{}, however, and we envision eventually building alternate bindings.

From each type in a \pads{} description, the compiler generates 
\begin{itemize}
\setlength{\itemsep}{0ex plus0.2ex}
\item an in-memory representation, 
\item a mask, which allows users to customize generated functions,
\item a parse descriptor, which describes syntactic and
semantic errors detected during parsing, 
\item parsing and printing functions, and 
\item a broad collection of utility functions.
\end{itemize}
%
\setcounter{totalnumber}{1}
\setcounter{dbltopnumber}{1}
\renewcommand{\topfraction}{0.85}
\renewcommand{\textfraction}{0.1}
\renewcommand{\floatpagefraction}{0.75}
\begin{figure*}
\begin{tiny}
\chapter{Library}


Basic structure: csm and error descriptors.

Describe functions generated by padsc compiler.

\begin{verbatim}
init
cleanup
init_ed
cleanup_ed
read
copy
copy_ed
write

accumulator-related functions
init
cleanup
add
print
\end{verbatim}

Using the library: initializing the handle, etc.


Explain discipline structure of library, give pointer to 
chapter \chapref{chap:library-customization}.

\caption{Selected portions of the library generated for the \texttt{entry\_t}
  declaration from \dibbler{} data description.}
\label{figure:library}
\end{tiny}
\end{figure*}
To give a feeling for the library that \pads{} generates, 
\figref{figure:library} includes selected portions of the generated 
library for the \dibbler{} \cd{entry_t} declaration.

The \C{} declarations for the in-memory representation, the mask, 
and the parse descriptor all share the structure of the \pads{}
type declaration.  The mapping to \C{} for each is straightforward: 
\kw{Pstruct}s map to \C{} structs with appropriately mapped fields, 
\kw{Punion}s map to tagged unions coded as \C{} structs with a tag field 
and an embedded 
union, \kw{Parray}s map to a \C{} struct with a length field and an 
embedded sequence, \kw{Penums} map to \C{} enumerations, \kw{Poptions} 
to tagged unions, and \kw{Ptypedef}s to \C{} typedefs.  Masks include
auxiliary fields to control behavior at the level of a structured
type, and parse descriptors include extra fields to record the 
state of the parse, the number of detected errors, 
the error code of the first detected error, and the location of that error.

The parser takes a mask as an argument and returns an
in-memory representation and a parse descriptor.  
The mask allows the user to specify 
which constraints the parser should check and which portions of the
in-memory representation it should fill in.  This control allows the
description-writer to specify all known constraints about the data
without worrying about the run-time cost of verifying potentially
expensive constraints for time-critical applications.

Appropriate error-handling can be as important as processing
error-free data.  The parse descriptor marks which portions of the
data contain errors and characterizes the detected errors.
Depending upon the nature of the errors and the desired application,
programmers can take the appropriate action: halting the program,
discarding parts of the data, or repairing the errors.
If the mask requests
that a data item be verified and set, and if the parse descriptor
indicates no error, then the in-memory representation satisfies the
semantic constraints on the data.

Because we generate a parsing function for each type in a \pads{} description,
we support multiple-entry point parsing, which allows us to 
accommodate larger-scale data.
For a small file, programmers can define a \pads{} type that describes
the entire file and use that type's parsing function to read the whole
file with one call.  For larger-scale data, programmers can sequence
calls to parsing functions that read manageable portions of the file,
\eg{}, reading a record at a time in a loop.  The parsing code generated
for \kw{Parrays} allows users to choose between reading the entire array
at once or reading it one element at a time, again to support parsing
and processing very large data sources.

The ratio of the size of the data description to the size of the generated code gives a rough measure of the leverage of the
declarative description.  For the 
68~line \dibbler{} data description, the compiler yields a 1432~\texttt{.h} file
and a 6471~\texttt{.c} file.  This expansion comes from the extensive error checking in the generated parser and the number of generated utility functions.

We discuss details of the generated library in the following section
as we describe its uses.


\subsection{Example library use}
\label{sec:example:library-use}
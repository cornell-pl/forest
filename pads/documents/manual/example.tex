\chapter{Example}
\label{chap:example}
In this chapter, we use examples to give an overview of \pads{}.

\section{CLF: Common log format}
\label{sec:example:common-log-format}
Web servers use the Common Log Format (CLF) to log client
requests~\cite{wpp}.  Researchers use such logs to measure
properties of web workloads and to evaluate protocol changes
by "replaying" the user activity recorded in the log.
This ASCII format consists of a sequence of
records, each of which has seven fields: the host name or IP address
of the client making the request, the account associated with the
request on the client side, the name the user provided for
authentication, the time of the request, the actual request, the
\textsc{http} response code, and the number of bytes returned as a
result of the request.  The actual request has three parts: the
request method (\eg, \texttt{GET}, \texttt{PUT}), the requested
\textsc{uri}, and the protocol version.  In addition, the second and
third fields are often recorded only as a '-' character to indicate
the server did not record the actual data.  \figref{figure:clf-records}
shows a couple of typical records.

\begin{figure*}[t!]
\begin{small}
\begin{center}
\begin{code}
\chapter{Common log format example}
\section{pads description}
\section{c header file}
\end{code}
\vskip -2ex
\caption{Tiny example of web server log data.}
\label{figure:clf-records}
\end{center}
\end{small}
\end{figure*}

\begin{figure}
\begin{code}
\kw{Punion} client\_t \{
  Pip       ip;      /- 135.207.23.32
  Phostname host;    /- www.research.att.com
\};
\mbox{}
\kw{Punion} auth\_id\_t \{
  Pchar unauthorized : unauthorized == '-';
  Pstring(:' ':) id;
\};
\mbox{}
\kw{Pstruct} version\_t \{
  "HTTP/";
  Puint8 major; '.';
  Puint8 minor;
\};
\mbox{}
\kw{Penum} method\_t \{
    GET,    PUT,  POST,  HEAD,
    DELETE, LINK, UNLINK
\};
\mbox{}
bool chkVersion(version\_t v, method\_t m) \{
  \kw{if} ((v.major == 1) && (v.minor == 1)) \kw{return} true;
  \kw{if} ((m == LINK) || (m == UNLINK)) \kw{return} false;
  \kw{return} true;
\};
\mbox{}
\kw{Pstruct} request\_t \{
  '\\"';   method\_t       meth;
  ' ';    Pstring(:' ':) req\_uri;
  ' ';    version\_t      version :
                  chkVersion(version, meth);
  '\\"';
\};
\mbox{}
\kw{Ptypedef} Puint16\_FW(:3:) response\_t :
         response\_t x => \{ 100 <= x && x < 600\};
\mbox{}
\kw{Precord} \kw{Pstruct} entry\_t \{
         client\_t       client;
   ' ';  auth\_id\_t      remoteID;
   ' ';  auth\_id\_t      auth;
   " ["; Pdate(:']':)   date;
   "] "; request\_t      request;
   ' ';  response\_t     response;
   ' ';  Puint32        length;
\};
\mbox{}
\kw{Psource} \kw{Parray} clt\_t \{
  entry\_t [];
\}
\end{code}

\caption{\pads{} description for web server log data.}
\label{figure:wsl}
\end{figure}


\section{\padsl{} description}
\label{sec:example:padsl-description}
\figref{figure:wsl} gives the \pads{} description for CLF web server logs.  
We will use this example to illustrate various features of the \pads{}
language.  In \pads{} descriptions, types are declared before they are
used, so the type that describes the totality of the data source
appears at the bottom of the description. 

\kw{Pstruct}s describe fixed sequences of data with unrelated types.
In the CLF description, the type declaration for \cd{version_t}
illustrates a simple \kw{Pstruct}. It starts with a string literal
that matches the constant \cd{HTTP/} in the data source.  It then has
two unsigned integers recording the major and minor version numbers
separated by the literal character \kw{'.'}.  \pads{} supports
character, string, and regular expression literals, which are
interpreted with the ambient character encoding. The type
\cd{request_t} similarly describes the request portion of a CLF
record.  In addition to physical format information, this \kw{Pstruct}
includes a semantic constraint on the \cd{version} field.
Specifically, it requires that obsolete methods \cd{LINK} and
\cd{UNLINK} occur only under HTTP/1.1.  This constraint illustrates
the use of predicate functions and the fact that earlier fields are in
scope during the processing of later fields, as the constraint refers
to both the \cd{meth} and \cd{version} fields in the \kw{Pstruct}.
\chapref{chap:structs} describes \kw{Pstruct}s in detail.

\kw{Punion}s describe variation in the data source.  For example, the
\cd{client_t} type in the CLF description indicates that the first
field in a CLF record can be either an IP address or a hostname.
During parsing, the branches of a \kw{Punion} are tried in order; the
first branch that parses without error is taken.  The \cd{auth_id_t}
type illustrates the use of a constraint: the branch \cd{unauthorized}
is chosen only if the parsed character is a dash.  \pads{} also
supports a \textit{switched} union that uses a selection expression to
determine the branch to parse.  Typically, this expression depends
upon already-parsed portions of the data source.

\pads{} provides \kw{Parray}s to describe varying-length sequences of data all 
with the same type.  The \cd{eventSeq_t} declaration in the \dibbler{} data description
uses a \kw{Parray} to characterize the sequence of events an
order goes through during processing.  This declaration indicates that the elements
in the sequence have type \cd{event_t}.  It also specifies that the elements will
be separated by vertical bars, and that the sequence will be terminated by 
an end-of-record marker (\kw{Peor}).  In general, \pads{} provides a rich collection
of array-termination conditions: reaching a maximum size, finding a terminating
literal (including end-of-record and end-of-source),
or satisfying a user-supplied predicate over the already-parsed portion of 
the \kw{Parray}.  Finally, this type declaration includes a \kw{Pwhere} clause
to specify that the sequence of timestamps must be in sorted order.
It uses the \kw{Pforall} construct to express this constraint.
In general, the body of a \kw{Pwhere} clause can be any boolean expression.
In such a context for arrays, the pseudo-variable \cd{elts} is bound to the in-memory representation of the sequence and \cd{length} to its length.

Returning to the CLF description in \figref{figure:wsl}, the \kw{Penum} type \cd{method_t} describes
a collection of data literals.  During parsing, \pads{} interprets these
constants using the ambient character encoding.  The \kw{Ptypedef} 
\cd{response_t} describes possible server response codes in CLF data by adding
the constraint that the three-digit integer must be between 100 and 600.

The \cd{order_header_t} type in the \dibbler{} data description contains several
anonymous uses of the \kw{Popt} type.  This type is syntactic sugar for a 
stylized use of a \kw{Punion} with two branches: the first with the indicated type, and the second with the ``void'' type, which  
always matches but never consumes any input.



\section{Generated library}
\label{sec:example:generated-library}
Explain selected outputs of library.

\section{Example library use}
\label{sec:example:library-use}
\chapter{Ptypedefs}
\label{chap:typedefs}
\ptypedef{}s allow additional constraints to be added to an existing type.

\section{Syntax}
\begin{tabular}{rcl}
\nont{typedef\_predicates} & \is{}  & identifier identifier \cd{=>} \{ \nont{predicate} \}\\
\nont{typedef\_ty}    & \is{} & \Ptypedef{} \nont{p\_ty} identifier \opt{\nont{p\_formals}} \opt{\cd{:} \nont{typedef\_predicates}};\\[4ex]
\end{tabular}

\noindent
We explain the meaning of this syntax in the remainder of this chapter.
All non-terminals not defined in this grammar fragment were
defined previously, as follows.
Predicates (\term{predicate}) are defined in \secref{sec:common-predicates},
\padsl{} types (\nont{p\_ty}) and parameter lists (\nont{p\_formals})
in \secref{sec:common-parameterization}. 

\subsection{Examples}
\Ptypedef{} predicates (\nont{typedef\_predicates}) are written over a variable bound to the
in-memory representation of the base type of the \Ptypedef{}.
For example, the declaration
\input{code/simple-typedef}
defines a new type \cd{bid_t} to be a \cd{Puint32} with the additional
constraint that any legal bid must be greater than 100.  We read
this constraint ``If \cd{x} is a \cd{bid_t}, then \cd{x} must be
greater than 100.''  In the body of the constraint, the variable
\cd{x} has the in-memory represention type of the \Ptypedef{}.

Like all other \pads{} types, \Ptypedef{}s can be parameterized.  The declaration
\input{code/parameterized-typedef}
introduces a new type \cd{pn_t}.  The base type for this declaration
is 64-bit unsigned integers represented in the source as a sequence of
\cd{len} ASCII digits.  The declaration adds the constraint
that any \cd{pn_t} must have a value less than \cd{hi}.  In general,
the constraint can be any integer-valued expression.


\section{Generated library}
\subsection{In-memory representation}
\label{sec:typedefs-rep}
\input{code/typedef.TypedefRep.tex}

\subsection{Mask}
\label{sec:typedefs-masks}
\input{code/typedef.TypedefCSM.tex}

\subsection{Parse descriptor}
\label{sec:typedefs-parse-descriptors}
\input{code/typedef.TypedefPD.tex}

\subsection{Operations}
\input{code/typedef.TypedefOps.tex}
\subsubsection{read}
The error codes for \Popt{}s are:

\tskip{}
\begin{tabular}{lp{4in}}
 \cd{P_NO_ERR}                 & Indicates no error occurred\\[1ex]
 \cd{P_TYPEDEF_CONSTRAINT_ERR} & Indicates that the typedef constraint failed.\\[1ex]
\end{tabular}

\noindent

\subsubsection{Accumulator functions}
Accumulator functions for \Ptypedef{}s are described in
\chapref{chap:accumulators}. 

\chapter{Library customization}
\label{chap:library-customization}
\cutname{library_customization.html}
The \pads{} core library is parameterized by a main discpline and by
an IO discipline that allow users to customize the behavior of the
core system and the IO subsystem, respectively.  One aspect of
controlling IO is choosing what constitues a \textit{Precord} in the
data source.

\section{The \pads{} discipline}
A \pads{} handle, \ie{}, a value of type \cd{P\_t}, contains a field
named \cd{my\_disc} of type \cd{Pdisc\_t}, which appears in
\figref{fig:pdisc}. The various fields of \cd{my\_disc} allow users to
customize aspects of the \pads{} system.  We describe these fields in
the following sections.

\begin{figure}
\inputCode{code/pdisc}
\caption{The \texttt{Pdisc\_t} type, which allows users to customize the
  behavior of the \pads{} system.}
\label{fig:pdisc}
\end{figure}

\subsection{Version}
\label{sec:library-customization-version}
This field stores the interface version of the \pads{} library.
The \C{} macro \cd{P\_VERSION} refers to the current version.

\subsection{Control flags}
\label{sec:library-customization-control-flags}
The \cd{flags} field is a combination of bit fields, described in the
following sections.  At the moment, only one such field is in use.

\subsubsection{White space}
\label{sec:library-customization-white-space}
If the \cd{P\_WSPACE\_OK} bit is set in the \cd{flags} field, 
then leading white space is allowed for variable-width ASCII
integers.  Similarly, leading and/or trailing white space is allowed
for fixed-width ASCII integers.

\subsection{Character encodings}
\label{sec:library-customization-character-encodings}
\pads{} supports ASCII and EBCDIC encodings.
The \cd{def\_charset} field indicates the character set for
interpreting base types with no explicit character encoding by storing
one of the following values:
\begin{verbatim}
 Pcharset_ASCII
 Pcharset_EBCDIC
\end{verbatim}


\subsection{Copying strings} 
\label{sec:library-customization-copy-strings}
 If \cd{copy\_string} field in \pads{} discipline is non-zero, the string read functions
 copy the strings found into the supplied representation, otherwise they do not.
 Instead, the target \cd{Pstring} points to memory managed by the current
 IO discipline.  This sharing avoids copying and speeds programs that
 will never reference an old record after a new one is read in.
Field \cd{copy\_strings} should only be set to zero for record-based IO disciplines where
strings from record K are not used after \cd{P\_io\_next\_rec} has been called to move
the IO cursor to record K+1.  Note: \cd{Pstring\_preserve} can be used to
force a string that is using sharing to make a copy so that the string is 'preserved'
(remains valid) across calls to \cd{P\_io\_next\_rec}.

\subsection{Scanning extent}
\label{sec:library-customization-scanning-extent}
When scanning for a literal or regular expression match, how far
should the scan/match go before giving up?    If a record-based IO discipline is
used, scanning and matching is limited to the scope of a single record.  In
addition, the following 4 \cd{Pdisc\_t} fields can be used to provide further
constraints on scan/match scope.  

\begin{description}
\item[\cd{match\_max}] This field specifies the maximum number of
  bytes that will be included in an inclusive pattern match attempt
  (see, \eg{} data type \cd{Pstring\_ME}).  If set to zero, no
  \cd{match\_max} constraint is imposed for a record-based IO
  discipline (other than finding end-of-record), whereas a built-in
  soft limit of \cd{P\_BUILTIN\_MATCH\_MAX} characters is imposed for
  non-record-based IO disciplnes.  (The built-in limit is soft because
  if the match happens to get more than \cd{P\_BUILTIN\_MATCH\_MAX}
  characters in a single IO discipline read call it will go ahead and
  consider all of them.  In contrast, if the discipline
  \cd{match\_max} is set explicitly to value \texttt{K}, then this is
  a hard limit: the match will only consider \texttt{K} characters
  even if more are available.)

\item[\cd{numeric\_max}] This field specifies the maximum number of
   bytes that will be included in an attempt to read a character-based
   representation of a number.  If non-zero, the field should be set
   large enough to cover any leading white space (if allowed by
   \cd{P\_WSPACE\_OK}), an optional +/- sign, and the digits (dot
   \etc{} for floats) that make up the numeric value.  A value of zero
   for \cd{numeric\_max} results in an end-of-record constraint for
   record-based IO disciplines and in a soft limit of
   \cd{P\_BUILTIN\_NUMERIC\_MAX} bytes for non-record-based IO
   disciplines.

\item[\cd{scan\_max}] This field specifies the maximum number of bytes
   that will be considered by a normal scan that is looking for a
   terminating literal or a terminating regular expressin (see, \eg{}
   data type \cd{Pstring\_SE}.).  Note that this includes both the
   bytes skipped plus the bytes used for the match.  A \cd{scan\_max}
   of zero results in an end-of-record constraint for record-based IO
   disciplines and in a soft limit of \cd{P\_BUILTIN\_SCAN\_MAX} bytes
   for non-record-based IO disciplines.
  
\item[\cd{panic\_max}] This field specifies the maximum number of
  bytes that will be considered by when 
  parsing hits a 'panic' state and is looking for a synchronizing
  literal or pattern.  See, for example, termination conditions for
  user-defined array types.  A \cd{panic\_max} of zero results in an
  end-of-record constraint for record-based IO disciplines and in a
  soft limit of \cd{P\_BUILTIN\_PANIC\_MAX} bytes for non-record-based IO
  disciplines.
\end{description}

For non-record-based IO disciplines, the default soft limits may
be either too small or too large for a given input type.  It is
important for these cases to determine appropriate hard limit settings.

The built-in soft limits for use with non-record-based IO disciplines are
as follows.  Although you can change them and recompile the PADS library,
it is easier to simply set up the correct hard limits in the discipline.

\inputCode{code/scanlimits}

\subsection{File open function}
The field \cd{fopen\_fn} stores the file open function used by
\cd{P\_io\_fopen}. If this field is NULL, the default function
\cd{P\_fopen} is used.  

A \cd{Pfopen\_fn} takes arguments (source, mode) and 
opens file source in the specified mode and returns 
an SFIO stream on success or NULL on failure.

\inputCode{code/pfopen_fn}
%
\noindent
It should normally have the the same semantics as the call
\cd{sfopen(NiL, string, mode)}, except that for the string constants
 it should return an existing SFIO stream:

\begin{tabular}{lcl}
   \cd{\literal{"/dev/stdin"}}  &\quad $\longrightarrow{}$ \quad&   \cd{sfstdin}\\
   \cd{\literal{"/dev/stdout"}} &\quad $\longrightarrow{}$ \quad&   \cd{sfstdout}\\
   \cd{\literal{"/dev/stderr"}} &\quad $\longrightarrow{}$ \quad&   \cd{sfstderr}\\
\end{tabular}

For \cd{\literal{"/dev/stdin"}}, mode \cd{\literal{"r"}} (read) is expected.
For \cd{\literal{"/dev/stdout}} or \cd{\literal{"/dev/stderr"}}, mode
\cd{\literal{"a"}} (append-only) is expected. 
If you use some other mode for these three cases, the function 
should attempt to apply mode to the specified SFIO stream;
it should return NULL if this fails, otherwise the specified stream.

\subsection{Error function}
The field \cd{error\_fn} stores the error reporting function. 
If this field is NULL, the default function \cd{P\_error} is used.  
The type for this function is:

\inputCode{code/errorfn}
%
\noindent
Error functions are a lot like the standard \cd{printf} funtion, with
two additional arguments, \cd{libnm} and \cd{level}.  For example, where
one might use \cd{printf} as follows:

\begin{centercode}
     printf("count: \%d\newl{}", 10);
\end{centercode}
%
\noindent
one  can do the same thing with \cd{P_error} using:
\begin{centercode}
     P\_error(NULL, P\_LEV\_INFO, "count: \%d", 10);
\end{centercode}
%
\noindent
Note that \cd{P\_error} automatically adds a newline, so we did not
have to include a \cd{\literal{"\newl{}"}} in the format string, as we did with
the \cd{printf} example.

The first argument to an error function, \cd{libnm},  is normally
\cd{NULL} (it is for the name of the library calling the error
function).

The second argument, \cd{level}, is used to specify the severity of
the error.  Level \cd{P\_LEV\_INFO} is used for an informative (non-error)
message, \cd{P\_LEV\_WARN} is used for a warning, \cd{P\_LEV\_ERR} for a
non-fatal error, and \cd{P\_LEV\_FATAL} for a fatal error.
One can 'or' in the following flags (as in
\cd{P\_LEV\_WARN|P\_FLG\_PROMPT)}:

\begin{tabular}{lcl}
\cd{P\_FLG\_PROMPT} & \quad &  do not emit a newline\\
\cd{P\_FLG\_SYSERR} & \quad &  add a description of \cd{errno} (\cd{errno} should be a system error)\\
\cd{P\_FLG\_LIBRARY} & \quad & error is from library\\
\end{tabular}

Library messages are forced if the environment variable
\cd{ERROR\_OPTIONS} includes the string ``library'';
SYSERR (\cd{errno}) messages are forced if it includes the string ``system.''

For convenience, if the first argument, library name \cd{libnm}, is non-NULL,
then flag \cd{P\_FLG\_LIBRARY} is automatically or'd into \cd{level}.  

\subsection{Endian-ness}
\label{sec:library-customization-endian}
The field \cd{d\_endian} stores the endianness of the data.  It can
have the value \cd{PbigEndian} or \cd{PlittleEndian}.  
If \cd{d\_endian} does not equal the endiannes of the machine running
the parsing code, the bye order of binary integeeres is swapped by the
binary integer read function. 

\subsection{Accumulator customization}
\label{sec:library-customization-accumulator}
The fields \cd{acc\_max2track}, \cd{acc\_max2rep}, and
\cd{acc\_pct2rep} allow the user to customize accumulator behavior.  
\secref{sec:accumulators-customization} describes these fields in
detail.

\subsection{Input time zone}
\label{sec:library-customization-input-time-zone}
The field \cd{in\_time\_zone} specifies the default time zone for
string-based time input, used for input date strings 
that do not have time zone information in them.     For example, the date
\texttt{15/Oct/1997:18:46:51} has no time zone information.  If
\cd{in\_time\_zone} is set  to \literal{\cd{"UTC"}}, 
then this date/time would be assumed to be a UTC time.
In contrast, regardless of the \cd{in\_time\_zone} setting, the date
\texttt{15/Oct/1997:18:46:51 -0700}
will always be interpreted as being in a time zone seven hours
earlier than UTC time.

The \cd{in\_time\_zone} is passed to the \cd{tmzone} function, so it
must be a time zone description that \cd{tmzone}
understands. Intuitively, it accepts three-letter strings such as 
\literal{\cd{"PST"}}  and \literal{\cd{"EDT"}}  as well as strings
denoting numeric offsets from UTC time, such as \literal{\cd{"-0500"}}.
\chapref{chap:library-use} describes the legal time zone designation strings.
Documentation for the \cd{tmzone} function appears on the web page:
\ifthenelse{\boolean{hevea}}{
\myurl{www.research.att.com/\~gsf/man/man3/tm.html}}{
\myurl{www.research.att.com/~gsf/man/man3/tm.html}}


Before calling \cd{P\_open}, the discipline field \cd{disc->in\_time}
can be initialized directly.  After calling \cd{P\_open}, however, it
must be changed by passing the pads handle and a time zone string to 
\cd{P\_set\_in\_time\_zone}, \eg{},

\begin{centercode}
    P\_set\_in\_time\_zone(pads,\literal{"PST"});
\end{centercode}

This will set \cd{pads->disc->in\_time\_zone}, and will also update
an internal representation of the time zone maintained as part of
the pads state.

\subsection{Output time zone}
\label{sec:library-customization-output-time-zone}
This field specifies the output time zone for formatted time output.
Regardless of the time zone used to read in a time,
\cd{disc->output\_time\_zone} controls which time zone is used when
formatting the time for output.  For example, a time that is read as 6am UTC time
would be formatted as 1am if the \cd{output\_time\_zone} is \cd{\literal{"-0500"}}.
Note that in the normal case you should use the same time zone
for both input and output, unless you are intentially translating
times from one time zone to another one. The format of output time
zone specification strings is the same as for input time zone.
\chapref{chap:library-use} describes the legal time zone designation strings.

Before calling \cd{P\_open}, the discipline field \cd{disc->in\_time}
can be initialized directly.  After calling \cd{P\_open}, however, it
must be changed by passing the pads handle and a time zone string to 
\cd{P\_set\_output\_time\_zone}, \eg{},

\begin{centercode}
    P\_set\_output\_time\_zone(pads,\literal{"CDT"});
\end{centercode}

This will set \cd{pads->disc->output\_time\_zone}, and will also update
an internal representation of the output time zone maintained as part of
the pads state.

\subsection{Input formats}
\label{sec:library-customization-input-formats}
The \cd{in\_formats} field of the discipline allows one to specify
default input formats for some special types where there is 
in no 'obvious' default. \figref{fig:input-formats} contains the type
of this field.
\begin{figure}
\inputCode{code/pinformats}
\caption{The \texttt{Pin\_formats\_t} type, which allows users to specify the
  input format of various \pads{} base types. Each of the fields must
  be a non-null string with a format understood by the \texttt{tmdate} function.}
\label{fig:input-formats}
\end{figure}
The current entries are:

\begin{description}
\item[\cd{in\_formats.timestamp}]
This field contains a format string specifying the input timestamp
format, for use with \cd{Ptimestamp} and its variants.  Alternatives
can be given using \cd{\%|}, and the special \cd{\%\&} format can be
used to indicate all formats that can be parsed by the
\cd{tmdate} fuction.  The default,

\inputCode{code/timestamp-format}
%
\noindent
allows for timestamps of these forms:
\begin{verbatim}
 091172+230202
 091172+23:02:02
 09111972+230202
 09111972+23:02:02
\end{verbatim}
 and also allows for all date/time formats parsed by \cd{tmdate}.
Documentation for the \cd{tmdate} function appears on the web page:
\ifthenelse{\boolean{hevea}}{
\myurl{www.research.att.com/\~gsf/man/man3/tm.html}}{
\myurl{www.research.att.com/~gsf/man/man3/tm.html}}
 
\item[\cd{in\_formats.date}]
 A format string specifying the input date format, for use with
 \cd{Pdate} and its variants.  The default, 

\inputCode{code/date-format}
%
\noindent
 allows for dates of these two forms:
\begin{verbatim}
   091172
   09111972
\end{verbatim}
and also allows for all date formats parsed by \cd{tmdate}.
Documentation for the \cd{tmdate} function appears on the web page:
\ifthenelse{\boolean{hevea}}{
\myurl{www.research.att.com/\~gsf/man/man3/tm.html}}{
\myurl{www.research.att.com/~gsf/man/man3/tm.html}}

\item[\cd{in\_formats.tme}]
 A format string specifying the input time format, for use with
 \cd{Ptime} and its variants.  The default, 

\inputCode{code/time-format}
%
\noindent
 allows for times of these two forms:
\begin{verbatim}
  230202
  23:02:02
\end{verbatim}
and also allows for all date formats parsed by \cd{tmdate}.
Documentation for the \cd{tmdate} function appears on the web page:
\ifthenelse{\boolean{hevea}}{
\myurl{www.research.att.com/\~gsf/man/man3/tm.html}}{
\myurl{www.research.att.com/~gsf/man/man3/tm.html}}

\end{description}

\subsection{Output formats}
\label{sec:library-customization-output-formats}
The \cd{out\_formats} field of the discipline allows one to specify
default output formats for some special types where there is 
in no 'obvious' default. \figref{fig:output-formats} contains the type
of this field.
\begin{figure}
\inputCode{code/poutformats}
\caption{The \texttt{Pout\_formats\_t} type, which allows users to specify the
  output format of various \pads{} base types. Each of the fields must
  be a non-null string with a format understood by the \texttt{fmttime} function.}
\label{fig:output-formats}
\end{figure}
The current entries are:
\begin{description}
\item[\cd{out\_formats.timestamp}]
\item[\cd{out\_formats.timestamp\_explicit}]
These two values specifying the default output formats for the \cd{Ptimestamp}
and \cd{Ptimestamp\_explicit} families of types, respectively.  The normal use is for these formats
to describe both the date and time of day.  Some examples:

\inputCode{code/timestamp-output-format}
%
\noindent
Documentation for the \cd{tmdate} function appears on the web page:
\ifthenelse{\boolean{hevea}}{
\myurl{www.research.att.com/\~gsf/man/man3/tm.html}}{
\myurl{www.research.att.com/~gsf/man/man3/tm.html}}


\item[\cd{out\_formats.date}]
\item[\cd{out\_formats.date\_explicit}]
These two values specifying the default output formats for the \cd{Pdate}
and \cd{Pdate\_explicit} families of types, respectively.  The normal use is for these formats
to describe the date but not the time of day.  Some examples:

\inputCode{code/date-output-format}
%
\noindent
Documentation for the \cd{tmdate} function appears on the web page:
\ifthenelse{\boolean{hevea}}{
\myurl{www.research.att.com/\~gsf/man/man3/tm.html}}{
\myurl{www.research.att.com/~gsf/man/man3/tm.html}}



\item[\cd{out\_formats.time}]
\item[\cd{out\_formats.time\_explicit}]
These two values specifying the default output formats for the \cd{Ptime}
and \cd{Ptime\_explicit} families of types, respectively.  The normal use is for these formats
to describe a time of day but not the date.  Some examples:

\inputCode{code/time-output-format}
%
\noindent
Documentation for the \cd{tmdate} function appears on the web page:
\ifthenelse{\boolean{hevea}}{
\myurl{www.research.att.com/\~gsf/man/man3/tm.html}}{
\myurl{www.research.att.com/~gsf/man/man3/tm.html}}
\end{description}

\subsection{Writing invalid values}
\label{library-customization-writing-invalid}
Write functions take a parse descriptor and a value.  The value is valid if the
parse descriptor's \cd{errCode} is set to \cd{P\_NO\_ERR}.  The value
has been filled in if the \cd{errCode} is
\cd{P\_USER\_CONSTRAINT\_VIOLATION}.  For other \cd{errCodes}, the value
should be assumed to contain garbage.  For invalid values, the write function must still
write SOME value.  For every type, one can specify an \cd{inv\_val} helper function
that produces an invalid value for the type, to be used by the type's write
functions.  If no function is specified, then a default invalid value is used,
where there are two cases: if the \cd{errCode} is \cd{P\_USER\_CONSTRAINT\_VIOLATION}, then
the current invalid value is used; for any other \cd{errCode}, a default invalid
value is used.

The map from write functions to \cd{inv\_val} functions is found in
the discipline in the field \cd{inv\_val\_fn\_map}, which maps
values of type  \cd{const char *} (the string form of the type name)
to \cd{Pinv\_val\_fn} functions.  If the \cd{inv\_val\_fn\_map} field
is \cd{NULL}, no mappings are used.

An invalid value function that handles type \cd{T} values always takes
four arguments:
\begin{itemize}
\item The \cd{P\_t*} handle
\item  A pointer to a type \cd{T} parse descriptor
\item  A pointer to the invalid type \cd{T} representation
\item  A \cd{va\_list} argument that encodes the extra type parameters (if any)
      associated with the type.  For example,
      type \cd{Pa\_int32\_FW(:<width>:)} has a single type parameter (width) of type \cd{Puint32},
      so \cd{va\_list} would encode a single \cd{Puint32} argument.
\end{itemize}
 Arguments two and three use \cd{void*} types to enable the table to be used with arbitrary types,
including user-defined types.  One must cast these \cd{void*} args to the appropriate
 pointer types before use; see the example below.  The function should
replace the invalid value with a new value, and return \cd{P\_OK} on
success and \cd{P\_ERR} if a replacement value has not been set.

Use \cd{P\_set\_inv\_val\_fn} to set a function pointer,
\cd{P\_get\_inv\_val\_fn} to do a lookup:

\inputCode{code/invfn}

\textbf{Example}: suppose an \cd{Pa\_int32} field has an attached constraint that requires the
value must be at least negative thirty.  If a value of negative fifty is read, \cd{errCode} will be
\cd{P\_USER\_CONSTRAINT\_VIOLATION}. If no \cd{inv\_val} function has been provided, then the
\cd{Pa\_int32} write function will output \cd{\literal{-50}}.  If the read function fails to read even a
valid integer, the \cd{errCode} will be \cd{P\_INVALID\_A\_NUM}, and the \cd{Pa\_int32} write
function will output \cd{P\_MIN\_INT32} (the default invalid value for all \cd{int32} write
functions). If one wanted to correct all user contraint cases to use value \cd{\literal{-30}}, and
to use \cd{P\_MAX\_INT32} for other invalid cases, one could provide an \cd{inv\_val}
helper function to do so:

\inputCode{code/invfn-example}
%

Note that for a type \cd{T} with three forms, \cd{P\_T}, \cd{Pa\_T}, and \cd{Pe\_T}, there
is only one entry in the \cd{inv\_val\_fn\_map}, under string
\cd{\literal{"P_T"}}.  For example, use 
\cd{\literal{"Pint32"}} to specify an invalid value function for all of these types: \cd{Pint32},
\cd{Pa\_int32}, and \cd{Pe\_int32}.

An \cd{inv\_val\_fn} for a string type should use \cd{Pstring\_copy}, \cd{Pstring\_cstr\_copy},
\cd{Pstring\_share}, or \cd{Pstring\_cstr\_share} to fill in the value of the \cd{Pstring*} param.


\section{The IO Discpline}
\label{sec:io-discipline}
IO discipline values, which have type \cd{Pio\_disc\_t}, control the
'raw' reading of data from a file or from some other data source.  
The \pads{} system provides a collection of functions for generating
various IO disciplines, corresponding to various kinds of record
structures: new-line terminated, fixed width, IBM-style (initial data
indicating size of record, followed by payload), \etc{}  In addition,
the discipline indicates if the data source is seekable (a file) or
not (a stream).

To use an IO discipline, the user first creates one by invoking
a creation function supplied by the \pads{} system.  The resulting IO
discipline is then installed by passing it as an argument to either
\cd{P\_open} or to \cd{P\_set\_io\_disc}.

\begin{description}
\item
[\small{\cd{Pio\_disc\_t * P\_fwrec\_make(size\_t leader\_len, size\_t data\_len, size\_t trailer\_len)}}]
 Instantiates an instance of \cd{fwrec}, a discipline for fixed-width
 records.  The parameter \cd{data\_len} specifies the number of data bytes per record,
 while \cd{leader\_len} and \cd{trailer\_len} specify the number of bytes that
 occur before and after the data bytes within each record (either or
 both can be zero).  Thus the total record size in bytes is the sum
 of the three arguments.  

\item[\small{\cd{Pio\_disc\_t * P\_fwrec\_noseek\_make(size\_t leader\_len,
       size\_t data\_len, size\_t trailer\_len)}}]
Instantiates an instance of \texttt{fwrec\_noseek}, a version of \texttt{norec}
that does not require that the SFIO stream is seekable.

\item[\small{\cd{Pio\_disc\_t * P\_ctrec\_make(Pbyte termChar, size\_t block\_size\_hint);}}]
Instantiates an instance of \cd{ctrec}, a discipline for
character-terminated variable-width records. Argument \texttt{termChar} is the
character that marks the end of a record. Argument
\texttt{block\_size\_hint} suggests a block size to use, if the
discipline chooses to do fixed block-sized reads 'under the covers'.
It may be ignored by the discipline.
For ASCII newline-terminated records use, \literal{\cd{'\newl{}'}} or
\cd{P\_ASCII\_NEWLINE} 
as the term character.  For \cd{EBCDIC} newline-terminated records, use
\cd{P\_EBCDIC\_NEWLINE} as the term character.


\item[\small{\cd{Pio\_disc\_t * P\_ctrec\_noseek\_make(Pbyte termChar,
      size\_t block\_size\_hint)}}] 
Instantiates an instance of \cd{ctrec\_noseek}, a version of \cd{norec}
that does not require that the SFIO stream is seekable.

\item[\small{\cd{Pio\_disc\_t * P\_nlrec\_make(size\_t block\_size\_hint)}}]
Shorthand for calling the corresponding \cd{ctrec} make function with
\literal{\cd{'\newl{}'}} as \cd{termChar}.

\item[\small{\cd{Pio\_disc\_t * P\_nlrec\_noseek\_make(size\_t block\_size\_hint)}}]
Shorthand for calling the corresponding \cd{ctrec} make function with
\literal{\cd{'\newl{}'}} as \cd{termChar}.



\item[\small{\cd{Pio\_disc\_t * P\_vlrec\_make(int blocked, size\_t avg\_rlen\_hint)}}]
 Instantiates an instance of \cd{vlrec}, a discipline for IBM-style
 variable-length records with record length specified at the start
 of each record.  If blocked is set (\cd{!= 0}) then the records are
 grouped into blocks, where each block has a length given at the
 start of each block.  Argument \cd{avg\_rlen\_hint} is a hint as to what the
 average record length is, to help the discipline allocate memory.
 It should include the four bytes at the start of each record used for
 the record length.  It may be ignored by the discipline.
 

\item[\small{\cd{Pio\_disc\_t * P\_vlrec\_noseek\_make(int blocked,
 size\_t avg\_rlen\_hint)}}] Instantiates an instance of
 \cd{vlrec\_noseek}, a version of \cd{vlrec} that does not require
 that the SFIO stream is seekable.


\item[\small{\cd{Pio\_disc\_t * P\_norec\_make(size\_t block\_size\_hint)}}]
Instantiates an instance of \cd{norec}, a raw bytes discipline that
does not use records.  Argument \cd{block\_size\_hint} is a hint as to what block size
to use, if the discipline chooses to do fixed block-sized reads
'under the covers'.  It may be ignored by the discipline.


\item[\small{\cd{Pio\_disc\_t * P\_norec\_noseek\_make(size\_t block\_size\_hint)}}]
Instantiates an instance of \cd{norec\_noseek}, a version of \cd{norec}
that does not require that the SFIO stream is seekable.


\end{description}



\subsection{Closing an IO discipline}
When an IO discipline is no longer needed, the user should unmake it.
The function \cd{P\_io\_disc\_unmake} explicitly deallocates an IO
discipline. In addition, the function \cd{P\_close}
deallocates the installed IO discipline.  
The function \cd{P\_set\_io\_disc} deallocates the previously
installed discipline.
If desired, an IO discipline can be preserved using
\cd{P\_close\_keep\_io\_disc} or \cd{P\_set\_io\_disc\_keep\_old}, in
which case it can be re-used in a future \cd{P\_open} or
\cd{P\_set\_io\_disc} call. 

\subsection{Implementations}
Implementations of the standard IO disciplines can be found in
\texttt{libpads/default\_io\_disc.c}.  Anyone planning to implement a new IO
discipline should consult \texttt{default\_io\_disc.c}.



\section{Adding new base types}
\label{sec:library-adding-new-base-types}

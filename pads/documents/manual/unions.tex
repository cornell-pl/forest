\chapter{Punions}
\label{chap:unions}
\Punion{}s are used to express variations in data.  \pads{}
supports two forms of union: switched and in-place.  The first form
supports data sources where there is an indication (\ie, a switch) in
the data prior to the union indicating which alternative should be
chosen.  The second form supports data sources where no such switch is
present.  In this case, the read code tries the branches in order
until it finds one in which no errors occurred during parsing.
\section{Syntax}
\begin{tabular}{rcl}
\nont{union\_field} & \is{} & \nont{full\_field} \alt{} \nont{comp\_field}\\[1ex]
\nont{branch}     & \is{} & \Pcase{} \nont{expression} : \nont{union\_field}
                    \alt{}  \Pdefault : \nont{union\_field}\\[1ex]
\nont{branches}   & \is{} & \nont{branch} \alt{} \nont{branch} \nont{branches} \\[1ex]
\nont{switched}   & \is{} & \Pswitch{} (\nont{expression})\{ \nont{branches} \}\\[1ex]
\nont{in\_place}  & \is{} & \nont{union\_field} \alt{} \nont{union\_field} \nont{in\_place}\\[1ex]
\nont{union\_bdy} & \is{} & \nont{switched} \alt{} \nont{in\_place}\\[1ex]
\nont{union\_ty}  & \is{} & \Punion{} identifier \opt{\nont{formals}} \{ \nont{union\_bdy} \} \\[4ex]
\end{tabular}

\noindent
We explain the meaning of this syntax in the remainder of this chapter.
All non-terminals not defined in this grammar fragment were
defined previously, as follows.
Full fields (\nont{full\_field}) 
appear in \secref{sec:structs-full-fields}, 
computed fields (\nont{comp\_field}) in
\secref{sec:structs-computed-fields}, and
parameter lists (\nont{formals}) in \secref{sec:common-parameterization}.
Expressions (\nont{expression}) represent any \C{} expression. 


\subsection{Example: Switched \Punion{}s}
The \pads{} declarations in \figref{fig:switched-union} describe data
which uses an integer tag to determine the format of the rest of the
data. 
The \Pstruct{} \cd{choice} specifies the integer field \cd{which}
should be passed to the switched \Punion{} \cd{branches}.   
The \cd{branches} declaration describes three alternatives, depending
upon the value of the tag \cd{which}.  
%
\begin{figure}
\input{code/union.switch}
\caption{Switched \Punion{} for describing data variations determined
  by tags earlier in the data source.}
\label{fig:switched-union}
\end{figure}
%
A tag value of \cd{1} indicates an unsigned integer will follow:
\begin{verbatim}
 1 4
\end{verbatim}%
while a tag
value of \cd{2} indicates a string terminated by an end-of-record
mark:  
\begin{verbatim}
 2 hello
\end{verbatim}%
Any other value for the tag will fall into the default clause of the
union, which indicates that no further data
exists:
\begin{verbatim}
 3
\end{verbatim}%

\subsection{Example: In-place \Punion{}s}
The following in-place \Punion{} describes a data fragment that is
either a resolved or a symbolic IP address:
\input{code/host_t}
The (omitted) types \cd{nIP} and \cd{sIP} describe named and symbolic
IP addresses, respectively.
The comments embedded in the description give an example of each of the two
forms.   With in-line \Punion{}s, the parser tries each of the branches
in turn until it finds one that matches the data without any errors.

\subsection{\Pswitch{}}
The expression on which a switched \Punion{} branches can be any \C{}
expression of integer type (as in a \kw{switch} statement in \C{}).
Typically, this expression is computed from a parameter to the
switched \Punion{}.

\subsection{Branches}
Each branch in either a switched or in-place \Punion can have one of
two forms: a \Pcase{} statement, or a \Pdefault{} statement.  


\section{Generated library}
\subsection{In-memory representation}
\label{sec:unions-rep}
\subsection{Mask}
\label{sec:unions-masks}
\subsection{Parse descriptor}
\label{sec:unions-parse-descriptors}
\subsection{Operations}
init/cleanup rep
init/cleanup ed
\subsubsection{read}
  error codes
\subsubsection{Accumulator functions}


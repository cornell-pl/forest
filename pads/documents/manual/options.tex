\chapter{Popts}
\label{chap:opts}
\section{Syntax}
\begin{tabular}{rcl}
\nont{union\_field} & \is{} & \nont{full\_field} \alt{} \nont{comp\_field} \alt{} \nont{literal\_field}\\[1ex]
\nont{branch}     & \is{} & \Pcase{} \nont{expression} : \nont{union\_field}
                    \alt{}  \Pdefault : \nont{union\_field}\\[1ex]
\nont{branches}   & \is{} & \nont{branch} \alt{} \nont{branch} \nont{branches} \\[1ex]
\nont{switched}   & \is{} & \Pswitch{} (\nont{expression})\{ \nont{branches} \}\\[1ex]
\nont{in\_place}  & \is{} & \nont{union\_field} \alt{} \nont{union\_field} \nont{in\_place}\\[1ex]
\nont{union\_bdy} & \is{} & \nont{switched} \alt{} \nont{in\_place}\\[1ex]
\nont{union\_ty}  & \is{} & \Punion{} identifier \opt{\nont{p\_formals}} \{ \\
&& \quad \nont{union\_bdy} \\
&& \}\ \opt{ \Pwhere{} \ \{\ \nont{predicate}\ \}}; \\[4ex]

\end{tabular}

\noindent
We explain the meaning of this syntax in the remainder of this chapter.
All non-terminals not defined in this grammar fragment were
defined previously, as follows.
Full fields (\nont{full\_field}) 
appear in \secref{sec:structs-full-fields}, 
computed fields (\nont{comp\_field}) in
\secref{sec:structs-computed-fields}, 
literals (\nont{literal\_fields}) in \secref{sec:common-literals}
and
\padsl{} parameter lists (\nont{p\_formals}) in \secref{sec:common-parameterization}.
Expressions (\nont{expression}) represent any \C{} expression. 


\subsection{Example}

\section{Generated library}


\subsection{In-memory representation}
\label{sec:opts-rep}

\subsection{Mask}
\label{sec:opts-masks}

\subsection{Parse descriptor}
\label{sec:opts-parse-descriptors}

\subsection{Operations}


\subsubsection{Read function}

The error codes for \Popts{}s are:

\tskip{}
\begin{tabular}{lp{4in}}
 \cd{PDC\_NO\_ERR}                 & Indicates no error occurred\\[1ex]
 \cd{PDC\_UNION\_MATCH\_ERR}         & Indicates that no branch of the
                                    union parsed without error.\\[1ex]
\end{tabular}

\noindent

\subsubsection{Accumulator functions}
Accumulator functions for \Popts{}s are described in \chapref{chap:accumulators}. 


\chapter{Penums}
\label{chap:enums}
\cutname{enums.html}
\Penum{}s allow a fixed collection of source strings to be converted into
integers in-memory.

\section{Syntax}
\begin{tabular}{rcl}
\nont{p\_enum\_prefix}      & \is{}  & \kw{Pprefix} ( identifier ) \\
\nont{p\_raw\_enum\_field}  & \is{}  & \nont{p\_literal} \opt{= expression } \\
\nont{p\_enum\_field}       & \is{}  & \nont{p\_raw\_enum\_field}, \opt{p\_comment}\\
\nont{p\_last\_enum\_field} & \is{}  & \nont{p\_raw\_enum\_field} \opt{p\_comment}\\
\nont{p\_enum\_fields}   & \is{}  & \nont{p\_last\_enum\_field} \\
                         & \alt{} & \nont{p\_enum\_field} \nont{p\_enum\_fields} \\
\nont{enum\_ty}    & \is{} & \Penum{} identifier \opt{\nont{p\_formals}} \opt{\nont{p\_enum\_prefix}} \{ \nont{p\_enum\_fields} \} ;\\[4ex]
\end{tabular}

\subsection{Example}
If the physical representation consists of the following
strings:\texttt{S\_init}, \texttt{S\_lec}, 
\texttt{S\_care}, \texttt{S\_for}, \texttt{S\_if}, and
\texttt{S\_tpv},
then we can use the specification: 

\inputCode{code/enum}
%
\noindent
to describe the data. 
The labels in the enumeration correspond to the strings in the
physical representation.  To support physical representations that are
not valid \C{} identifiers, the \Pfrom{} clause allows the user to
specify the physical string separately from the enumeration label,
\eg{}, \texttt{my\_for} in place of \texttt{for}.  The argument 
to the \Pfrom{} clause can be any character, string, or regular expression.
The \Pprefix{} clause indicates that all the
labels in the enumeration are prefixed by the specified string in the
physical representation.  \pads{} assigns integer values to the labels
in the enumeration in the same fashion that \C{} does, so the first
label is given the value zero, and each successive label is given the
next higher value.  The user can specify a value for a given label
using the equal expression syntax.  The compiler does not check that
all labels are given distinct values. An artifact of the embedding
into \C{} is that all labels must be globally distinct.  If two
different enumerations contain the same string, one of them must be
differentiated using a \Pfrom{} clause.


\section{Generated library}
We use the \texttt{orderStates} example to illustrate the data
structures and functions generatef for \Penum{}s.
\subsection{In-memory representation}
\label{sec:enums-rep}
The following \C{} code is the generated in memory representation for
the \texttt{orderStates} \Penum{}. The associated values are computed
at \pads{} compile time and so the associated values must be
compile-time constants, as in \C{}.

\inputCode{code/enumRep}

\subsection{Mask}
\label{sec:enums-masks}
The mask for a \Penum{} is simply a base mask:

\inputCode{code/enumMask}

\subsection{Parse descriptor}
\label{sec:enums-parse-descriptors}
The parse descriptor for a \Penum{} is simply a base parse descriptor:

\inputCode{code/enumPD}

\subsection{Operations}
The operations for \Penum{}s are those described in
\chapref{chap:common-features}, with one addition.
The \texttt{orderStates2str} function converts the integer
representation of a \Penum{} into a \C{} string.
These functions appear in \figref{figure:enum-ops}.
\begin{figure}
\inputCode{code/enumOps}
\caption{Prototypes of operations generated for
  the \Penum{} \texttt{orderStates}.}
\label{figure:enum-ops}
\end{figure}

\subsubsection{Read Function}
The error codes for \Penum{}s are:

\tskip{}
\begin{center}
\begin{tabular}{l|p{4in}}
Code                           & Meaning \\ \hline
 \cd{P_NO_ERR}                 & Indicates no error occurred\\[1ex]
 \cd{P_ENUM_MATCH_ERR} & Indicates that the no branch of the enum matched.\\[1ex]
\end{tabular}
\end{center}
\subsubsection{Accumulator functions}
Accumulator functions for \Penum{}s are described in \chapref{chap:accumulators}.

\subsubsection{Histogram functions}
Histogram functions for \Penum{}s are described in
\chapref{chap:histogram}. 

\subsubsection{Clustering functions}
Clustering functions for \Penum{}s are described in
\chapref{chap:cluster}. 

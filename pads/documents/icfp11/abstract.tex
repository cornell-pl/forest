A {\em filestore} is a structured collection of data files housed
in a conventional hierarchical file system.  Many applications
use filestores as a poor-man's data base and correct execution of 
these applications typically requires that filestore structure 
and attributes satisfy a variety of invariants.  Unfortunately, while
creating filestores is expedient, maintaining their invariants is 
usually tedious and error-prone.  Unfortunately, current programming 
languages provide little support for documenting assumptions about 
file stores, detecting errors, or safely loading from or storing to 
them.

This paper describes the design, implementation and semantics
of \forest{}, a new domain-specific language
embedded in \haskell{} for documenting and managing filestores.
The core of the \forest{} design is a declarative filestore
description language.  This description language uses a type-based
metaphore to specify the structure, attributes and invariants
required of filestores.  \forest{} descriptions
have first-class status with \haskell{} and programmers
may refer to them directly to obtain a variety of
useful utilities for reliably connecting to data on disk.
\forest{} also exploits \haskell{}'s powerful generic programming
infrastructure to make it easy for third-party developers to build
tools that work with any \forest{} description.  We illustrate the use
of this infrastructure to build a number of powerful tools, including a
file system
visualizer, a file access checker, and a
number of description-directed variants of
standard shell tools.  Finally, core \forest{} has a precise,
formal semantics inspired by classical tree logics.  We use the semantics
to prove a variety of round-tripping laws for \forest{}'s loading
and storing infrastructure.

\newif\ifdraft\drafttrue
\newif\ifcolor\colortrue

% For per-person control of tex'ing, put commands like \twocolfalse
% in a file called texdirectives.tex, which we read at this point (if
% it exists).  Note that this file should be left out of the SVN
% repository. 
\makeatletter \@input{texdirectives} \makeatother

\documentclass[nocopyrightspace]{sigplanconf}

\usepackage{alltt}
\usepackage{balance}
\usepackage{amsmath}
\usepackage{amsthm}
\usepackage{amssymb}
\usepackage{code}
\usepackage{color}
\usepackage{tikz}
\usepackage[normalem]{ulem}
\usepackage{url}
\usepackage{verbatim}
\usepackage{pervasives}

\newcommand{\cut}[1]{}
\newcommand{\note}[1]{\textbf{Note:}#1}

\newcommand{\appref}[1]{Appendix~\ref{#1}}
\newcommand{\secref}[1]{Section~\ref{#1}}
\newcommand{\tblref}[1]{Table~\ref{#1}}
\newcommand{\figref}[1]{Figure~\ref{#1}}
\newcommand{\listingref}[1]{Listing~\ref{#1}}
\newcommand{\bftt}[1]{{\ttfamily\bfseries{}#1}}
\newcommand{\kw}[1]{\bftt{#1}}
\newcommand{\eg}{{\em e.g.}}
\newcommand{\cf}{{\em cf.}}
\newcommand{\ie}{{\em i.e.}}
\newcommand{\etc}{{\em etc.\/}}

\newcommand{\hscomment}[1]{{\color{red}\{- #1 -\}}}


\title{Transactional Forest}

\authorinfo{Submission \#274}{}{\vspace*{-4cm}}

\begin{document}

\maketitle

\begin{abstract}
Many applications rely on the file system to store persistent data,
but current programming languages lack convenient constructs for
manipulating file system data. Previous work on the Forest language
developed a type-based abstraction for file systems in which the
programmer writes a high-level specification describing the expected
structure of the file system, and the compiler generates an in-memory
representation for the data and accompanying ``load'' and ``store''
functions. Unfortunately Forest does not provide any consistency
guarantees so if multiple applications are manipulating the file
system concurrently---by far the common case---it can produce
incorrect results.

This paper presents Transactional Forest: an extension to Forest that
enriches the language with seralizable transactions. We present the
design of the language, which is based on a new ``atomic'' construct
and a monad that tracks effects. We formalize the semantics of POSIX
file systems in a simple core calculus and prove the correctness of
our implementation. We discuss our implementation in Haskell and
illustrate its use on a substantial case study: the Soil and Water
Assessment Tool (SWAT), which is a modeling tool used by numerous
hydrologists and environmental scientists.
\end{abstract}

\section{Introduction}
\label{sec:introduction}

%
% File systems are a thing
%
Many applications today use file systems to store persistent
data. There are numerous reasons that programmers choose to use file
systems instead of systems specifically designed for managing data,
such as key-value stores or relational databases. File systems are
ubiquitous and have a low barrier to entry, being bundled with all
major operating systems. Programs can manipulate the file system
directly, through standard APIs such as POSIX, and do not have to
first perform tasks such as setting up database accounts, creating
tables, defining schemeas, or loading data into the system. File
system data is portable across operating systems and can be easily
replicated and transferred between them, since it is not ``locked in''
to a custom database representation. For this reason, a large number
of data-intensive users such as scientists have come to rely on file
systems for storing their data.

%
% But they kind of suck
%
At the same time, file systems have several serious limitations that
create practical hurdles in many programs. First, many applications
divide their data across multiple files and directories on the file
system. Having to write explicit code to open, read, and write these
files is tedious for programmers and complicates the logic of many
applications. For example, consider a system that stores web server
logging data in files whose name records the year and month when the
data was recorded and whose contents records the actual requests
received by the web server. Even a simple task such as tabulating up
the total number of requests in a given date range will require
opening and reading a large number of files. Second, APIs such as
POSIX lack constructs for documenting assumptions about the structure
of the file system. Applications whose correctness depends on specific
sets of files and directories being organized in a particular way
essentially have no way to declare and enforce those constraints
(except by writing code to manually traverse the file system and
verify the presence of certain files). Simple errors such as a
misnamed directory or a missing file can easily lead to
application-level errors, but can be difficult diagnose. Third, few
file systems offer constructs for ensuring consistency in the presence
of multiple concurrent users. Although POSIX file locks can be used to
control access to specific files, these primitives are complicated to
use and, like any pessimistic scheme, reduce the degree of concurrency
that can be achieved by the system. 

%
% Forest 1.0
%
Previous work on Forest~\cite{forest-icfp:fisher+} developed
type-based abstractions for file systems. The programmer writes a
high-level specification that describes the expected structure of the
file system, and the Forest compiler automatically generates an
in-memory representation for the data as well as accompanying ``load''
and ``store'' functions. In addition, the framework provides generic
tools for visualizing, summarizing, and validating file system data
described in Forest. Forest solves the first two issues discussed
above: it streamlines applications, since they can be written against
high-level datatypes rather than having to use low-level file system
APIs; and it also provides mechanisms for automatically detecting
situations when assumptions about the structure of the file system
have been violated. However, Forest's ``load'' and ``store'' functions
are strictly best effort and do not provide any guarantees about
consistency. When there are multiple concurrent users of the file
system, low-level file system operations may be interleaved
arbitrarily, leading to inconsistent results.

%
% SWAT
%
As an example to illutrate the ways that consistency failures can lead
to incorrect results, consider the following real-world
application. The widely used Soil and Water Assessment Tool (SWAT)~\cite{SWAT} 
models the impact of various land management practices in
large watersheds. An environmental scientist might use SWAT to predict
the effects on local rivers and streams that would result from
changing crops and fertilizers or
replacing farmland with a suburban development. SWAT represents
data about the watershed in a structured directory with a large
collection of files, each recorded in a master ``index'' file and
stored with a specific name and extension in the directory and with a
specific, although varying structure. An attractive feature of SWAT is
that it includes tools and interfaces for connecting to other datasets
and frameworks for modeling watershed dynamics, which allows
scientists calibrate their results and crosscheck predictions.

%
% SWAT Workflow
%
In a typical workflow using SWAT, a scientist might wish to modify
some of the parameters stored in certain data files within certain
parameters, until the overall model is calibrated to a given
theoretical model or external dataset. Another common workflow is
finding the value of a set of input parameters (e.g., land use) to
optimize a set of output parameters (e.g., phosporus runoff). In both
of these cases, the scientist needs to explore the set of parameters
to optimize some value---concretely, this entails modifying the input
parameters stored in ASCII text files, running the SWAT binary to
compute derived data, and then comparing the output parameters, which
are also stored in ASCII text files. Although this is a
straightforward optimization problem, it is not generally feasible to
use techniques such as gradient descent since the function being
computed is encoded as a black-box function and may not even be
convex. Hence, typically scientists solve these problems by exhaustive
trial-and-error search.

%
% Consistency failure
%
Ideally, to improve performance, the search for optimal parameters
could be done in parallel, using multiple threads to explore the
search space. However, since the data is stored on the file system,
this is not safe in general since writes to the file system performed
by multiple threads may be interleaved in arbitrary order, which can
easily lead to incorrect results. For example, if the optimal value
terminates first but a later thread writes its parameters into the
SWAT files, then the computation would halt with sub-optimal
parameters.

%
% Transactions to the rescue!
%
It is not hard to see that the root cause of this problem is Forest's
``best effort'' approach to loading and storing data on the file
system. This paper develops a new version of Forest, Transactional
Forest (TxForest) that remedies this problem by providing a more
powerful abstractions that offer strong consistency
guarantees. TxForest provides an ``atomic'' construct and a new
monadic API for accessing data on the file system. By using these
features, programmers gain the guarantee that all of the low-level
read and write operations needed to implement a transaction will be
executed as if they ran from start to finish in isolation, or not at
all---i.e., TxForest ensures serializability. This has all of the
usual benefits in that programmers can reason locally about the
behavior of each transaction without worrying about harmful
interference from other transactions that might be executing
concurrently.

%
% Implementation
%
There are many possible ways of implementing serializable transactions
including pessimistic schemes based on locking, optimistic schemes
based on buffering reads and writes using a log, and emerging schemes
such as coordination-free transactions. In principle, any of these
could be used as the basis for an implementation of TxForest. We have
built a full working prototype in Haskell based on an optimistic
scheme. It leverages the power of Haskell's type system to track file
system effects in a monad, and provides constructs for executing those
effects atomically, failing if some other transaction has caused their
assumptions to become invalid.

%
% Formalization
%
To establish the correctness of our implementation, we formalized the
subset of POSIX used by the TxForest compiler and run-time system in a
simple calculus and proved that it guarantees serializability.
Although this calculus only models a subset of the overall POSIX API,
it captures a number of its essential features. Hence, we hope that it
might be useful for other work on file system based abstractions.

%
% Evaluation
%
Through a collaboration with environmental scientists, we have built a
TxForest description that captures SWAT data files and used it to
automate the task of searching for optimal parameters, as described
above. This application demonstrates that even our unoptimized version
of TxForest is a useful tool that can be used to solve practical
real-world problems. 

%
% Contributions
%
Overall, the contributions of this paper are as follows:
\begin{itemize}
\item We make the case for developing language-based file system
  abstractions with strong consistency guarantees.
\item We describe the design of TxForest, a particular language that
  realizes these goals in a Haskell DSL.
\item We formalize the key elements of TxForest and POSIX in a core
  calculus and prove that our implementation provides strong
  consistency.
\item We present a prototype implementation of TxForest, and discuss
  using it to build a real-world application for representing and
  optimizing SWAT data. 
\end{itemize}
%
The rest of this paper is structured as follows. The next section
reviews Forest and provides further motivation for our
design. Section~\ref{sec:txforest} presents our design for
TxForest. Section~\ref{sec:formalization} develops a formal model of
POSIX and establishes the correctness of a reference implementation
based on optimistic transactions. Section~\ref{sec:implementation}
discusses our implementation. Section~\ref{sec:swat} describes our
experience building a SWAT application in TxForest. We discuss related
work in Section~\ref{sec:related} and conclude in
Section~\ref{sec:conclusion}.


\begin{itemize}
\item Background
\begin{itemize}
\item Forest Language (primitives, load, store)
\item Running example 
\item Problems with naive semantics
\item Transactions to the rescue
\end{itemize}
\item TxForest Language
\begin{itemize}
\item Atomic construct 
\item Transaction monad
\item Varieties of failure
\item Revised running example
\item Guarantees
\end{itemize}
\item Featherweight POSIX
\begin{itemize}
\item Discussion of Sewell-eque formalism vs. core calculi  
\item Showcase subtlety (e.g., weak locking primitives?)
\item Define translation from TxForest to IMPOSIX (defines TxForest's semantics)
\item Reference implementations (data structure locks, lockf, etc.)
\item Prove serializability for fully Forested programs
\end{itemize}
\item Implementation
\item SWAT Case Study
\item Related Work
\item Conclusion
\end{itemize}

\section{Background}
\label{sec:Background}

%%\item Background
%%\begin{itemize}
%%\item Forest Language (primitives, load, store)
%%\item Running example 
%%\item Problems with naive semantics
%%\item Transactions to the rescue
%%\end{itemize}

To explain the design of TxForest, we use a running example drawn from
the field of environmental science, specifically SWAT.  This modeling
tool allows researchers to explore tradeoffs related to land use in a
given watershed.  The model incorporates data related to the topology
of the watershed, current land use, historic precipitation and
temperature levels, measurements of water purity at various locations,
\textit{etc}.  This data is stored in a large collection of files and
directories in the file system.  A typical SWAT installation, which is
specific to a given watershed, contains thousands of files.  The data
and required formats are documented in a 654~page manual available on
the web~\cite{SWAT-IO-Documentation}.

The Forest language was designed to describe filestores such as SWAT.
The language includes primitives for describing files, directories,
symbolic links, and associated meta-data such as names, owners,
groups, permissions, sizes, and timestamps.  File contents may be
represented as unstructured strings in a variety of encodings or as
structured data using Pads
descriptions~\cite{fisher+:pads,fisher-walker:icdt}. Forest is
implemented using Haskell's quasi-quotation
mechanism~\cite{Mainland:quasi}.

\begin{figure}
\begin{code}
[txforest|
  \kw{data} Preamble\_f  = File Preamble
  data BSN\_f       = File SwatFile
  data PCP\_f       = File PCP
  data TMP\_f       = File TMP

  data Flow\_f      = File Flow                  
  data RCH\_f       = File RCH                   
  data Deviation\_f = File Deviation     
  ...                                        
  data Swat\_d = Directory 
     \{ cio \kw{is} "file.cio" :: Preamble\_f      
     , basin is <| getBasinPath cio |> :: BSN\_f          
     , pcps  is [f :: PCP\_f | f <- matches (GL "*.pcp") ] 
     , tmps  is [f :: TMP\_f | f <- matches (GL "*.tmp") ] 
     , ...
     \}
|]
\end{code}
\caption{(Partial) TxForest SWAT description. }
\label{fig:SWAT-description}
\end{figure}


\figref{fig:SWAT-description}
shows a (partial) TxForest description of the SWAT filestore that we use
as a running example in this paper.
The TxForest declarations appear within \texttt{[txforest|...|]}
quasiquotation brackets, allowing us to embed TxForest within Haskell
code files.  The SWAT description starts by declaring a number of
different file formats by applying the \texttt{File} constructor to
a PADS format description.  (For brevity, we omit the PADS
descriptions).   For example, the TxForest file format
\texttt{Preamble\_f}, which contains SWAT configuration information,
is described by the PADS type \texttt{Preamble}.
The other TxForest file formats in the figure use PADS types to
describe the format of files storing 
physical information about the watershed basin (\texttt{BSN\_f}),
daily precipitation data (\texttt{PCP\_f}),
daily temperature data (\texttt{TMP\_f}),
measured waterflow out of the basin (\texttt{Flow\_f}),
modeled waterflow out of the basin (\texttt{RCH\_f}),
and the deviation between the measured and modeled outflows (\texttt{Deviation\_f}).

Given these file formats, the description specifies the root SWAT
directory, \texttt{Swat\_d}.  This directory includes the master
watershed file, which has internal name \texttt{cio}, on-disk name
\texttt{file.cio}, and a file format described by
\texttt{Preamble\_f}.  The next component of the directory is the
\texttt{basin} field, whose format is described by the \texttt{BSN\_f}
file type.  The on-disk name of the \texttt{basin} component is given
in the \texttt{cio} file.  The Haskell code \texttt{getBasinPath cio},
included in TxForest using the anti-quotation notation
\texttt{<|...|>} extracts this path information from the configuration
file.  The next component in the SWAT directory is internally named
\texttt{pcps}; this component stores daily precipitation information.
It is stored on disk as a collection of files each of which has the
\texttt{PCP\_f} file format and the \texttt{.pcp} file extension.  The
SWAT description uses TxForest's file comprehension and Glob matching
facilities to bind the \texttt{pcp} internal name to the list of files
matching this specification.  Similarly, the \texttt{tmps} component
specifies the collection of files that contain daily temperature
information.

Given such a description, the TxForest compiler generates Haskell data
structures for storing in-memory representations of the on-disk data
and associated metadata, including any discrepancies detected between
the TxForest description and the actual filestore.  TxForest also
generates functions for (lazily) loading the contents of the filestore
into the generated Haskell data structures and for writing the
information in these Haskell data structures back to the datastore in
the required formats.

\cut{Writing structures to disk is a two-step
process.  In step one, a \textit{manifest} function writes the
structure into a temporary space and notes any errors.  In step two, a
\textit{store} function copies the temporary store into the correct
location.  This two-step process allows TxForest to detect errors
without corrupting the mainline file store and lets users determine
whether the errors should halt the writing process.}

The original Forest implementation did not prevent concurrent loads
and stores from causing inconsistencies or corruption.  As the number
of users manipulating a filestore grows, such a laissez-faire approach
becomes untenable.  In this paper, we address this weakness by
integrating transactions into Forest.  Transactional Forest 
ensures that all accesses to the filestore mediated by a Forest
description will see and maintain a consistent view by aborting and
restarting transactions that would otherwise have created a
conflict. The design of the description language is unchanged: the
TxForest description in \figref{fig:SWAT-description} is also a valid
Forest description.  The difference between TxForest and Forest lies
in the generated code for loading and storing files and in how the
user accesses the in-memory representations.

The process that SWAT users follow to calibrate their models
illustrates the utility of transactions.  The goal of calibration is
to tune various model parameters stored in the filestore so that the
outflow predicated by SWAT matches the measured outflow as closely as
possible.  Because SWAT is a black-box executable developed over
decades, researchers perform this calibration by searching through the
parameter space: they set the parameters in the filestore, run SWAT,
calculate the deviation between the modeled and predicted value, and
repeat until they find a set of parameters with an acceptable
deviation.  Using TxForest, we are able to parallelize this search.
Each thread atomically copies the SWAT datastore to thread-local
space, sets the parameters, invokes SWAT on the thread-local
filestore, checks if the calculuated deviation is the best found by
any thread so far, and if so, atomically writes the new parameter
values back into the SWAT filestore.


\cut{
An example query that researchers using this tool might ask is ``what
type of land use assignment to a given area of a watershed keeps corn
yield above a threshold, maintains housing capacity above another
threshold, and minimizes nitrate levels in nearby streams.''  The SWAT
approach to solving such queries involves a concurrent black-box
optimization process in which each thread reads the current values of
all relevant parameters from the file system, computes the current
value of the optimization function, and makes local changes, and
re-runs the optimization function. If the new result is higher than
the old one, the tool writes those changes back into the file system.
Figure ~\ref{fig:SWAT-opt-code} shows the key TxForest code that
replicates this process.
}

\begin{figure}
\begin{code}
iterateSWAT :: Path -> IO ()
iterateSWAT workingDir = do
  atomically (copySWATData workingDir)
  newBasin <- modifyBasinParams workingDir
  runSWAT workingDir
  newDeviation <- getNewDeviation workingDir
  atomically (updateDeviation newBasin newDeviation)
\end{code}
\caption{TxForest code to determine if new model parameters reduce the
deviation between modeled and measured outflows.}
\label{fig:SWAT-opt-code}
\end{figure}






\section{TxForest Language}
\label{sec:txforest}
% Atomic construct 
% Transaction monad
% Varieties of failure
% Revised running example
% Guarantees 

In order to facilitate the construction of general transactions, TxForest programmers are able to use transactional constructs to manipulate the file system, side-by-side with the rich computations over ordinary data structures offered by its host language.
This coupling of transactional and pure functional code is elegantly supported by the type system.

\paragraph{Transactions}
In Haskell, I/O actions with irrevocable side-effects, such as reading/writing to files or managing threads, are typed as operations in the primitive \cd{IO} monad.
Akin to \emph{software transactional memory}~\cite{HaskellSTM}, forest memory transactions perform tentative file store operations that can be rolled  back at any time. Therefore, they live within an explicitly different \cd{FTM} forest transactional monad.
One can execute a TxForest transaction atomically with respect to other concurrent transactions by placing it inside an \cd{atomic} function with type:
\begin{code}
atomic :: FTM a -> IO a
\end{code}
As a bonus, the type-level distinction between monads prevents non-transactional actions from being run inside a transaction.

For the Haskell aficionados, the \cd{FTM} monad is also an instance of \cd{MonadPlus} (blocking and choice), \cd{MonadThrow} and \cd{MonadCatch} (throwing and catching user-defined exceptions).

\paragraph{Transactional variables}
TxForest interacts with the file system by means of shared \emph{forest transactional variables}. Variable types are declared with the \cd{var} keyword within the TxForest sublanguage. For each variable type declaration, the TxForest compiler generates an instance of the \cd{TxForest} type class:
\begin{code}
class TxForest args ty rep | ty -> rep, ty -> args where
  new         :: args -> FilePath -> FTM ty
  read        :: ty -> FTM rep
  writeOrElse :: ty -> rep -> b
              -> (Manifest -> FTM b) -> FTM b
\end{code}
In the above signature, a variable of TxForest type \cd{ty} has a Haskell data representation of type \cd{rep}. A \cd{new} variable can be declared with argument data consistent with its forest type and rooted at the argument file path. A \cd{read} (lazily) loads the corresponding slice of the file system into memory and a \cd{writeOrElse} attempts to store a Haskell data structure on disk.
Following this special interface, a transaction is able to log all file store effects.

\paragraph{Errors}
Since TxForest descriptions define richer structured views of file stores, specific classes of \emph{forest errors} become evident to programmers, who can respond in application-specific ways.
One such example is the tentative nature of \cd{writeOrElse}. Forest dependent types may impose certain data dependencies on the underlying Haskell representations that can not be statically checked by the type system. For example, all the site files listed in a SWAT log of Figure~\ref{fig:SWAT-description} must have names matching the \cd{time} pattern. (Precisely, only values that could be read from a file store are deemed valid representations.) If these dependencies are not met, the write is aborted with a \emph{manifest error}, and a user-supplied alternate procedure is executed instead.

Nevertheless, a file store does not need to conform perfectly to its associated TxForest description.
Instead, TxForest (lazily) computes a summary of \emph{validation errors}. These may flag, for instance, that a required file can not be found or that an arbitrarily complex user-specified TxForest constraint is not satisfied.
At any point, a programmer can explicitly demand the validation of the whole file store bound to a transactional variable by calling:
\begin{code}
validate :: TxForest args ty rep => ty -> FTM ForestErr
\end{code}

%illustrate using the example code:
%new/read/write
%swat properties as (read-only) embedded monadic expressions

\paragraph{Guarantees}
TxForest is designed to the obey the following principles:
\begin{itemize} 
	\item Transactions are serializable. Successful transactions are guaranteed to run in serial order and failing transactions roll back and retry again;
	\item Transactional operations are transparent, as if they were performed on the file system. All the transactional variables are kept consistent with the same file system;
	\item Transactional variables are lazy. The content of a variable is only loaded from the file system when explicitly read or (recursively) validated;
	\item Transactional reads and writes preserve data on round-trips. Reading a variable and immediately writing it back always succeeds and keeps the file system unchanged; and writing succeeds as long as reading the resulting file system yields the same in-memory representation.
\end{itemize}






\section{Featherweight Posix}
\label{sec:posix}

We present a core calculus for working with POSIX, which we call IMPOSIX.
We start by discussing some of the reasons why one might want a core calculus
rather than a more advanced, Sewell-esque formalism followed by some
subtleties arising from using POSIX as the underlying filesystem model for 
Forest (\ref{subsec:posix-discussion}).
We then describe the semantics of our core calculus and some of the differences
from standard POSIX (\ref{subsec:posix-semantics}).
In \ref{subsec:posix-translation} we define a translation from TxForest to IMPOSIX 
before finally proving serializability in fully Forested programs (\ref{subsec:posix-proof}).

\subsection{Discussion}
\label{subsec:posix-discussion}

The motivation for using a core calculus rather than a Sewell-esque formalism
is largely simplicity. Sewell-esque formalism are fantastic for large
real-world proofs of complicated systems, but in some situations you don't
need the level of detail or even accuracy that they offer and the simplicity
of dealing with just a core calculus of POSIX is beneficial.

In choosing POSIX as the filesystem model underlying Forest, we
believe we have the large benefit of being immediately applicable to
real-world systems since many filesystems do use POSIX as the core
model. A toy filesystem may have offered more power, but would also have a much
higher barrier to adoption, which goes against one of Forest's core creeds.

However, this also presents a number of subtle problems. For example,
POSIX solely offers advisory file locking (as opposed to mandatory).
There are a variety of arguably good reasons why they choose to do this,
but the effect on Transactional Forest is that we can't offer
transactionality with respect to arbitrary processes making changes on the
filesystem. We could be transactional w.r.t. those adhering to advisory
file locks and we are inherently transactional within Forest threads.

\subsection{Semantics}
\label{subsec:posix-semantics}

In our core calculus, we include the POSIX operations,
open, close, read, readdir, write, remove, test, and lockf.
In most cases these work similarly to POSIX, but with some simplifications,
particularly in regards to errors. For example open cannot fail and
read can only get one type of error whether it fails due to the argument
not being a file descriptor or the file descriptor not pointing to a file.
lockf works on whole files instead of pieces of them and simply allow 
locking and unlocking while test is slightly more powerful in that it
can not only tell you if a path has a file, a directory, or nothing,
but also if it is a directory, whether or not it is empty. 
This is largely done for simplicity, but in practice one could
simply check by opening and trying to read a directory instead. The
exact semantics are described in Figure~\ref{fig:posix-semantics}
along with a standard IMP-like construction for command and
expression evaluation.

\begin{figure}
\caption{POSIX Semantics goes here}
\label{fig:posix-semantics}
\end{figure}

\subsection{Translation to IMPOSIX}
\label{subsec:posix-translation}



\subsection{Proof of Serializability}
\label{subsec:posix-proof}

We define a compilation function, which runs on the IMPOSIX translation
and turns it into our transactional code.



Now we move on to proving that compiled atomic statements exhibit mutual
serializability.

\begin{comment}

\begin{figure*}
\begin{minipage}{.5\linewidth}
\begin{displaymath}
\begin{array}{l@{\quad}l@{\,}c@{\,}ll@{}}
 \textrm{Switch} & \S & ::= & \mkS(\sw,\pts,\SF,\inp,\outp,\inm,\outm)  \\
 \textrm{Controller} & \C & ::= & \mkC(\Cst,\Cin,\Cout) \\
 \textrm{Link} & \Lt & ::= & 
  \mkL((\sw_{\mathit{src}},\pt_{\mathit{src}}),\pks,
       (\sw_{\mathit{dst}},\pt_{\mathit{dst}})) \\
 \textrm{Link to Controller} & \Mt & ::= & \mkM(\sw,\CSL,\SCL) \\
\end{array}
\end{displaymath}
\centerline{\textbf{Devices}}
\end{minipage}\begin{minipage}{.5\linewidth}
\begin{displaymath}
\begin{array}{l@{\quad}l@{\,}c@{\,}ll@{}}
 \textrm{Ports on switch} & \pts & \in & \{\pt\} \\
 \textrm{Input/output buffers} & \inp,\outp & \in & 
  \multiset{(\pt,\pk)} \\
 \textrm{Messages from controller} & 
  \inm & \in & \multiset{\CS} \\
 \textrm{Messages to controller} & 
  \outm & \in & \multiset{\SC} \\
\end{array}
\end{displaymath}
\centerline{\textbf{Switch Components}}
\end{minipage}

\end{figure*}
\begin{comment}

\begin{minipage}{.5\linewidth}
\begin{displaymath}
\begin{array}{l@{\quad}l@{\,}c@{\,}ll@{}}
 \textrm{Controller state} & \Cst & & \\
 \textrm{Controller input relation} & \Cin & \in 
   \sw \times \SC \times \Cst \crel \Cst \\
 \textrm{Controller output relation} & \Cout & \in 
   \Cst \crel \sw \times \CS \times \Cst \\
\end{array}
\end{displaymath}
\centerline{\textbf{Controller Components}}
\end{minipage}\begin{minipage}{.5\linewidth}
\begin{displaymath}
\begin{array}{@{}l@{\quad}l@{\,}c@{\,}l@{\,}l@{}}
& \textrm{Message queue from controller} & 
  \CSL & \in & \queue{\CS_1\cdots\CS_n} \\
& \textrm{Message queue to controller} & 
  \SCL & \in & \queue{\SC_1\cdots\SC_n} \\
\end{array}
\end{displaymath}
\centerline{\textbf{Controller Link}}
\end{minipage}

%\begin{minipage}{.5\linewidth}
\begin{displaymath}
\begin{array}{l@{\quad}l@{\,}c@{\,}ll@{}}
 \textrm{From controller} & 
  \textrm{\CS} & ::= & \FlowMod{\SFMod} \mid \PktOut{\pt~\pk} \mid \BarrierRequest{n}\\
 \textrm{To controller} &
  \textrm{\SC} & ::= & \PktIn{\pt~\pk} \mid \BarrierReply{n} \\
 \textrm{Table update} & \SFMod & ::= & \addflow{\prio}{\Patt}{\Action} 
            \alt \delflow{\Patt}
\end{array}
\end{displaymath}
\centerline{\textbf{Abstract OpenFlow Protocol}}
%\end{minipage}

\infrule[Fwd]
{\SFint{\SF}(\pt,\pk) \crel 
 (\multiset{\pt'_1\cdots \pt'_n}, \multiset{\pk'_1\cdots \pk'_m})}
{\begin{array}{ll}
 & \mkS(\sw,\pts,\SF,\multiset{(\pt,\pk)} \uplus \inp,\outp,\inm,\outm) \\
 \obsstep{(\sw,\pt,\pk)} &
 \mkS(\sw,\pts,\SF,\inp,\multiset{(\pt'_1,\pk)\cdots (\pt'_n,\pk)} \uplus \outp,
  \inm, \multiset{\PktIn~\pt~\pk'_1 \cdots \PktIn~\pt~\pk'_m} \uplus\outm)
 \end{array}}

\squeezev\squeezev
\infrule[Wire-Send]
{}
{\begin{array}{ll}
 & \mkS(\sw,\pts,\SF,\inp,\multiset{(\pt,\pk)} \uplus \outp,\inm,\outm) 
    \parcomp
   \mkL((\sw,\pt),\pks,(\sw',\pt')) \\
 \taustep &
   \mkS(\sw,\pts,\SF,\inp,\outp,\inm,\outm) 
    \parcomp
   \mkL((\sw,\pt),\queue{\pk}\app\pks,(\sw',\pt'))
  \end{array}}
 
\squeezev\squeezev
\infrule[Wire-Recv]
{}
{\begin{array}{ll}
 & \mkL((\sw',\pt'),\pks\app\queue{\pk},(\sw,\pt)) \parcomp
   \mkS(\sw,\pts,\SF,\inp,\outp,\inm,\outm)
   \\
 \taustep &
   \mkL((\sw',\pt'),\pks,(\sw,\pt)) \parcomp
   \mkS(\sw,\pts,\SF,\multiset{(\pt,\pk)} \uplus \inp,\outp,\inm,\outm)
\end{array}}

\squeezev\squeezev
\infrule[Add]
{\squeezeh}
{\squeezeh\mkS(\sw,\pts,\SF,\inp,\outp,\multiset{\!\FlowMod{\addflow{m}{\Patt}{\Action}}\!}\uplus\inm,\outm)
 \taustep
 \mkS(\sw,\pts,\SF \uplus \multiset{(m,\Patt,\Action)},\inp,\outp,\inm,\outm)}

\squeezev\squeezev
\infrule[Del]
{\SF_{rem} = \multiset{(\prio',\Patt',\Action') \mid 
            \textrm{$(\prio',\Patt',\Action') \in \SF$ and $\Patt \ne \Patt'$}}}
{\begin{array}{ll}
& \mkS(\sw,\pts,\SF,\inp,\outp,\multiset{\FlowMod{\delflow{\Patt}}}\uplus\inm,
      \outm) 
 \taustep 
 \mkS(\sw,\pts,
      \SF_{rem},
      \inp,\outp,\inm,\outm)
\end{array}}

\squeezev\squeezev
\infrule[PktOut]
{\pt \in \pts}
{\mkS(\sw,\pts,\SF,\inp,\outp,\multiset{\PktOut{\pt~\pk}}\uplus\inm,\outm)
 \taustep
 \mkS(\sw,\pts,\SF,\inp,\multiset{(\pt,\pk)} \uplus \outp,\inm,\outm)}

\squeezev\squeezev
\infrule[Ctrl-Send]
{\Cout(\Cst) \crel (\sw,\CS,\Cst')}
{\mkC(\Cst,\Cin,\Cout) \parcomp 
 \mkM(\sw,\CSL,\SCL)
 \taustep
 \mkC(\Cst',\Cin,\Cout) \parcomp 
 \mkM(\sw,\queue{\CS} \app \CSL,\SCL)}

\squeezev\squeezev
\infrule[Ctrl-Recv]
{\Cin(\sw,\Cst,\SC) \crel \Cst'}
{\mkC(\Cst,\Cin,\Cout)
 \parcomp 
 \mkM(\sw,\CSL,\SCL\app\queue{\SC})
 \taustep
 \mkC(\Cst',\Cin,\Cout)
 \parcomp 
 \mkM(\sw,\CSL,\SCL)}


\squeezev\squeezev
\infrule[Switch-Recv-Ctrl]
{\CS\ne\BarrierRequest{n}}
{\begin{array}{ll}
 &
 \mkM(\sw,\CSL\app\queue{\CS},\SCL)
 \parcomp
 \mkS(\sw,\pts,\SF,\inp,\outp,\inm,\outm) \\
 \taustep &
 \mkM(\sw,\CSL,\SCL)
 \parcomp
 \mkS(\sw,\pts,\SF,\inp,\outp,\multiset{\CS}\uplus\inm,\outm)
 \end{array}}

\squeezev\squeezev
\infrule[Switch-Recv-Barrier]
{}
{\begin{array}{ll}
 &
 \mkM(\sw,\CSL\app\queue{\BarrierRequest{n}},\SCL)
 \parcomp
 \mkS(\sw,\pts,\SF,\inp,\outp,\emptymset,\outm) \\
 \taustep &
 \mkM(\sw,\CSL,\SCL)
 \parcomp
 \mkS(\sw,\pts,\SF,\inp,\outp,\emptymset,\multiset{\BarrierReply{n}}\uplus\outm)
 \end{array}}

\squeezev\squeezev
\infrule[Switch-Send-Ctrl]
{}
{\begin{array}{ll}
 & 
 \mkS(\sw,\pts,\SF,\inp,\outp,\inm,\multiset{\SC} \uplus \outm) \parcomp
 \mkM(\sw,\CSL,\SCL) \\
 \taustep &
 \mkS(\sw,\pts,\SF,\inp,\outp,\inm,\outm) \parcomp
 \mkM(\sw,\CSL,\queue{\SC}\app\SCL)
 \end{array}}

\squeezev\squeezev
\infrule[Congruence]
{\Sys_1 \taustep \Sys_1'}
{\Sys_1 \parcomp \Sys_2 \taustep \Sys_1' \parcomp \Sys_2}
\caption{Featherweight OpenFlow syntax and semantics.}
\label{fig:fwof}
\end{figure*}
\end{comment}

\cut{
\begin{itemize}
\item Compiler
  \begin{itemize}
  \item Camlp4, tcc, and fmlc (generate typedefs and prefeed defs) 
  \end{itemize}

\item Runtime system
  \begin{itemize}
  \item data structure of feed items: iData, meta data, etc
  \item implementation of feed/stream: lazy list
  \item fetching mechanism: eager fetching vs. lazy consumption, 
    http\_client library, batch fetching
  \item parse using padsML easy lib
  \item concurrency
  \item error handling
  \item discussion of selected combinators: local pairing, 
    dependent pairing (separate thread/queue)
  \end{itemize}

\item Tools library
  \begin{itemize}
  \item use of generic tool framework and feeds runtime lib
  \item use of several external ocaml libs: rrdtools, xml\_light
  \end{itemize}

\item Future work (shall we include???)
  \begin{itemize}
  \item expose meta data to the surface language
  \item a second (simplified) prefeed def with type defs only
  \end{itemize}

\item Experiments
  \begin{itemize}
  \item performance metrics: throughput, network/system latency
  \item setup (mac powerbook g4, 100Mb ethernet connection, 
    comon spec, comon nodes, random selection of nodes)
  \item two tables and graphs: throughput peaks at 
    200 nodes (chunk size), sys latency almost constant,
    system is scalable to comon (842 nodes)
  \end{itemize}
\end{itemize}
}

The \padsd{} implementation has three parts: the compiler, the runtime
system, and the built-in tools library. We describe these
parts in turn and then evaluate the overall system performance and design.

%%In this section, we describe
%%these parts and evaluate the performance of the system. We conclude
%%with a discussion of our choice to design a language
%%extension to \ocaml{} as opposed to providing a library.
%We will show that the system can easily scale
%to support PlanetLab-sized applications with 
%hundreds of nodes. 

\paragraph*{The Compiler.}
The \padsd{} compiler consists of
\cd{tcc}, the tool configuration compiler for .tc files, 
and 
\cd{fmlc}, the compiler for feed declarations (.fml files). 
Both compilers convert their sources into \ocaml{} code, which is then
compiled
and linked to the runtime libraries.  We implemented both tools with
\camlp{}, the \ocaml{} preprocessor. 

% \begin{figure}[t]
% \centering
% \begin{codebox}
% let simple_comon =
% \{\kw{frep} = fun ff ->
%  ff.\kw{all}
%  \{Combinators.\kw{format} = Comon_format.Source.parse;
%   \kw{print} = Comon_format.Source.print;
%   \kw{format_rep} = Comon_format.Source.tyrep; 
%   \kw{incremental} = false;
%   \kw{header_format} = None; 
%   \kw{locations} = sites;
%   \kw{schedule} =
%     Schedule.{\kw every} (Time.now (), 10., 
%                     Schedule.default_duration, 60.);
%   \kw{has_records} = Comon_format.__PML__has_records; 
%   \kw{pp} = None\}\}
% \end{codebox}
% \caption{Code fragment of compiled simple\_comon feed}\label{fig:compiledcomon}
% \end{figure}

%A source program is parsed into an abstract
%syntax tree defined by the Camlp4 extended syntax, and the
%code generation is done through the quotation system. 
The \cd{fmlc} compiler performs code generation in two steps.
First, the code generator emits the
type declarations for each feed.  
%% FIXME: is what is meant by representations clear here? (ksf)
Second, it generates representations for each feed description.  
The compiler constructs these representations
by extracting elements from the concurrently
generated \padsml{} libraries
and using polymorphic combinators to build structured 
descriptions.  
%Figure \ref{fig:compiledcomon} shows a fragment of
%the compiled code for
%the simple CoMon feed in Figure \ref{fig:simplecomon}.
%%While a programmer could use our combinator library directly, the
%%surface syntax provides a simple veneer that reduces the barrier to
%%entry substantially.
% in a
%lazy fashion (that is, only generate the declaration if the
%feed is used in the rest of the description).
%In the second step, 
%,
%also known as the ``prefeed''. Schedules which are
%definite times such as ``2008/09/30:12:00:00'' or ``5 mins'' are
%converted to floating point number of seconds at compile time
%to make the code more efficient.
%All O'Caml expressions embedded in the fml file are
%included without change in the compiled code. 

\paragraph*{The Runtime System.}
We implement each \padsd{} feed as a lazy list of feed items. 
Following the semantics in \secref{sec:semantics}, 
a feed item is a (meta-data, payload) pair, 
although the implementation has a more refined notion of meta-data
that includes more detailed error information.
% for HTTP errors, late item arrivals, parse errors, \etc{}

% Late arrival, (3)
% %{{\small{ 
% \begin{verbatim}
% 1: Misc HTTP error
% 2: Late arrival
% 3: Ssh host required
% 4: Remote command required for ssh
% 5: Bad message
% \end{verbatim}
% %}}} \normalsize

% % This means the feed 
% % is actually a function {\tt next}. It is evaluated 
% % only when the user program attempts to take an
% % item from the feed. The function {\tt next} returns an
% % item plus a new {\tt next} function which represents
% % the tail of the feed.
% %a data item of polymorphic type 'a and 
% %a meta data structure that corresponds 'a. 
% %The type of a feed is also the type of its data item. 
% %The type of a base feed has an option type.
% The meta data is a tree structure in which
% each node is tagged with a meta header. 
% The tree structure (also known as the meta body) resembles 
% the structure of the data item, i.e.  if the data item is a pair, 
% then the meta body is also a pair of meta data. Each leaf
% of the meta body corresponds to a base feed. Meta information
% such as the scheduled time, arrival time, location and 
% errors is stored in the leaves. The header records the summary 
% of meta information within the subtree. 

The \padsd{} runtime system is a multi-threaded concurrent
system that follows the  master-worker implementation strategy. 
%Each base feed is created and maintained by a separate 
%worker thread, and a master thread drives the combination of 
%base feeds into compound feeds.  
Each worker thread either fetches data from a specified
location and parses the data into an internal representation (the {\em rep}),
%(known as the {\em rep})
or synthesizes its data by calling a
generator function.  Using error conditions, location, scheduled time
and arrival time, the worker generates the appropriate meta-data,
pairs it with the rep and pushes the feed item onto a queue. 
%And then it generates the idata from the rep and the meta data,
%and pushes the idata into the concurrent queue.
%The workers communicated with
%the master through a concurrent queue.
The master thread pops the feed item from the queue on demand, \ie{},
when the user program requests the data. 
The worker thread is {\em eager}, which guarantees that all 
data will be fetched and archived, but the master thread is 
{\em lazy}, which allows application programs to process only relevant
data. 

%The concurrency control is implemented by the O'Caml threads
%library using standard mutex and condition variables.

We used the \ocamlnet{} library~\cite{ocamlnet2} to implement
the fetching engine. It batches concurrent fetch requests into groups
of 200, a size which balances maximizing throughput with avoiding
overwhelming the operating system with too many open sockets.


%Given a list of locations to fetch from at any one time, 
%the system converts the list into batches and fetches up to 200 locations
%per batch. The choice of 200 is a trade-off between maximizing
%the throughput and avoiding overwhelming the operating system
%with too many open sockets. 
%If the system fails to fetch an
%item due to network or system error, a feed item is still created
%with the appropriate error code written in the meta-data. 

% The \padsd{} system does not 
% automatically filter out erroneous items because errors often 
% provides very important information to users who monitors 
% distributed systems. Users could optional choose to filter
% out bad feed items using the Feed library functions.

% Compound feeds are created within the generic tool framework by
% combining base feeds using various combinator functions.
% These functions takes idatas from two feeds and creates a new idata often
% by comparing the timestamps in the two idatas. While this is done
% lazily in most of the combinators, it is not the case for
% dependent pairs. This is because the dependent feed needs to be
% created {\em eagerly} when each item from the {\em depending}
% feed arrives. Therefore we add another layer of ``pseudo-fetching"
% for the dependent feed. Here a separate thread is created for each
% dependent feed, which actively takes data from the depending
% feed, creates dependent items, and push to another concurrent
% queue of its own.  
 
\paragraph*{Tools Library.}
As explained in Section~\ref{sec:programming}, we implemented the
\padsd{} off-the-shelf tool suite using our generic tool framework. 
%Many of the tools rely upon format-specific libraries generated 
%from format specifications.
Some tools depend upon auxiliary tools.  
For instance, the feed selector calls a data selector built using
the \padsml{} generic tool framework \cite{padsml-padl} for base feeds.
%For instance, the
%feed selector relies upon a data selector generated by type-directed
%compilation of the \padsml{} description.
Other tools depend upon external libraries. For instance, the
\cd{feed2rrd} tool requires the RRD round-robin database~\cite{rrdtool} and
the \cd{feed2rss} tool uses the XML-Light package~\cite{xmllight} for
parsing and printing XML.
%The system also provides an Feed interface to a number of
%useful runtime functions such as map and fold for advanced
%users to program against the feeds.

\cut{
\begin{table*}[th]
\begin{center}
\begin{tabular}{|l|r|r|r|r|r|r|r|r|r|r|r|r|}\hline
Num of nodes&	50&	100&	150&	200&	250&	300&	350&	400&	450&	500&	550&	600 \\ \hline\hline
Net latency per node (secs)&	9&	4&	4&	4&	8.6&	5.3&	19.1&	19.5&	14.4&	7.8&	12&	13.3 \\ \hline
Sys latency per node (secs)&	0&	0&	0&	0.3&	0.2&	0.4&	0.3&	0.1&	0.3&	0.4&	0.2&	0.7 \\ \hline
%Total Latency (secs)&	9&	4.04&	4&	4.3&	8.8&	5.8&	19.4&	19.6&	14.7&	8.2&	12.3&	14 \\ \hline
Total fetch time (secs)&	9&	5&	4&	5&	23&	9&	22&	23&	26&	14&	27&	28 \\ \hline	
Throughput (items/sec)&	5.6&	20&	37.5&	40&	10.9&	33.3&	15.9&	17.4&	17.3&	35.7&	20.4&	21.4 \\ \hline
\end{tabular}
\end{center}
\caption{Performance of CoMon without archiving}
\label{tab:comon-noarch}
\end{table*}


\begin{table*}
\begin{center}
\begin{tabular}{|l|r|r|r|r|r|r|r|r|r|r|r|r|}\hline
Num of nodes&	50&	100&	150&	200&	250&	300&	350&	400&	450&	500&	550&	600 \\ \hline\hline
Net latency per node (secs)&	16&	4&	4&	4&	18.9&	6&	20.6&	22&	8.4&	13&	21.8&	21.3 \\ \hline
Sys latency per node (secs)&	0.8&	1.28&	1.4&	1.8&	1.9&	1.5&	1.6&	1.3&	1.9&	1.7&	1.7&	2.2 \\ \hline
%Total Latency (secs)&	16.8&	5.28&	5.4&	5.8&	20.8&	7.5&	22.2&	23.3&	10.3&	14.7&	23.56&	23.5 \\ \hline
Total fetch time (secs)&	17&	6&	7&	7&	27&	12&	27&	30&	19&	33&	43&	43 \\ \hline
Throughput (items/sec)&	2.9&	16.7&	21.4&	28.6&	9.3&	25&	13&	13.3&	23.7&	15.2&	12.8&	14 \\ \hline
\end{tabular}
\end{center}
\caption{Performance of CoMon with archiving}
\label{tab:comon-arch}
\end{table*}
}

\paragraph*{Experiments.} \label{sec:experiments}

\begin{figure}[t]
\begin{center}
\epsfig{file=throughput.eps, width=\columnwidth}
\epsfig{file=latency.eps, width=\columnwidth}
\caption{Average throughput and latencies per node}
\label{fig:throughput}
\shrink
\end{center}
\end{figure}

To assess performance, we measure 
the average time to fetch a data item (termed {\em network latency}), 
the average time to prepare the data item for consumption
after fetching it (termed {\em system latency}),
and the {\em throughput} of the system for the CoMon feed
description in \figref{fig:feedcomon}. 
The throughput measures the average
number of items fetched and processed per second. 
%We picked
%the comon example because it is a real life
%application that involves fetching from large number of 
%nodes (800+) at the same time, which can be viewed as a stress test. 

All the experiments were conducted on a Mac Powerbook G4 computer
with a 1.67GHz CPU and 2GB memory running Mac OS X 10.4.
%connected to the Internet via 11Mb/s wireless ethernet.
In each experiment, we randomly selected 16 subsets of PlanetLab
nodes, with increasing size from 50 to 800 in increments of 50.
%%In each experiment, we randomly selected 50 nodes,
%%100 nodes, up to 800 PlanetLab nodes
%\footnote{List taken from http://www.cs.princeton.edu/~vivek/node\_list\_all. 
%Nodes could come and go sporadically.} 
%%to form 16 node sets. 
For each set, we applied the profiler tool for the CoMon feed
twice, once without archiving and once with it, to measure the
throughput and latencies as the system fetched from these node lists. 
We repeated the experiment ten times and calculated the average values.

\figref{fig:throughput} shows the average throughput
and the average network and system latencies.
The throughput is maximized when fetching from 200 nodes because
the system supports up to 200 concurrent fetches.
%Generally, 
%the throughput goes up with the number of nodes
%until 200 and then starts declining and saturating
%as it approaches 800. Locally, 
%the throughput hits peaks at multiples of 200 since the system supports 
%up to 200 concurrent fetches.
%is the size of the
%fetching batch. At multiples of 200, the concurrent
%fetching is maximally utilized. 
%An anomaly occurred at 400 nodes, as a number of nodes were unreachable
%because of DNS failures.
% at the time of
%the experiment. 
Archiving adds to the overhead of the system and hence
reduces the throughput and increases network and system latencies. 
Note that while network latency increases with the number of nodes,  
system latency remains almost constant and relatively
low, showing that the \padsd{} runtime system adds
little overhead to the inevitable network fetching
cost. Despite the random network delays in these experiments,
the network latency is generally linear in the number of nodes. 
The system, which we have not tried to optimize, was able to fetch
data from 
800~nodes and archive the results in under 70~seconds, well under the
5 minute turnaround time currently supported by CoMon. 
Taken together, these results suggest that \padsd{} is
capable of supporting PlanetLab-scale monitoring. 

%Tables \ref{tab:comon-noarch} and \ref{tab:comon-arch}
%shows the results from two different scenarios:
%one in which the user program simply creates the comon feed without applying
%any tools on it, and one in which the comon feed is created
%and archived. In both scenarios, the system was
%able to fetch from up to 600 nodes within one minute, which is
%significantly shorter than the 5-minute turnaround time in the real
%CoMon system.
%\begin{figure}[th]
%\begin{center}
%\epsfig{file=latency.eps, width=\columnwidth}
%\caption{Average latencies per node}
%\label{fig:latency}
%\end{center}
%\end{figure}

\cut{
We plot the average throughput in Figure \ref{fig:throughput}
and the system and network latencies in Figure \ref{fig:latency}. 
As expected, both the throughput and latency suffer a little with 
archiving taking place. The throughput peaks when fetching from
200 nodes because 200 is the size of the fetching batch and
at 200 nodes, the concurrent fetching engine is maximally utilized.
The experiment for 450 nodes exhibits an anomaly as the archived
experiment takes less time than the un-archived experiment.
This is probably due to sudden delays in some of the nodes when the
no-archiving experiment is run. One obvious take-away from
Figure \ref{fig:latency} is that while the network latency may
vary greatly depending on the network condition, the system latency
stays almost constant at relatively low levels. This shows that
the \padsd{} system runtime adds little overhead to the 
inevitable network cost, which means the system could scale to
large applications. 
}

\paragraph*{Language or Library.}
A natural question that frequently arises for domain-specific languages
is whether the system is better
implemented as a library or as a language extension.  The strongest reason
for us to implement our system as a language extension is that
O'Caml  (and C, and SML, and, in fact, most functional and imperative languages) 
have poor support for generic, type-directed programming.  Unfortunately, 
many of
our key tools, including our parsers, printers, database loaders, selectors,
etc, are generic programs defined over the types of the feeds that our 
specifications
generate.  By defining a language extension, we are free to invoke a
compiler to assemble the code fragments comprising the needed applications
in a type-correct way.  

Now, in theory, the compiler is not 100\% essential to the generation of
our generic programs, but in practice, it is an enormous advantage to the 
average programmer.  
After spending months studying this issue, the best alternative we have devised
that does not use compiler support is to require that
programmers write their specification code inside of functors parameterized
in the appropriate way.  These functors can then be passed off to other 
functors implementing appropriate tool interfaces.  However this 
functor programming style is extremely hard to learn, to use and to 
explain.
%\footnote{Some of the smarter members of the group have attempted
%to explain it multiple times to one of the dumber members (David Walker)
%and he still has trouble understanding how it works.}  
Avoiding these
complications by creating a language-level interface seems to be a good,
practical solution to the problem.  For more insight into the precise
issues at hand, we recommend reading related work on the construction of
the \padsml{} infrastructure~\cite{padsml-padl} 
as well as
Hinze's work~\cite{hinz:icfp04} on 
generic programming.

Two secondary issues influencing our choice of language over library
are that (1) we could choose a pleasing and concise syntax for both
our feed and tool specifications and (2) this approach allows
smooth integration with \padsml{}, which itself is a successful 
language extension.  On the latter point, developing a system in which
data locality, temporal availability, format and properties are
all specified in one place and in one seemlessly integrated syntax was
an important goal.  We believe it improves the user programming
experience significantly.

\cut{%%%%%%%%%%%%%%%%%%

\subsection{Language or Library}
% reason (4) managing functors & modules?
Our feed language is a veneer on \ocaml{} built with the \camlp{}
preprocessor.   
A natural question is whether the system would be better
implemented as a library rather than a language extension.  For the reasons
described in the following paragraphs, we chose to present our work as a
language. 

\textbf{\textit{Automatic elimination of boilerplate code.}}
The compiler eliminates boilerplate code by
(a) generating both type declarations and values from 
descriptions (particularly record types and datatypes), something
that cannot be done in a library, (b) packaging definitions
in modules for name-space management and functor usage, 
(c) automatically filling in defaults for values omitted from
configuration files, and (d) generating complete, stand-alone executables from
declarative descriptions and configurations. 
%Much of the boilerplate 
%elimination is achieved by parsing fml and config files, filling in 
%defaults and automatically generating driver programs for the naive 
%user that string together generic tools.  Other boilerplate is avoided
%by generating both types and values automatically from descriptions
%(particularly datatype descriptions) and packaging them inside modules.

\textbf{\textit{Syntax and simplicity of coding style.}}  The underlying
interfaces are {\em very} higher-order, which, without surface sugar,
would force a complex coding style on the off-the-shelf user.
For instance, almost every line of a description would be translated
to an increasingly nested combinator application, and every variable binding
would induce a use of higher-order abstract syntax. 

\textbf{\textit{Generic programming.}} \ocaml{} (and most other 
potential host languages)
has no direct support for
the generic programming needed to implement the tool suite.  After considerable
study, the most effective way we have found to provide the required 
generic programming interface involves judicious use of unsafe casts
under the covers.  By generating type representations using the compiler,
we guarantee these casts cannot go wrong.

% \textbf{\textit{Compile-time schedule analysis.}} The synchronous
% pair and list feed comprehension combinators are intended for subfeeds
% that share the same schedule.  Using them otherwise is likely an error.
% We anticipate future work on schedule analysis to detect such errors 
% and to help us optimize our implementation.  
% %Such analysis is easier in language framework.  


\textbf{\textit{Integration with \pads{}.}}  Core
\pads{}~\cite{fisher+:pads,fisher+:popl06,fisher+:dirttoshovels,mandelbaum+:pads-ml} has had success as a language extension on top of C as well as
\ocaml{}.  Its purpose is to describe and document properties of ad hoc
data sources as well as to facilitate generation of local, single-source 
tools.  Extending such descriptions to include source location, 
availability and 
access mode helps complete the documentation in a single centralized 
specification and through a uniform notation.  It gives off-the-shelf
users everything they need in a single language.  Forcing a division 
of the specification into part library/part language would ruin its
cohesiveness, particularly in the context of
dependent feeds where there is tight interplay between access mode, location,
schedule and format.


%Having made our case for a language extension, the bulk of our implementation
%{\em is} a collection of libraries.  Hence, the intrepid hacker may eschew our
%surface syntax and program directly against the interfaces underneath.
%Whatever the programmer chooses, the central contribution of this work
%are the abstractions we provide.  Moreover
Though we believe our current design is well motivated,
we also believe the ideas presented here can transcend
their current implementation.  By defining a compact feed calculus 
with a precise semantics, we allow the possibility for
others to embed our abstractions directly in a language
such as Haskell that provides superior support for generic programming. 
%%and the ability to avoid boilerplate by automatically deriving 
%%implementations from type classes.

}%%%%%%%%%%%%%%%%%%%%

\section{SWAT: A Case Study}
\label{sec:swat}
In this section, we describe a case study we have built to show the
use of TxForest in a real-world application.  As mentioned earlier,
SWAT is a tool used by environmental scientists to study the impact of
land use.  Each SWAT datastore contains data related to a single
watershed.  The datastore includes information about topology, land
use, vegetation, precipitation, temperatures, measured waterflows,
\etc{}.  \figref{fig:SWAT-description} gives a TxForest description of
a portion of a SWAT datastore.  

Before environmental scientists can use SWAT to answer questions about
possible crop yields or pollution effects resulting from various land
use policies, they must first calibrate the system so that predicted
values match measured ones.  A key portion of the model that must be
calibrated is the predicted outflow from the watershed.  Basin
parameters that are particularly important in determining outflow
include \texttt{SURLAG}, which is the 
 \textit{surface runoff lag coefficient} that models how much surface
 water reaches the main basin channel 
and \texttt{ESCO}, which is the 
 \textit{soil evaporation compensation factor} that models how ground
 characteristics might impede or enhance evaporation.

Scientists use SWAT as a black box, so to perform calibration, they
perform a search over the parameter space, seeking to minimize the
deviation between the measured outflow and the predicted outflow.
The measured outflow is available in a data file supplied by the
United States Geological Survey.  The predicted outflow is computed by
running SWAT over the datastore with the parameters set to the desired
values.  If the deviation is higher than desired, the scientists
randomly perturb the parameters and rerun the calibration.  

We used TxForest to parallelize this process.  At a high level, the
code spawns a number of threads, each of which searches a portion of
the parameter space.  Each thread atomically copies the SWAT filestore
to a thread-local location.  It then sets the Basin parameters to a
new point in the search space, calls the SWAT executable to calculate
the impact of these parameters, computes the resulting deviation from
the measured flow, and atomically copies the values back to the SWAT
filestore if the new deviation is less than the previously seen
minimum. \figref{fig:SWAT-code} shows the most important pieces of
this code. 

\begin{figure}
\begin{code}
\hscomment{Start numThread optSWAT threads.}
main :: IO ()
main = do
  setDeviation deviationMax deviationFile  
  replicateM_ numThreads \$ forkIO \$ optSWAT

\mbox{}
\hscomment{Each thread runs iterateSwat numIteration
   times in a thread-local space.}
optSWAT :: IO ()
optSWAT = do
  workingDir <- getThreadTempDirectory
  replicateM_ numIterations \$ iterateSWAT workingDir

\mbox{}
\hscomment{Each iteration atomically copies the filestore
   to its working space, modifies the Basin parameters,
   runs SWAT on the working space, calculates the new 
   deviation, and updates the global deviation atomically.}
iterateSWAT :: FilePath -> IO ()
iterateSWAT workingDir = do
  atomically (copySWATData workingDir)
  newBsn <- modifyBsnParams workingDir
  runSWAT workingDir
  newDeviation <- getNewDeviation workingDir
  atomically (updateDeviation newBsn newDeviation)

\mbox{}
\hscomment{Copy SWAT filestore to working filestore.}
copySWATData :: FilePath -> FTM TxVarFS ()
copySWATData workingDir = do
    (orig :: Universal_d TxVarFS) <- new () swatDataDir
    (copy :: Universal_d TxVarFS) <- new () workingDir
    copyOrError orig copy ``Failed to copy SWAT data.''

\mbox{}
\hscomment{Modify basin parameters in working filestore,
   including SURLAG and ESCO parameters.}
modifyBsnParams :: FilePath -> IO SwatLines
modifyBsnParams workingDir = ...

\mbox{}
\hscomment{Calculuate deviation between measured and predicted 
   outflows in the working filestore.}
getNewDeviation :: FilePath -> IO Double
getNewDeviation workingDir = ...

\mbox{}
\hscomment{Run SWAT executable on working filestore. }
runSWAT :: FilePath -> IO ()
runSWAT workingDir = do
  setCurrentDirectory workingDir
  callProcess swatExe []

\mbox{}
\hscomment{Replace the basin file in SWAT filestore if the 
   newDeviation is smaller than the current one.}
updateDeviation :: SwatLines -> Double -> FTM TxVarFS ()
updateDeviation newBsn newDeviation =  do
  (dInfo :: Deviation_f TxVarFS) <- new () deviationFile
  (devMd, Deviation currentDeviation) <- read dInfo
  guard (currentDeviation > newDeviation)
  (swatRep :: Swat_d TxVarFS) <- new () \$ swatDataDir
  (dirMd, dir) <- read swatRep
  (bsnMd, _  ) <- read \$ basin dir
  writeOrError devInfo     (devMd, Deviation newDeviation) 
               ``Failed to write new deviation.''
  writeOrError (basin dir) (bsnMd, SwatFile  newBsn)       
               ``Failed to update SWAT Basin data.''


\end{code}
\caption{TxForest code for calibrating SWAT.}
\label{fig:SWAT-code}
\end{figure}


\note{To be fixed later: Why do we store the deviation in a file
  instead of in the program doing the search?}

\note{We need to have some results about running this process.  How
  fast is it?  How does it compare with what the scientists were doing
previosly?   Ie, is it in the same ballpark?}

Researchers have been studying {\em grammar induction}, the process of
inferring descriptions of text-based data, for decades.  Nevertheless,
the work we present in this paper represents an important and novel 
contribution to the field for three key reasons:

\begin{enumerate}
\item Our system solves {\em a new end-to-end problem} not treated in
past work --- the problem of generating an extensible suite of fully
functional data processing tools directly from ad hoc data.  
%%We can
%%currently generate an XML translator, a normalizing reformatter, a
%%graphing tool, a full query engine allowing users to write arbitrary XQueries
%%against the ad hoc data, an accumulator tool, and
%%programming libraries for parsing, printing and data validation.
Generating this suite requires the combination of three elements:
grammar induction, automatic intermediate representation generation
and type-directed programming.  A key contribution of this work is the
conception, development and evaluation of this end-to-end system.

%%After surveying
%%experts at the CAGI 2007 workshop on grammar induction, where we
%%presented a two-page overview of our system~\cite{burke+:cagi07}, and
%%searching the literature, we could find no existing system that
%%provides this end-to-end functionality.

\item Past work on grammar induction has focused primarily on
either (1) theoretical problems, (2) natural language processing, 
(3) web page analysis, or
(4) XML typing.  Our work tackles an understudied domain, that of complex system
logs and other ad hoc data sources.  Since ad hoc data has
different characteristics from the previously studied domains, naive
adaptations of the existing algorithms are unlikely to be %the most
effective.  
%%As the evaluation in this paper shows, our system is tuned
%%to perform well on ad hoc data, particularly system logs and
%%networking data.  
%%One of the conclusions of the chair of the CAGI 2007
%%workshop, presented in the final discussion session of the workshop,
%%was that ``ad hoc data'' was indeed a new domain for the study of
%%grammar induction and that more research in this area was an important
%%future direction for the community.

\item  From a technical standpoint, we developed a new top-down 
structure-discovery algorithm and showed how to combine that 
productively with a classic bottom-up rewriting system based on 
the minimum description length principle. We demonstrate that our
new algorithm has good practical properties on ad hoc data sources:  
it usually infers correct descriptions on a small amount of training
data and its performance scales linearly relative to the amount of training
data used.
\end{enumerate}

\noindent
%We presented a two-page overview of our system~\cite{burke+:cagi07} at
%the CAGI 2007 workshop on grammar induction. 
In the rest of this section, we analyze
the most closely related work in more depth.

\paragraph*{Traditional Grammar Induction.}
Classic grammar induction algorithms \cite{vidal:gisurvey} 
can be divided into two classes: those that require both
positive and negative examples to discover a grammar and those that
only require positive examples. The problem our system solves is the latter;
negative examples of ad hoc data sources are not available in
practice.  Consequently, effective theoretical algorithms for learning
from both positive and negative
examples such as RPNI~\cite{rpni}
%~\cite{lemay+:tree-transducers,rpni,raeymaekers+:learning-tree-languages},
are not applicable in our context.

Unfortunately, an early result by \citet{gold:inference} showed
that perfect grammar induction is impossible for any superfinite class
of languages when the algorithm has no access to negative examples.  A
{\em superfinite} class of languages is any set of languages that
includes all finite languages and at least one infinite
language. Hence, all the most familiar classes of languages, including
regular expressions, context free grammars and PADS are superfinite.
There are two main tactics one can use to avoid this negative
result: 
(1) use domain knowledge to explicitly limit the class of languages to a
non-superfinite class, or
(2) give up on perfect language identification and instead settle for {\em approximate
identification}~\cite{wharton:approximate-language-identification}
through the use of probabilistic language models.

Examples of non-trivial, non-superfinite
language classes with known inference algorithms include
k-reversible languages~\cite{angluin:revesible-language-inference},
%k-testable regular languages~\cite{garcia+:k-testable-languages},
SOREs and CHAREs~\cite{bex+:dtd-inference}.
None of these languages and the associated algorithms 
are a good fit for inferring PADS descriptions (even the
regular subset of PADS without dependencies and constraints).  
For example, ad hoc data is unlikely to be reversible and hence
k-reversible languages are not relevant.  
%K-testable regular languages are
%more relevant, but algorithms for inferring them
%operate by finding a finite automaton and converting that 
%automaton into a regular expression.  Unfortunately, the conversion process
%often leads to overly verbose regular expressions, sometimes 
%exponential in the size of the automaton~\cite{bex+:dtd-inference}. 
SOREs are a subset of the k-testable
regular languages with a linear-size translation from automata to
regular expressions, but they carry the restriction that each symbol
in the regular expression appear at most once.  A cursory glance at
our hand-written PADS descriptions reveals that many such descriptions
include repeated use of the same symbol.  Finally, it appears that
CHAREs restrict the nesting of regular expression operators too severely to 
be of much use to us.  For example, when $a$, $b$, and $c$ are atomic symbols,
even the simple expression $(ab + c)*$ is not a CHARE.

Given the difficulty of finding useful non-superfinite language classes,
it is reasonable to turn to algorithms for approximate
inference that use probabilistic models.    
Classic examples of such procedures include work by~\citet{stolcke94inducing} 
%%{\em insert other references here -- see Hong thesis related work
%%for other work...} 
and 
\citet{hong:thesis}.  These and a number of other algorithms
operate by repeatedly rewriting a candidate grammar (or set of candidate
grammars) until an objective function is optimized.
If the training data for the learning system is the strings
$s_1$, $s_2$, $\ldots$, $s_n$, these algorithms normally start their
process using the grammar $s_1 + s_2 + \cdots + s_n$.  Consequently,  
an enormous number of different rewrites may apply to the
initial candidate grammar.  Our structure refinement
phase avoids these problems 
because it is preceded by a highly efficient
histogram-based structure-discovery algorithm 
that identifies a good candidate grammar from which to start the search.  
%%Another interesting, non-standard element of our algorithm is the way 
%%it is tuned to include specialized rules for finding constraints and 
%%rewrite tokens.
%%These rules are very useful in the domain of ad hoc data; different
%%considerations are appropriate in other domains such as XML or HTML.

%% is also tuned in a variety of ways to make it effective

%% The effectiveness of structure-discovery allows us to 
%% simplify our search algorithm and cut down the search space we 
%% look at substantially.  In addition to worrying about
%% performance considerations, we tuned our structure refinement phase
%% specifically for ad hoc data by including domain-specific rules
%% for finding constraints and 
%% dependencies as well as those for introducing constants, enumerations and
%% good basic types/tokens that cannot be found effectively at earlier stages.
%% Some of these rules are needed in our system, but not other systems
%% that work in different domains, because
%% tokenization is highly ambiguous in ad hoc data.
%% Our initial tokenization and structure-discovery algorithms often 
%% over-generalize and this over-generalization must be undone during
%% the rewriting phase.  {\em NOTE: end of that paragraph was highly  run-on}

Another category of algorithms are those that learn various kinds of
automata as opposed to regular expressions or 
grammars~\cite{denis:learning-regular-languages,rpni,raeymaekers+:learning-tree-languages}.  
One difficulty with adapting these algorithms to our task is that 
we would need to convert the inferred automata into a 
grammatical representation so that we can
present the result to users and funnel it
to our tool-generation infrastructure.
Unfortunately, in theory, conversion from automata into
regular expressions can result in an exponential blowup in the
size of the representation.
Moreover, a substantial blowup appears to be relatively common in
practice~\cite{bex+:dtd-inference}.  Consequently, these algorithms
are not appropriate for our domain.



%% developed another system for information extraction from web pages
%% based on learning ($k$,$l$)-Contextual Tree Languages.  They show that
%% these tree languages can be learned from positive examples 
%% (from which one may infer they are not superfinite) and apply
%% their techniques to the problem of information extraction.  One of the
%% difficulties they face involves estimating the parameters involved in

  
%% {\em NOTE: I'm leaving out reference to denis:learning-regular-languages (see
%% pads.bib file).
%% Vincent Danos mentioned it but it contains no references to any real data.
%% It's interesting theoretical result that infers a new kind of automaton.
%% This automaton will likely have the same potential difficulties 
%% (possibly exponential explosion -- I haven't prove that though) 
%% when conversion to regular expressions happens. Anyway, I just didn't
%% want to bother studying the paper because they're so much other stuff
%% that is more relevant.  basically, I just couldn't figure out how to cram
%% the reference in elegantly.  I actually couldn't even figure out when I skimmed
%% the paper whether or not it uses both positive and negative examples.
%% It gets compared to RPNI, so I think it must use negative examples.
%% }

%% One disadvantage of such
%% techniques is that the initial state is large (representing
%% the entire training data set explicitly) and the search space is 
%% enormous.  Nevertheless, bottom-up state-merging is often used because
%% it has been difficult to find an effective state-splitting algorithm.
%% Our histogram-based structure-discovery procedure is a new state-splitting
%% algorithm that appears to work well on ad hoc data when coupled with
%% bottom-up rewriting.


%% The classic grammar induction problem~\cite{vidal:gisurvey} requires we find an
%% algorithm that discovers a grammar $G$ given a set of
%% positive examples $R+$ (example strings in the language to be inferred)
%% and a set of negative examples $R-$ (example strings {\em not}
%% in the language to be inferred).  To be more specific, in the limit,
%% as the sets of positive and negative examples grow, the
%% algorithm is expected to converge on the language that defines them.  
%% Unfortunately, very early on,
%% Gold~\cite{gold:inference} proved a key negative result about this problem:  If
%% the algorithm is presented with no negative examples, grammar
%% induction for any super-finite class of languages is impossible.
%% A {\em super-finite} class of languages is any set of languages
%% that includes all finite languages and at least one infinite language.
%% All the most familiar classes of languages, including regular expressions, 
%% context free grammars and PADS, fall into this class.

%% Traditional
%% Some traditional grammar induction algorithms assume that
%% both positive and negative training data
%% One way to categorize research in traditional
%% grammar induction is to ask whether the research in question
%% assumes that both positive and negative training data is available
%% or whether only positive training data is available.

%% analyze the assumptions made
%% about the training data.  
%% Very early in the study of grammar induction, Gold proved a key
%% negative result:


%% Other researchers have defined grammar induction algorithms that use
%% bottom-up rewriting to search through description space for an optimal
%% description.  Many of these techniques, such as 
%% require the availability of both
%% positive and negative examples.  In our context, negative examples
%% never exist, making such techniques inapplicable.
%% % since Gold's early result proved the
%% %impossibility of {\em perfect} grammar induction for any useful family of
%% %languages when no negative examples are
%% %available~\cite{gold:inference}.  
%% However, others, such as Stolcke and
%% Omohundro~\cite{stolcke94inducing} and Hong~\cite{hong01using}, do not
%% assume the existence of negative examples.  These and a number of other systems
%% search through solution space using
%% state-merging rewriting rules.  One disadvantage of such
%% techniques is that the initial state is large (representing
%% the entire training data set explicitly) and the search space is 
%% enormous.  Nevertheless, bottom-up state-merging is often used because
%% it has been difficult to find an effective state-splitting algorithm.
%% Our histogram-based structure-discovery procedure is a new state-splitting
%% algorithm that appears to work well on ad hoc data when coupled with
%% bottom-up rewriting.

% State-merging rewriting rules seem to be
% more popular than 
% that use bottom-up rewriting to find good grammars may suffer from the
% problem of running into local maxima.  The rewriting component of our
% algorithm can also run into a local maximum, but because we start with
% a relatively good candidate generated from our recursive, top-down
% algorithm, this does not appear to be much of a problem for us.  We
% also believe that combining top-down structure-discovery with
% bottom-up rewriting has the potential to deal with larger data sources
% than a pure bottom-up approach.  Our empirical experiments demonstrate
% that the top-down structure-discovery phase is extremely efficient
% when compared with the cost of rewriting.  However, proposals for
% bottom-up-only inference techniques use the (possibly enormous) data
% source itself as the first description.  We are unaware of other
% systems that combine two techniques similar to ours.


%% ; De La Higuera
%% surveys some recent trends~\cite{higuera01current}.  However,
%% our system is unique in two important ways.  First, our inference
%% algorithm does not stand alone; it is part of the more general \pads{}
%% programming environment.  The fusion of the
%% \pads{} system, including its automatic data representation generation,
%% its error detection facilities, its generic programming environment, 
%% and its powerful tool suite, together with grammar induction
%% is one of our key contributions.  Second, many researchers have
%% focused either on grammar induction for natural language processing or
%% for information extraction from \xml{} or \html{} documents.  In
%% contrast, we focus on ad hoc data sources such as system logs and
%% scientific data sets. Ad hoc data is substantially less
%% structured syntactically than \xml{}, and yet, unlike natural language, it is
%% possible to assign our data sources accurate, compact descriptions. After
%% searching the literature and consulting
%% with experts in grammar induction at the CAGI 2007
%% workshop, where we presented a two page overview of our system~\cite{burke+:cagi07},
%% we could find no existing work comparable to ours.

% Third, from a
% technical standpoint, we developed a new top-down structure-discovery
% algorithm and showed how to combine that productively with a
% classic bottom-up rewriting systems based on the minimum description
% length principle.  In what follows, we compare our system more
% specifically to the most closely related work of other researchers.

\paragraph*{Information Extraction.}

The basic goal of an information extraction system is to find and
separate the interesting and relevant bits of information (the
needles) from a haystack of data.  Such systems are fundamentally
different from ours, in that they choose which bits of information to
extract, while we learn a description of the entirety of a data
source, leaving the choice about which pieces are interesting to
down-stream applications.  Of course, this option is only feasible
because we target ad hoc data, which is fairly structured and dense in
useful information, rather than web pages or free text, which are the
usual targets for information extraction systems. 

A common approach to information extraction involves an inductive
learning process in which a user manually tags the relevant data in sample documents.
An example might be highlighting product names and prices on a
collection of shopping web pages from a particular site.  The learning
system then uses these labelled documents in two ways: first, to
decide which bits of information should be extracted from the page
(\ie, product names and prices), and second, to construct a
\textit{wrapper} function to extract those bits of information from
similar pages.  Soderland's WHISK system (\citeyear{soderland:whisk}) is an
example of such an extraction system.  It is particularly general as
it makes few assumptions about the form of the source text,
operating over structured data, stylized text such as Craig's List
descriptions, or free-form text.  WHISK differs from our system in
that it requires user labeling and then only extracts a collection of
tuples from the data source rather than returning the complete
structure of the data source.


Kushmerick and
colleagues (\citeyear{kushmerick-phd1997,KushmerickWD97:Wrapper}) focus on
more structured data to reduce the amount of labeling required during
training.  In particular, this work assumes the labelled pages conform
to one of six different templates, the most well-developed of which
has the form of a header, followed by a sequence of K-tuples each of
which is flanked by a pair of begin and end tags, followed by a
trailer.  For such documents, the system generates a wrapper to
extract the K-tuples.  
% To limit the amount of labeling, the system
% has provisions to automatically tag the desired tuples using {\it
% recognizers}, which are imperfect, but reusable heuristics for finding
% atomic pieces of data such as country names or phone numbers.  Such
% recognizers mean the user has to select which set of recognizers to
% use for a particular extraction task instead of labeling pages by
% hand.  The system requires one recognizer to be {\it perfect}, meaning
% it generates neither false positives nor false negatives.  It then
% uses a process called {\it corroboration} to correct the mistakes of
% the other recognisers.  The system is not robust in the presence of
% missing data, and it is not clear how it would handle multiple
% instances of the same kind of data within a single tuple. This
% approach differs from ours in that it requires the data to comform to
% one of a fixed collection of templates.  In addition, the 
% templates that support corroboration will only return relational data,
% whereas our system will return semi-structured data.
The use of fixed templates and the primary focus on relational data makes this
work quite different from ours.

\citet{muslea+:active-learning} tackle a similar
problem, but strive to reduce the amount of labeling by having the
learning system chose which documents to have the user label,
selecting documents by their probative value.  \citet{borkar+:text-segmentation} uses hand-labelled training
examples and a user-specified set of desired features to train Hidden
Markov Models to select the desired features from similar documents.
This work is quite successful at learning to select the relevant
features of addresses and bibliographic citations from a variety of
input formats. 
% Various researchers have leveraged the syntactic
% regularity and verbosity of XML/HTML to reduce the amount of user
% annotations required to train information extraction systems targetted
% at web pages~\cite{Ambite+:ariadne,doorenbos+:shopbot}.
% Work by Ireson {\em et al.}~\cite{ireson+:ml-evaluation} investigates
% how information extraction systems should be evaluated.  Soderland's
% WHISK paper~\cite{soderland:whisk} and Kushmerick's
% theis~\cite{kushmerick-phd1997} both contain detailed descriptions of
% other information extraction systems.  
In general, systems that depend
upon labeling are unlikely to be helpful in our context; rather than
spending time explicitly labeling documents, the user might as well
write a PADS description by hand.

% Another type of information extraction system strives to provide a
% high-level semantic characterization of the content of natural
% language documents to guide information retrieval
% queries~\cite{gubanov+:structural-text-search,rus+:information-capture}.
% This work differs from ours in that it is building a semantic rather
% than a syntatic description of the source data.

More closely related are various efforts to identify tabular data 
either from free-form text~\cite{Ng+:texttables,Pinto+:texttables} or
from web pages~\cite{Lerman+:webtables}.  These approaches typically
use hand-labelled examples to train machine learning systems to
identify the tables.  They then use heuristics specific to tabular
data to extract the tuples contained within those tables.  The portion
of this work related to identifying structured data from within more
free-form documents is complementary to ours.  The portion responsible
for deconstructing the identified tables uses more specific
domain-knowledge related to the form of tables than we do.

Web pages generated in response to queries tend to be formed by
sloting the resulting tuples into a standard template.  Another line
of work aims to separate such templates from the payload
data~\cite{arasu+:sigmod03,Cresenzi+:roadrunner}.  
Arasu and Garcia-Molina %~\cite{arasu+:sigmod03}
use a top-down grammar induction
algorithm somewhat similar to our rough structure-inference phase
(though it does not use histograms),
but has no description-rewriting engine.  
%However, in certain ways, Arasu has a much easier task than we do as html
%documents have far more regular structure than ad hoc data sources do.
This algorithm exploits the hierarchical nesting
structure of \xml{} documents in essential ways
and so cannot be applied directly to ad hoc data.  
%For example,
%we use histograms to summarize the contents of data chunks whereas
%Arasu does not.  In addition, a substantial portion of our system
%is a description rewriting engine, which Arasu seems not to need.  






% For further reading on
% information extraction from web pages, Hong's
% thesis~\cite{hong:thesis} includes an informative survey.  Though,
% Arasu's work and TSIMMIS appear more closely related to our work than
% the others Hong mentions.


\paragraph*{XML Type Inference.}
Many researchers have studied the problem of learning
a schema such as a DTD or XSchema from a collection 
of XML
documents~\cite{bex+:dtd-inference,bex+:inferring-xml-schema,fernau:learning-xml,garofalakis+:xtract}.  
At a high level, this task is similar to the format inference component of our system.  
However, the details differ because XML has different characteristics
from ad hoc data.  One difference is that XML documents come in a well-nested tree 
shape, with obvious delimiters defining the structure.  
A second important difference is that the appropriate tokenization for
a given ad hoc data source is often not known in advance.  
%%One of our strategies for dealing with
%%these ambiguities is to define simple approximate tokens for use
%%in the tokenization phase, but then to employ a collection of 
%%rules to improve token ({\em i.e.}, base type) choices in the rewriting phase
%%when more contextual information is available.  
In contrast,
tokens in XML documents are clearly demarcated using angle bracket syntax.
%%A third difference is that XML documents are often organized such that
%%the structure of a child node is dependent on its parent or grandparent.
%%In contrast, in the flatter ad hoc data we have considered, dependencies
%%generally arise between siblings -- some data item to the left influences
%%the structure of data to the right.  
As a result of these differences,
XML inference algorithms cannot be used ``off-the-shelf'' for understanding
the structure of ad hoc data.  They must be modified, tuned and
empirically evaluated on this new task.

One line of research on schema inference for XML makes use of the 
observation that 99\% of the content models for XML nodes are defined as
SOREs or CHAREs~\cite{martens+:expressiveness-xml-schema}. 
%(recall, these
%are heavily restricted forms of regular expressions).  
This observation allows \citet{bex+:dtd-inference} to define
an efficient algorithm for inferring concise DTDs.  Later 
\citet{bex+:inferring-xml-schema} build on this work 
by showing how to infer $k$-local XML Schema definitions also based on
SORES.  A $k$-local definition allows node content to depend on the parent
tag, grandparent tag, etc. (up to $k$ levels for some fixed $k$).
As mentioned earlier, hand-written PADS descriptions do not generally obey
the SOREs or CHAREs restriction, nor are they generally arranged with a nesting
structure that suggests $k$-local inference will be particularly useful.
The successful application of these techniques to XML data reinforces 
the idea that the ad hoc data we analyze has quite different characteristics
from XML, and therefore the ad hoc data inference problem merits study
independent of the XML inference problem.

XTRACT~\cite{garofalakis+:xtract} is another system for inferring DTDs
for XML documents.  It operates in three phases: generalization,
factoring and MDL optimization.  The first phase plays a role similar to
our structure discovery phase in that it generates a
collection of candidate structures from a series of XML examples.
This generalization phase searches for patterns in XML
data; it is tuned using the authors' knowledge of common DTD
structures.  Factoring decreases the size of generated candidate DTDs;
some of the factoring rules resemble our rewriting rules.
Finally, they tackle the MDL optimization problem by mapping the
problem into an instance of the NP-complete Facility Location Problem,
which they solve using a quadratic approximation algorithm.
Our MDL-guided rewriting problem considers a more general set of
rewriting rules and hence we cannot reuse their technique.

%% Another related problem of great interest in the XML and database world
%% involves finding a mapping between two data sources with different schema.
%% \citet{doan+:disparate-data-sources} is one example amongst
%% many which attempts to solve this problem using a machine learning approach.
%% While some of our PADS tools do involve translations between
%% different formats, our learning system does not attempt to discover
%% translation tools for which the output is guaranteed to match 
%% the characteristics of a second data set.


\paragraph*{Other work.}
Potter's Wheel~\cite{raman+:potterwheel} is a system that attempts to
help users find and purge errors from
relational data sources.  It does so through the use of a spread-sheet
style interface, but in the background, a grammar inference algorithm
infers the structure of the input data, which may be ``ad hoc,'' 
somewhat like ours.  This inference algorithm operates by
enumerating all possible sequences of base types that appear
in the training data.  
%As in our work,
%users can specify custom base types, and search for a description
%is based on the minimum description length principle.  
Since Potter's Wheel is aimed at processing
relational data, they only infer \cd{struct} types
as opposed to enumerations, arrays, switches or unions.  

The TSIMMIS project~\cite{chawathe+:tsimmis} aims to
allow users to manage and query collections of heterogeneous, ad hoc
data sources.  TSIMMIS sits on top of the Rufus
system~\cite{shoens+:rufus}, which supports automatic classification
of data sources based on features such as the presence of certain
keywords, magic numbers appearing at the beginning of files and file
type.  
%The sources are classified using categories such as ``email''
%and ``C program.''  
This sort of classification is materially
different from the syntactic analysis we have developed.

\section{Conclusion} 
\label{sec:conclusion}

Ad hoc data is pervasive and valuable: in industry, in medicine, and
in scientific research.  Such data tends to have poor documentation,
to contain various kinds of errors, and to be voluminous.  Unlike
well-behaved data in standardized relational or \xml{} formats, such
data has little or no tool support, forcing data analysts and
scientists to waste valuable time writing brittle custom code, even if
all they want to do is convert their data into a well-behaved format.
To improve the situation, various researchers have developed data
description languages such as \pads{}, \datascript{}, and
\packettypes{}.  Such languages allow analysts to write terse,
declarative descriptions of ad hoc data.  A compiler then generates a
parser and customized tools.  Because these languages are tailored to
their domain, they can provide useful services automatically while a
more general purpose tool, such as \lex{}/\yacc{} or \perl{}, cannot.

In the spirit of Landin, we have taken the first steps toward
specifying a semantics for this class of languages by defining the
data description calculus \ddc{}.  This calculus, which is a dependent
type theory with a simple set of orthogonal primitives, is expressive
enough to describe the features of \pads{}, \datascript{}, and
\packettypes{}.  In keeping with the spirit of the data description
languages, our semantics is transformational: instead of simply
recognizing a collection of input strings, we specify how to transform
those strings into canonical in-memory representations annotated with
error information.  Furthermore, we prove that the error information
is meaningful, allowing analysts to rely on the error summaries rather
than having to re-vet the data by-hand.

We have already used the semantics to identify bugs in the
implementation of \padsc{} and to highlight areas where \padsc{}
sacrifices safety for speed.  We have also used the semantics as a guide
for the design of a whole new language, \padsml{}, designed
specifically for functional programmers.  In the future, we hope 
\ddc{} will serve as a solid foundation for the next 700 data 
description languages to come.
 
\bibliographystyle{plain} 
\balance  
\bibliography{main}

\section*{Appendix}
\label{sec:appendix}
\appendix
\section{Language Syntax}
{\allowdisplaybreaks
\noindent
{\bf Syntax of data descriptions and other types}
\label{app:syntax-dd}
\begin{bnf}
\name{Constants} \meta{k} \::= \mcd{true} \| \mcd{false} \| \mcd{()} \| ...
\\
\name{Type Variables} \meta{\alpha}
\\
\name{Type Names} \meta{t}
\\
\name{\Core{} Types} \meta{T} \::= 
  \alpha 
\| {Pbase} 
\| M 
\nlalt \ppair x {T_1} {T_2} 
\| \precord {\nont{ffts}} 
\| \nont{tas}\;t(M) 
\nlalt \pset x T M 
\nlalt \parray T {M_{sep}} {M_{term}} 
\\
\name{\Core{} Datatypes} \meta{D} \::= 
  \mcd{datatype}\; \nont{tps}\; t(x{:}F) = \nont{b} \nlalt
  \mcd{type}\; \nont{tps}\; t(x{:}F) = T
\\
\name{Type Parameters} \meta{tps} \::= \cdot \| \alpha \| (\nont{tvs})
\\
\name{} \meta{tvs} \::= \alpha \| \alpha,\, \nont{tvs}
\\
\name{Type Arguments} \meta{tas} \::= \cdot \| T \| (\nont{ts})
\\
\name{} \meta{ts} \::= T \| T,\, \nont{tss}
\\
\name{} \meta{b} \::= \nont{cs} \| \mcd{case}\; M\; \mcd{of}\; \nont{ccs}
\\
\name{} \meta{cs} \::= c\;\mcd{of}\;T \| c\;\mcd{of}\;T \cvb \nont{cs}
\\
\name{} \meta{ccs} \::= 
  \nont{pat} \Rightarrow c\;\mcd{of}\;T \nlalt
  \nont{pat} \Rightarrow c\;\mcd{of}\;T \cvb \nont{ccs}
\\
\name{\Core{} Field Types} \meta{ffts} \::= \nont{fft} \| \nont{fft};\;\nont{ffts}
\\
\name{\Core{} Field Type} \meta{fft} \::= T \| x = T
\end{bnf}
%\begin{bnf}
\name{Constants} \meta{k} \::= \mcd{true} \| \mcd{false} \| \mcd{()} \| ...
\\
\name{Type Variables} \meta{\alpha}
\\
\name{Type Names} \meta{t}
\\
\name{\Core{} Types} \meta{T} \::= 
  \alpha 
\| {Pbase} 
\| M 
\nlalt \ppair x {T_1} {T_2} 
\| \precord {\nont{ffts}} 
\| \nont{tas}\;t(M) 
\nlalt \pset x T M 
\nlalt \parray T {M_{sep}} {M_{term}} 
\\
\name{\Core{} Datatypes} \meta{D} \::= 
  \mcd{datatype}\; \nont{tps}\; t(x{:}F) = \nont{b} \nlalt
  \mcd{type}\; \nont{tps}\; t(x{:}F) = T
\\
\name{Type Parameters} \meta{tps} \::= \cdot \| \alpha \| (\nont{tvs})
\\
\name{} \meta{tvs} \::= \alpha \| \alpha,\, \nont{tvs}
\\
\name{Type Arguments} \meta{tas} \::= \cdot \| T \| (\nont{ts})
\\
\name{} \meta{ts} \::= T \| T,\, \nont{tss}
\\
\name{} \meta{b} \::= \nont{cs} \| \mcd{case}\; M\; \mcd{of}\; \nont{ccs}
\\
\name{} \meta{cs} \::= c\;\mcd{of}\;T \| c\;\mcd{of}\;T \cvb \nont{cs}
\\
\name{} \meta{ccs} \::= 
  \nont{pat} \Rightarrow c\;\mcd{of}\;T \nlalt
  \nont{pat} \Rightarrow c\;\mcd{of}\;T \cvb \nont{ccs}
\\
\name{\Core{} Field Types} \meta{ffts} \::= \nont{fft} \| \nont{fft};\;\nont{ffts}
\\
\name{\Core{} Field Type} \meta{fft} \::= T \| x = T
\end{bnf}
%%% Local Variables: 
%%% mode: latex
%%% TeX-master: "paper"
%%% End: 


\noindent
{\bf Syntax of terms}
\label{app:syntax-terms}
\begin{bnf}
\name{Types} \meta{F} \::= 
  T           \descr{type of \pvalue{}} 
\nlalt \nont{base} \descr{values of ordinary base types} 
\nlalt \mcd{PD}    \descr{PD type}
\nlalt F * F       \descr{ordinary pairs} 
\nlalt \{\nont{fts}\}     \descr{ordinary records} 
\nlalt F \-> F     \descr{functions}
\nlalt \pstream F  \descr{streams}
\\
\name{Field Types} \meta{fts} \::= x = F \| x=F,\;\nont{fts}
\\
%\end{bnf}
%\begin{bnf}
\name{Parse Descriptors} \meta{pd} \::=   
  G \| B \| N \| S \| U
\\
\name{\Core{} Terms} \meta{N} \::=  
       Pbase[M_1](M_2)                \descr{base type constructor}
\nlalt \langle M \rangle              \descr{unit value (with singleton type M)}
\nlalt (x{=}{M_1} \mathrel{**} {M_2}) \descr{pair}
\nlalt \lcr \nont{fs} \rcr            \descr{record}
\nlalt c[M_1](M_2)                    \descr{data type constructor}
\nlalt \{x = {M_1} \cvb {M_2}\}       \descr{constrained type, with
  $M_2$ the constraint}
\nlalt \mcd{Parray}(M, M_{sep}, M_{term})   \descr{array; first element is stream}
\end{bnf}

\newpage

\begin{bnf}
\name{Terms} \meta{M} \::= 
       x                        \descr{variable}
\nlalt N                        \descr{\core{} terms}
\nlalt k                        \descr{constants}
\nlalt \nont{pd}                      \descr{pd value}
\nlalt (M_1 * M_2)              \descr{ordinary pair}
\nlalt \{\nont{fs}\}        \descr{ordinary record}
\nlalt \tfun {x_1}{x_2}{F_1}{F_2}{M}     \descr{recursive function x1 with arg x2}
\nlalt \mcd{nil}                \descr{empty stream}
\nlalt M_1 \mathrel{::} M_2     \descr{cons}
\nlalt \mcd{case}\;M\;\mcd{of}\;\nont{ms} \descr{deconstructors}
\nlalt M_1\;(M_2)               \descr{function application}
\nlalt \mcd{op}\;M                     \descr{additional uninteresting operations}
\nlalt \mcd{let}\;x = M_1\;\mcd{in}\;M_2           \descr{computation in host language}
\nlalt \mcd{cast}\;(M : T)             \descr{type annot/dependent cast?}
\\
\name{Fields} \meta{fs} \::= x = M \| x = M; \nont{fs}
\\
\name{Matches}\meta{ms} \::= 
  \nont{pat} \Rightarrow M \| \nont{pat} \Rightarrow M \cvb \nont{ms}
\end{bnf}
%{\small
\begin{verbatim}
M ::=  x                        // variable
     | Pbase[M1](M2)            // base type constructor; M1 is
                                   an argument to the type; M2 computes the rep)
     | c(M)                     // data type constructor with parameter M
     | (x:M1 ** M2)             // pair
     | {fields}                 // record
     | {x = M1 | M2}            // set-type; M2 is the predicate
     | Parray(M, Msep, Mterm)   // array; first element is stream
     | k                        // constants
     | let x = M in M           // computation in host language
     | <M>                      // unit value given singleton type M
     | (M1 * M2)                // ordinary pair
     | nil                      // empty list
     | M1 :: M2                 // cons
     | case M of MS             // deconstructors
     | fun x1(x2:F1):F2 = M     // recursive function x1 with arg x2
     | M1 (M2)                  // function application
     | cast (M : T)             // type annot/dependent cast?
     | op M                     // additional uninteresting operations
     
fields ::= x = M | x = M; fields

pd ::=   G    // good
     |   B    // bad
     |   N    // nested error
     |   S    // semantic error
     |   U    // unknown
\end{verbatim}
}

  
\newpage

\noindent
{\bf Syntax of patterns}
\label{app:syntax-pat}
\begin{bnf}
% \name{Parse Descriptors} \meta{pd} \::=   
%          G    \descr{good}
% \nlalt   B    \descr{bad}
% \nlalt   N    \descr{nested error}
% \nlalt   S    \descr{semantic error}
% \nlalt   U    \descr{unknown}
\name{\Core{} Patterns} \meta{fpat} \::=
x \| \nont{Pbase}(\nont{pat})
\nlalt \langle \nont{pat} \rangle             \descr{singleton}
\nlalt (\nont{fpat} \mathrel{**} \nont{fpat})   \descr{\core{} pair}
\nlalt \lcr \nont{ffps} \rcr                  \descr{\core{} record}
\nlalt c(\nont{fpat})                          \descr{constructor}
\nlalt \{\nont{fpat} \cvb \nont{cpat}\}        \descr{type constaint}
\nlalt \mcd{Parray}(\nont{pat}, x_{sep}, x_{term}) \descr{array with stream, sep and term.}
\nlalt \nont{fpat}\langle\langle\nont{pdpat}\rangle\rangle
\\
\name{Patterns}\meta{pat} \::= 
       \nont{fpat} \descr{\core{} pattern}
\nlalt k \| \nont{pdpat}                      \descr{constants and parse descriptors}
\nlalt (\nont{pat} * \nont{pat})              \descr{normal pair}
\nlalt \{\nont{fps}\}                        \descr{record}
\nlalt \mcd{nil} \| \nont{pat}_1 \mathrel{::} \nont{pat}_2 \descr{stream}
\\
\name{\Core{} Field Pattern} \meta{ffps} \::= x = \nont{fpat} \| x = \nont{fpat};\;\nont{ffps}
\\
\name{Constraint Pattern} \meta{cpat} \::= x \| \mcd{true} \| \mcd{false}
\\
\name{PD Pattern} \meta{pdpat}\::= x \| \nont{pd}
\\
\name{Field Pattern} \meta{fps} \::= x = \nont{pat} \| x = \nont{pat};\;\nont{fps}
\end{bnf}
%{\small
\begin{bnf}
% \name{Parse Descriptors} \meta{pd} \::=   
%          G    \descr{good}
% \nlalt   B    \descr{bad}
% \nlalt   N    \descr{nested error}
% \nlalt   S    \descr{semantic error}
% \nlalt   U    \descr{unknown}
\name{\Core{} Patterns} \meta{fpat} \::=
x \| \nont{Pbase}(\nont{pat})
\nlalt \langle \nont{pat} \rangle             \descr{singleton}
\nlalt (\nont{fpat} \mathrel{**} \nont{fpat})   \descr{\core{} pair}
\nlalt \lcr \nont{ffps} \rcr                  \descr{\core{} record}
\nlalt c(\nont{fpat})                          \descr{constructor}
\nlalt \{\nont{fpat} \cvb \nont{cpat}\}        \descr{type constaint}
\nlalt \mcd{Parray}(\nont{pat}, x_{sep}, x_{term}) \descr{array with stream, sep and term.}
\nlalt \nont{fpat}\langle\langle\nont{pdpat}\rangle\rangle
\\
\name{Patterns}\meta{pat} \::= 
       \nont{fpat} \descr{\core{} pattern}
\nlalt k \| \nont{pdpat}                      \descr{constants and parse descriptors}
\nlalt (\nont{pat} * \nont{pat})              \descr{normal pair}
\nlalt \{\nont{fps}\}                        \descr{record}
\nlalt \mcd{nil} \| \nont{pat}_1 \mathrel{::} \nont{pat}_2 \descr{stream}
\\
\name{\Core{} Field Pattern} \meta{ffps} \::= x = \nont{fpat} \| x = \nont{fpat};\;\nont{ffps}
\\
\name{Constraint Pattern} \meta{cpat} \::= x \| \mcd{true} \| \mcd{false}
\\
\name{PD Pattern} \meta{pdpat}\::= x \| \nont{pd}
\\
\name{Field Pattern} \meta{fps} \::= x = \nont{pat} \| x = \nont{pat};\;\nont{fps}
\end{bnf}
}

%%% Local Variables: 
%%% mode: latex
%%% TeX-master: "paper"
%%% End: 


\noindent
{\bf Syntax of programs}
\label{app:syntax-prog}
\begin{bnf}
\name{Program} \meta{prog} \::= 
  M              
 \nlalt D\; \nont{prog}          \descr{type declaration}
 \nlalt \mcd{val}\;x = M\;\mcd{prog} \descr{value declaration}
\end{bnf}
%{\small
\begin{verbatim}
prog ::= M              
       | D prog          // type declaration
       | val x = M prog  // value declaration
\end{verbatim}
}

}
%%% Local Variables: 
%%% mode: latex
%%% TeX-master: "paper"
%%% End: 


\end{document}
